\documentclass{exercise}

\institute{Applied and Computational Mathematics}
\title{Hausaufgabenübung 1}
\author{Joshua Feld, 406718}
\course{Mathematische Grundlagen IV}
\professor{Torrilhon}
\semester{Sommersemester 2022}
\program{CES (Bachelor)}

\begin{document}
    \maketitle


    \section*{Aufgabe 1}

    \begin{problem}
        Man berechne in Abhängigkeit von den Werten \(\alpha \in \R\) und \(\beta \in \R\) alle Lösungen des Randwertproblems
        \[
            -u''\parentheses*{x} = 1, \quad \forall x \in \parentheses*{0, 1}
        \]
        zusammen mit den Randbedingungen
        \begin{enumerate}
            \item \(u\parentheses*{0} = \alpha\), \(u\parentheses*{1} = \beta\),
            \item \(u\parentheses*{0} = \alpha\), \(\frac{\partial u}{\partial n}\parentheses*{1} = \beta\),
            \item \(\frac{\partial u}{\partial n}\parentheses*{0} = \alpha\), \(u\parentheses*{1} = \beta\),
            \item \(\frac{\partial u}{\partial n}\parentheses*{0} = \alpha\), \(\frac{\partial u}{\partial n}\parentheses*{1} = \beta\).
        \end{enumerate}
    \end{problem}

    \subsection*{Lösung}
    Integration ergibt \(u'\parentheses*{x} = c_1 - x\) und \(u\parentheses*{x} = -\frac{1}{2}x^2 + c_1 x + c_2\).
    Folglich sind
    \begin{align*}
        u\parentheses*{0} &= c_2, & \frac{\partial u}{\partial n}\parentheses*{0} = -u'\parentheses*{0} = -c_1,\\
        u\parentheses*{1} &= c_1 + c_2 - \frac{1}{2}, & \frac{\partial u}{\partial n}\parentheses*{1} = u'\parentheses*{1} = c_1 - 1.
    \end{align*}
    \begin{enumerate}
        \item
        \begin{align*}
            \alpha &\stackrel{!}{=} u\parentheses*{0} = c_2,\\
            \beta &\stackrel{!}{=} u\parentheses*{1} = c_1 + c_2 - \frac{1}{2}.
        \end{align*}
        Es gibt genau eine Lösung
        \[
            c_2 = \alpha, \quad c_1 = \beta + \frac{1}{2} - c_2 = \beta - \alpha + \frac{1}{2},
        \]
        also
        \[
            u\parentheses*{x} = -\frac{1}{2}x^2 + \parentheses*{\beta - \alpha + \frac{1}{2}}x + \alpha
        \]
        \item
        \begin{align*}
            \alpha &\stackrel{!}{=} u\parentheses*{0} = c_2,\\
            \beta &\stackrel{!}{=} \frac{\partial u}{\partial n}\parentheses*{1} = c_1 - 1.
        \end{align*}
        Es gibt genau eine Lösung
        \[
            c_2 = \alpha, \quad c_1 = \beta + 1,
        \]
        also
        \[
            u\parentheses*{x} = -\frac{1}{2}x^2 + \parentheses*{\beta + 1}x + \alpha.
        \]
        \item
        \begin{align*}
            \alpha &\stackrel{!}{=} \frac{\partial u}{\partial n}\parentheses*{0} = -c_1,\\
            \beta &\stackrel{!}{=} u\parentheses*{1} = c_1 + c_2 - \frac{1}{2}.
        \end{align*}
        Es gibt genau eine Lösung
        \[
            c_1 = -\alpha, \quad c_2 = \alpha + \beta + \frac{1}{2},
        \]
        also
        \[
            u\parentheses*{x} = -\frac{1}{2}x^2 - \alpha x + \alpha + \beta + \frac{1}{2}.
        \]
        \item
        \begin{align*}
            \alpha &\stackrel{!}{=} \frac{\partial u}{\partial n}\parentheses*{0} = -c_1,\\
            \beta &\stackrel{!}{=} \frac{\partial u}{\partial n}\parentheses*{1} = c_1 - 1.
        \end{align*}
        Aus den Gleichungen folgt
        \[
            c_1 = -\alpha \quad \text{und} \quad c_1 = \beta + 1.
        \]
        Falls \(-\alpha = \beta + 1\) gibt es unendlich viele Lösungen mit
        \[
            u\parentheses*{x} = -\frac{1}{2}x^2 - \alpha x + c_2, \quad c_2 \in \R,
        \]
        ansonsten gibt es keine Lösung.
    \end{enumerate}


    \section*{Aufgabe 2}

    \begin{problem}
        Bestimmen Sie zu den Daten
        \begin{center}
            \begin{tabular}{rcccc}
                \toprule
                \(x_k\) & \(0\) & \(\frac{\pi}{2}\) & \(\pi\) & \(\frac{3\pi}{2}\)\\
                \midrule
                \(y_k\) & \(3\) & \(2\) & \(3\) & \(4\)\\
                \bottomrule
            \end{tabular}
        \end{center}
        ein komplexes trigonometrisches Polynom der Form
        \[
            T_4\parentheses*{y; x} = \sum_{k = 0}^3 c_k e^{ikx},
        \]
        das die Daten interpoliert, also \(T_4\parentheses*{y; x_k} = y_k\), für \(k = 0, \ldots, 3\).
        Stellen Sie dazu das zugehörige lineare Gleichungssystem auf und lösen Sie es direkt.
    \end{problem}

    \subsection*{Lösung}
    Das lineare Gleichungssystem, das dem Interpolationsproblem entspricht, lautet:
    \[
        \begin{pmatrix}
            1 & 1 & 1 & 1\\
            1 & i & -1 & -i\\
            1 & -1 & 1 & -1\\
            1 & -i & -1 & i
        \end{pmatrix}\begin{pmatrix}
            c_0\\c_1\\c_2\\c_3
        \end{pmatrix} = \begin{pmatrix}
            3\\2\\3\\4
        \end{pmatrix}.
    \]
    Dieses lässt sich mit dem Gaußschen Eliminationsverfahren lösen, wobei sich im letzten Schritt die Matrix
    \[
        \parentheses*{\begin{array}{cccc|c}
            1 & 1 & 1 & 1 & 3\\
            0 & -2 & 0 & -2 & 0\\
            0 & 0 & -2 & -2i & -1\\
            0 & 0 & 0 & 4i & 2
        \end{array}}.
    \]
    ergibt.
    Nun erhalten wir durch Rückwärtseinsetzen
    \begin{align*}
        4ic_3 = 2 &\iff c_3 = -\frac{i}{2},\\
        -2c_2 - 2ic_3 = -1 &\iff 2c_2 + 1 = 1 \iff c_2 = 0,\\
        -2c_1 - 2c_3 = 0 &\iff c_1 = -c_3 = \frac{i}{2},\\
        c_0 + c_1 + c_2 + c_3 = 3 &\iff c_0 = 3.
    \end{align*}
    Das komplexe trigonometrische Polynom, welches die gegebenen Daten interpoliert ist also
    \[
        T_4\parentheses*{y; x} = 3 + \frac{i}{2}\parentheses*{e^{ix} - e^{3ix}}. 
    \]


    \section*{Aufgabe 3}

    \begin{problem}
        Beweisen Sie das folgende Lemma aus der Vorlesung:
        \begin{quote}
            \textbf{\sffamily Lemma.} Für \(j, k \in \Z\) gilt
            \[
                \angles*{e_j, e_k} = \frac{1}{2\pi}\int_0^{2\pi}e^{ijx}e^{-ikx}\d x = \delta_{jk}.
            \]
        \end{quote}
    \end{problem}

    \subsection*{Lösung}
    Mithilfe der Identität \(e^{ix} = \cos x + i\sin x\) lässt sich die Aussage leicht nachrechnen:
    \begin{align*}
        \angles*{e_j, e_k} &= \frac{1}{2\pi}\int_0^{2\pi}e^{ijx}e^{-ikx}\d x\\
        &= \frac{1}{2\pi}\int_0^{2\pi}e^{i\parentheses*{j - k}x}\d x\\
        &= \frac{1}{2\pi}\int_0^{2\pi}\cos\parentheses*{\parentheses*{j - k}x}\d x + \frac{i}{2\pi}\underbrace{\int_0^{2\pi}\sin\parentheses*{\parentheses*{j - k}x}\d x}_{= 0\ \forall j, k \in \Z}\\
        &= \frac{1}{2\pi}\int_0^{2\pi}\cos\parentheses*{\parentheses*{j - k}x}\d x\\
        &= \frac{1}{2\pi}\begin{cases}
            2\pi, & \text{falls }j = k,\\
            0, & \text{sonst}
        \end{cases}\\
        &= \delta_{jk}.
    \end{align*}
\end{document}
