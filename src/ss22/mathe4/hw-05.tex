\documentclass{exercise}

\institute{Applied and Computational Mathematics}
\title{Hausaufgabenübung 5}
\author{Joshua Feld, 406718}
\course{Mathematische Grundlagen IV}
\professor{Torrilhon \& Berkels}
\semester{Sommersemester 2022}
\program{CES (Bachelor)}

\begin{document}
    \maketitle


    \section*{Aufgabe 1}

    \begin{problem}
        Lösen Sie das Randwertproblem auf \(\Omega = \parentheses*{0, a} \times \parentheses*{0, b}\)
        \begin{align*}
            \Delta u\parentheses*{x, y} &= 0, \quad \parentheses*{x, y} \in \Omega,\\
            u\parentheses*{0, y} = u\parentheses*{a, y} &= 0, \quad y \in \brackets*{0, b},\\
            u\parentheses*{x, 0} &= u_0\sin\parentheses*{\frac{2\pi}{a}x}, \quad x \in \brackets*{0, a},\\
            u\parentheses*{x, b} &= u_b\sin\parentheses*{\frac{2\pi}{a}x}, \quad x \in \brackets*{0, a},
        \end{align*}
        mit \(u_0, u_b \in \R\).
    \end{problem}

    \subsection*{Lösung}
    Wir lösen dieses Randwertproblem mit dem Separationsansatz:
    \[
        u\parentheses*{x, y} = X\parentheses*{x} \cdot Y\parentheses*{y}, \quad X, Y \ne 0.
    \]
    Einsetzen in die DGL liefert
    \[
        Y\parentheses*{y}X''\parentheses*{x} + X\parentheses*{x}Y''\parentheses*{y} = 0 \iff \underbrace{\frac{X''\parentheses*{x}}{X\parentheses*{x}}}_{\text{nur von }x\text{ abh.}} + \underbrace{\frac{Y''\parentheses*{y}}{Y\parentheses*{y}}}_{\text{nur von }y\text{ abh.}} = 0.
    \]
    Diese Gleichung kann nur erfüllt werden, wenn
    \[
        \frac{X''}{X} = -\lambda^2 \iff X''\parentheses*{x} + \lambda^2 X\parentheses*{x} = 0, \quad \frac{Y''}{Y} = \lambda^2 \iff Y''\parentheses*{y} - \lambda^2 Y\parentheses*{y} = 0, \quad \lambda \in \R^+.
    \]
    Lösungen der ODEs für \(\lambda > 0\):
    \begin{align*}
        \chi_1\parentheses*{t} = t^2 + \lambda^2 = 0 \implies t_{1, 2} = \pm i\lambda &\implies X\parentheses*{x} = \tilde{C}_1 e^{i\lambda x} + \tilde{C}_2 e^{-i\lambda x} = C_1\sin\parentheses*{\lambda x} + C_2\cos\parentheses*{\lambda x},\\
        \chi_2\parentheses*{t} = t^2 - \lambda^2 = 0 \implies t_{1, 2} = \pm\lambda &\implies Y\parentheses*{y} = \tilde{C}_3 e^{\lambda y} + \tilde{C}_4 e^{-\lambda y} = C_3\sinh\parentheses*{\lambda y} + C_4\cosh\parentheses*{\lambda y}.
    \end{align*}
    Randbedingungen:
    \begin{enumerate}[label=(\roman*)]
        \item 
        \begin{align*}
            u\parentheses*{0, y} = C_2 Y\parentheses*{y} = 0 &\xRightarrow{Y \ne 0} C_2 = 0,\\
            u\parentheses*{a, y} = C_1\sin\parentheses*{\lambda x} \cdot Y\parentheses*{y} = 0 &\xRightarrow[C_1 \ne 0]{Y \ne 0} \lambda_n = \frac{n\pi}{a}, \quad n \in \N.
        \end{align*}
        Damit folgen direkt die Eigenfunktionen
        \[
            X_n = C_1^{\parentheses*{n}}\sin\parentheses*{\frac{n\pi}{a}x}.
        \]
        Aus diskreten Werten für \(\lambda_n\) erhalten wir auch
        \[
            Y_n = C_3^{\parentheses*{n}}\sinh\parentheses*{\frac{n\pi}{a}y} + C_4^{\parentheses*{n}}\cosh\parentheses*{\frac{n\pi}{a}y}.
        \]
        Damit haben wir eine abzählbare Menge von Lösungen
        \[
            u_n\parentheses*{x, y} = \parentheses*{C_3^{\parentheses*{n}}\sinh\parentheses*{\frac{n\pi}{a}y} + C_4^{\parentheses*{n}}\cosh\parentheses*{\frac{n\pi}{a}y}} \cdot C_1^{\parentheses*{n}}\sin\parentheses*{\frac{n\pi}{a}x}.
        \]
        Da die Laplace-Gleichung linear ist, können wir eine Linearkombination von \(u_n\parentheses*{x, y}\) als Lösung verwenden:
        \[
            u\parentheses*{x, y} = \sum_{n = 0}^\infty\parentheses*{a_n\sinh\parentheses*{\lambda_n y} + b_n\cosh\parentheses*{\lambda_n y}} - \sin\parentheses*{\lambda_n x}.
        \]
        \item
        \[
            u\parentheses*{x, 0} = \sum_{n = 0}^\infty b_n\underbrace{\cosh\parentheses*{0}}_{= 1}\sin\parentheses*{\lambda_n x} \stackrel{!}{=} u_0\sin\parentheses*{\frac{2\pi}{a}x}, \quad \forall x \in \brackets*{0, a}.
        \]
        Ein Koeffizientenvergleich ergibt
        \[
            b_2 = u_0 \quad \text{und} \quad b_n = 0
        \]
        für alle \(n \ne 2\).
        \item
        \begin{align*}
            u\parentheses*{x, b} &= \sum_{n = 0}^\infty\parentheses*{a_n\sinh\parentheses*{\frac{n\pi}{a}b} + b_n\cosh\parentheses*{\frac{n\pi}{a}b}} \cdot \sin\parentheses*{\frac{n\pi}{a}x} \stackrel{!}{=} u_b\sin\parentheses*{\frac{2\pi}{a}x}, \quad \forall x \in \brackets*{0, a},\\
            \xRightarrow{\text{(ii)}} u_b\sin\parentheses*{\frac{2\pi}{a}x} &= \sum_{n = 0}^\infty a_n\sinh\parentheses*{\frac{n\pi}{a}b}\sin\parentheses*{\frac{n\pi}{a}x} + u_0\sin\parentheses*{\frac{2\pi}{a}x}\cosh\parentheses*{\frac{2\pi}{a}b}.
        \end{align*}
        Nochmaliger Koeffizientenvergleich führt zu
        \[
            a_2 = \frac{u_b - u_0\cosh\parentheses*{\frac{2\pi}{a}y}}{\sinh\parentheses*{\frac{2\pi}{a}b}} \quad \text{und} \quad a_n = 0 \forall n \ne 2.
        \]
    \end{enumerate}
    Insgesamt ergibt sich also
    \[
        u\parentheses*{x, y} = \parentheses*{u_0\cosh\parentheses*{\frac{2\pi}{a}y} + a_2\sinh\parentheses*{\frac{2\pi}{a}y}}\sin\parentheses*{\frac{2\pi}{a}x}.
    \]


    \section*{Aufgabe 2}

    \begin{problem}
        Sei \(f: \R \setminus \braces*{0} \to \R\) mit \(f\parentheses*{x} := \frac{x}{\sin\parentheses*{x}}\) gegeben.
        Man kann zeigen, dass \(f\) in \(0\) glatt fortgesetzt werden kann.
        Im Beweis eines Lemmas der Numerik-Vorlesung wird die Taylorentwicklung von \(f\) um \(x = 0\)
        \[
            f\parentheses*{x} = 1 + \frac{1}{6}x^2 + \mathcal{O}\parentheses*{x^4}
        \]
        verwendet.
        Leiten Sie diese Taylorentwicklung her.
        Die Existenz der glatten Fortsetzung von \(f\) dürfen Sie dabei ohne Beweis verwenden.
    \end{problem}

    \subsection*{Lösung}
    Wir müssen die Ableitungen in der Taylorentwicklung
    \[
        f\parentheses*{x} = \sum_{n = 0}^3\frac{f^{\parentheses*{n}}\parentheses*{0}}{n!}x^n + \mathcal{O}\parentheses*{x^4}
    \]
    bestimmen, wobei \(f^{\parentheses*{n}}\parentheses*{0}\) hier die Fortsetzung von \(f\) in \(x = 0\) bezeichnet
    \[
        f^{\parentheses*{n}}\parentheses*{0} = \lim_{x \to 0}f^{\parentheses*{n}}\parentheses*{x}.
    \]
    Wir bemerken zunächst, dass \(f\) gerade bezüglich \(x = 0\) ist
    \[
        f\parentheses*{-x} = -\frac{x}{\sin\parentheses*{-x}} = \frac{x}{\sin\parentheses*{x}} = f\parentheses*{x},
    \]
    sodass die ungeraden Ableitungen verschwinden:
    \[
        f^{\parentheses*{1}}\parentheses*{0} = 0, \quad f^{\parentheses*{3}}\parentheses*{0} = 0.
    \]
    Wir bestimmen mithilfe der Regel von l'Hopital
    \begin{align*}
        f^{\parentheses*{0}}\parentheses*{0} &= \lim_{x \to 0}f^{\parentheses*{0}}\parentheses*{x} = \lim_{x \to 0}\frac{x}{\sin\parentheses*{x}} = \lim_{x \to 0}\frac{1}{\cos\parentheses*{x}} = 1,\\
        f^{\parentheses*{2}}\parentheses*{0} &= \lim_{x \to 0}f^{\parentheses*{2}}\parentheses*{x}\\
        &= \lim_{x \to 0}\frac{x\sin^2\parentheses*{x} - 2\sin\parentheses*{x}\cos\parentheses*{x} + 2x\cos^2\parentheses*{x}}{\sin^3\parentheses*{x}}\\
        &= \lim_{x \to 0}\frac{x - 2\sin\parentheses*{x}\cos\parentheses*{x} + x\cos^2\parentheses*{x}}{\sin^3\parentheses*{x}}\\
        &= \lim_{x \to 0}\frac{1 + 2\sin^2\parentheses*{x} - 2\cos^2\parentheses*{x} + \cos^2\parentheses*{x} - 2x\cos\parentheses*{x}\sin\parentheses*{x}}{3\sin^2\parentheses*{x}\cos\parentheses*{x}}\\
        &= \lim_{x \to 0}\frac{3\sin\parentheses*{x} - 2x\cos\parentheses*{x}}{3\sin\parentheses*{x}\cos\parentheses*{x}}\\
        &= \lim_{x \to 0}\frac{3\cos\parentheses*{x} -2\cos\parentheses*{x} + 2x\sin\parentheses*{x}}{-3\sin^2\parentheses*{x} + 3\cos^2\parentheses*{x}}\\
        &= \lim_{x \to 0}\frac{\cos\parentheses*{x} + 2x\sin\parentheses*{x}}{-3\sin^2\parentheses*{x} + 3\cos^2\parentheses*{x}}\\
        &= \frac{1}{3}.
    \end{align*}
    Womit die geforderte Taylorentwicklung gezeigt ist:
    \begin{align*}
        f\parentheses*{x} &= \sum_{n = 0}^3\frac{f^{\parentheses*{n}}\parentheses*{0}}{n!}x^n + \mathcal{O}\parentheses*{x^4}\\
        &= f^{\parentheses*{0}}\parentheses*{0} + f^{\parentheses*{1}}\parentheses*{0}x + \frac{1}{2}f^{\parentheses*{2}}\parentheses*{0}x^2 + \frac{1}{6}f^{\parentheses*{3}}\parentheses*{0}x^3 + \mathcal{O}\parentheses*{x^4}\\
        &= 1 + 0x + \frac{1}{2} \cdot \frac{1}{3}x^2 + \frac{1}{6} \cdot 0x^3 + \mathcal{O}\parentheses*{x^4}\\
        &= 1 + \frac{1}{6}x^2 + \mathcal{O}\parentheses*{x^4}.
    \end{align*}


    \section*{Aufgabe 3}

    \begin{problem}
        Diskretisieren Sie das Randwertproblem
        \[
            y'' + y' + y = 0, \quad y'\parentheses*{0} = 1, \quad y\parentheses*{1} = 0
        \]
        mit Schrittweite \(h = \frac{1}{n}, n \in \N\).
    \end{problem}

    \subsection*{Lösung}
    Bei der Diskretisierung mit \(h = \frac{1}{n}\) ergeben sich \(n\) Intervalle mit den Randpunkten \(x_0, \ldots, x_n\).
    Durch Hinzunahme eines weiteren fiktiven Punktes \(x_{-1}\) erhöht sich die Anzahl an Punkten auf \(n + 2\) und die Punkte \(x_0, \ldots, x_{n - 1}\) sind innere Punkte.
    Wir benutzen zur Diskretisierung der ersten und zweiten Ableitung die Differenzenformeln von zweiter Ordnung.
    Es ergibt sich
    \[
        2\parentheses*{y_{i + 1} - 2y_i + y_{i - 1}} + h\parentheses*{y_{i + 1} - y_{i - 1}} + 2h^2 y_i = 0
    \]
    und damit
    \[
        \parentheses*{2 + h}y_{i + 1} + \parentheses*{-4 + 2h^2}y_i + \parentheses*{2 - h}y_{i - 1} = 0, \quad i = 0, \ldots, n - 1.
    \]
    Die Anwendung der symmetrischen Differenzenformel für die erste Ableitung im Punkt \(x_0\) ergibt die Randwertgleichung
    \[
        y_1 - y_{-1} = 2h,
    \]
    die zweite Randgleichung ist
    \[
        y_n = 0.
    \]
    Wir eliminieren die Variablen \(y_{-1}\) und \(y_n\): Einsetzen ergibt
    \begin{align*}
        4y_1 + \parentheses*{-4 + 2h^2}y_0 &= \parentheses*{2 - h} \cdot 2h,\\
        \parentheses*{2 + h}y_{i + 1} + \parentheses*{-4 + 2h^2}y_i + \parentheses*{2 - h}y_{i - 1} &= 0, \quad i = 1, \ldots, n - 2,\\
        \parentheses*{-4 + 2h^2}y_{n - 1} + \parentheses*{2 - h}y_{n - 2} &= 0.
    \end{align*}
    Das Gleichungssystem hat somit die Form
    \[
        \begin{pmatrix}
            -4 + 2h^2 & 4 & 0 & \cdots & 0\\
            2 - h & -4 + 2h^2 & 2 + h & 0 & \vdots\\
            0 & \ddots & \ddots & \ddots & 0\\
            \vdots & 0 & 2 - h & -4 + 2h^2 & 2 + h\\
            0 & \cdots & 0 & 2 - h & -4 + 2h^2
        \end{pmatrix}\begin{pmatrix}
            y_0\\y_1\\\vdots\\y_{n - 2}\\y_{n - 1}
        \end{pmatrix} = \begin{pmatrix}
            \parentheses*{2 - h} \cdot 2h\\0\\\vdots\\0\\0
        \end{pmatrix}.
    \]
\end{document}
