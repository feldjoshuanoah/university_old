\documentclass{exercise}

\institute{Applied and Computational Mathematics}
\title{Hausaufgabenübung 6}
\author{Joshua Feld, 406718}
\course{Mathematische Grundlagen IV}
\professor{Torrilhon \& Berkels}
\semester{Sommersemester 2022}
\program{CES (Bachelor)}

\begin{document}
    \maketitle


    \section*{Aufgabe 1}

    \begin{problem}
        Bestimmen Sie eine Lösung der Wärmeleitungsgleichung mit periodischen Randbedingungen:
        \begin{align*}
            \partial_t u\parentheses*{t, x} - \Delta u\parentheses*{t, x} &= 0, \quad t > 0, x \in \parentheses*{-\pi, \pi}\\
            u\parentheses*{t, -\pi} &= u\parentheses*{t, \pi},\\
            \partial_x u\parentheses*{t, -\pi} &= \partial_x u\parentheses*{t, \pi},\\
            u\parentheses*{0, x} &= \cos\parentheses*{2x}.
        \end{align*}
        Nutzen Sie dazu den Seperationsansatz.
    \end{problem}

    \subsection*{Lösung}
    Der Separationsansatz ist \(u\parentheses*{x, t} = X\parentheses*{x}T\parentheses*{t}\). Umgeformt in DGL ergibt sich
    \[
        \frac{T'\parentheses*{t}}{T\parentheses*{t}} = \frac{X''\parentheses*{x}}{X\parentheses*{x}} = -\lambda^2 \equiv \text{konst.}
    \]
    Daraus folgt
    \begin{align}
        X''\parentheses*{x} + \lambda^2 X\parentheses*{x} &= 0 \quad \text{mit} \quad X\parentheses*{-\pi} = X\parentheses*{\pi}, X'\parentheses*{-\pi} = X'\parentheses*{\pi},\label{eq:1}\\
        T'\parentheses*{t} + \lambda^2 T\parentheses*{t} &= 0.\label{eq:2}
    \end{align}
    Wir nutzen den Ansatz \(X\parentheses*{x} = Ce^{\mu x}\).
    Einsetzen in \eqref{eq:1} ergibt die Gleichung
    \[
        \mu^2 Ce^{\mu x} + \lambda^2 Ce^{\mu x} = 0.
    \]
    Daraus ergeben sich die Eigenwerte
    \[
        \mu_1 = \lambda i, \quad \mu_2 = -\lambda i,
    \]
    also
    \[
        X\parentheses*{x} = C_1 e^{i\lambda x} + C_2 e^{-i\lambda x}.
    \]
    Durch Anwendung der Randwertbedingungen
    \begin{align*}
        X\parentheses*{-\pi} = C_1 e^{-i\lambda\pi} + C_2 e^{i\lambda\pi} &= C_1 e^{i\lambda\pi} + C_2 e^{-i\lambda\pi} = X\parentheses*{\pi}\\
        X'\parentheses*{-\pi} = i\lambda\parentheses*{C_1 e^{-i\lambda\pi} - C_2 e^{i\lambda\pi}} &= i\lambda\parentheses*{C_1 e^{i\lambda\pi} - C_2 e^{-i\lambda\pi}} = X'\parentheses*{\pi}
    \end{align*}
    folgen die Gleichungen
    \begin{align}
        C_2\parentheses*{e^{i\lambda\pi} - e^{-i\lambda\pi}} = C_1\parentheses*{e^{i\lambda\pi} - e^{-i\lambda\pi}},\label{eq:2}\\
        C_2\parentheses*{e^{i\lambda\pi} - e^{-i\lambda\pi}} = C_1\parentheses*{e^{-i\lambda\pi} - e^{-i\lambda\pi}}.\label{eq:3}
    \end{align}
    Aus \(\eqref{eq:2} - \eqref{eq:3}\) folgt
    \[
        \sin\parentheses*{\lambda\pi} = 0 \iff \lambda \in \Z
    \]
    und somit ergibt sich schließlich
    \[
        X\parentheses*{x} = \sum a_n\cos\parentheses*{nx} + b_n\sin\parentheses*{nx}.
    \]
    Mit dem Ansatz \(T\parentheses*{t} = Ke^{-\lambda^2 t}\) und dem Anfangswert
    \[
        X\parentheses*{x}T\parentheses*{0} = \cos\parentheses*{2x} = X\parentheses*{x}K
    \]
    folgt
    \[
        X\parentheses*{x} = \frac{1}{K}\cos\parentheses*{2x}.
    \]
    Man müsste \(\cos\parentheses*{2x}\) in einer Fourierreihe entwickeln, das ist aber einfach: Wähle \(a_2 = \frac{1}{K}\) und die restlichen Koeffizienten \(a_n, b_n\) gleich Null, dann gilt schlussendlich
    \[
        u\parentheses*{x, t} = e^{-4t}\cos\parentheses*{2x}.
    \]


    \section*{Aufgabe 2}

    \begin{problem}
        Gegeben sei die Transportgleichung
        \begin{align*}
            u_x\parentheses*{x, y} + u_y\parentheses*{x, y} &= 0, \quad \parentheses*{x, y} \in \R \times \R,\\
            u\parentheses*{0, y} &= \cos\parentheses*{y}, \quad y \in \R.
        \end{align*}
        \begin{enumerate}
            \item Ermitteln Sie mithilfe der Charakteristikenmethode einen Kandidaten \(u\) für die Lösung dieser Transportgleichung.
            \item Verifizieren Sie, dass Ihr Kandidat \(u\) aus Teil a) die Transportgleichung erfüllt.
        \end{enumerate}
    \end{problem}

    \subsection*{Lösung}
    \begin{enumerate}
        \item Wir suchen einen charakteristischen Weg \(\parentheses*{x\parentheses*{s}, y\parentheses*{s}}\) mit \(\parentheses*{x\parentheses*{0}, y\parentheses*{0}} = \parentheses*{0, y_0}\), entlang dem sich das Anfangswertproblem auf eine gewöhnliche DGL reduziert.
        Wir definieren
        \[
            z\parentheses*{s} := u\parentheses*{x\parentheses*{s}, y\parentheses*{s}}, \quad s \in \R
        \]
        und berechnen
        \[
            z'\parentheses*{s} = x'\parentheses*{s}u_x\parentheses*{x\parentheses*{s}, y\parentheses*{s}} + y'\parentheses*{s}u_y\parentheses*{x\parentheses*{s}, y\parentheses*{s}}.
        \]
        Nehmen wir \(x'\parentheses*{s} = 1\) und \(y'\parentheses*{s} = 1\), dann ist mit der obigen PDE \(z'\parentheses*{s} = 0\).
        Daher ist \(z = z\parentheses*{s}\) eine konstante Funktion von \(s\) für jede Wahl von \(y_0\).
        Die Differentialgleichungen für \(x'\parentheses*{s}\) und \(y'\parentheses*{s}\) liefern
        \[
            x'\parentheses*{s} = 1 \quad \text{und} \quad y'\parentheses*{s} = 1.
        \]
        Es folgt
        \[
            x\parentheses*{s} = s + \alpha_1 \quad \text{und} \quad y\parentheses*{s} = s + \alpha_2.
        \]
        Weiterhin erhalten wir
        \[
            0 = x\parentheses*{0} = 0 + \alpha_1 \implies x\parentheses*{s} = s
        \]
        und
        \[
            y_0 = y\parentheses*{0} = 0 + \alpha_2 \implies y\parentheses*{s} = s + y_0.
        \]
        Da \(z'\parentheses*{s} = 0\), ist \(u\parentheses*{x\parentheses*{s}, y\parentheses*{s}}\) konstant für gegebene \(x_0\) und \(y_0\) und mit
        \[
            y\parentheses*{s} = s + y_0 \implies y_0 = y\parentheses*{s} - s\text{ und }x\parentheses*{s} = s
        \]
        folgt
        \[
            u\parentheses*{x\parentheses*{s}, y\parentheses*{s}} = u\parentheses*{0, y_0} = \cos\parentheses*{y_0} = \cos\parentheses*{y\parentheses*{s} - s} = \cos\parentheses*{y\parentheses*{s} - x\parentheses*{s}}.
        \]
        Somit gilt
        \[
            u\parentheses*{x, y} = \cos\parentheses*{y\parentheses*{s} - x\parentheses*{s}}, \quad \parentheses*{x, y} \in \R \times \R.
        \]
        \item Wir berechnen
        \begin{align*}
            u_x &= \sin\parentheses*{y\parentheses*{s} - x\parentheses*{s}},\\
            u_y &= -\sin\parentheses*{y\parentheses*{s} - x\parentheses*{s}}\\
            \implies u_x + u_y &= 0.
        \end{align*}
        Außerdem wird die Randbedingung \(u\parentheses*{0, y} = \cos\parentheses*{y}\) erfüllt.
    \end{enumerate}


    \section*{Aufgabe 3}

    \begin{problem}
        Eine Matrix \(A \in \C^{n \times n}\) heißt strikt diagonaldominant, falls für alle \(i = 1, \ldots, n\) gilt:
        \[
            \sum_{j = 1, i \ne j}^n\absolute*{a_{ij}} < \absolute*{a_{ii}}.
        \]
        Zeigen Sie, dass jede strikt diagonaldominante Matrix invertierbar ist.
    \end{problem}

    \subsection*{Lösung}
    Betrachte die Gershgorin-Kreise der Matrix \(A\):
    \[
        G_i = \braces*{x \in \C : \absolute*{x - a_{ii}} \le \sum_{i = 1, i \ne j}^n\absolute*{a_{ij}}}.
    \]
    Da \(\sum_{j = 1, i \ne j}^n \absolute*{a_{ij}} < \absolute*{a_{ii}}\) für alle \(i = 1, \ldots, n\) gilt, wissen wir, dass \(0 \not\in G_i\) für alle \(i = 1, \ldots, n\) gilt.
    Da \(\sigma\parentheses*{A} \subset \bigcup_{i = 1}^n G_i\), folgt somit \(0 \not\in \sigma\parentheses*{A}\).
    Damit ist \(A\) invertierbar.


    \section*{Aufgabe 4}

    \begin{problem}
        Gegeben sei das Problem
        \begin{align*}
            -\parentheses*{a_{11}\partial_x^2 u\parentheses*{x, y} + 2a_{12}\partial_x \partial_y u\parentheses*{x, y} + a_{22}\partial_x^2 u\parentheses*{x, y}} &= f\parentheses*{x, y}, \quad x \in \Omega,\\
            u\parentheses*{x, y} &= g\parentheses*{x, y}, \quad x \in \partial\Omega.
        \end{align*}
        Für den diskreten Differentialoperator sei folgender Differenzenstern gegeben:
        \[
            \frac{1}{h^2}\begin{bmatrix}
                a_{12}^- & -\parentheses*{a_{22} - \absolute*{a_{12}}} & -a_{12}^+\\
                -\parentheses*{a_{11} - \absolute*{a_{12}}} & 2\parentheses*{a_{11} + a_{22} - \absolute*{a_{12}}} & -\parentheses*{a_{11} - \absolute*{a_{12}}}\\
                -a_{12}^+ & -\parentheses*{a_{22} - \absolute*{a_{12}}} & a_{12}^-
            \end{bmatrix}
        \]
        mit \(a_{12}^+ = \max\braces*{a_{12}, 0}\) und \(a_{12}^- = \min\braces*{a_{12}, 0}\).
        Zeigen Sie, dass die resultierende Matrix für
        \[
            a_{11} > 0, \quad a_{22} > 0, \quad \absolute*{a_{11}} > \absolute*{a_{12}}, \quad \absolute*{a_{22}} > \absolute*{a_{12}}
        \]
        eine \(M\)-Matrix ist.
        Hierbei können Sie die Irreduzibilität als gegeben annehmen.
    \end{problem}

    \subsection*{Lösung}
    Eine Zeile der Matrix \(A\) für einen Gitterknoten, dessen Nachbarknoten nicht auf dem Rand liegen, ist gegeben als
    \begin{align*}
        \frac{1}{h^2}&\left[\ldots, -a_{12}^+, -\parentheses*{a_{22} - \absolute*{a_{12}}}, a_{12}^-, \ldots, -\parentheses*{a_{11} - \absolute*{a_{12}}}, 2\parentheses*{a_{11} + a_{22} - \absolute*{a_{12}}},\right.\\
        &\left.-\parentheses*{a_{11} - \absolute*{a_{12}}}, \ldots, a_{12}^-, -\parentheses*{a_{22} - \absolute*{a_{12}}}, -a_{12}^+, \ldots\right],
    \end{align*}
    wobei \(2\parentheses*{a_{11} + a_{22} - \absolute*{a_{12}}}\) auf der Diagonalen liegt.
    Anhand der Definition von \(a_{12}^+\) und \(a_{12}^-\) und den Bedingungen an \(a_{11}\), \(a_{12}\) und \(a_{22}\) erkennt man schnell, dass alle Diagonaleinträge positiv und alle anderen Einträge negativ sein müssen.
    Es handelt sich also um eine \(L\)-Matrix.
    Als nächstes zeigen wir Diagonaldominanz.
    Wir machen eine Fallunterscheidung:
    \begin{itemize}
        \item \(a_{12} \ge 0\): Damit gilt \(a_{12}^- = 0\) und \(a_{12}^+ = a_{12}\).
        Somit erhalten wir
        \begin{align*}
            t
        \end{align*}
        \item \(a_{12} < 0\): Damit gilt \(a_{12}^- = a_{12}\) und \(a_{12}^+ = 0\).
        Somit erhalten wir
        \begin{align*}
            t
        \end{align*}
    \end{itemize}
    Damit gilt für jede Zeile in der keiner der Nachbarknoten ein Randknoten ist:

    \[
        \sum_{j = 1, i \ne j}^n \absolute*{a_{ij}} = \absolute*{a_{ii}}.
    \]
    Falls einer der Nachbarknoten ein Randknoten ist, trägt dieser nicht zur Matrix \(A\) bei und entfällt damit in den Summen.
    Damit folgt für die zugehörigen Zeilen \(\parentheses*{A_{ij}}_{j = 1, \ldots, n}\)
    \[
        \sum_{j = 1, i \ne j}^n \absolute*{a_{ij}} < \absolute*{a_{ii}}.
    \]
    Die Matrix \(A\) ist also diagonaldominant und irreduzibel diagonaldominant.
    Da \(A\) eine irreduzibel diagonaldominante \(L\)-Matrix ist, ist \(A\) eine \(M\)-Matrix.
\end{document}
