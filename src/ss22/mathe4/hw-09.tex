\documentclass{exercise}

\institute{Applied and Computational Mathematics}
\title{Hausaufgabenübung 9}
\author{Joshua Feld, 406718}
\course{Mathematische Grundlagen IV}
\professor{Torrilhon \& Berkels}
\semester{Sommersemester 2022}
\program{CES (Bachelor)}

\begin{document}
    \maketitle


    \section*{Aufgabe 1}
    
    \begin{problem}
        Gegeben sei das Problem
        \begin{align*}
            \partial_t u &= \partial_x^2 u + q_0, \quad x \in \parentheses*{0, L}, t > 0,\\
            u\parentheses*{0, t} = u\parentheses*{L, t} &= 0,\\
            u\parentheses*{x, 0} &= 0
        \end{align*}
        mit konstantem Quellterm \(q_0\).
        \begin{enumerate}
            \item Berechnen Sie die Lösung \(u\parentheses*{x, t}\).
            \item Berechnen Sie die stationäre Lösung \(\lim_{t \to \infty}u\parentheses*{x, t} = u_\infty\parentheses*{x}\).
        \end{enumerate}
    \end{problem}
    
    \subsection*{Lösung}
    \begin{enumerate}
        \item Die zum Problem gehörenden Eigenfunktionen und Eigenwerte erfüllen
        \[
            -\Delta\phi = \lambda\phi, \quad \phi\parentheses*{0} = \phi\parentheses*{L} = 0
        \]
        und laut Normierung damit
        \[
            \phi_k\parentheses*{x} = \sqrt{\frac{2}{L}}\sin\parentheses*{\frac{k\pi}{L}x}, \quad \lambda_k = \parentheses*{\frac{k\pi}{L}}^2.
        \]
        In der Gleichung für \(u\parentheses*{x, t}\) tritt die Quelle
        \[
            q\parentheses*{x, t} = q_0
        \]
        auf, die in den Eigenfunktionen entwickelt wird:
        \[
            q_0\parentheses*{x} = \sum_{k = 1}^\infty \beta_k \phi_k\parentheses*{x},
        \]
        mit
        \[
            \beta_k = \int_0^L q_0\phi_k\d x = q_0\frac{L}{k\pi}\sqrt{\frac{2}{L}}\parentheses*{1 - \cos\parentheses*{k\pi}} = \begin{cases}
                \frac{2q_0\sqrt{2L}}{k\pi}, & \text{falls }k\text{ ungerade},\\
                0, & \text{sonst}.
            \end{cases}
        \]
        Für die Lösung \(u\parentheses*{x, t}\) wird nun der Ansatz
        \[
            u\parentheses*{x, t} = \sum_{k = 1}^\infty \alpha_k\parentheses*{t}\phi_k\parentheses*{x}
        \]
        verwendet, wobei \(\alpha_k\) zeitabhängig ist, weil \(u\parentheses*{x, t}\) zeitabhängig ist.
        Beim Einsetzen muss darauf geachtet werden, dass die Zeitableitung auf \(\alpha_k\parentheses*{t}\) wirkt.
        Wir erhalten
        \begin{align*}
            \sum_{k = 1}^\infty \parentheses*{\partial_t \alpha_k\parentheses*{t}\phi_k\parentheses*{x} - \alpha_k\parentheses*{t}\partial_x^2 \phi_k\parentheses*{x}} &= \sum_{k = 1}^\infty \parentheses*{\alpha_k'\parentheses*{t}\phi_k\parentheses*{x} + \alpha_k\parentheses*{t}\lambda_k \phi_k\parentheses*{x} - \beta_k \phi_k\parentheses*{x}}\\
            &= \sum_{k = 1}^\infty \parentheses*{\alpha_k'\parentheses*{t} + \alpha_k\parentheses*{t} - \beta_k}\phi_k\parentheses*{x} = 0.
        \end{align*}
        Da die Eigenfunktionen orthonormal sind, muss für \(k > 1\) die gewöhnliche Differentialgleichung
        \[
            \alpha_k'\parentheses*{t} + \alpha_k\parentheses*{t}\lambda_k - \beta_k = 0
        \]
        gelten.
        Die allgemeine Lösung lautet
        \[
            \alpha_k\parentheses*{t} = \frac{\beta_k}{\lambda_k} + C_k e^{-\lambda_k t}.
        \]
        Die Konstante \(C_k\) wird durch die Anfangsbedingung für \(\alpha_k\parentheses*{0}\) bestimmt.
        Dazu muss die Anfangsbedingung für \(u\parentheses*{x, t}\) entwickelt werden.
        Hier ist dies allerdings ganz einfach möglich:
        \[
            u\parentheses*{x, 0} = \sum_{k = 1}^\infty \alpha_k\parentheses*{0}\phi_k\parentheses*{x} \stackrel{!}{=} 0.
        \]
        Also muss gelten: \(\alpha_k\parentheses*{0} = 0\).
        Also folgt \(C_k = -\frac{\beta_k}{\lambda_k}\).
        Die Lösung \(u\) lautet ingsgesamt
        \[
            u\parentheses*{x, t} = \sum_{k = 1}^\infty \frac{\beta_k}{\lambda_k}\parentheses*{1 - e^{-\lambda_k t}}\sqrt{\frac{2}{L}}\sin\parentheses*{\frac{k\pi}{L}x} = \sum_{\substack{k = 1,\\k\text{ ungerade}}}^\infty 4q_0 L^2 \frac{1 - e^{-\lambda_k t}}{k^3 \pi^3}\sin\parentheses*{\frac{k\pi}{L}x}.
        \]
        \item Im Fall \(t \to \infty\) geht der Term mit der Exponentialfunktion gegen \(0\):
        \[
            u_\infty\parentheses*{x} = \sum_{\substack{k = 1,\\k\text{ ungerade}}}^\infty \frac{4q_0 L^2}{k^3 \pi^3}\sin\parentheses*{\frac{k\pi}{L}x}.
        \]
        Alternativ kann die stationäre Lösung auch direkt aus der Gleichung mit \(\frac{\partial}{\partial t} \to 0\) bestimmt werden.
        Zu Lösen ist nun
        \[
            \partial_x^2 u + q_0 = 0.
        \]
        Zweimaliges Integrieren gibt
        \[
            u_\infty\parentheses*{x} = -\frac{q_0}{2}x^2 + C_2 x + C_1.
        \]
        Die Konstanten ergeben sich aus den Randbedingungen:
        \begin{align*}
            u_\infty\parentheses*{0} = 0 &\implies C_1 = 0,\\
            u_\infty\parentheses*{L} = 0 &\implies C_2 = \frac{q_0 L}{2}.
        \end{align*}
        Also gilt
        \[
            u_\infty\parentheses*{x} = \frac{q_0}{2}x\parentheses*{L - x}.
        \]
        Eine Entwicklung dieser Funktion ergibt gerade die obige Summendarstellung.
    \end{enumerate}


    \section*{Aufgabe 2}
    
    \begin{problem}
        Lösen Sie das Problem
        \begin{align*}
            \partial_t u &= \partial_x^2 u, \quad x \in \parentheses*{0, L}, t > 0,\\
            u\parentheses*{0, t} &= 0,\\
            u\parentheses*{L, t} &= ate^{-at},\\
            u\parentheses*{x, 0} &= u_0 x\parentheses*{L - x},
        \end{align*}
        mit Konstanten \(a\) und \(u_0\).

        \emph{Hinweis: Folgen Sie dem in der Vorlesung vorgestellten Weg.}
    \end{problem}
    
    \subsection*{Lösung}
    Die Randbedingungen müsen homogenisiert werden. Dazu basteln wir eine Hilfsfunktion \(r\parentheses*{x, t}\), für die die Randbedingungen erfüllt und transformieren
    \[
        u\parentheses*{x, t} \to v\parentheses*{x, t} + r\parentheses*{x, t}.
    \]
    Bei praktisch allen Problemen, die wir betrachten kommt als Hilfsfunktion eine lineare oder konstante Funktion in Frage, da für diese \(\partial_x^2 r = 0\) gilt.
    Für die Aufgabe bietet sich
    \[
        r\parentheses*{x, t} = \frac{x}{L}ate^{-at}
    \]
    an, denn \(r\parentheses*{0, t} = 0\) und \(r\parentheses*{L, t} = ate^{-at}\).
    Wir definieren
    \[
        u\parentheses*{x, t} = v\parentheses*{x, t} + \frac{x}{L}ate^{-at} \iff v\parentheses*{x, t} = u\parentheses*{x, t} - \frac{x}{L}ate^{-at}
    \]
    und von nun an ist \(v\) die gesuchte Funktion.
    Einsetzen in die Gleichung und Anfangs- und Randbedingungen ergibt eine partielle Differentialgleichung für \(v\), nämlich
    \[
        \partial_t\parentheses*{v + \frac{x}{L}ate^{-at}} = \partial_x^2\parentheses*{v + \frac{x}{L}ate^{-at}},
    \]
    also gilt
    \[
        \partial_t v - \partial_x^2 v = -\frac{x}{L}\parentheses*{1 - at}ae^{-at}
    \]
    mit
    \[
        v\parentheses*{0, t} = v\parentheses*{L, t} = 0, \quad v\parentheses*{x, 0} = u_0 x\parentheses*{L - x}.
    \]
    Die Randbedingungen sind jetzt homogen.
    Die Eigenfunktionen \(\phi_k\) müssen folgendes Problem lösen:
    \[
        -\partial_x^2 \phi_k = \lambda_k \phi_k, \quad \phi\parentheses*{0} = \phi\parentheses*{L} = 0.
    \]
    Die Eigenfunktionen und Eigenwerte lauten nun
    \[
        \phi_k\parentheses*{x} = \sqrt{\frac{2}{L}}\sin\parentheses*{\frac{k\pi}{L}x}, \quad \lambda_k = \parentheses*{\frac{k\pi}{L}}^2.
    \]
    In der Gleichung für \(v\) tritt eine Art Quellterm auf, die durch die Homogenisierung entstand:
    \[
        q\parentheses*{x, t} = -\frac{x}{L}\parentheses*{1 - at}ae^{-at}.
    \]
    In Eigenfunktionen entwickelt erhalten wir
    \[
        q\parentheses*{x, t} = \sum_{k = 1}^\infty \beta_k\parentheses*{t}\sqrt{\frac{2}{L}}\sin\parentheses*{\frac{k\pi}{L}x}
    \]
    mit
    \begin{align*}
        \beta_k\parentheses*{t} = \int_0^L -\frac{x}{L}\parentheses*{1 - at}ae^{-at}\sqrt{\frac{2}{L}}\sin\parentheses*{\frac{k\pi}{L}x}\d x\\
        &= -\parentheses*{1 - at}ae^{-at}\sqrt{\frac{2}{L}}\int_0^L \frac{x}{L}\sin\parentheses*{\frac{k\pi}{L}x}\d x\\
        &= -\parentheses*{1 - at}ae^{-at}\sqrt{\frac{2}{L}}\frac{L}{k\pi}\parentheses*{-1}^k\\
        &= \parentheses*{-1}^k \frac{\sqrt{2L}}{k\pi}\parentheses*{1 - at}ae^{-at}.
    \end{align*}
    Für die Lösung \(v\parentheses*{x, t}\) wird nun der Ansatz
    \[
        v\parentheses*{x, t} = \sum_{k = 1}^\infty \alpha_k\parentheses*{t}\phi_k\parentheses*{x}
    \]
    gemacht.
    Durch Einsetzen ergibt sich
    \begin{align*}
        \sum_{k = 1}^\infty \parentheses*{\partial_t \alpha\parentheses*{t}\phi_k\parentheses*{x} - \alpha_k\parentheses*{t}\partial_x^2 \phi_k\parentheses*{x} - \beta_k\parentheses*{t}\phi_k\parentheses*{x}} &= 0\\
        \iff \sum_{k = 1}^\infty \parentheses*{\alpha'\parentheses*{t}\phi_k\parentheses*{x} - \alpha_k\parentheses*{t}\lambda_k \phi_k\parentheses*{x} - \beta_k\parentheses*{t}\phi_k\parentheses*{x}} &= 0\\
        \iff \sum_{k = 1}^\infty \parentheses*{\alpha'\parentheses*{t} - \alpha_k\parentheses*{t}\lambda_k - \beta_k\parentheses*{t}} &= 0.
    \end{align*}
    Da die Eigenfunktionen orthogonal sind, muss für \(k > 0\) gelten:
    \[
        \alpha'\parentheses*{t} - \alpha_k\parentheses*{t}\lambda_k - \beta_k\parentheses*{t} = 0.
    \]
    Die allgemeine Lösung lautet
    \[
        \alpha_k\parentheses*{t} = \alpha_k\parentheses*{0}e^{-\lambda_k t} + \int_0^t e^{-\lambda_k\parentheses*{t - \tau}}\beta_k\parentheses*{\tau}\d\tau.
    \]
    Das Integral können wir weiter ausrechnen:
    \begin{align*}
        \int_0^t e^{-\lambda_k\parentheses*{t - \tau}}\beta_k\parentheses*{\tau}\d\tau &= e^{-\lambda_k t}\int_0^t e^{\lambda_k \tau}\parentheses*{-1}^k \frac{\sqrt{2L}}{k\pi}\parentheses*{1 - a\tau}ae^{-a\tau}\d\tau\\
        &= e^{-\lambda_k t}\parentheses*{-1}^k \frac{\sqrt{2L}}{k\pi}a\int_0^t e^{\lambda_k \tau}\parentheses*{1 - a\tau}e^{-a\tau}\d\tau\\
        &= e^{-\lambda_k t}\parentheses*{-1}^k \frac{\sqrt{2L}}{k\pi}a\parentheses*{\int_0^t e^{\lambda_k \tau}e^{-a\tau}\d\tau - a\int_0^t \tau e^{\lambda_k \tau}e^{-a\tau}\d\tau}\\
        &= e^{-\lambda_k t}\parentheses*{-1}^k \frac{\sqrt{2L}}{k\pi}a\frac{\parentheses*{\lambda_k - \parentheses*{\lambda_k - a}at}e^{\lambda_k - at} - \lambda_k}{\parentheses*{\lambda_k - a}^2}\\
        &= \parentheses*{-1}^k \frac{\sqrt{2L}}{k\pi}a\frac{\sqrt{2L}}{k\pi}a\frac{\parentheses*{\lambda_k - \parentheses*{\lambda_k - a}at}e^{-at} - \lambda_k e^{-\lambda_k t}}{\parentheses*{\lambda_k - a}^2}.
    \end{align*}
    Es bleibt die Anfangsbedingung für \(\alpha_k\parentheses*{0}\) zu bestimmen.
    Dazu muss die Anfangsbedingung für \(v\parentheses*{x, t}\) entwickelt werden:
    \[
        v\parentheses*{x, 0} = \sum_{i = 1}^k \alpha_k\parentheses*{0}\phi_k\parentheses*{x} \stackrel{!}{=} u_0 x\parentheses*{L - x}
    \]
    mit
    \[
        \alpha_k\parentheses*{0} = \int_0^L u_0 x\parentheses*{L - x}\sqrt{\frac{2}{L}}\sin\parentheses*{\frac{k\pi}{L}x}\d x = \begin{cases}
            u_0 \sqrt{2L}\frac{4L^2}{k^3 \pi^3}, & \text{falls }k\text{ ungerade},\\
            0, & \text{falls }k\text{ gerade}.
        \end{cases}
    \]
    Die Lösung \(v\parentheses*{x, t}\) lautet insgesamt
    \begin{align*}
        v\parentheses*{x, t} &= \sum_{k = 1}^\infty \alpha_k\parentheses*{t}\phi_k\parentheses*{x}\\
        &= \sum_{\substack{k = 1,\\k\text{ ungerade}}}^\infty \frac{8u_0 L^2}{k^3 \pi^3}e^{-\lambda_k t}\sin\parentheses*{\frac{k\pi}{L}x} + \sum_{k = 1}^\infty \parentheses*{-1}^k \frac{2a}{k\pi}\frac{\parentheses*{\lambda_k - \parentheses*{\lambda_k - a}at}e^{-at} - \lambda_k e^{-\lambda_k t}}{\parentheses*{\lambda_k - a}^2}\sin\parentheses*{\frac{k\pi}{L}x}
    \end{align*}
    und die Lösung für \(u\parentheses*{x, t}\) folgt aus
    \[
        u\parentheses*{x, t} = v\parentheses*{x, t} + \frac{x}{L}ate^{-at}.
    \]


    \section*{Aufgabe 3}
    
    \begin{problem}
        Bestimmen Sie die Koeffizienten \(a\), \(b\) und \(c\), sodass die finite Differenz
        \[
            af\parentheses*{x} + bf\parentheses*{x + h} + cf\parentheses*{x + 2h} \approx f'\parentheses*{x}
        \]
        die erste Ableitung von \(f\) an der Stelle \(x\) mit möglichst hoher Ordnung approximiert.
        Bestimmen Sie die Konsistenzordnung.
    \end{problem}
    
    \subsection*{Lösung}
    Wir betrachten die Taylorentwicklungen für \(f\parentheses*{x + h}\) und \(f\parentheses*{x + 2h}\) wie im Folgenden:
    \begin{align*}
        f\parentheses*{x} &= f\parentheses*{x},\\
        f\parentheses*{x + h} &\approx f\parentheses*{x} + f'\parentheses*{x} \cdot h + f''\parentheses*{x} \cdot \frac{h^2}{2} + f'''\parentheses*{x} \cdot \frac{h^3}{6},\\
        f\parentheses*{x + 2h} &\approx f\parentheses*{x} + f'\parentheses*{x} \cdot 2h + f''\parentheses*{x} \cdot 2h^2 + f'''\parentheses*{x} \cdot \frac{4h^3}{3}.
    \end{align*}
    Daraus erhalten wir folgende Gleichungssystem für die Koeffizienten \(a\), \(b\) und \(c\):
    \begin{itemize}
        \item \(f\parentheses*{x}\)-Term: \(a + b + c = 0\),
        \item \(f'\parentheses*{x}\)-Term: \(hb + 2hc = 1\),
        \item \(f''\parentheses*{x}\)-Term: \(\frac{h^2}{2}b + 2h^2 c = 0\).
    \end{itemize}
    Wir multiplizieren die zweite Zeile mit \(\frac{1}{h}\), die dritte mit \(\frac{2}{h^2}\) und erhalten folgendes Gleichungssystem:
    \[
        \begin{pmatrix}
            1 & 1 & 1\\
            0 & 1 & 2\\
            0 & 1 & 4
        \end{pmatrix}\begin{pmatrix}
            a\\
            b\\
            c
        \end{pmatrix} = \begin{pmatrix}
            0\\
            \frac{1}{h}\\
            0
        \end{pmatrix}.
    \]
    Das Gauß'sche Eliminationsverfahren lautet
    \begin{align*}
        \begin{pmatrix}
            1 & 1 & 1\\
            0 & 1 & 2\\
            0 & 1 & 4
        \end{pmatrix}\begin{pmatrix}
            a\\
            b\\
            c
        \end{pmatrix} = \begin{pmatrix}
            0\\
            \frac{1}{h}\\
            0
        \end{pmatrix} &\xrightarrow{\text{Eli. 2. Splt.}} \begin{pmatrix}
            1 & 1 & 1\\
            0 & 1 & 2\\
            0 & 0 & 2
        \end{pmatrix}\begin{pmatrix}
            a\\
            b\\
            c
        \end{pmatrix} = \begin{pmatrix}
            0\\
            \frac{1}{h}\\
            -\frac{1}{h}
        \end{pmatrix}\\
        &\xrightarrow{\text{Eli. 3. Splt.}} \begin{pmatrix}
            1 & 1 & 0\\
            0 & 1 & 0\\
            0 & 0 & 2
        \end{pmatrix}\begin{pmatrix}
            a\\
            b\\
            c
        \end{pmatrix} = \begin{pmatrix}
            \frac{1}{2h}\\
            \frac{2}{h}\\
            -\frac{1}{h}
        \end{pmatrix}\\
        &\xrightarrow{\text{Eli. 2. Splt.}} \begin{pmatrix}
            1 & 0 & 0\\
            0 & 1 & 0\\
            0 & 0 & 2
        \end{pmatrix}\begin{pmatrix}
            a\\
            b\\
            c
        \end{pmatrix} = \begin{pmatrix}
            -\frac{3}{2h}\\
            \frac{2}{h}\\
            -\frac{1}{h}
        \end{pmatrix} \iff \begin{pmatrix}
            a\\
            b\\
            c
        \end{pmatrix} = \begin{pmatrix}
            -\frac{3}{2h}\\
            \frac{4}{2h}\\
            -\frac{1}{2h}
        \end{pmatrix}.
    \end{align*}
    Verwenden wir diese Koeffizienten zusammen mit den Taylor-Entwicklungen von \(f\parentheses*{x + h}\) und \(f\parentheses*{x + 2h}\), so erhält man:
    \[
        af\parentheses*{x} + bf\parentheses*{x + h} + cf\parentheses*{x + 2h} \approx f'\parentheses*{x} + f'''\parentheses*{x}\parentheses*{\frac{4}{2h} \cdot \frac{h^3}{6} - \frac{1}{2h} \cdot \frac{4h^3}{3}} = f'\parentheses*{x} - \frac{h^2}{3}f'''\parentheses*{x}.
    \]
    Damit ist die Konsistenzordnung \(2\).


    \section*{Aufgabe 4}
    
    \begin{problem}
        Gegeben ist das Konvektions-Diffusionsproblem: Gesucht ist \(u \in C^2\parentheses*{0, 1}\) mit
        \begin{align*}
            -u''\parentheses*{x} + 10u'\parentheses*{x} + u\parentheses*{x} &= f\parentheses*{x}, \quad x \in \parentheses*{0, 1},\\
            u\parentheses*{0} = u\parentheses*{1} &= 0.
        \end{align*}
        Dieses Problem soll mithilfe einer finite Differenzen Methode auf einem regelmäßgien Gitter der Schrittweite \(h\) und den Gitterpunkten \(0 = x_0 < x_1 < \cdots < x_n = 1\) approximiert und in ein lineares Gleichungssystem der Form \(A_h u_h = b_h\) überführt werden.
        Dazu sollen die Differenzenquotienten
        \begin{align*}
            u'\parentheses*{x_i} &\approx \frac{u\parentheses*{x_{i + 1}} - u\parentheses*{x_{i - 1}}}{2h},\\
            u''\parentheses*{x_i} &\approx \frac{u\parentheses*{x_{i + 1}} - 2u\parentheses*{x_i} + u\parentheses*{x_{i - 1}}}{h^2}
        \end{align*}
        benutzt werden.
        \begin{enumerate}
            \item Bestimmen Sie \(A_h\) und \(b_h\).
            \item Geben Sie eine Bedingung an, unter der \(A_h\) diagonaldominant ist.
            \item Wie müsste das Problem diskretisiert werden, um ohne Zusatzbedingung eine diagonaldominante Matrix \(A_h\) zu erhalten?
            Wie sähe die Matrix aus?
        \end{enumerate}
    \end{problem}
    
    \subsection*{Lösung}
    \begin{enumerate}
        \item Sei \(h = \frac{1}{n}\) und \(x_i = ih\) für \(i = 0, \ldots, n\).
        Weiter sei \(u_i = u\parentheses*{x_i}\).
        Dann gilt für \(i = 2, \ldots, n - 2\)
        \begin{align*}
            -\frac{u_{i + 1} - 2u_i + u_{i - 1}}{h^2} + 10 \cdot \frac{u_{i + 1} - u_{i - 1}}{2h} + u_i &= f\parentheses*{ih}\\
            \implies \frac{1}{2h^2}\parentheses*{\parentheses*{-2 + 10h}u_{i + 1} + \parentheses*{4 + 2h^2}u_i + \parentheses*{-2 - 10h}u_{i - 1}} &= f\parentheses*{ih}.
        \end{align*}
        Da \(u_0 = u\parentheses*{0} = 0\) und \(u_n = u\parentheses*{1} = 0\) gilt für \(i = 1\) bzw. \(i = n - 1\)
        \begin{align*}
            \frac{1}{2h^2}\parentheses*{\parentheses*{-2 + 10h}u_{2} + \parentheses*{4 + 2h^2}u_1} &= f\parentheses*{h},\\
            \frac{1}{2h^2}\parentheses*{\parentheses*{4 + 2h^2}u_{n - 1} + \parentheses*{-2 - 10h}u_{n - 2}} &= f\parentheses*{\parentheses*{n - 1}h}.
        \end{align*}
        Insgesamt erhalten wir das Gleichungssystem \(A_h u_h = b_h\) mit
        \[
            A_h = \frac{1}{h^2}\begin{pmatrix}
                2 + h^2 & -1 + 5h\\
                -1 - 5h & 2 + h^2 & -1 + 5h\\
                & \ddots & \ddots & \ddots\\
                & & -1 - 5h & 2 + h^2 & -1 + 5h\\
                & & & -1 - 5h & 2 + h^2
            \end{pmatrix}
        \]
        und
        \[
            b_h = \begin{pmatrix}
                f\parentheses*{h}\\
                f\parentheses*{2h}\\
                \vdots\\
                f\parentheses*{\parentheses*{n - 2}h}\\
                f\parentheses*{\parentheses*{n - 1}h}
            \end{pmatrix}, \quad u_h = \begin{pmatrix}
                u_1\\
                u_2\\
                \vdots
                u_{n - 2}\\
                u_{n - 1}
            \end{pmatrix}.
        \]
        \item Für \(h \le \frac{1}{5}\) gilt
        \begin{align*}
            \absolute*{-1 + 5h} \le 2 &< 2 + h^2\\
            \absolute*{-1 - 5h} + \absolute*{-1 + 5h} = 1 + 5h + 1 - 5h = 2 &< 2 + h^2\\
            \absolute*{-1 - 5h} \le 2 &< 2 + h^2.
        \end{align*}
        Damit gilt \(\sum_{j \ne i}\absolute*{a_{i, j}} < a_{i, i}\) für \(i = 1, \ldots, n - 1\) und \(A_h\) ist diagonaldominant.
        \item Wird für die Approximation von \(u'\parentheses*{x_i}\) anstatt der zentralen Differenzen, die Upwind-Diskretisierung
        \[
            u'\parentheses*{x_i} \approx \frac{u\parentheses*{x_i} - u\parentheses*{x_{i - 1}}}{h}
        \]
        benutzt, so erhalten wir die Matrix
        \[
            A_h = \frac{1}{h^2}\begin{pmatrix}
                2 + 10h + 2h^2 & -1\\
                -1 - 10h & 2 + 10h + 4h^2 & -1\\
                & \ddots & \ddots & \ddots\\
                & & -1 - 10h & 2 + 10h + 2h^2 & -1\\
                & & & -1 - 10h & 2 + 10h + h^2
            \end{pmatrix}.
        \]
        Mit dieser Diskretisierung ist \(A_h\) eine diagonaldominante Matrix, da
        \[
            \sum_{j \ne i}\absolute*{a_{i, j}} \le 2 + 10h < 2 + 10h + 4h^2 = a_{i, i}, \quad i = 1, \ldots, n - 1.
        \]
    \end{enumerate}
\end{document}
