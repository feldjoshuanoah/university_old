\documentclass{exercise}

\institute{Applied and Computational Mathematics}
\title{Hausaufgabenübung 13}
\author{Joshua Feld, 406718}
\course{Mathematische Grundlagen IV}
\professor{Torrilhon \& Berkels}
\semester{Sommersemester 2022}
\program{CES (Bachelor)}

\begin{document}
    \maketitle


    \section*{Aufgabe 1}
    
    \begin{problem}
        Sei \(f \in S\parentheses*{\R}\) eine ungerade Funktion, d.h. \(f\parentheses*{x} = -f\parentheses*{-x}\) für alle \(x \in \R\).
        Die Sinustransformierte von \(f\parentheses*{x}\) ist definiert durch
        \[
            \mathcal{F}_s\parentheses*{f\parentheses*{x}}\parentheses*{\xi} := \sqrt{\frac{2}{\pi}}\int_0^\infty f\parentheses*{x}\sin\parentheses*{x\xi}\d x.
        \]
        Weiterhin bezeichnen
        \[
            \mathcal{F}\parentheses*{f\parentheses*{x}}\parentheses*{\xi} := \frac{1}{\sqrt{2\pi}}\int_{-\infty}^\infty f\parentheses*{x}e^{-ix\xi}\d x
        \]
        die Fouriertransformierte und
        \[
            \mathcal{F}^{-1}\parentheses*{g\parentheses*{x}}\parentheses*{\xi} := \frac{1}{\sqrt{2\pi}}\int_{-\infty}^\infty g\parentheses*{x}e^{ix\xi}\d x
        \]
        die inverse Fouriertransformierte.
        \begin{enumerate}
            \item Zeigen Sie den Zusammenhang
            \[
                \mathcal{F}\parentheses*{f\parentheses*{x}} = -i\mathcal{F}_s\parentheses*{f\parentheses*{x}}.
            \]
            \item Benutzen Sie das Ergebnis aus a) um die Parseval-Gleichung
            \[
                \int_0^\infty \absolute*{f\parentheses*{x}}^2 \d x = \int_0^\infty \absolute*{\mathcal{F}_s\parentheses*{f\parentheses*{x}}\parentheses*{\xi}}^2 \d\xi
            \]
            zu zeigen.
            \item Finden Sie einen Zusammenhang zwischen der Sinus- und der inversen Fouriertransformation, und zeigen Sie
            \[
                f\parentheses*{x} = \mathcal{F}_s\parentheses*{\mathcal{F}_s\parentheses*{f\parentheses*{x}}}.
            \]
        \end{enumerate}
    \end{problem}
    
    \subsection*{Lösung}
    \begin{enumerate}
        \item Wir berechnen folgende Integrale mit \(f\parentheses*{x} = -f\parentheses*{-x}\)
        \begin{align*}
            \int_{-\infty}^\infty f\parentheses*{x}\sin\parentheses*{x\xi}\d x &= 2\int_0^\infty f\parentheses*{x}\sin\parentheses*{x\xi}\d x = \sqrt{2\pi}\mathcal{F}_s\parentheses*{f\parentheses*{x}}\parentheses*{\xi},\\
            \int_{-\infty}^\infty f\parentheses*{x}\cos\parentheses*{x\xi}\d x &= 0,
        \end{align*}
        um die Relation zwischen der Sinustransformierten und der Fourier-Transformation herzustellen:
        \begin{align*}
            \mathcal{F}\parentheses*{f\parentheses*{x}}\parentheses*{\xi} &= \frac{1}{\sqrt{2\pi}}\int_{-\infty}^\infty f\parentheses*{x}\exp\parentheses*{-i\xi x}\d x\\
            &= \frac{1}{\sqrt{2\pi}}\int_{-\infty}^\infty f\parentheses*{x}\parentheses*{\cos\parentheses*{x\xi} - i\sin\parentheses*{x\xi}}\d x\\
            &= \frac{1}{\sqrt{2\pi}}\int_{-\infty}^\infty f\parentheses*{x}\parentheses*{-i\sin\parentheses*{x\xi}}\\
            &= -i\mathcal{F}_s\parentheses*{f\parentheses*{x}}\parentheses*{\xi}.
        \end{align*}
        \item Wir benutzen die Parsevalsche Formel für die Fouriertransformierte, um zu zeigen, dass
        \begin{align*}
            \int_0^\infty \absolute*{f\parentheses*{x}}^2 \d x &= \frac{1}{2}\int_{-\infty}^\infty \absolute*{f\parentheses*{x}}^2 \d x\\
            &= \frac{1}{2}\int_{-\infty}^\infty \absolute*{\mathcal{F}\parentheses*{f\parentheses*{x}}\parentheses*{\xi}}^2 \d\xi\\
            &= \frac{1}{2}\int_{-\infty}^\infty \absolute*{-i\mathcal{F}_s\parentheses*{f\parentheses*{x}}\parentheses*{\xi}}^2 \d\xi\\
            &= \int_0^\infty \absolute*{\mathcal{F}\parentheses*{f\parentheses*{x}}\parentheses*{\xi}}^2 \d\xi.
        \end{align*}
        \item Analog zu a) gilt die folgende Beziehung zwischen der Sinus- und der inversen Fourier-Transformation
        \[
            \mathcal{F}^{-1}\parentheses*{f\parentheses*{x}}\parentheses*{\xi} = i\mathcal{F}_s\parentheses*{f\parentheses*{x}}\parentheses*{\xi}.
        \]
        Wir folgern mithilfe des Fourierschen Umkehrsatzes
        \[
            f\parentheses*{x} = \mathcal{F}^{-1}\parentheses*{\mathcal{F}\parentheses*{f\parentheses*{x}}} = \mathcal{F}^{-1}\parentheses*{-i\mathcal{F}_s\parentheses*{f\parentheses*{x}}} = i\mathcal{F}_s\parentheses*{-i\mathcal{F}_s\parentheses*{f\parentheses*{x}}} = \mathcal{F}_s\parentheses*{\mathcal{F}_s\parentheses*{f\parentheses*{x}}}.
        \]
    \end{enumerate}


    \section*{Aufgabe 2}
    
    \begin{problem}
        Gegeben sei das Anfangswertproblem der inhomogenen Wärmeleitungsgleichung
        \begin{align}
            u_t\parentheses*{t, x} - \Delta u\parentheses*{t, x} &= f\parentheses*{t, x}, \quad t > 0, x \in \R, \label{eq:1}\\
            u\parentheses*{0, x} &= 0, \quad x \in \R. \nonumber
        \end{align}
        \begin{enumerate}
            \item Zeigen Sie, dass die Fourier-Transformation von \(f\parentheses*{x} = \exp\parentheses*{-x^2}\) bzgl. \(x\) lautet:
            \[
                \mathcal{F}\parentheses*{f\parentheses*{x}}\parentheses*{\xi} = \sqrt{\pi}\exp\parentheses*{-\frac{\xi^2}{4}}.
            \]
            \emph{Hinweis: Sie dürfen \(\int_\R \exp\parentheses*{-x^2}\d x = \sqrt{\pi}\) verwenden.}
            \item Beweisen Sie, dass die Fouriertransformierte bzgl. \(x\) von \(g\parentheses*{x} = \exp\parentheses*{-\frac{x^2}{4\parentheses*{t - s}}}\) für \(t > s\) durch
            \[
                \mathcal{F}\parentheses*{g\parentheses*{x}}\parentheses*{\xi} = \sqrt{4\pi\parentheses*{t - s}}\exp\parentheses*{\parentheses*{s - t}\xi^2}
            \]
            gegeben ist.
            \item Leiten Sie eine gewöhnliche Differentialgleichung für
            \[
                \hat{u}\parentheses*{t, \xi} = \mathcal{F}\parentheses*{u\parentheses*{t, x}}\parentheses*{\xi}
            \]
            hier, indem Sie die Fourier-Transformation bzgl. \(x\) auf \eqref{eq:1} anwenden.
            Lösen Sie diese Differentialgleichung für \(\hat{u}\parentheses*{t, \xi}\).
            \item Zeigen Sie, dass die Lösung des ursprünglichen Problems \eqref{eq:1} wie folgt lautet:
            \[
                u\parentheses*{t, x} = \int_0^t \int_\R \frac{1}{\sqrt{4\pi\parentheses*{t - s}}}\exp\parentheses*{-\frac{\parentheses*{x - y}^2}{4\parentheses*{t - s}}}f\parentheses*{s, y}\d y\d s.
            \]
        \end{enumerate}
    \end{problem}
    
    \subsection*{Lösung}
    \begin{enumerate}
        \item Definiert man \(\hat{f}\parentheses*{\xi} := \mathcal{F}\parentheses*{f\parentheses*{x}}\parentheses*{\xi}\), so gilt
        \begin{align*}
            \frac{\d}{\d\xi}\hat{f}\parentheses*{\xi} &= \mathcal{F}\parentheses*{-ixf\parentheses*{x}}\parentheses*{\xi}\\
            &= i\mathcal{F}\parentheses*{-xf\parentheses*{x}}\parentheses*{\xi}\\
            &= \frac{i}{2}\mathcal{F}\parentheses*{-2x\exp\parentheses*{-x^2}}\parentheses*{\xi}\\
            &= \frac{i}{2}\mathcal{F}\parentheses*{\frac{\d}{\d x}\exp\parentheses*{-x^2}}\parentheses*{\xi}\\
            &= \frac{i}{2}i\xi\mathcal{F}\parentheses*{f\parentheses*{x}}\parentheses*{\xi}\\
            &= -\frac{\xi}{2}\hat{f}\parentheses*{\xi}.
        \end{align*}
        Dises ODE besitzt die Lösung
        \[
            \hat{f}\parentheses*{\xi} = \hat{f}\parentheses*{0}\exp\parentheses*{-\frac{\xi^2}{4}},
        \]
        mit
        \[
            \hat{f}\parentheses*{0} = \int_\R \exp\parentheses*{-x}\d x = \sqrt{\pi}.
        \]
        Daher ist
        \[
            \hat{f}\parentheses*{\xi} = \sqrt{\pi}\exp\parentheses*{-\frac{\xi^2}{4}}.
        \]
        \item Für \(g\parentheses*{x} = \exp\parentheses*{-\frac{x^2}{4\parentheses*{t - s}}}\) und \(\alpha := \sqrt{4\parentheses*{t - s}}, t > s\) erhält man
        \begin{align*}
            \mathcal{F}\parentheses*{g\parentheses*{x}}\parentheses*{\xi} &= \int_\R \exp\parentheses*{-\frac{x^2}{4\parentheses*{t - s}}}\exp\parentheses*{-ix\xi}\d x\\
            &= \int_\R \exp\parentheses*{-\parentheses*{\frac{x}{\alpha}}^2}\exp\parentheses*{-i\frac{x}{\alpha}\alpha\xi}\d x\\
            &= \alpha\int_\R \exp\parentheses*{-y^2}\exp\parentheses*{-iy\alpha\xi}\d y.
        \end{align*}
        Daher ist
        \[
            \mathcal{F}\parentheses*{g\parentheses*{x}}\parentheses*{\xi} = \alpha\mathcal{F}\parentheses*{f\parentheses*{x}}\parentheses*{\alpha\xi} = \alpha\sqrt{\pi}\exp\parentheses*{-\frac{\xi^2 \alpha^2}{4}} = \sqrt{4\parentheses*{t - s}\pi}\exp\parentheses*{-\parentheses*{t - s}\xi^2}.
        \]
        \item Wendet man die Fourier-Transformation auf \eqref{eq:1} an, so folgt
        \begin{align*}
            \partial_t\hat{u}\parentheses*{t, \xi} + \xi^2 \hat{u}\parentheses*{t, \xi} &= \hat{f}\parentheses*{t, \xi},\\
            \hat{u}\parentheses*{0, \xi} &= 0.
        \end{align*}
        Es handelt sich hierbei um eine \emph{inhomogene} gewöhnliche DGL 1. Ordnung, die
        \[
            \hat{u}_h\parentheses*{t, \xi} = C\exp\parentheses*{-t\xi^2}
        \]
        als Lösung der \emph{homogenen} Gleichnug besitzt.
        Variation der Konstanten führt auf den Ansatz
        \[
            \hat{u}\parentheses*{t, \xi} = C\parentheses*{t}\exp\parentheses*{-t\xi^2}
        \]
        in DGL
        \[
            C'\parentheses*{t}\exp\parentheses*{-t\xi^2} \underbrace{- C\parentheses*{t}\xi^2 \exp\parentheses*{-t\xi^2} + \xi^2 C\parentheses*{t}\exp\parentheses*{-t\xi^2}}_{= 0} = \hat{f}\parentheses*{t, \xi}.
        \]
        Daher ist
        \[
            C\parentheses*{t} = \int_0^t \hat{f}\parentheses*{s, \xi}\exp\parentheses*{s\xi^2}\d s + K, \quad K \in \R.
        \]
        Aufgrund des Anfangswerts \(0 = \hat{u}\parentheses*{0, \xi} = K\) erhält man insgesamt
        \begin{align*}
            \hat{u}\parentheses*{t, \xi} &= \int_0^t \hat{f}\parentheses*{s, \xi}\exp\parentheses*{\parentheses*{s - t}\xi^2}\d s\\
            &= \int_0^t \frac{\hat{f}\parentheses*{s, \xi}\mathcal{F}\parentheses*{\exp\parentheses*{-\frac{x^2}{4\parentheses*{t - s}}}}\parentheses*{\xi}}{\sqrt{4\pi\parentheses*{t - s}}}\d s\\
            &= \int_0^t \frac{\mathcal{F}\parentheses*{f\parentheses*{s, x}\exp\parentheses*{-\frac{x^2}{4\parentheses*{t - s}}}}\parentheses*{\xi}}{\sqrt{4\pi\parentheses*{t - s}}}\d s.
        \end{align*}
        \item Eine Rücktransformation liefert die Lösung \(u\parentheses*{t, x}\) wie im Folgenden
        \begin{align*}
            u\parentheses*{t, x} &= \mathcal{F}^{-1}\parentheses*{\hat{u}\parentheses*{t, \xi}}\parentheses*{x}\\
            &= \int_0^t \frac{\parentheses*{f\parentheses*{s, x}\exp\parentheses*{-\frac{x^2}{4\parentheses*{t - s}}}}\parentheses*{x}}{\sqrt{4\pi\parentheses*{t - s}}}\d s\\
            &= \int_0^t \frac{1}{\sqrt{4\pi\parentheses*{t - s}}}\int_\R \exp\parentheses*{-\frac{\parentheses*{x - y}^2}{4\parentheses*{t - s}}}f\parentheses*{s, y}\d y\d s.
        \end{align*}
    \end{enumerate}


    \section*{Aufgabe 3}
    
    \begin{problem}
        Gegeben seien die symmetrisch positiv definite Matrix \(A\) und der Vektor \(b\):
        \[
            A = \begin{pmatrix}
                1 & 2 & 1\\
                2 & 5 & 2\\
                1 & 2 & 2
            \end{pmatrix} \quad \text{und} \quad b = \begin{pmatrix}
                1\\
                1\\
                1
            \end{pmatrix}.
        \]
        \begin{enumerate}
            \item Zur Lösung des Gleichungssystems \(Ax = b\) führen Sie \emph{zwei Schritte} des CG-Verfahrens mit dem Startvektor \(x_0 = \parentheses*{1, 0, 0}^T\) durch.
            \item Schätzen Sie ab, wie viele Schritte notwendig sind, um eine Genauigkeit des Verfahrens kleiner als \(10^{-8}\) zu erreichen.
        \end{enumerate}
    \end{problem}
    
    \subsection*{Lösung}
    \begin{enumerate}
        \item Startwerte: \(x_0 = \begin{pmatrix}
            1\\
            0\\
            0
        \end{pmatrix}\) und \(d_0 = r_0 = b - Ax_0 = \begin{pmatrix}
            0\\
            -1\\
            0
        \end{pmatrix}\).
        \begin{enumerate}
            \item \(x_1\) mit \(k = 0\):
            \begin{align*}
                \alpha_0 &= \frac{r_0^T r_0}{d_0^T Ad_0} = \frac{1}{5},\\
                x_1 &= x_0 + \alpha_0 d_0 = \begin{pmatrix}
                    1\\
                    -\frac{1}{5}\\
                    0
                \end{pmatrix},\\
                r_1 &= r_0 - \alpha_0 Ad_0 = \begin{pmatrix}
                    \frac{2}{5}\\
                    0\\
                    \frac{2}{5}
                \end{pmatrix},\\
                \beta_1 &= \frac{r_1^T r_1}{r_0^T r_0} = 0,32,\\
                d_1 &= r_1 + \beta_1 d_0 = \begin{pmatrix}
                    \frac{2}{5}\\
                    -0,32\\
                    \frac{2}{5}
                \end{pmatrix}.
            \end{align*}
            \item \(x_2\) mit \(k = 1\):
            \[
                \alpha_1 = 1,11, \quad x_2 = \begin{pmatrix}
                    1,44\\
                    -0,55\\
                    0,44
                \end{pmatrix}, \quad r_2 = \begin{pmatrix}
                    0,22\\
                    0\\
                    -0,22
                \end{pmatrix}, \quad \beta_2 = 3,3058, \quad d_2 = \begin{pmatrix}
                    1,5423\\
                    -1,0579\\
                    1,1023
                \end{pmatrix}.
            \]
            (Die Werte für \(\beta_2\) und \(d_2\) werden erst im dritten Schritt verwendet und müssen an dieser Stelle deswegen nicht unbedingt bestimmt werden.)
        \end{enumerate}
        \item Da das CG-Verfahren endlich ist und für ein lineares Gleichungssystem der Dimension \(n\) spätestens nach \(n\) Schritten (bis auf Rundungsfehler) die exakte Lösung liefert, erhält man nach drei Schritten die exakte Lösung.
    \end{enumerate}


    \section*{Aufgabe 4}
    
    \begin{problem}
        Gegeben sei das lineare Gleichungssystem \(Ax = b\) mit
        \[
            A = \begin{pmatrix}
                1 & -2\\
                1 & -4
            \end{pmatrix}, \quad b = \begin{pmatrix}
                10\\
                -1
            \end{pmatrix}.
        \]
        \begin{enumerate}
            \item Führen Sie jeweils \emph{einen Schritt} des Jacobi-Verfahrens und Gauß-Seidel-Verfahrens mit den Startvektor \(x^0 = \parentheses*{1, 1}^T\) durch.
            \item Das SOR-Verfahren lässt sich in Matrix-Darstellung wie folgt formulieren
            \[
                x^{k + 1} = x^k - \omega D^{-1}\parentheses*{-Lx^{k + 1} + \parentheses*{D - U}x^k - b},
            \]
            wobei \(D\) (Diagonal Matrix), \(L\) (Lower Matrix) und \(U\) (Upper Matrix) genau wie beim Gauß-Seidel-Verfahren definiert sind.
            Im Unterschied zum Gauß-Seidel-Verfahren wird die Korrektur mit einem Relaxationsparameter \(\omega > 0\) multipliziert.

            Führen Sie \emph{einen Schritt} des SOR-Verfahrens mit Relaxationsparameter \(\omega = 1,5\) mit dem Startwert \(x^0 = \parentheses*{1, 1}^T\) durch.
            \item Zeigen Sie die Konvergenz des SOR-Verfahrens mit Relaxationsparameter \(\omega = 1,5\).
            \item Vergleichen Sie die Lösungen mit der wahren Lösung des Gleichungssystems.
            Welches Verfahren hat den kleinsten bzw. größten Fehler in der \(1\)-Norm?
        \end{enumerate}
    \end{problem}
    
    \subsection*{Lösung}
    \begin{enumerate}
        \item \(A\) lässt sich zerlegen in \(D - L - U\) mit \(A = D - L - U\) wie folgt
        \[
            D = \begin{pmatrix}
                1 & 0\\
                0 & -4
            \end{pmatrix}, \quad L = \begin{pmatrix}
                0 & 0\\
                -1 & 0
            \end{pmatrix}, \quad U = \begin{pmatrix}
                0 & 2\\
                0 & 0
            \end{pmatrix}.
        \]
        Das Residuum für den Startvektor \(x^0\) beträgt \(r^0 = Ax^0 - b = \parentheses*{-11, 2}^T\).
        \begin{itemize}
            \item Jacobi-Verfahren:
            \[
                x^1 = x^0 - D^{-1}\parentheses*{Ax^0 - b} = x^0 - D^{-1}r^0 = \begin{pmatrix}
                    1\\
                    1
                \end{pmatrix} - \begin{pmatrix}
                    1 & 0\\
                    0 & -\frac{1}{4}
                \end{pmatrix}\begin{pmatrix}
                    -11\\
                    -2
                \end{pmatrix} = \begin{pmatrix}
                    12\\
                    \frac{1}{2}
                \end{pmatrix}.
            \]
            \item Gauß-Seidel-Verfahren:
            \[
                x^1 = x^0 - \parentheses*{D - L}^{-1}\parentheses*{Ax^0 - b} = x^0 - \parentheses*{D - L}^{-1}r^0.
            \]
            Pro Iterationsschritt ist hier das Gleichungssystem \(\parentheses*{D - L}v = r^0\), bzw.
            \[
                \begin{pmatrix}
                    1 & 0\\
                    1 & -4
                \end{pmatrix}\begin{pmatrix}
                    v_1\\
                    v_2
                \end{pmatrix} = \begin{pmatrix}
                    -11\\
                    -2
                \end{pmatrix}
            \]
            zu lösen.
            Die Lösung ist \(v = \parentheses*{-11, -\frac{9}{4}}^T\) und folglich
            \[
                x^1 = x^0 - v = \parentheses*{12, \frac{13}{4}}^T.
            \]
            \item SOR-Verfahren:
            \[
                x^1 = \parentheses*{I - \parentheses*{\omega^{-1}D - L}^{-1}A}x^0 + \parentheses*{\omega^-1 D - L}^{-1}b = \begin{pmatrix}
                    \frac{35}{2}\\
                    \frac{103}{16}
                \end{pmatrix}.
            \]
        \end{itemize}
        \item Die Eigenwerte der Iterationsmatrix mit \(\omega = 1,5\)
        \[
            M_\omega = I - \parentheses*{\omega^{-1}D - L}^{-1}A = \begin{pmatrix}
                -\frac{1}{2} & 3\\
                -\frac{3}{16} & \frac{5}{8}
            \end{pmatrix}
        \]
        sind
        \[
            \lambda_{1, 2} = \frac{1}{16} \pm \frac{3\sqrt{7}}{16}i
        \]
        und damit
        \[
            \rho\parentheses*{M_\omega} = \frac{1}{2} < 1.
        \]
        Daher konvergiert das SOR-Verfahren.
        \item Die Lösung des linearen Gleichungssystems ist \(\begin{pmatrix}
            21\\
            \frac{11}{2}
        \end{pmatrix}\).
        \begin{itemize}
            \item Der Fehler im Jacobi-Verfahren ist somit
            \[
                \norm*{\begin{pmatrix}
                    12\\
                    \frac{1}{2}
                \end{pmatrix} - \begin{pmatrix}
                    21\\
                    \frac{11}{2}
                \end{pmatrix}}_1 = \norm*{\begin{pmatrix}
                    -9\\
                    -5
                \end{pmatrix}}_1 = 9 + 5 = 14.
            \]
            \item Der Fehler im Gauß-Seidel-Verfahren ist
            \[
                \norm*{\begin{pmatrix}
                    12\\
                    \frac{13}{4}
                \end{pmatrix} - \begin{pmatrix}
                    21\\
                    \frac{11}{2}
                \end{pmatrix}}_1 = \norm*{\begin{pmatrix}
                    -9\\
                    -\frac{9}{4}
                \end{pmatrix}}_1 = 9 + \frac{9}{4} = 11,25.
            \]
            \item Der Fehler im SOR-Verfahren ist
            \[
                \norm*{\begin{pmatrix}
                    \frac{35}{2}\\
                    \frac{103}{16}
                \end{pmatrix} - \begin{pmatrix}
                    21\\
                    \frac{11}{2}
                \end{pmatrix}}_1 = \norm*{\begin{pmatrix}
                    -\frac{7}{2}\\
                    \frac{15}{16}
                \end{pmatrix}}_1 = \frac{7}{2} + \frac{15}{16} = \frac{71}{16} = 4,4375.
            \]
        \end{itemize}
        Somit ist der Fehler in der \(1\)-Norm für das SOR-Verfahren am kleinsten und für das Jacobi-Verfahren am größten.
    \end{enumerate}
\end{document}
