\documentclass{lecture}

\institute{Applied and Computational Mathematics}
\title{Vorlesung 1}
\author{Joshua Feld, 406718}
\course{Mathematische Grundlagen IV}
\professor{Torrilhon \& Berkels}
\semester{Sommersemester 2022}
\program{CES (Bachelor)}

\begin{document}
    \maketitle


    \section*{Diskrete Fourier-Transformation}

    Die Fourier-Transformation ist ein wichtiges Werkzeug sowohl in der Analysis als auch in der Numerik.
    Grob gesprochen zerlegt die Fourier-Transformation ein Signal (z.B. einen Ton) in seine ``Frequenzen''.
    Beispiele für Anwendungen sind:
    \begin{itemize}
        \item Lösung von partiellen Differentialgleichungen durch Reihenansatz bzw. durch die Fourier-Transformation
        \item Approximation von Funktionen durch Orthonormalsysteme (siehe Mathe 1 und 2)
        \item Schnelle Berechnung von Produkten großer Zahlen und Faltungen
        \item Schnelle Interpolation
        \item Datenkompression (z.B. nutzen MP3 und JPEG Varianten der Fourier-Transformation)
    \end{itemize}
    Der praktische Erfolg beruht auf einer speziellen Einsicht, die den Aufwand der Berechnung der Fourier-Transformierten von \(O\parentheses*{n^2}\) auf \(O\parentheses*{n\log n}\) (wobei \(n\) die Anzahl der Datenpunkte ist) reduziert.
    Die sogenannte Fast Fourier Transformation (FFT) gilt als einer der wichtigsten Algorithmen überhaupt.
    Zugeschrieben wird die Idee Cooley und Tukey (1965); allerdings gab es die Idee auch schon 1805 von Gauss, um die Bahnen von Asteroiden bestimmen zu können.
    Im Folgenden leiten wir zunächst die diskrete Fourier Transformation als Lösung einer Interpolationsaufgabe her und dann den Algorithmus für die FFT in seinen Grundzügen.


    \section*{Trigonometrsiche Interpolation}

    Wir beginnen direkt in komplexer Notation und beschränken uns auf periodische Funktionen, d.h. Funktionen \(f: \R \to \C\), so dass es ein \(P > 0\) gibt mit \(f\parentheses*{x} = f\parentheses*{x + P}\) für alle \(x \in \R\).
    \(P\) nennt man Periode.
    Ohne Beschränkung der Allgemeinheit sei \(P = 2\pi\).
    In diesem Abschnitt seien daher (wenn nicht anders angegeben) alle Funktionen der Art \(f: \brackets*{0, 2\pi} \to \C\).
    Wir benutzen die Notation \(\overline{f\parentheses*{x}}\) für die zu \(f\parentheses*{x}\) komplex konjugierte Zahl.

    Wir erinnern uns, dass
    \[
        \angles*{f, g} := \frac{1}{2\pi}\int_0^{2\pi}f\parentheses*{x}\overline{g\parentheses*{x}}\d x
    \]
    ein inneres Produkt zweier Funktionen \(f\) und \(g\) in \(L^2\parentheses*{\brackets*{0, 2\pi}; \C}\) definiert.
    \begin{definition}
        Für \(j \in \Z\) sei
        \[
            e_j: \brackets*{0, 2\pi} \to \C, x \mapsto e^{ijx},
        \]
        wobei \(i \in \C\) die imaginäre Einheit bezeichnet.
        Die Funktionen \(e_j, j \in \Z\), heißen \emph{Grundschwingungen}.
    \end{definition}
    \begin{lemma}\label{lemma:1}
        Für \(j, k \in \Z\) gilt
        \[
            \angles*{e_j, e_k} = \frac{1}{2\pi}\int_0^{2\pi}e^{ijx}e^{-ikx}\d x = \delta_{j, k}.
        \]
    \end{lemma}
    \begin{proof}
        Mit Hilfe der Identität \(e^{ix} = \cos x + i \sin x\) lässt sich die Aussage leicht nachrechnen (Übung).
    \end{proof}
    \begin{remark}
        Lemma \ref{lemma:1} zeigt, dass die Grundschwingungen \(\parentheses*{e_j}_{j \in \Z}\) ein Orthonormalsystem bzgl. \(\angles*{\cdot, \cdot}\) sind.
        Wir können nun auf den von einer Menge von Grundschwingungen aufgespannten Raum projizieren.
        Für \(n \in \N\) seien
        \[
            S_n\parentheses*{f; x} := \sum_{\absolute*{k} \le n}\angles*{f, e_k}e^{ikx}
        \]
        und
        \[
            U_{2n + 1} := \setspan\braces*{e_k : \absolute*{k} \le n}.
        \]
        Damit ist \(S_n\parentheses*{f; \cdot} = \sum_{\absolute*{k} \le n}\angles*{f, e_k}e_k \in U_{2n + 1}\) die Orthogonalprojektion von \(f \in L^2\parentheses*{\brackets*{0, 2\pi}; \C}\) auf den Unterraum \(U_{2n + 1}\) und stellt somit eine Approximation von \(f\).
        Es gilt
        \[
            \angles*{f, e_k} = \frac{1}{2\pi}\int_0^{2\pi}f\parentheses*{x}e^{-ikx}\d x.
        \]
        In diesem Sinne geben die Koeffizienten an, wie viel von der \(k\)-ten ``Frequenz'' \(e_k\) im Signal \(f\) enthalten ist.
        Die Koeffizienten \(\angles*{f, e_k}\) heißen \emph{Fourier-Koeffizienten}.
        Die Projektion \(S_n\parentheses*{f; \cdot}\) nennt man \emph{Fourier-Teilsumme}.
    \end{remark}
    \begin{remark}
        Man kann zeigen, dass das Orthonormalsystem der Grundschwingungen \(\parentheses*{e_j}_{j \in \Z}\) auch vollständig ist, d.h. eine Art Basis von \(L^2\parentheses*{\brackets*{0, 2\pi}; \C}\) bildet.
        Genauer gilt
        \[
            S_n\parentheses*{f, \cdot} \xrightarrow{n \to \infty} f \quad \text{für alle }f \in L^2\parentheses*{\brackets*{0, 2\pi}; \C}.
        \]
        Man beachte, dass die Konvergenz nicht punktweise gilt, sondern in \(L^2\) (mehr dazu im Theorieteil).
        Die dabei entstehende Reihe
        \[
            \sum_{k \in \Z}\angles*{f, e_k}e^{ikx}
        \]
        nennt man \emph{Fourier-Reihe}.
    \end{remark}
    \begin{remark}
        Um die Fourier-Teilsummen numerisch zu berechnen, könnte man prinzipiell die Integrale in den Fourier-Koeffizienten durch Quadraturformeln approximieren.
        Dies ist aber sehr aufwendig.
        Wir betrachten im Folgenden den Weg über Interpolation, der aber auch im Nachgang als Quadratur interpretiert werden kann.
        Wir betrachten äquidistante Stützstellen
        \[
            x_j = \frac{2\pi j}{n}, \quad j = 0, \ldots, n - 1
        \]
        im Intervall \(\brackets*{0, 2\pi}\).
        Die komplexe Zahl
        \[
            \varepsilon_n := e^{-\frac{2\pi i}{n}}
        \]
        heißt \emph{\(n\)-te Einheitswurzel}, denn es gilt
        \[
            \parentheses*{\varepsilon_n^j}^n = \varepsilon_n^{jn} = e^{-2\pi ij} = 1, \quad j = 0, \ldots, n - 1,
        \]
        d.h. die Zahlen \(\varepsilon_n^j, j = 0, \ldots, n - 1\) teilen den Einheitskreis in gleich große Stücke.
    \end{remark}

    Zur Interpolation verwenden wir die folgenden Funktionenräume.
    \begin{definition}
        Für \(m \in \N\) sei
        \[
            \mathcal{T}_m := \setspan\braces*{e_j : 0 \le j < m}.
        \]
        Die Elemente von \(\mathcal{T}_m\) heißen \emph{(komplexe) trigonometrische Polynome vom Grad (kleiner-gleich) \(m - 1\)}.
    \end{definition}
    \begin{remark}
        Da die Grundschwingungen \(\parentheses*{e_j}_{j \in \Z}\) ein Orthonormalsystem bilden, sind sie insbesondere linear unabhängig und es folgt \(\dim\mathcal{T}_m = m\).
        Ferner gilt
        \[
            \mathcal{T}_m = \braces*{\brackets*{0, 2\pi} \to \C, x \mapsto \sum_{j = 0}^{m - 1}c_j e^{ijx} : c_0, \ldots, c_{m - 1} \in \C}.
        \]
        Setzt man \(z = e^{ix}\), so hat jedes Element von \(\mathcal{T}_m\) die Form
        \[
            \sum_{j = 0}^{m - 1}c_j z^j,
        \]
        d.h. die Elemente von \(\mathcal{T}_m\) sind komplexe Polynome, die wegen \(\absolute*{e^{ix}} = 1\) nur auf dem Einheitskreis in der komplexen Ebene ausgewertet werden.
        Zusammen mit der Identität \(e^{ix} = \cos x + i\sin x\) ist dies der Grund, dass man hier von komplexen trigonometrischen Polynomen spricht.
    \end{remark}
    \begin{theorem}
        \begin{enumerate}
            \item Seien \(p_1 \in \mathcal{T}_m\) und \(p_2 \in \mathcal{T}_n\).
            Dann gilt \(p_1 p_2 \in \mathcal{T}_{m + n - 1}\).
            \item Sei \(0 \ne p \in \mathcal{T}_m\).
            Dann hat \(p\) höchstens \(m -1\) verschiedene Nullstellen in \(\left[0, 2\pi\right)\).
        \end{enumerate}
    \end{theorem}
    \begin{proof}
        \begin{enumerate}
            \item Wegen \(p_1 \in \mathcal{T}_m\) und \(p_2 \in \mathcal{T}_n\) gibt es Koeffizienten \(c_j\) und \(d_k\) mit
            \[
                p_1\parentheses*{x} = \sum_{j = 0}^{m - 1}c_j e^{ijx} \quad \text{und} \quad p_2\parentheses*{x} = \sum_{k = 0}^{n - 1}d_k e^{ikx}.
            \]
            Daraus folgt
            \[
                p_1\parentheses*{x}p_2\parentheses*{x} = \sum_{j = 0}^{m - 1}\sum_{k = 0}^{n - 1}c_j d_k e^{i\parentheses*{j + k}x} \in \mathcal{T}_{m + n - 1}.
            \]
            \item Angenommen \(p = \sum_{j = 0}^{m - 1}c_j e^{ijx} \not\equiv 0\) hat \(m\) verschiedene Nullstellen \(x_1, \ldots, x_m \in \left[0, 2\pi\right)\).
            Wir betrachten das komplexe Polynom
            \[
                q: \C \to \C, z \mapsto \sum_{j = 0}^{m - 1}c_j z^j.
            \]
            Dann gilt
            \[
                q\parentheses*{e^{ix_k}} = p\parentheses*{x_k} = 0 \quad \text{für alle }k \in \braces*{1, \ldots, m}.
            \]
            Ferner sind wegen \(x_k \in \left[0, 2\pi\right)\) die \(e^{ix_k}\) verschieden, damit hat das komplexe Polynom \(q\) \(m\) verschiedene Nullstellen, ist aber vom Grad \(m - 1\).
            Somit muss \(q = 0\) gelten.
            Damit folgt aber auch, dass die Koeffizenten \(c_j\) alle Null sein müssen und somit folgt auch \(p = 0\).
            Ein Widerspruch zu unserer Annahme \(p \not\equiv 0\).
        \end{enumerate}
    \end{proof}

    Wir betrachten nun die folgende Interpolationsaufgabe bzgl. der trigonometrischen Polynome.
    \begin{problem}
        Gegeben seien Werte \(y_k \in \C, k = 0, \ldots, n - 1\), sowie Punkte \(x_k = \frac{2\pi k}{n} \in \left[0, 2\pi\right), k = 0, \ldots, n - 1\).
        Finde \(T_n \in \mathcal{T}_n\), so dass
        \[
            T_n\parentheses*{x_k} = y_k, \quad k = 0, \ldots, n - 1.
        \]
    \end{problem}
    \begin{remark}
        Sei \(f: \brackets*{0, 2\pi} \to \C\) eine Funktion.
        Durch die Wahl \(y_k = f\parentheses*{x_k}\) bedeutet die Interpolationsaufgabe in diesem Fall ein \(T_n \in \mathcal{T}_n\) zu finden, das an den Stellen \(x_k\) mit \(f\) übereinstimmt.
        Die Punkte \(x_k\) nennt man auch Stützstellen.
    \end{remark}
    \begin{remark}
        Da es sich letztlich nach Substitution um Polynome handelt, kann man die Lösbarkeit dieser Aufgabe (also Existenz und Eindeutigkeit des trigonometrischen Interpolationspolynoms) genauso beweisen wie im Fall des Standard-Interpolationspolynoms (also mithilfe des Lagrange-Polynoms, sowie des Fundamentalsatzes der Algebra; siehe Mathe 1).
    \end{remark}

    Es gibt jedoch eine einfachere Möglichkeit die Existenz zu zeigen.
    Als Vorbereitung darauf zeigen wir zunächst eine Hilfsaussage.
    \begin{lemma}
        Für \(m \in \braces*{-n + 1, \ldots, n - 1}\), gilt
        \[
            \frac{1}{n}\sum_{j = 0}^{n - 1}e^{-i\frac{2\pi mj}{n}} = \delta_{m, 0}.
        \]
    \end{lemma}
    \begin{proof}
        Für \(m = 0\) ergibt sich sofort
        \[
            \frac{1}{n}\sum_{j = 0}^{n - 1}e^{-i\frac{2\pi \cdot 0 \cdot j}{n}} = \frac{1}{n}\sum_{j = 0}^{n - 1}1 = 1.
        \]
        Für den Fall \(m \ne 0\) erinnern wir uns an die Summenformel der geometrischen Reihe:
        Es gilt
        \[
            \sum_{j = 0}^{n - 1}q^j = \frac{1 - q^n}{1 - q} \quad \text{für }q \ne 1.
        \]
        Da \(\absolute*{m} \le n - 1\), gilt \(e^{-i\frac{2\pi m}{n}} \ne 1\) und es folgt, dass
        \[
            \frac{1}{n}\sum_{j = 0}^{n - 1}e^{-i\frac{2\pi mj}{n}} = \frac{1}{n}\sum_{j = 0}^{n - 1}\parentheses*{e^{-i\frac{2\pi m}{n}}}^j = \frac{1}{n}\frac{1 - e^{-i \cdot 2\pi m}}{1 - e^{-i\frac{2\pi m}{n}}} = \frac{1}{n}\frac{1 - 1}{1 - e^{-i\frac{2\pi m}{n}}} = 0.
        \]
    \end{proof}
    \begin{theorem}
        Seien \(y = \parentheses*{y_k}_{k = 0}^{n - 1} \equiv \parentheses*{y_0, \ldots, y_{n - 1}}^T \in \C^n\) und \(x_k = \frac{2\pi k}{n}\) für \(k = 0, \ldots, n - 1\).
        Ferner sei
        \[
            d_j\parentheses*{y} := \frac{1}{n}\sum_{l = 0}^{n - 1}y_l e^{-ijx_l} = \frac{1}{n}\sum_{l = 0}^{n - 1}y_l \varepsilon_n^{jl}, \quad j = 0, \ldots, n - 1.
        \]
        Dann erfüllt das trigonometrische Polynom
        \[
            T_n\parentheses*{y; x} := \sum_{j = 0}^{n - 1}d_j\parentheses*{y}e^{ijx}
        \]
        die Interpolationsbedingungen \(T_n\parentheses*{y; x_k} = y_k\) für alle \(k = 0, \ldots, n - 1\).
        Ferner ist \(T_n\parentheses*{y; \cdot}\) das einzige Element aus \(\mathcal{T}_n\), das diese Interpolationsbedingungen erfüllt.
    \end{theorem}
    \begin{remark}
        Für den Fall \(y_k = f\parentheses*{x_k}\) beachte man die formale Ähnlichkeit zur Fourier-Teilsumme.
        Die Fourier-Koeffizienten werden hier durch eine Rechteckregel approximiert.
    \end{remark}
\end{document}
