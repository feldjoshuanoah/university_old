\documentclass{lecture}

\institute{Applied and Computational Mathematics}
\title{Vorlesung 2}
\author{Joshua Feld, 406718}
\course{Mathematische Grundlagen IV}
\professor{Torrilhon \& Berkels}
\semester{Sommersemester 2022}
\program{CES (Bachelor)}

\begin{document}
    \maketitle


    \section*{Partielle Differentialgleichungen}

    Wenn Mathematik die Sprache der Natur ist, dann sind PDEs die am meisten genutzten Vokabeln.
    Fast alle technischen Prozesse, die Felder in Ort und Zeit enthalten (z.B. die Temperatur bei \(\parentheses*{\bm{x}, t}\)), können durch PDEs beschrieben werden.

    \begin{definition}
        Eine Differentialgleichung, deren Unbekannte nach mehreren Variablen abgeleitet wird, heißt \emph{partielle Differentialgleichung}.
        Die allgemeine Form ist
        \[
            F\parentheses*{x, u, Du, \ldots, D^p u} = 0.
        \]
        mit \(u: G \to \R^m\) und dem Grundgebiet \(G\), wobei \(x = \parentheses*{x_1, \ldots, x_n}^T \in G \subset \R^n\).
    \end{definition}

    \begin{remark}
        \begin{enumerate}
            \item \(Du\) ist die Jacobi-Matrix \(\parentheses*{\frac{\partial u_i}{\partial x_j}}_{i = 1, \ldots, m; j = 1, \ldots, n}\).
            \item \(D^p u\) ist eine \(p\)-te Ableitung von \(u\), wobei \(p = \parentheses*{p_1, \ldots, p_n}\) ein Multi-Index ist, mit
            \[
                D^p = \parentheses*{\frac{\partial}{\partial x_1}}^{p_1}\parentheses*{\frac{\partial}{\partial x_2}}^{p_2}\cdots\parentheses*{\frac{\partial}{\partial x_n}}^{p_n}.
            \]
            \item Ähnlich zu ODEs unterscheiden wir skalare PDEs \(u: G \to \R\) und Systeme von PDEs mit \(m\) Gleichungen für \(m\) Unbekannte.
            Ebenso ist die Ordnung der PDE die höchste Ableitung \(\absolute*{p} = p_1 + \ldots + p_n\).
            \item In Anwendungen ist das Grundgebiet typischerweise im dreidimensionalen Raum \(G = \Omega \subset \R^d, d = 1, 2, 3\) oder es besteht aus Raum und Zeit \(G = \Omega \times \brackets*{0, T}\) mit Variablen \(\parentheses*{x, t} \in \Omega \times \R^+\).
            Für die Ableitungen nach dem Ort wird dann ``\(\nabla\)'' und \(\Delta = \sum_{k = 1}^n \frac{\partial^2}{\partial x_k^2}\) (``Laplace-Operator'') benutzt.
        \end{enumerate}
    \end{remark}

    \begin{example}
        \begin{enumerate}
            \item Die \emph{Transportgleichung} hat im eindimensionalen Raum die Form
            \[
                \partial_t u\parentheses*{x, t} + a\partial_x u\parentheses*{x, t} = 0,
            \]
            mit der Transportgeschwindigkeit \(a \in \R\) und \(u: G \to \R, \parentheses*{x, t} \mapsto u\parentheses*{x, t}\), wobei \(G = \R \times \brackets*{0, T}\).
            Im Allgemeinen Fall ändert sich die Form zu
            \[
                \partial_t u\parentheses*{\bm{x}, t} + \bm{a} \cdot \nabla u\parentheses*{\bm{x}, t} = 0,
            \]
            mit den Transportvektor \(\bm{a} = \parentheses*{a_1, \ldots, a_n}^T \in \R^n\) und \(u: G \to \R, \parentheses*{\bm{x}, t} \mapsto u\parentheses*{\bm{x}, t}\), wobei \(G = \R^n \times \brackets*{0, T}\).
            \item Die \emph{Poisson-Gleichung} lautet allgemein
            \[
                \Delta u\parentheses*{\bm{x}} = f\parentheses*{\bm{x}}
            \]
            mit der ``Quelle'' \(f: G \to \R\) und \(u: G \to \R, \bm{x} \mapsto u\parentheses*{\bm{x}}\), wobei \(G \subset \R^n\) (d.h. die Zeit spielt keine Rolle).
            Ist \(f \equiv 0\), so wird die Gleichung zur \emph{Laplace-Gleichung}.
            \item Die \emph{Diffusionsgleichung} im eindimensionalen Raum ist
            \[
                \partial_t u\parentheses*{x, t} - \kappa\partial_{xx}u\parentheses*{x, t} = f\parentheses*{x, t},
            \]
            mit dem Diffusionskoeffizienten \(\kappa \in \R\) und \(u: G \to \R, \parentheses*{x, t} \mapsto u\parentheses*{x, t}\), wobei \(G = \R \times \brackets*{0, T}\).
            In \(n\) Dimensionen verändert sich die Gleichung zu
            \[
                \partial_t u\parentheses*{\bm{x}, t} - \kappa\Delta u\parentheses*{\bm{x}, t} = f\parentheses*{\bm{x}, t}
            \]
            mit \(u: G \to \R, \parentheses*{\bm{x}, t} \mapsto u\parentheses*{\bm{x}, t}\), wobei \(G = \R^n \times \brackets*{0, T}\).
            \item Die \emph{Wellengleichung} ist definiert als
            \[\partial_{tt}u\parentheses*{\bm{x}, t} - c^2 \Delta u\parentheses*{\bm{x}, t} = f\parentheses*{\bm{x}, t},\]
            mit der Wellengeschwindigkeit \(c \in \R\) und \(u: G \to \R, \parentheses*{\bm{x}, t} \mapsto u\parentheses*{\bm{x}, t}\), wobei \(G = \R^n \times \brackets*{0, T}\).
            \item Die \emph{Navir-Stokes-Gleichungen} aus der Fluiddynamik sind
            \begin{align*}
                \partial_t \bm{u} + \bm{u} \cdot \nabla\bm{u} + \nabla p &= \nu\Delta\bm{u} + \bm{f},\\
                \nabla\bm{u} &= 0,
            \end{align*}
            mit der Viskosität \(\nu \in \R\) und dem Kraftfeld \(\bm{f}\).
            Die Strömungsgeschwindigkeit \(\bm{u}: G \to \R^d\) und der Druck \(p: G \to \R\) sind unbekannt.
            Das Grundgebiet ist hier \(G = \Omega \times \brackets*{0, T}\) mit \(\Omega \subset \R^d\).
            \item Die \emph{Maxwell-Gleichungen} aus der Elektrodynamik sind
            \begin{align*}
                \partial_t \bm{B} + \nabla\bm{E} &= 0, & \nabla\bm{B} = 0,\\
                \partial_t \bm{E} + \nabla\bm{B} &= -\bm{j}, & \nabla\bm{E} = \rho,
            \end{align*}
            mit der Ladungsdichte \(\rho \in \R\) und der Stromdichte \(\bm{j}\).
            Das elektrische Feld \(\bm{E}: G \to \R^d\) und das magnetische Feld \(\bm{B}: G \to \R^d\) sind unbekannt, wobei \(G = \Omega \times \brackets*{0, T}\) mit \(\Omega \subset \R^d\).
        \end{enumerate}
    \end{example}


    \section*{Anfangs- und Randwertprobleme}

    Ähnlich zu ODEs bestimmen die PDEs die unbekannte Funktion nicht vollständig.
    Bei PDEs ist die allgemeine Lösung \(\infty\)-dimensional.

    \begin{example}
        Für ODEs ist der einfachste Fall \(u''\parentheses*{x} = 0\).
        Wenn wir diese Gleichung zwei Mal integrieren, so erhalten wir \(u'\parentheses*{x} = c_1\) und \(u\parentheses*{x} = c_1 x + c_2\) mit \(c_1, c_2 \in \R\).
        Für PDEs wollen wir nun den Fall \(\partial_{xy}u\parentheses*{x, y} = 0\) betrachten.
        Es ergibt sich \(\partial_y u\parentheses*{x, y} = \tilde{c}_1\parentheses*{y}\) und folglich \(u\parentheses*{x, y} = \int \tilde{c}_1\parentheses*{y}\d y + \tilde{c}_2\parentheses*{x}\), wobei \(\tilde{c}_1, \tilde{c}_2\) zwei beliebige \(\infty\)-dimensionale Funktionen \(\R \to \R\) sind.
    \end{example}

    Lösungen partieller Differentialgleichungen werden also nicht durch einzelne Werte sondern durch Vorgabe von Funktionen festgelegt.
    
    \begin{definition}
        \begin{enumerate}
            \item Enthält die PDE die Zeit, z.B. \(u: \Omega \times \brackets*{0, T} \to \R^n\), werden die \emph{Anfangswerte} \(u\parentheses*{\bm{x}, 0} = u_0\parentheses*{\bm{x}}\) vorgegeben.
            \item Ist das Grundgebiet berandet, z.B. \(\Omega \subset \R^n\), so werden \emph{Randbedingungen} \(\left.u\parentheses*{\bm{x}}\right|_{x \in \partial\Omega} = g\parentheses*{\bm{x}}\) vorgegeben.
        \end{enumerate}
        Dies führt zu \emph{Anfangs-} und \emph{Randwertproblemen}.
    \end{definition}
\end{document}
