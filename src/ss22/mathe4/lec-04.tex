\documentclass{lecture}

\institute{Applied and Computational Mathematics}
\title{Vorlesung 4}
\author{Joshua Feld, 406718}
\course{Mathematische Grundlagen IV}
\professor{Torrilhon \& Berkels}
\semester{Sommersemester 2022}
\program{CES (Bachelor)}

\begin{document}
    \maketitle


    \begin{example}
        Betrachte \(\Omega = \brackets*{0, 1}^2 \subset \R^2, u: \Omega \times \brackets*{0, T} \to \R\) mit
        \[
            \partial_t u - \Delta u = f, \quad x \in \Omega, t \in \brackets*{0, T}
        \]
    \end{example}

    \begin{remark}
        \begin{enumerate}
            \item Die Anzahl an Rand- und Anfangsbedingungen hängt von der partiellen Differentialgleichung ab.
            \item Die Poission-Gleichung \(\Delta u = f\) auf einem Gebit \(\Omega\) erwartet pro Randpunkt \(x \in \partial\Omega\) eine Randbedingung.
            Es wird unterschieden zwischen
            \begin{enumerate}
                \item Dirichlet-Bedingung: \(u\parentheses*{x} = g\parentheses*{x}, x \in \partial\Omega\),
                \item Neumann-Bedingung: \(\partial_n u\parentheses*{x} = h\parentheses*{x}, x \in \partial\Omega\), wobei \(\partial_n = n \cdot \nabla\),
                \item Robin-Bedingung: \(\alpha\parentheses*{x}u\parentheses*{x} + \beta\parentheses*{x}\partial_n u\parentheses*{x} = r\parentheses*{x}, x \in \partial\Omega\).
            \end{enumerate}
        \end{enumerate}
    \end{remark}
\end{document}
