\documentclass{exercise}

\institute{Applied and Computational Mathematics}
\title{Altklausur 1}
\author{Joshua Feld, 406718}
\course{Mathematische Grundlagen IV}
\professor{Torrilhon \& Berkels}
\semester{Sommersemester 2022}
\program{CES (Bachelor)}

\begin{document}
    \maketitle


    \section*{Aufgabe 1}

    \begin{problem}
        Gegeben sei die Gleichung
        \begin{align*}
            u_x\parentheses*{x, y} + 2yu_y\parentheses*{x, y} &= y, \quad x \in \R^+, y \in \R,\\
            u\parentheses*{0, y} &= y, \quad y \in \R.
        \end{align*}
        \begin{enumerate}
            \item Bestimmen Sie die Charakteristiken der Gleichung.
            \item Bestimmen Sie die Lösung der Gleichung.
            \begin{enumerate}
                \item Formulieren Sie dazu zunächst die Differentialgleichung, die die Lösung entlang der Charakteristiken erfüllt.
                \item Lösen Sie anschließend diese Differentialgleichung.
            \end{enumerate}
            \item Überprüfen Sie, ob die in b) bestimmte Lösung die Gleichung erfüllt.
        \end{enumerate}
    \end{problem}

    \subsection*{Lösung}
    \begin{enumerate}
        \item
        \item
        \begin{enumerate}
            \item
            \item
        \end{enumerate}
        \item
    \end{enumerate}


    \section*{Aufgabe 2}

    \begin{problem}
        \begin{enumerate}
            \item Bestimmen Sie die distributionelle Ableitung der folgenden Distribution:
            \[
                \angles*{T_f, \varphi} = \int_\R f\parentheses*{x}\varphi\parentheses*{x}\d x \quad \forall\varphi \in \mathcal{D}\parentheses*{\R},
            \]
            mit
            \[
                f: \R \to \R, x \mapsto \begin{cases}
                    x^2, & \text{falls }\absolute*{x} \ge 1,\\
                    0, & \text{sonst}.
                \end{cases}
            \]
            \item Zeigen Sie, dass partielle Ableitungen von Distributionen vertauscht werden können, analog zu klassischen Funktionen:
            \[
                \angles*{\frac{\partial}{\partial x_i}\frac{\partial}{\partial x_j}T, \varphi} = \angles*{\frac{\partial}{\partial x_j}\frac{\partial}{\partial x_i}T, \varphi} \quad \forall T \in \mathcal{D}'\parentheses*{\R^n}, \varphi \in \mathcal{D}\parentheses*{\R^n}.
            \]
            \item Zeigen Sie, dass \(T_H' = \delta\), wobei
            \[
                H: \R \to \R, x \mapsto \begin{cases}
                    1, & \text{falls }x > 0,\\
                    0, & \text{falls }x \le 0
                \end{cases}
            \]
            die Heaviside-Funktion und \(\delta\) die Dirac-Distribution sind.
        \end{enumerate}
    \end{problem}

    \subsection*{Lösung}
    \begin{enumerate}
        \item
        \item
        \item
    \end{enumerate}


    \section*{Aufgabe 3}

    \begin{problem}
        Bestimmen Sie die Fouriertransformierten folgender Funktionen:
        \begin{enumerate}
            \item \(f\parentheses*{x} = e^{-\frac{x^2}{2}}\),
            \item \(f\parentheses*{x} = H\parentheses*{1 - x^2}\).
            Dabei ist \(H\) die Heaviside-Funktion
            \[
                H: \R \to \R, x \mapsto \begin{cases}
                    1, & \text{falls }x > 0,\\
                    0, & \text{falls }x \le 0.
                \end{cases}
            \]
        \end{enumerate}
        \emph{Hinweis: \(\int_\R e^{-\frac{x^2}{2}}\d x = \sqrt{2\pi}\).}
    \end{problem}

    \subsection*{Lösung}
    \begin{enumerate}
        \item
        \item
    \end{enumerate}


    \section*{Aufgabe 4}

    \begin{problem}
        Gegeben sei die Funktionenfolge
        \[
            f_k\parentheses*{x} = \sqrt{x^2 + \frac{1}{k}} \quad \text{für} \quad x \in \Omega = \parentheses*{-1, 1}, k = 1, 2, \ldots
        \]
        \begin{enumerate}
            \item Zeigen Sie, dass \(f_k \in C^1\parentheses*{\Omega}\) für alle \(k = 1, 2, \ldots\) gilt und bestimmen Sie die \(\infty\)-Norm
            \[
                \norm*{f_k}_\infty = \sup_{x \in \Omega}\absolute*{f_k\parentheses*{x}}.
            \]
            \item Zeigen Sie, dass \(\parentheses*{f_k}_{k \in \N}\) bezüglich dieser Norm eine Cauchyfolge ist.
            \item Zeigen Sie, dass der Grenzwert nicht in \(\parentheses*{C^1, \norm*{\cdot}_\infty}\) liegt.
            \item Folgern Sie, dass \(\parentheses*{C^1, \norm*{\cdot}_\infty}\) kein Banachraum ist.
        \end{enumerate}
    \end{problem}

    \subsection*{Lösung}


    \section*{Aufgabe 5}

    \begin{problem}
        Bestimmen Sie zu den Daten
        \begin{center}
            \begin{tabular}{lcccc}
                \toprule
                \(k\) & \(0\) & \(1\) & \(2\) & \(3\)\\
                \midrule
                \(x_k\) & \(0\) & \(\frac{\pi}{2}\) & \(\pi\) & \(\frac{3}{2}\pi\)\\
                \(f\parentheses*{x_k}\) & \(3\) & \(1\) & \(2\) & \(3\)\\
            \end{tabular}
        \end{center}
        ein trigonometrisches Polynom der Form
        \[
            T_4\parentheses*{f; x} = \sum_{j = 0}^{n - 1}d_j\parentheses*{f}e^{ijx},
        \]
        das die Daten interpoliert, also die Bedingung
        \[
            T_4\parentheses*{f; x_k} = f\parentheses*{x_k}
        \]
        für \(k = 0, \ldots, 3\) erfüllt.
    \end{problem}

    \subsection*{Lösung}


    \section*{Aufgabe 6}

    \begin{problem}
        Gegeben ist das Konvektions-Diffusionsproblem: Gesucht ist \(u \in C^2\parentheses*{0, 1}\) mit
        \begin{align*}
            -u''\parentheses*{x} + 10u'\parentheses*{x} + u\parentheses*{x} &= f\parentheses*{x}, \quad \text{für }x \in \parentheses*{0, 1},\\
            u\parentheses*{0} = u\parentheses*{1} &= 0.
        \end{align*}
        Dieses Problem soll mithilfe einer finite Differenzen Methode auf einem regelmäßigen Gitter der Schrittweite \(h\) und den Gitterpunkten \(0 = x_0 < x_1 < \cdots < x_n = 1\) approximiert und in ein lineares Gleichungssystem der Form \(A_h u_h = b_h\) überführt werden.
        Dazu sollen die Differenzenquotienten
        \begin{align*}
            u'\parentheses*{x_i} &\approx \frac{u\parentheses*{x_{i + 1}} - u\parentheses*{x_{i - 1}}}{2h},\\
            u''\parentheses*{x_i} &\approx \frac{u\parentheses*{x_{i + 1}} - 2u\parentheses*{x_i} + u\parentheses*{x_{i - 1}}}{h^2}
        \end{align*}
        benutzt werden.
        \begin{enumerate}
            \item Bestimmen Sie \(A_h\) und \(b_h\).
            \item Geben Sie eine Bedingung an, unter der \(A_h\) diagonaldominant ist.
            \item Wie müsste das Problem diskretisiert werden, um ohne Zusatzbedingung eine diagonaldominante Matrix \(A_h\) zu erhalten?
            Wie sähe die Matrix aus?
        \end{enumerate}
    \end{problem}

    \subsection*{Lösung}
    \begin{enumerate}
        \item
        \item
        \item
    \end{enumerate}


    \section*{Aufgabe 7}

    \begin{problem}
        \begin{enumerate}
            \item Bestimmen Sie die Koeffizienten \(a\), \(b\) und \(c\), so dass die finite Differenz
            \[
                af\parentheses*{x} + bf\parentheses*{x + h} + cf\parentheses*{x + 2h} \approx f'\parentheses*{x}
            \]
            die erste Ableitung von \(f\) an der Stelle \(x\) mit möglichst hoher Ordnung approximiert.
            Bestimmen Sie die Konsistenzordnung.
            \item Gegeben sei folgende Approximation von \(f''\parentheses*{x}\) für ausreichend glattes \(f\):
            \[
                \frac{2}{h_1 + h_2}\parentheses*{\frac{f\parentheses*{x + h_2} - f\parentheses*{x}}{h_2} - \frac{f\parentheses*{x} - f\parentheses*{x - h_1}}{h_1}}.
            \]
            Bestimmen Sie die Konsistenzordnung.
        \end{enumerate}
    \end{problem}

    \subsection*{Lösung}
    \begin{enumerate}
        \item
        \item
    \end{enumerate}


    \section*{Aufgabe 8}

    \begin{problem}
        Gegeben seien die beiden Matrizen
        \[
            A = \begin{pmatrix}
                1 & 0 & 1\\
                0 & 2 & 0\\
                1 & 0 & 2
            \end{pmatrix}, \quad B = \begin{pmatrix}
                1 & -1\\
                -1 & 2
            \end{pmatrix}.
        \]
        \begin{enumerate}
            \item Führen Sie zwei Schritte des konjugierten Gradientenverfahrens (CG) zur Lösung des Gleichungssystems \(Ax = b\) mit rechter Seite \(b = \parentheses*{2, 1, 0}^T\) und dem Startvektor \(x_0 = \parentheses*{0, 0, 0}^T\) durch, d.h. berechnen Sie \(x_2\).
            \item Bestimmen Sie die Iterationsmatrix \(T\) des Gauß-Seidel-Verfahrens für die Matrix \(B\).
            \item Konvergiert das Gauß-Seidel-Verfahren für die Matrix \(B\) für beliebige Startvektoren?
            Erklären Sie Ihre Antwort mithilfe von Aufgabenteil b).
            \item Wie viele Schritte des Gauß-Seidel-Verfahrens sind bei der Matrix \(B\) nötig, um den Startfehler garantiert um den Faktor \(R = 1024\) zu reduzieren?
            D.h. geben Sie ein möglichst kleines \(k\) an, so dass für beliebige Startvektoren \(x_0\) gilt:
            \[
                \norm*{x_k - x}_\infty < \frac{1}{1024}\norm*{x_0 - x}_\infty.
            \]
            \emph{Hinweis: Betrachten Sie die quadrierte Iterationsmatrix \(T^2\), die zwei Schritten des Gauß-Seidel-Verfahrens entspricht.}
        \end{enumerate}
    \end{problem}

    \subsection*{Lösung}
\end{document}
