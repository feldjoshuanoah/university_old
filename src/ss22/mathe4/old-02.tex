\documentclass{exercise}

\institute{Applied and Computational Mathematics}
\title{Altklausur 2}
\author{Joshua Feld, 406718}
\course{Mathematische Grundlagen IV}
\professor{Torrilhon \& Berkels}
\semester{Sommersemester 2022}
\program{CES (Bachelor)}

\begin{document}
    \maketitle


    \section*{Aufgabe 1}

    \begin{problem}
        \begin{enumerate}
            \item Klassifizieren Sie die PDEs
            \begin{enumerate}
                \item \(\partial_t u + \Delta u + u^2 = x\),
                \item \(\partial_t u - u\partial_x u = x^2\),
                \item \(\partial_t u - \partial_x u = x^2\),
                \item \(\partial_{tt}u - \Delta u = 0\).
            \end{enumerate}
            \item Lösen Sie mit der Methode der Charakteristiken das folgende Anfangswertproblem: \(\Omega = \R^+ \times \R, a > 0, \Gamma\parentheses*{r} = \parentheses*{0, r}\) für \(r \in \R\) und \(u_0\parentheses*{r} = r^3\)
            \begin{align*}
                \partial_t u\parentheses*{t, x} + a\partial_x u\parentheses*{t, x} &= u\parentheses*{t, x}, \quad \text{in }\Omega,\\
                u\parentheses*{\Gamma\parentheses*{x}} &= u_0\parentheses*{r}, \quad \text{auf }\partial\Omega.
            \end{align*}
        \end{enumerate}
    \end{problem}

    \subsection*{Lösung}
    \begin{enumerate}
        \item
        \begin{enumerate}
            \item
            \item
            \item
            \item
        \end{enumerate}
        \item
    \end{enumerate}


    \section*{Aufgabe 2}

    \begin{problem}
        \begin{enumerate}
            \item Zeigen Sie, dass
            \[
                \phi_{j, k}\parentheses*{x, y} = \cos\parentheses*{j\pi x}\sin\parentheses*{k\pi y}, \quad j = 0, 1, \ldots, k = 1, 2, \ldots
            \]
            die Eigenfunktionen des Laplaceoperators auf \(\Omega = \brackets*{0, 1}^2\) sind.
            Wie lauten die Eigenwerte?
            \item Entwickeln Sie die Funktion
            \[
                f\parentheses*{x, y} = \frac{1}{2}\parentheses*{\sin\parentheses*{4\pi y} - \cos\parentheses*{4\pi x}\sin\parentheses*{10\pi y}}
            \]
            in den Eigenfunktionen \(\phi_{j, k}\).
            \item Geben Sie die Lösung des Anfangsrandwertproblems
            \begin{align*}
                \partial_t u\parentheses*{t, x, y} &= \Delta u\parentheses*{t, x, y} + f\parentheses*{x, y}, \quad \text{in }\Omega,\\
                u\parentheses*{t, x, y} &= 0, \quad y \in \braces*{0, 1}, x \in \parentheses*{0, 1},\\
                \partial_t u\parentheses*{t, x, y} &= 0, \quad x \in \braces*{0, 1}, y \in \parentheses*{0, 1},\\
                u\parentheses*{t, x, y} &= 0, \quad \text{für }t = 0
            \end{align*}
            an.
        \end{enumerate}
    \end{problem}

    \subsection*{Lösung}
    \begin{enumerate}
        \item
        \item
        \item
    \end{enumerate}


    \section*{Aufgabe 3}

    \begin{problem}
        \begin{enumerate}
            \item Bestimmen Sie die distributionelle Ableitung der folgenden Distribution:
            \[
                T_f \phi := \parentheses*{f, \phi}, \quad \phi \in \mathcal{D}\parentheses*{\R},
            \]
            wobei
            \[
                f: \R \to \R, x \mapsto \begin{cases}
                    0, & \text{falls }\absolute*{x} \ge 1,\\
                    0, & \text{falls }x = 0,\\
                    x + 1, & \text{falls }0 < x < 1,\\
                    x - 1, & \text{falls }-1 < x < 0.
                \end{cases}
            \]
            \item Zeigen Sie, dass partielle Ableitungen von Distributionen vertauscht werden können, analog zu klassischen Funktionen:
            \[
                \angles*{\frac{\partial}{\partial x_i}\frac{\partial}{\partial x_j}T, \varphi} = \angles*{\frac{\partial}{\partial x_j}\frac{\partial}{\partial x_i}T, \varphi} \quad \forall T \in \mathcal{D}'\parentheses*{\R^n}, \varphi \in \mathcal{D}\parentheses*{\R^n}.
            \]
            \item Finden Sie eine Lösung \(U\) der Gleichung
            \[
                -\Delta U = \frac{\partial}{\partial x_1}\delta, \quad \text{in }\mathcal{D}'\parentheses*{\R^3},
            \]
            wobei \(\delta\) die Dirac-Distribution bedeutet.
        \end{enumerate}
    \end{problem}


    \section*{Aufgabe 4}

    \begin{problem}
        \begin{enumerate}
            \item Sei \(f \in L^1\parentheses*{\R^n}\) und \(c \in \R^n\).
            Zeigen Sie, dass
            \[
                \mathcal{F}\parentheses*{f\parentheses*{x + c}}\parentheses*{\zeta} = e^{i\angles*{\zeta, c}}\mathcal{F}\parentheses*{f}\parentheses*{\zeta} \quad \forall\zeta \in \R^n
            \]
            gilt, wobei \(\mathcal{F}\) die Fourier-Transformation bedeutet.
            \item Seien \(f, g \in S\).
            Zeigen Sie, dass
            \[
                \mathcal{F}\parentheses*{f * g} = \mathcal{F}\parentheses*{f} \cdot F\parentheses*{g},
            \]
            wobei \(f * g\) die Faltung von \(f\) und \(g\) bedeutet und \(S\) der Schwartzsche Raum ist.
            \item Berechnen Sie die Fourier-Transformation der Dirac-Distribution.
        \end{enumerate}
    \end{problem}

    \subsection*{Lösung}
    \begin{enumerate}
        \item
        \item
        \item
    \end{enumerate}


    \section*{Aufgabe 5}

    \begin{problem}
        Wir betrachten das Randwertproblem
        \begin{align}
            -\varepsilon u''\parentheses*{x} + u\parentheses*{x} &= x^4, \quad \text{für }x \in \parentheses*{0, 1},\label{eq:1}\\
            u\parentheses*{0} = u\parentheses*{1} &= 0,\nonumber
        \end{align}
        für \(\varepsilon \ne 0\) und diskretisieren das Gebiet im Inneren mit einem äquidistanten Gitter mit \(N \in \N\) Gitterpunkten \(x_n\):
        \[
            x_n = nh, \quad h = \frac{1}{N + 1}, \quad n = 1, \ldots, N.
        \]
        Zur Approximation der Lösung des Randwertproblems nutzen wir die Methode der finiten Differenzen, im Speziellen:
        \[
            u''\parentheses*{x_n} \approx \frac{u\parentheses*{x_n + h} - 2u\parentheses*{x_n} + u\parentheses*{x_n - h}}{h^2}
        \]
        für \(1 \le n \le N\).
        \begin{enumerate}
            \item Geben Sie das resultierende lineare Gleichungssystem \(A_h u_h = b_h\) an.
            \item Seien \(\lambda_{\text{min}}\parentheses*{A_h}\) und \(\lambda_{\text{max}}\parentheses*{A_h}\) der kleinste bzw. größte Eigenwert der Matrix \(A_h\) aus Teil a).
            Zeigen Sie, dass
            \[
                \min\parentheses*{1, 1 + \frac{4\varepsilon}{h^2}} < \lambda_{\text{min}}\parentheses*{A_h}, \quad \lambda_{\text{max}}\parentheses*{A_h} < \max\parentheses*{1, 1 + \frac{4\varepsilon}{h^2}}
            \]
            gilt.
            \item Zeigen Sie, dass für \(\varepsilon > 0\) die Matrix \(A_h\) für beliebiges \(N\) invertierbar ist.
            \item Was können Sie über die Invertierbarkeit der Matrix \(A_h\) und die Lösung des Randwertproblems \eqref{eq:1} sagen, wenn \(\varepsilon = 0\)?
        \end{enumerate}

        \emph{Hinweis:
        \begin{itemize}
            \item Die Eigenwerte der \(N \times N\) Tridiagonalmatrix
            \[
                \begin{pmatrix}
                    0 & 1 & 0 & \cdots & 0\\
                    1 & 0 & 1 & \ddots & \vdots\\
                    0 & 1 & \ddots & \ddots & 0\\
                    \vdots & \ddots & \ddots & \ddots & 1\\
                    0 & \cdots & 0 & 1 & 0
                \end{pmatrix}
            \]
            lauten
            \[
                \lambda_n = 2\cos\parentheses*{\frac{n\pi}{N + 1}}, \quad n = 1, \ldots, N.
            \]
            \item Sei \(\alpha \in \R\) und \(A = \alpha I + B\), wobei \(I \in \R^{N \times N}\) die Einheitsmatrix und \(B \in \R^{N \times N}\) eine beliebige andere Matrix ist.
            Dann gilt für das Spektrum \(\sigma\parentheses*{A}\) von \(A\):
            \[
                \lambda \in \sigma\parentheses*{B} \iff \parentheses*{\alpha + \lambda} \in \sigma\parentheses*{A}.
            \]
            \item Denken Sie an den Zusammenhang zwischen der Determinante einer Matrix und dem Produkt ihrer Eigenwerte.
        \end{itemize}}
    \end{problem}

    \subsection*{Lösung}
    \begin{enumerate}
        \item 
        \item 
        \item 
        \item 
    \end{enumerate}


    \section*{Aufgabe 6}

    \begin{problem}
        Gegeben sei das lineare Gleichungssystem \(Ax = b\) mit
        \[
            A = \begin{pmatrix}
                8 & 1 & 1\\
                2 & 7 & 1\\
                1 & -1 & 5
            \end{pmatrix}, \quad b = \begin{pmatrix}
                20\\
                26\\
                4
            \end{pmatrix},
        \]
        samt Lösung \(x^* = \parentheses*{2, 3, 1}^T\).
        \begin{enumerate}
            \item Begründen Sie, warum das Jacobi-Verfahren gegen die Lösung des linearen Gleichungssystems konvergiert für jeden beliebigen Startwert.
            \item Führen Sie einen Schritt des Jacobi-Verfahrens durch.
            Nutzen Sie den Startwert \(x_0 = \parentheses*{1, 2, 1}^T\).
            \item Wie viele Schritte sind höchstens für das Jacobi-Verfahren notwendig, um ausgehend von \(x_0 = \parentheses*{1, 2, 1}^T\) den Fehler im Startwert in der \(\infty\)-Norm um den Faktor \(R = 10^3\) zu reduzieren?
            \item Wie viele Schritte erwarten Sie sind notwendig in Teil c), falls man anstatt des Jacobi-Verfahrens das Gauß-Seidel Verfahren verwendet?
            Begründen Sie Ihre Antwort.
            Es sind keine Rechnungen nötig.
        \end{enumerate}
        \emph{Hinweis:
        \begin{itemize}
            \item Sie können in Teil c) die Näherung \(\frac{3}{7} \approx \frac{1}{2}\) verwenden.
            \item Sei \(M = \parentheses*{m_{i, j}}_{\substack{i = 1, \ldots, m,\\j = 1, \ldots, n}} \in \R^{m \times n}\), dann gilt
            \[
                \norm*{M}_\infty = \max_{1 \le i \le m}\sum_{j = 1}^n \absolute*{m_{i, j}}.
            \]
        \end{itemize}}
    \end{problem}


    \section*{Aufgabe 7}

    \begin{problem}
        Bestimmen Sie zu den Daten
        \begin{center}
            \begin{tabular}{lcccc}
                \toprule
                \(x_k\) & \(0\) & \(\frac{\pi}{2}\) & \(\pi\) & \(\frac{3}{2}\pi\)\\
                \midrule
                \(f\parentheses*{x_k}\) & \(3\) & \(0\) & \(-1\) & \(0\)\\
            \end{tabular}
        \end{center}
        ein trigonometrisches Polynom der Form
        \[
            T_4\parentheses*{f; x} = \sum_{j = 0}^{n - 1}d_j\parentheses*{f}e^{ijx},
        \]
        das die Daten interpoliert, also die Bedingung
        \[
            T_4\parentheses*{f; x_k} = f\parentheses*{x_k}
        \]
        für \(k = 0, \ldots, 3\) erfüllt.
    \end{problem}

    \subsection*{Lösung}


    \section*{Aufgabe 8}

    \begin{problem}
        Gegeben sei die folgende Differentialgleichung
        \[
            u''\parentheses*{x} - u'\parentheses*{x} + u\parentheses*{x} = 2x - 1 - x^2,
        \]
        für \(x \in \parentheses*{0, 1}\) mit den gemischten Randbedingungen
        \[
            u\parentheses*{1} = 0, \quad u'\parentheses*{0} = 0.
        \]
        Diskretisieren Sie das Randwertproblem mittels den finiten Differenzen:
        \begin{align*}
            u'\parentheses*{x} &\approx \frac{u\parentheses*{x + h} - u\parentheses*{x - h}}{2h},\\
            u''\parentheses*{x} &\approx \frac{u\parentheses*{x + h} - 2u\parentheses*{x} + u\parentheses*{x - h}}{h^2},
        \end{align*}
        auf einem gleichmäßigen Gitter mit Schrittweite \(h = \frac{1}{N}\) und Gitterpunkten \(x_n = nh, n = 0, \ldots, N\).
        Approximieren Sie die Ableitung am linken Randpunkt mithilfe eines zusätzlichen (fiktiven) Punktes außerhalb des Gebiets, den Sie hinterher wieder geeignet entfernen.
        \begin{enumerate}
            \item Geben Sie das resultierende lineare Gleichungssystem \(A_h y_h = b_h\) an.
            \item Die exakte Lösung des Randwertproblems lautet \(u\parentheses*{x} = 1 - x^2\).
            Bestimmen Sie den Konsistenzfehler der finiten Differenz Methode bzgl. der \(\infty\)-Norm.
        \end{enumerate}
    \end{problem}

    \subsection*{Lösung}
    \begin{enumerate}
        \item
        \item
    \end{enumerate}
\end{document}
