\documentclass{exercise}

\institute{Applied and Computational Mathematics}
\title{Altklausur 3}
\author{Joshua Feld, 406718}
\course{Mathematische Grundlagen IV}
\professor{Torrilhon \& Berkels}
\semester{Sommersemester 2022}
\program{CES (Bachelor)}

\begin{document}
    \maketitle


    \section*{Aufgabe 1}
    
    \begin{problem}
        Sei \(\Omega = \R^+ \times \R\).
        Sei \(u_0\parentheses*{x} = 1 - x\) und \(\Gamma\parentheses*{r} = \parentheses*{0, r}, r \in \R\).
        Berechnen Sie die Lösung des Anfangswertproblems
        \begin{align*}
            \partial_t u - u\partial_x u &= 0, \quad \text{auf }\Omega,\\
            u\parentheses*{\Gamma\parentheses*{r}} &= u_0\parentheses*{r}, \quad r \in \R
        \end{align*}
        und überprüfen Sie explizit, ob die Anfangsbedingung sowie die PDE erfüllt ist.
    \end{problem}
    
    \subsection*{Lösung}


    \section*{Aufgabe 2}
    
    \begin{problem}
        \begin{enumerate}
            \item Zeigen Sie, dass
            \[
                \phi_{j, k}\parentheses*{x, y} = 2\sin\parentheses*{j\pi x}\sin\parentheses*{k\pi y}, \quad j, k = 1, 2, \ldots
            \]
            die Eigenfunktionen des Laplaceoperators auf \(\Omega = \brackets*{0, 1}^2\) mit Nullrandbedingungen sind.
            Wie lauten die Eigenwerte?
            \item Entwickeln Sie die Funktion
            \[
                f\parentheses*{x, y} = x\parentheses*{1 - x}y\parentheses*{1 - y}
            \]
            in den Eigenfunktionen \(\phi_{j, k}\).
            \item Geben Sie die Lösung des Anfangsrandwertproblems
            \begin{align*}
                \partial_t u &= \Delta u + f, \quad \text{in }\Omega,\\
                u &= 0, \quad \text{auf }\partial\Omega,\\
                u &= 0, \quad \text{bei }t = 0
            \end{align*}
            an.
        \end{enumerate}
    \end{problem}
    
    \subsection*{Lösung}
    \begin{enumerate}
        \item
        \item
        \item
    \end{enumerate}


    \section*{Aufgabe 3}
    
    \begin{problem}
        \begin{enumerate}
            \item Bestimmen Sie die distributionelle Ableitung der folgenden Distribution:
            \[
                T_f \phi := \angles*{f, \phi},
            \]
            wobei
            \[
                f: \R \to \R, x \mapsto \begin{cases}
                    0, & \text{falls }x < 0,\\
                    1 + x + x^2, & \text{falls }x \ge 0.
                \end{cases}
            \]
            Dabei ist \(\phi \in \mathcal{D}'\parentheses*{\R}\).
            \item Zeigen Sie: Eine Fundamentallösung der Differentialgleichung
            \[
                -\frac{\d^2}{\d x^2}u\parentheses*{x} = f\parentheses*{x}, \quad x \in \R
            \]
            ist
            \[
                G\parentheses*{x} = -\frac{1}{2}\absolute*{x},
            \]
            d.h. es gilt im distributionellen Sinne
            \[
                -\frac{\d^2}{\d x^2}G = \delta.
            \]
        \end{enumerate}
    \end{problem}
    
    \subsection*{Lösung}
    \begin{enumerate}
        \item
        \item
    \end{enumerate}


    \section*{Aufgabe 4}
    
    \begin{problem}
        \begin{enumerate}
            \item Seien \(X, Y\) zwei normierte Vektorräume und \(T: X \to Y\) eine lineare Abbildung.
            Geben Sie die Definition eines linearen Operators an und beweisen Sie, dass
            \[
                T\text{ beschränkt} \iff T\text{ stetig}.
            \]
            \item Gegeben sei die wesentlich beschränkte Funktion \(V: \R \to \R, V \in L^\infty\parentheses*{\R}\).
            Betrachten Sie den Multiplikationsoperator
            \[
                T: L^2\parentheses*{\R} \to L^2\parentheses*{\R}, f \mapsto Vf,
            \]
            wobei \(Vf\) eine Multiplikation im Lebesgue-Sinne mit \(\parentheses*{Vf}\parentheses*{x} = V\parentheses*{x} \cdot f\parentheses*{x}\) a.e. darstellt.
            Zeigen Sie
            \begin{enumerate}
                \item \(T\) ist selbstadjungiert,
                \item \(T\) ist stetig.
            \end{enumerate}
        \end{enumerate}
    \end{problem}
    
    \subsection*{Lösung}


    \section*{Aufgabe 5}
    
    \begin{problem}
        Bestimmen Sie zu den Daten
        \begin{center}
            \begin{tabular}{lcccc}
                \toprule
                \(x_k\) & \(0\) & \(\frac{\pi}{2}\) & \(\pi\) & \(\frac{3}{2}\pi\)\\
                \midrule
                \(f\parentheses*{x_k}\) & \(6\) & \(2 + 2i\) & \(2\) & \(2 - 2i\)\\
            \end{tabular}
        \end{center}
        ein trigonometrisches Polynom der Form
        \[
            T_4\parentheses*{f; x} = \sum_{j = 0}^{n - 1}d_j\parentheses*{f}e^{ijx},
        \]
        das die Daten interpoliert, also die Bedingung
        \[
            T_4\parentheses*{f; x_k} = f\parentheses*{x_k}
        \]
        für \(k = 0, \ldots, 3\) erfüllt.
    \end{problem}
    
    \subsection*{Lösung}


    \section*{Aufgabe 6}

    \begin{problem}
        Gegeben ist das Konvektions-Diffusionsproblem: Gesucht ist \(u \in C^2\parentheses*{0, 1}\) mit
        \begin{align*}
            -\frac{1}{\pi^2}u''\parentheses*{x} + u'\parentheses*{x} &= f\parentheses*{x}, \quad \text{für }x \in \parentheses*{0, 1},\\
            u\parentheses*{0} = u\parentheses*{1} &= 0.
        \end{align*}
        Dieses Problem soll mithilfe einer finite Differenzen Methode auf einem regelmäßigen Gitter der Schrittweite \(h\) und den Gitterpunkten \(0 = x_0 < x_1 < \cdots < x_n = 1\) approximiert und in ein lineares Gleichungssystem der Form \(A_h u_h = b_h\) überführt werden.
        Dazu sollen die Differenzenquotienten
        \begin{align*}
            u'\parentheses*{x_i} &\approx \frac{u\parentheses*{x_{i + 1}} - u\parentheses*{x_i}}{h},\\
            u''\parentheses*{x_i} &\approx \frac{u\parentheses*{x_{i + 1}} - 2u\parentheses*{x_i} + u\parentheses*{x_{i - 1}}}{h^2}
        \end{align*}
        benutzt werden.
        \begin{enumerate}
            \item Bestimmen Sie \(A_h\) und \(b_h\).
            \item Geben Sie eine Bedingung an, unter der \(A_h\) diagonaldominant ist.
            \item Wie müsste das Problem diskretisiert werden, um ohne Zusatzbedingung eine diagonaldominante Matrix \(A_h\) zu erhalten?
            Wie sähe die Matrix aus?
        \end{enumerate}
    \end{problem}

    \subsection*{Lösung}
    \begin{enumerate}
        \item
        \item
        \item
    \end{enumerate}


    \section*{Aufgabe 7}
    
    \begin{problem}
        Gegeben sei das lineare Gleichungssystem \(Ax = b\) mit
        \[
            A = \begin{pmatrix}
                2 & -1 & 1\\
                -1 & 2 & -1\\
                1 & -1 & 2
            \end{pmatrix}, \quad b = \begin{pmatrix}
                4\\
                -1\\
                7
            \end{pmatrix}.
        \]
        Das lineare Gleichungssystem hat die Lösung \(x = \parentheses*{1, 2, 4}^T\).
        \begin{enumerate}
            \item Führen Sie jeweils einen Schritt des Jacobi- und des Gauß-Seidel-Verfahrens mit dem Startvektor \(x^0 = \parentheses*{1, 1, 1}^T\) durch.
            \item Gegeben sei nun die Richardson-Methode
            \[
                x^{k + 1} = x^k + \omega\parentheses*{b - Ax^k}
            \]
            für das Gleichungssystem \(Ax = b\).
            \begin{enumerate}
                \item Für welche Werte \(\omega\) konvergiert die Methode?
                \item Wie lautet der optimale Wert für \(\omega\), sodass das Verfahren unabhängig vom Startvektor \(x^0\) am schnellsten konvergiert?
            \end{enumerate}
        \end{enumerate}
    \end{problem}
    
    \subsection*{Lösung}


    \section*{Aufgabe 8}
    
    \begin{problem}
        \begin{enumerate}
            \item Gegeben sei folgende Approximation von \(f''\parentheses*{x}\) für ausreichend glattes \(f\):
            \[
                \frac{2}{h_1 + h_2}\parentheses*{\frac{f\parentheses*{x + h_2} - f\parentheses*{x}}{h_2} - \frac{f\parentheses*{x} - f\parentheses*{x - h_1}}{h_1}}.
            \]
            Bestimmen Sie die Ordnung dieser finiten Differenz als Approximation von \(f''\).
            \item Die erste Ableitung einer Funktion \(u\) an der Stelle \(\xi\) soll mithilfe einer finiten Differenz, die die Funktionsauswertungen von \(u\) an den Stellen \(\xi\), \(\xi + h_1\) und \(\xi + h_1 + h_2\) benutzt, approximiert werden, d.h.
            \[
                u'\parentheses*{\xi} = au\parentheses*{\xi} + bu\parentheses*{\xi + h_1} + cu\parentheses*{\xi + h_1 + h_2} + R\parentheses*{u, h},
            \]
            wobei \(h = \max\parentheses*{h_1, h_2}\) und \(R\parentheses*{u, h} = \mathcal{O}\parentheses*{h^p}\) der Fehler der finiten Differenz ist.
            Bestimmen Sie \(a, b, c\) so, dass \(p\) maximal wird.
            Wie groß ist dieses?

            \emph{Hinweis: Entwickeln Sie \(u\) un die Stelle \(\xi\).}
        \end{enumerate}
    \end{problem}
    
    \subsection*{Lösung}
\end{document}
