\documentclass{exercise}

\institute{Applied and Computational Mathematics}
\title{Altklausur 4}
\author{Joshua Feld, 406718}
\course{Mathematische Grundlagen IV}
\professor{Torrilhon \& Berkels}
\semester{Sommersemester 2022}
\program{CES (Bachelor)}

\begin{document}
    \maketitle


    \section*{Aufgabe 1}
    
    \begin{problem}
        \begin{enumerate}
            \item Lösen Sie das Anfangswertproblem: Suche \(u: \R \times \R^+ \to \R\), sodass
            \begin{align*}
                \partial_{tt}u - \partial_{xx}u &= 0, \quad \text{auf }\R,\\
                u\parentheses*{x, 0} &= 0, \quad x \in \R,\\
                \partial_t u\parentheses*{x, 0} &= \chi_{\brackets*{-1, 1}}\parentheses*{x}, \quad x \in \R,
            \end{align*}
            wobei \(\chi_{\brackets*{-1, 1}}\parentheses*{x} = \begin{cases}
                1, & \text{falls }x \in \brackets*{-1, 1},\\
                0, & \text{sonst}.
            \end{cases}\)
            \item Klassifizieren Sie die PDEs
            \begin{enumerate}
                \item \(-\Delta u + u^2 = f\),
                \item \(\partial_t u - \partial_x u + u^2 = f\).
            \end{enumerate}
        \end{enumerate}
    \end{problem}
    
    \subsection*{Lösung}
    \begin{enumerate}
        \item
        \item
        \begin{enumerate}
            \item
            \item
        \end{enumerate}
    \end{enumerate}


    \section*{Aufgabe 2}
    
    \begin{problem}
        Sei \(\Omega = \R^+ \times \R\).
        Sei \(u_0\parentheses*{x} = 1 + x\) und \(\Gamma\parentheses*{r} = \parentheses*{0, r}, r \in \R\).
        Lösen Sie das Anfangswertproblem
        \begin{align*}
            \partial_t u - u\partial_x u &= 0, \quad \text{auf }\Omega,\\
            u\parentheses*{\Gamma\parentheses*{r}} &= u_0\parentheses*{r}, \quad r \in \R.
        \end{align*}
    \end{problem}
    
    \subsection*{Lösung}


    \section*{Aufgabe 3}
    
    \begin{problem}
        Sei \(\tilde{f}: \brackets*{-1, 1} \to \R\) gegeben durch \(\tilde{f}\parentheses*{x} = \absolute*{x}\).
        Sei \(f: \R \to \R\) definiert als die periodische Fortsetzung von \(\tilde{f}\) auf \(\R\), d.h. für \(x \in \brackets*{2k - 1, 2k + 1}\) gilt \(f\parentheses*{x} = \absolute*{x - 2k} \forall k \in \Z\).
        \begin{enumerate}
            \item Berechnen Sie die erste Ableitung \(DT_f\) im Sinne der Distributionen.
            \item Berechnen Sie die zweite Ableitung \(D^2 T_f\) im Sinne der Distributionen.
            \item Berechnen Sie \(\parentheses*{D^2 T_f * \phi}\parentheses*{x}\).
            \item Lösen Sie die PDE im Sinne der Distributionen (in \(\R\))
            \[
                -\partial_{xx}U = \sum_{k \in \Z \setminus \braces*{0}}\frac{1}{k^2}\delta_k, \quad \text{in }\mathcal{D}',
            \]
            wobei \(\delta_k\parentheses*{x} = \delta\parentheses*{x - k}\) die Dirac-Delta Distribution in \(k\) ist.

            \emph{Hinweis: Die Fundamentallösung der Laplace-Gleichung in \(\R\) ist \(G\parentheses*{x} = \frac{1}{2}\absolute*{x}\).}
        \end{enumerate}
    \end{problem}
    
    \subsection*{Lösung}
    \begin{enumerate}
        \item
        \item
        \item
        \item
    \end{enumerate}


    \section*{Aufgabe 4}
    
    \begin{problem}
        \begin{enumerate}
            \item Seien \(X, Y\) zwei normierte Vektorräume und \(T: X \to Y\) eine lineare Abbildung.
            Geben Sie die Definition eines linearen Operators an und beweisen Sie, dass
            \[
                T\text{ beschränkt} \iff T\text{ stetig}.
            \]
            \item Betrachten Sie das Eigenwertproblem: Finde \(\parentheses*{\lambda, \phi}\), sodass
            \begin{align*}
                -\partial_{xx}\phi &= \lambda\phi, \quad \text{auf }\Omega = \parentheses*{0, 1},\\
                \phi\parentheses*{0} = \phi\parentheses*{1} &= 0.
            \end{align*}
            Geben Sie alle Eigenwerte und (normalisierten) Eigenfunktionen an.
            Berechnen Sie anhand der Spektralzerlegung die Norm
            \[
                \norm*{\nabla f}_{L^2\parentheses*{\Omega}}^2 = \int_0^1 \absolute*{\nabla f\parentheses*{x}}^2 \d x
            \]
            für \(f\parentheses*{x} = 2\sin\parentheses*{\pi x} - 4\sin\parentheses*{2\pi x}\).
        \end{enumerate}
    \end{problem}
    
    \subsection*{Lösung}
    \begin{enumerate}
        \item
        \item
    \end{enumerate}


    \section*{Aufgabe 5}
    
    \begin{problem}
        Sei \(n \in \N\) gegeben.
        Betrachten Sie äquidistante Stützstellen \(x_k = \frac{2\pi k}{n}\) für \(k = 0, \ldots, n - 1\) und zugehörige Daten \(y = \parentheses*{y_0, \ldots, y_{n - 1}}^T \in \C^n\).
        Ferner, sei \(\hat{y} = \parentheses*{\hat{y}_0, \ldots, \hat{y}_{n - 1}}^T \in \C^n\) die diskrete Fourier-Transformierte von \(y\), d.h. \(\hat{y}_k = d_k\parentheses*{y}\).
        \begin{enumerate}
            \item Zeigen Sie, dass trigonometrische Polynom
            \[
                T_n\parentheses*{y; x} = \sum_{j = 0}^{n - 1}\hat{y}_j e^{ijx}
            \]
            die Werte \(\parentheses*{x_k, y_k}, k = 0, \ldots, n - 1\) interpoliert.

            \emph{Hinweis: Nutzen Sie, dass \(\sum_{j = 0}^{n - 1}\parentheses*{\varepsilon_n}^{mj} = n \delta_{m, 0}\) für alle \(m \in \braces*{-n + 1, \ldots, n + 1}\), wobei \(\varepsilon_n\) die  \(n\)-te Einheitswurzel bezeichnet.}
            \item Bestimmen Sie zu den Werten
            \begin{center}
                \begin{tabular}{lcccc}
                    \toprule
                    \(x_k\) & \(0\) & \(\frac{\pi}{2}\) & \(\pi\) & \(\frac{3}{2}\pi\)\\
                    \midrule
                    \(f\parentheses*{x_k}\) & \(0\) & \(1\) & \(2\) & \(3\)\\
                \end{tabular}
            \end{center}
            die inverse (diskrete) Fourier-Transformierte \(\check{y} = \parentheses*{\check{y}_0, \ldots, \check{y}_{n - 1}}^T \in \C^n\) zu \(\hat{y}\).
        \end{enumerate}
    \end{problem}
    
    \subsection*{Lösung}
    \begin{enumerate}
        \item
        \item
    \end{enumerate}


    \section*{Aufgabe 6}

    \begin{problem}
        \begin{enumerate}
            \item Bestimmen Sie die Koeffizienten \(a\), \(b\) und \(c\), so dass die finite Differenz
            \[
                af\parentheses*{x} + bf\parentheses*{x + h} + cf\parentheses*{x - 2h} \approx f'\parentheses*{x}
            \]
            die erste Ableitung von \(f\) an der Stelle \(x\) mit möglichst hoher Ordnung approximiert.
            Bestimmen Sie die Konsistenzordnung.
            \item Gegeben sei folgende Approximation von \(f''\parentheses*{x}\) für ausreichend glattes \(f\):
            \[
                \frac{2}{h_1 + h_2}\parentheses*{\frac{f\parentheses*{x + h_2} - f\parentheses*{x}}{h_2} - \frac{f\parentheses*{x} - f\parentheses*{x - h_1}}{h_1}}.
            \]
            Bestimmen Sie die Konsistenzordnung.
        \end{enumerate}
    \end{problem}

    \subsection*{Lösung}
    \begin{enumerate}
        \item
        \item
    \end{enumerate}


    \section*{Aufgabe 7}
    
    \begin{problem}
        Gegeben sei das lineare Gleichungssystem \(Ax = b\) mit
        \[
            A = \begin{pmatrix}
                \alpha & 1 & 0\\
                1 & \alpha & -1\\
                0 & -1 & \alpha
            \end{pmatrix}, \quad b = \begin{pmatrix}
                0\\
                -1\\
                -2
            \end{pmatrix}.
        \]
        \begin{enumerate}
            \item Für welche Werte von \(\alpha \in \R\) konvergiert jeweils das Jacobi- und das Gauß-Seidel-Verfahren für beliebige Startvektoren \(x^0 \in \R^3\)?
        \end{enumerate}
        Sei nun \(\alpha = 3\) gegeben.
        Dann ist \(x^* = \frac{1}{21}\parentheses*{5, -15, -19}^T\) eine Lösung des linearen Gleichungssystems.
        \begin{enumerate}
            \item[b)] Führen Sie einen Schritt mit dem Gauß-Seidel-Verfahren durch.
            Verwenden Sie als Startvektor \(x^0 = \parentheses*{1, 0, 0}^T\).
            \item[c)] Zeigen Sie, dass die Voraussetzungen zur Anwendung des CG-Verfahrens gegeben sind und führen Sie einen Schritt mit dem CG-Verfahren durch.
            Verwenden Sie als Startvektor \(x^0 = \parentheses*{0, -1, -1}^T\).
            \item[d)] Welches Ergebnis erhält man nach drei Schritten mit dem CG-Verfahren?
        \end{enumerate}
    \end{problem}
    
    \subsection*{Lösung}
    \begin{enumerate}
        \item
        \item
        \item
        \item
    \end{enumerate}


    \section*{Aufgabe 8}

    \begin{problem}
        Gegeben ist das Konvektions-Diffusionsproblem: Gesucht ist \(u \in C^2\parentheses*{0, 1}\) mit
        \begin{align*}
            -u''\parentheses*{x} + 10u'\parentheses*{x} + u\parentheses*{x} &= f\parentheses*{x}, \quad \text{für }x \in \parentheses*{0, 1},\\
            u\parentheses*{0} = u\parentheses*{1} &= 0.
        \end{align*}
        Dieses Problem soll mithilfe einer finite Differenzen Methode auf einem regelmäßigen Gitter der Schrittweite \(h\) und den Gitterpunkten \(0 = x_0 < x_1 < \cdots < x_n = 1\) approximiert und in ein lineares Gleichungssystem der Form \(A_h u_h = b_h\) überführt werden.
        Dazu sollen die Differenzenquotienten
        \begin{align*}
            u'\parentheses*{x_i} &\approx \frac{u\parentheses*{x_{i + 1}} - u\parentheses*{x_{i - 1}}}{2h},\\
            u''\parentheses*{x_i} &\approx \frac{u\parentheses*{x_{i + 1}} - 2u\parentheses*{x_i} + u\parentheses*{x_{i - 1}}}{h^2}
        \end{align*}
        benutzt werden.
        \begin{enumerate}
            \item Bestimmen Sie \(A_h\) und \(b_h\).
            \item Geben Sie eine Bedingung an, unter der \(A_h\) diagonaldominant ist.
            \item Wie müsste das Problem diskretisiert werden, um ohne Zusatzbedingung eine diagonaldominante Matrix \(A_h\) zu erhalten?
            Wie sähe die Matrix aus?
        \end{enumerate}
    \end{problem}

    \subsection*{Lösung}
    \begin{enumerate}
        \item
        \item
        \item
    \end{enumerate}
\end{document}
