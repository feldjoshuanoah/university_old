\documentclass{exercise}

\institute{Applied and Computational Mathematics}
\title{Altklausur 5}
\author{Joshua Feld, 406718}
\course{Mathematische Grundlagen IV}
\professor{Torrilhon \& Berkels}
\semester{Sommersemester 2022}
\program{CES (Bachelor)}

\begin{document}
    \maketitle


    \section*{Aufgabe 1}
    
    \begin{problem}
        Gegeben ist das PDE-System 1. Ordnung für \(U: \Omega \subset \R^2 \times \R^+ \to \R^3\)
        \begin{align*}
            \partial_t u_1 + 3\partial_{x_1}u_2 + 2\partial_{x_2}u_1 + \partial_{x_2}u_2 &= 0,\\
            \partial_t u_2 + \partial_{x_1}u_2 + 3\partial_{x_2}u_2 &= 0,\\
            \partial_t u_3 + 2\partial_{x_1}u_2 + 3\partial_{x_1}u_3 + 2\partial_{x_2}u_3 &= 0.
        \end{align*}
        Zeigen Sie, dass das System in der Form
        \[
            \partial_t U + \vecdiv F\parentheses*{U} = 0
        \]
        geschrieben werden kann und überprüfen Sie ob das System hyperbolisch ist.
    \end{problem}
    
    \subsection*{Lösung}


    \section*{Aufgabe 2}
    
    \begin{problem}
        \begin{enumerate}
            \item Bestimmen Sie die Lösungen des Randwertproblems
            \[
                -u''\parentheses*{x} = 1 \quad \forall x \in \parentheses*{0, 1},
            \]
            in Abhängigkeit von den Werten \(\alpha, \beta \in \R\) für jede der folgenden Randbedingungen
            \begin{align}
                u\parentheses*{0} = \alpha \quad &\text{und} \quad u\parentheses*{1} = \beta,\nonumber\\
                u\parentheses*{0} = \alpha \quad &\text{und} \quad \frac{\partial u}{\partial n}\parentheses*{1} = \beta,\nonumber\\
                \frac{\partial u}{\partial n}\parentheses*{0} = \alpha \quad &\text{und} \quad u\parentheses*{1} = \beta,\nonumber\\
                \frac{\partial u}{\partial n}\parentheses*{0} = \alpha \quad &\text{und} \quad \frac{\partial u}{\partial n}\parentheses*{1} = \beta.\label{eq:1}
            \end{align}
            \item Das reine Neumann-Problem \eqref{eq:1} ist nicht für beliebige \(\alpha, \beta \in \R\) lösbar.
            Wie lautet die Bedingung an \(\alpha\) und \(\beta\) damit eine Lösung existiert?
            Ist die Lösung \(u\) dann eindeutig?
            Begründen Sie Ihre Antwort.
            \item Welche Bedingung muss eine Randfunktion \(g\) in dem Neumann-Problem der mehrdimensionalen Poisson-Gleichung
            \begin{align*}
                \Delta u &= f, \quad \text{in }\Omega \subset \R^n,\\
                \frac{\partial u}{\partial n} &= g, \quad \text{auf }\partial\Omega
            \end{align*}
            für gegebenes \(f\) erfüllen?

            \emph{Hinweis: \(\Delta\cdot = \vecdiv\parentheses*{\nabla\cdot}\).}
        \end{enumerate}
    \end{problem}
    
    \subsection*{Lösung}
    \begin{enumerate}
        \item
        \item
        \item
    \end{enumerate}


    \section*{Aufgabe 3}
    
    \begin{problem}
        \begin{enumerate}
            \item Berechnen Sie die ersten zwei distributionellen Ableitungen der Funktion \(f: \R \to \R\) mit
            \[
                f\parentheses*{x} = \begin{cases}
                    0, & \text{falls }x < 0,\\
                    x, & \text{falls }0 \le x \le 1,\\
                    1, & \text{falls }1 < x.
                \end{cases}
            \]
            \item Welche Eigenschaft muss die distributionelle Ableitung der Dirac-Funktion erfüllen?
            \item Zeigen Sie, dass die Funktion \(\gamma: \R^3 \setminus \braces*{0} \to \R\) mit
            \[
                \gamma\parentheses*{x} = -\frac{1}{4\pi}\frac{1}{r\parentheses*{x}}, \quad r\parentheses*{x} = \norm*{x}_2
            \]
            eine Fundamentallösung der dreidimensionalen Laplacegleichung
            \[
                -\Delta\gamma = \delta
            \]
            ist.
        \end{enumerate}
    \end{problem}
    
    \subsection*{Lösung}
    \begin{enumerate}
        \item
        \item
        \item
    \end{enumerate}


    \section*{Aufgabe 4}
    
    \begin{problem}
        Gegeben ist das Problem
        \begin{align*}
            \partial_t u\parentheses*{x, t} &= \partial_{xx}u\parentheses*{x, t} - u\parentheses*{x, t} + \sin\parentheses*{\pi x}, \quad x \in \parentheses*{0, 1}, t > 0,\\
            u\parentheses*{0, t} &= 0, \quad t > 0,\\
            u_x\parentheses*{1, t} &= 0, \quad t > 0,\\
            u\parentheses*{x, 0} &= a, \quad x \in \parentheses*{0, 1},
        \end{align*}
        wobei \(a \in \R\), d.h. die Anfangsbedingung ist eine konstante Funktion.
        \begin{enumerate}
            \item Berechnen Sie die zugehörigen Eigenwerte und die normierten Eigenfunktionen.
            \item Entwickeln Sie die Anfangsbedingung und den Quellterm in Eigenfunktionen.
            \item Berechnen Sie die Lösung des Problems mit dem Eigenfunktionen-Ansatz.
        \end{enumerate}
    \end{problem}
    
    \subsection*{Lösung}
    \begin{enumerate}
        \item
        \item
        \item
    \end{enumerate}


    \section*{Aufgabe 5}
    
    \begin{problem}
        Bestimmen Sie zu den Daten
        \begin{center}
            \begin{tabular}{lcccc}
                \toprule
                \(x_k\) & \(0\) & \(\frac{\pi}{2}\) & \(\pi\) & \(\frac{3}{2}\pi\)\\
                \midrule
                \(f\parentheses*{x_k}\) & \(6\) & \(2 + 2i\) & \(2\) & \(2 - 2i\)\\
            \end{tabular}
        \end{center}
        ein trigonometrisches Polynom der Form
        \[
            T_4\parentheses*{f; x} = \sum_{j = 0}^{n - 1}d_j\parentheses*{f}e^{ijx},
        \]
        das die Daten interpoliert, also die Bedingung
        \[
            T_4\parentheses*{f; x_k} = f\parentheses*{x_k}
        \]
        für \(k = 0, \ldots, 3\) erfüllt.
    \end{problem}
    
    \subsection*{Lösung}


    \section*{Aufgabe 6}
    
    \begin{problem}
        Die erste Ableitung einer Funktion \(u\) an der Stelle \(\xi\) soll mithilfe der Differenzenformel, die die Funktionsauswertungen von \(u\) an den Stellen \(\xi\), \(\xi + h_1\) und \(\xi + h_1 + h_2\) benutzt, approximiert werden, d.h.
        \[
            u'\parentheses*{\xi} = au\parentheses*{\xi} + bu\parentheses*{\xi + h_1} + cu\parentheses*{\xi + h_1 + h_2} + R\parentheses*{u, h},
        \]
        wobei \(h = \max\parentheses*{h_1, h_2}\) und \(R\parentheses*{u, h} = \mathcal{O}\parentheses*{h^p}\) der Fehler der Differenzenformel ist.
        Bestimmen Sie \(a, b, c\) so, dass \(p\) maximal wird.
        Wie groß ist dieses?

        \emph{Hinweis: Entwickeln Sie \(u\) um die Stelle \(\xi\).}
    \end{problem}
    
    \subsection*{Lösung}


    \section*{Aufgabe 7}
    
    \begin{problem}
        Gegeben sei das lineare Gleichungssystem \(Ax = b\) mit
        \[
            A = \begin{pmatrix}
                5 & 1 & 0\\
                1 & 5 & -1\\
                0 & -1 & 5
            \end{pmatrix}, \quad b = \begin{pmatrix}
                0\\
                -3\\
                -5
            \end{pmatrix}
        \]
        mit Lösung
        \[
            x = \frac{1}{23}\begin{pmatrix}
                4\\
                -20\\
                -27
            \end{pmatrix}.
        \]
        \begin{enumerate}
            \item Konvergieren das Jacobi- und Gauß-Seidel-Verfahren für jeden Startvektor \(x^0 \in \R^3\)?
            \item Führen Sie einen Schritt mit dem Gauß-Seidel-Verfahren durch.
            Verwenden Sie als Startvektor \(x^0 = \parentheses*{1, 0, 0}^T\).
            \item Zeigen Sie, dass die Voraussetzungen zur Anwendung des CG-Verfahrens gegeben sind und führen Sie einen Schritt mit dem CG-Verfahren durch.
            Verwenden Sie als Startvektor \(x^0 = \parentheses*{0, -1, -1}^T\).
            \item Welches Ergebnis erhält man nach drei Schritten mit dem CG-Verfahren?
        \end{enumerate}
    \end{problem}
    
    \subsection*{Lösung}
    \begin{enumerate}
        \item
        \item
        \item
        \item
    \end{enumerate}


    \section*{Aufgabe 8}
    
    \begin{problem}
        Gegeben sei die gewöhnliche Differentialgleichung
        \begin{align*}
            -u''\parentheses*{x} + c\parentheses*{x}u\parentheses*{x} &= f\parentheses*{x}, \quad x \in \parentheses*{0, 1},\\
            u\parentheses*{0} &= 1,\\
            u\parentheses*{1} &= 0,
        \end{align*}
        wobei \(c, f: \brackets*{0, 1} \to \R\) stetig sind mit
        \[
            c\parentheses*{x} > 0, \quad x \in \brackets*{0, 1}.
        \]
        \begin{enumerate}
            \item Nehmen Sie eine finite Differenzen Diskretisierung für diese Differentialgleichung vor, wobei das Intervall \(\brackets*{0, 1}\) äquidistant in vier Teilintervalle zerlegt werden soll, und stellen Sie das diskrete System in der Form
            \[
                A_h u_h = f_h
            \]
            auf.
            \item Zeigen Sie, dass die Matrix \(A_h\) symmetrisch und positiv definit (es genügt zu zeigen, dass die Eigenwerte positiv sind) ist.

            \emph{Hinweis: Gerschgorin-Kreise.}
        \end{enumerate}
    \end{problem}
    
    \subsection*{Lösung}
    \begin{enumerate}
        \item
        \item
    \end{enumerate}
\end{document}
