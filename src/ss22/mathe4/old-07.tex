\documentclass{exercise}

\institute{Applied and Computational Mathematics}
\title{Altklausur 7}
\author{Joshua Feld, 406718}
\course{Mathematische Grundlagen IV}
\professor{Torrilhon \& Berkels}
\semester{Sommersemester 2022}
\program{CES (Bachelor)}

\begin{document}
    \maketitle


    \section*{Aufgabe 1}
    
    \begin{problem}
        Bestimmen Sie die Lösung des folgenden Cauchy-Problems für die eindimensionale Wellengleichung
        \begin{align*}
            u_{tt} - u_{xx} &= 0, \quad t > 0, x \in \R,\\
            u\parentheses*{0, x} &= xe^x, \quad x \in \R,\\
            u_t\parentheses*{0, x} &= -x^2, \quad x \in \R.
        \end{align*}
    \end{problem}
    
    \subsection*{Lösung}
    Aus der Vorlesung wissen wir, dass die Lösung der gegebenen eindimensionalen Wellengleichung durch die d'Alembertsche Formel gegeben ist:
    \[
        u\parentheses*{t, x} = \frac{1}{2}\parentheses*{g\parentheses*{x + t} + g\parentheses*{x - t}} + \frac{1}{2}\int_{x - t}^{x + t}h\parentheses*{y}\d y
    \]
    mit den Anfangswert-Funktionen
    \[
        g\parentheses*{x} = xe^x \quad \text{und} \quad h\parentheses*{x} = -x^2.
    \]
    Damit lautet die Lösung
    \begin{align*}
        u\parentheses*{t, x} &= \frac{1}{2}\parentheses*{\parentheses*{x + t}e^{x + t} + \parentheses*{x - t}e^{x - t}} + \frac{1}{2}\int_{x - t}^{x + t}-y^2 \d y\\
        &= \frac{1}{2}\parentheses*{\parentheses*{x + t}e^{x + t} + \parentheses*{x - t}e^{x - t}} + \frac{1}{2} \cdot \left.\parentheses*{-\frac{1}{3}y^3}\right|_{x - t}^{x + t}\\
        &= \frac{1}{2}\parentheses*{\parentheses*{x + t}e^{x + t} + \parentheses*{x - t}e^{x - t}} + \frac{1}{6}\parentheses*{x - t}^3 - \frac{1}{6}\parentheses*{x + t}^3\\
        &= \frac{1}{2}\parentheses*{xe^x \cosh\parentheses*{t} + te^x \sinh\parentheses*{t}} - tx^2 - \frac{1}{3}t^3
    \end{align*}


    \section*{Aufgabe 2}
    
    \begin{problem}
        Wir betrachten Polarkoordinaten \(\parentheses*{r, \varphi}\) mit
        \[
            x = r\cos\varphi, \quad y = r\sin\varphi, \quad r > 0, \varphi \in \left[0, 2\pi\right).
        \]
        Sei \(u\) eine harmonische Funktion in Polarkoordinaten, d.h. \(u = u\parentheses*{r, \varphi}\), in der Kreisscheibe
        \[
            B = \braces*{\parentheses*{r, \varphi} : r < 2}
        \]
        mit \(u\parentheses*{2, \varphi} = 3\sin\varphi + 1\).
        \begin{enumerate}
            \item Bestimmen Sie \(u\parentheses*{0, 0}\).
            \item Bestimmen Sie \(\max_{\parentheses*{r, \varphi} \in \bar{B}}u\parentheses*{r, \varphi}\). 
        \end{enumerate}
    \end{problem}
    
    \subsection*{Lösung}
    \begin{enumerate}
        \item Wir benutzen die Mittelwerteigenschaft:
        \begin{align*}
            u\parentheses*{0, 0} &= \frac{1}{\omega_n r^{n - 1}}\int_{\partial B_r\parentheses*{0}}u\parentheses*{2, \varphi}\d\varphi, \quad r = 2, n = 2\\
            &= \frac{1}{2\pi \cdot 2}\int_{\partial B_2\parentheses*{0}}\parentheses*{3\sin\varphi + 1}\d\varphi\\
            &= \frac{1}{4\pi}\parentheses*{0 + 2\pi} = \frac{1}{2}.
        \end{align*}
        \item Wir benutzen das Maximumprinzip:
        \[
            \max_{\parentheses*{r, \varphi} \in \bar{B}}u\parentheses*{r, \varphi} = \max_{\parentheses*{r, \varphi} \in \partial B}u\parentheses*{r, \varphi} = \max_{\varphi \in \left[0, 2\pi\right)}u\parentheses*{2, \varphi} = \max_{\varphi \in \left[0, 2\pi\right)}3\sin\varphi + 1 = 4.
        \]
    \end{enumerate}


    \section*{Aufgabe 3}
    
    \begin{problem}
        Zeigen Sie: Eine Fundamentallösung der Differentialgleichung
        \[
            -\frac{\d^2}{\d x^2}u\parentheses*{x} = f\parentheses*{x}, \quad x \in \R
        \]
        ist
        \[
            G\parentheses*{x} = -\frac{1}{2}\absolute*{x},
        \]
        d.h. es gilt im distributionellen Sinne
        \[
            -\frac{\d^2}{\d x^2}G = \delta.
        \]
    \end{problem}
    
    \subsection*{Lösung}
    Eine Fundamentallösung \(G\) des  Laplace-operators in einer Raumdimension muss
    \[
        \phi\parentheses*{0} = \int_\R G\parentheses*{x}\Delta\phi\parentheses*{x}\d x \quad \forall\phi \in C_0^\infty\parentheses*{\R}
    \]
    erfüllen.
    Sei also \(\phi \in C_0^\infty\parentheses*{\R}\).
    Wir berechnen
    \begin{align*}
        \frac{1}{2}\int_{-\infty}^\infty \absolute*{x}\phi''\parentheses*{x}\d x &= -\frac{1}{2}\int_{-\infty}^0 x\phi''\parentheses*{x}\d x + \frac{1}{2}\int_0^\infty x\phi''\parentheses*{x}\d x\\
        &= \int_0^\infty x\phi''\parentheses*{x}\d x\\
        &= \left.x\phi'\parentheses*{x}\right|_0^\infty - \int_0^\infty \phi'\parentheses*{x}\d x\\
        &= \left.-\phi\parentheses*{x}\right|_0^\infty = \phi\parentheses*{0},
    \end{align*}
    wobei ausgenutzt wurde, dass \(\supp\parentheses*{\phi} \subset \brackets*{-R, R}\) für ein \(R \in \R\).


    \section*{Aufgabe 4}
    
    \begin{problem}
        Wir betrachten die Folge periodischer Funktionen \(\parentheses*{u_n}_{n \in \N}\)
        \[
            u_n\parentheses*{x} := e^{inx},
        \]
        sowie die drei Operatoren
        \begin{align*}
            \parentheses*{Iu}\parentheses*{x} &:= u\parentheses*{x}, \quad \text{(Identität)},\\
            \parentheses*{Du}\parentheses*{x} &:= \frac{\d}{\d x}u\parentheses*{x}, \quad \text{(Ableitung)},\\
            \parentheses*{Su}\parentheses*{x} &:= \int_0^x u\parentheses*{s}\d s, \quad \text{Stammfunktion}.
        \end{align*}
        \begin{enumerate}
            \item Untersuchen Sie, ob die Folgen \(\parentheses*{Iu_n}_{n \in \N}\), \(\parentheses*{Iu_n}_{n \in \N}\) und \(\parentheses*{Iu_n}_{n \in \N}\) in \(L^2\parentheses*{\brackets*{0, 2\pi}}\)
            \begin{enumerate}
                \item beschränkt oder unbeschränkt,
                \item konvergent oder divergent
            \end{enumerate}
            sind.
            \item Sind die Operatoren \(I\) und \(D\) beschränkt und/oder kompakt im \(L^2\)-Sinne?
            Verwenden Sie Ihre Ergebnisse aus Aufgabenteil a).
            \item (Dies ist eine Zusatzfrage, die nicht bewertet wird.)
            Stellen Sie eine allgemeine Vermutung über die Beschränktheit und Kompaktheit von \(S\) im \(L^2\)-Sinne auf.
        \end{enumerate}
    \end{problem}
    
    \subsection*{Lösung}


    \section*{Aufgabe 5}
    
    \begin{problem}
        Bestimmen Sie zu den Daten
        \begin{center}
            \begin{tabular}{lcccc}
                \toprule
                \(x_k\) & \(0\) & \(\frac{\pi}{2}\) & \(\pi\) & \(\frac{3}{2}\pi\)\\
                \midrule
                \(f\parentheses*{x_k}\) & \(2\) & \(0\) & \(2\) & \(0\)\\
                \bottomrule
            \end{tabular}
        \end{center}
        ein komplexes trigonometrisches Polynom der Form
        \[
            T_4\parentheses*{f; x} = \sum_{j = 0}^{n - 1}d_j\parentheses*{f}e^{ijx},
        \]
        das die Daten interpoliert, also die Bedingung
        \[
            T_4\parentheses*{f; x_k} = f\parentheses*{x_k}
        \]
        für \(k = 0, \ldots, 3\) erfüllt.
    \end{problem}
    
    \subsection*{Lösung}
    Ein trigonometrisches Polynom ist gegeben durch
    \[
        T_4\parentheses*{f; x} = \sum_{j = 0}^3 d_j\parentheses*{f}e^{ijx},
    \]
    mit
    \[
        d_j\parentheses*{f} = \frac{1}{4}\sum_{k = 0}^3 f\parentheses*{x_k}e^{-ijx_k} \quad \text{und} \quad e^{-\frac{h\pi i}{2}} = \parentheses*{-i}^n.
    \]
    Es gilt
    \begin{align*}
        d_0 &= \frac{1}{4}\sum_{k = 0}^3 f\parentheses*{x_k} = \frac{1}{4}\parentheses*{2 + 0 + 2 + 0} = 1,\\
        d_1 &= \frac{1}{4}\sum_{k = 0}^3 f\parentheses*{x_k}e^{-ix_k} = \frac{1}{4}\parentheses*{2 + 0 + 2i + 0} = \frac{1}{2} + \frac{1}{2}i,\\
        d_2 &= \frac{1}{4}\sum_{k = 0}^3 f\parentheses*{x_k}e^{-2ix_k} = \frac{1}{4}\parentheses*{2 + 0 + \parentheses*{-2} + 0} = 0,\\
        d_3 &= \frac{1}{4}\sum_{k = 0}^3 f\parentheses*{x_k}e^{-3ix_k} = \frac{1}{4}\parentheses*{2 + 0 + \parentheses*{-2i} + 0} = \frac{1}{2} - \frac{1}{2}i.
    \end{align*}
    Damit gilt
    \begin{align*}
        T_4\parentheses*{f; x} &= \frac{1}{2}\parentheses*{2 + \parentheses*{1 + i}e^{ix} + \parentheses*{1 - i}e^{3ix}}\\
        &= \frac{1}{2}\parentheses*{2 + \cos\parentheses*{x} + \cos\parentheses*{3x} - \sin\parentheses*{x} + \sin\parentheses*{3x} + i\parentheses*{\cos\parentheses*{x} - \cos\parentheses*{3x} + \sin\parentheses*{x} + \sin\parentheses*{3x}}}
    \end{align*}
    da nach dem Moivreschen Satz \(e^{ix} = \cos\parentheses*{x} + i\sin\parentheses*{x}\) gilt.


    \section*{Aufgabe 6}
    
    \begin{problem}
        Gesucht ist eine finite Differenzen Formel für die erste Ableitung \(u'\parentheses*{x_i}\) basierend auf den Werten von \(u\) bei \(x_{i - 1}\), \(x_i\) und \(x_{i + 2}\) eines äquidistanten Gitters mit Gitterweite \(\Delta x\)
        \[
            u'\parentheses*{x_i} = \underbrace{\alpha u_{i - 1} + \beta u_i + \gamma u_{i + 2}}_{=: \delta_i\brackets*{u}} + \mathcal{O}\parentheses*{\Delta x^p}.
        \]
        \begin{enumerate}
            \item Bestimmen Sie \(\alpha\), \(\beta\) und \(\gamma\) so, dass die Konsistenzordnung \(p\) größtmöglich ist und geben Sie \(p\) an.
            \item Zeigen Sie, dass für eine Funktion \(u \in C^3\parentheses*{I}\) mit \(I \subset \R\) eine Fehlerabschätzung der Form
            \[
                \absolute*{u'\parentheses*{x_i} - \delta_i\brackets*{u}} \le C\norm*{u'''}_{\infty, I}\Delta x^2
            \]
            gilt und geben Sie einen sinnvollen Wert für \(C\) an.
        \end{enumerate}
    \end{problem}
    
    \subsection*{Lösung}
    \begin{enumerate}
        \item
        \item
    \end{enumerate}


    \section*{Aufgabe 7}
    
    \begin{problem}
        Das Helmholtz-Problem: Gesucht ist \(u \in C^2\parentheses*{\parentheses*{0, 1}}\) so, dass
        \begin{align*}
            -u''\parentheses*{x} + u\parentheses*{x} &= f\parentheses*{x}, \quad x \in \parentheses*{0, 1},\\
            u\parentheses*{0} = u\parentheses*{1} &= 0,
        \end{align*}
        soll mithilfe einer zentralen finite Differenzen Methode auf einem regelmäßigen Gitter mit den Gitterpunkten \(0 = x_0 < \cdots < x_n = 1\) approximiert und in ein lineares Gleichungssystem \(Au = b\) überführt werden.
        \begin{enumerate}
            \item Bstimmen Sie die Matrix \(A\).
            \item Geben Sie Schranken für
            \begin{enumerate}
                \item die Eigenwerte von \(A\),

                \emph{Hinweis: Gerschgorin-Kreise.}
                \item die \(L_2\)-Norm von \(A\) und von \(A^{-1}\) und
                \item die Konditionszahl von \(A\) an.
            \end{enumerate}
            \item Bestimmen Sie die
            \begin{enumerate}
                \item \(L_\infty\)-Norm von \(A\) und
                \item \(L_1\)-Norm von \(A\).
            \end{enumerate}
        \end{enumerate}
    \end{problem}
    
    \subsection*{Lösung}
    \begin{enumerate}
        \item Die Schrittweite der finiten Differenzenmethode ist \(h = \frac{1}{n}\) und die Gitterpunkte liegen bei \(x_i = ih, i = 0, \ldots, n\).
        Aufgrund der Randbedingungen gilt \(u\parentheses*{x_0} = u\parentheses*{x_n} = 0\).
        Zur Approximation der zweiten Ableitung verwenden wir die Differenzenformel
        \[
            -u''\parentheses*{x_i} \approx \frac{-u\parentheses*{x_{i - 1}} + 2u\parentheses*{x_i} - u\parentheses*{x_{i + 1}}}{h^2}.
        \]
        Wir erhalten die Gleichungen
        \begin{align*}
            \frac{1}{h^2}\parentheses*{2u\parentheses*{x_1} - u\parentheses*{x_2}} + u\parentheses*{x_1} &= f\parentheses*{x_1},\\
            \frac{1}{h^2}\parentheses*{-u\parentheses*{x_{i - 1}} + 2u\parentheses*{x_i} - u\parentheses*{x_{i + 1}}} + u\parentheses*{x_i} &= f\parentheses*{x_i}, \quad i = 2, \ldots, n - 2,\\
            \frac{1}{h^2}\parentheses*{-u\parentheses*{x_{n - 2}} + 2u\parentheses*{x_{n - 1}}} + u\parentheses*{x_{n - 1}} &= f\parentheses*{x_{n - 1}},
        \end{align*}
        bzw.
        \[
            Au = b,
        \]
        wobei
        \[
            A = \frac{1}{h^2}\begin{pmatrix}
                2 + h^2 & -1 & 0 & \cdots & 0\\
                -1 & 2 + h^2 & -1 & \ddots & \vdots\\
                0 & \ddots & \ddots & \ddots & 0\\
                \vdots & \ddots & \ddots & \ddots & -1\\
                0 & \cdots & 0 & -1 & 2 + h^2
            \end{pmatrix}_{\parentheses*{n - 1} \times \parentheses*{n - 1}}, \quad u = \begin{pmatrix}
                u\parentheses*{x_1}\\
                \vdots\\
                u\parentheses*{x_{n - 1}}
            \end{pmatrix}, \quad b = \begin{pmatrix}
                f\parentheses*{x_1}\\
                \vdots\\
                f\parentheses*{x_{n - 1}}
            \end{pmatrix}.
        \]
        \item
        \begin{enumerate}
            \item Die Gerschgorin-Kreise \(K_i\) sind definiert als
            \[
                K_i = \braces*{z \in \C : \absolute*{z - a_{ii}} \le \sum_{\substack{j = 1\\j \ne i}}^n \absolute*{a_{ij}}},
            \]
            d.h.
            \begin{align*}
                K_{n - 1} = K_1 &= \braces*{z \in \C : \absolute*{z - \parentheses*{1 + \frac{2}{h^2}}} \le \frac{1}{h^2}},\\
                K_j &= \braces*{z \in \C : \absolute*{z - \parentheses*{1 + \frac{2}{h^2}}} \le \frac{2}{h^2}}, \quad j = 2, \ldots, n - 2.
            \end{align*}
            Da \(A\) symmetrisch ist, sind die Eigenwerte reell, also gilt für das Spektrum
            \[
                \sigma\parentheses*{A} \subset \bigcup_{i = 1}^{n - 1}K_i = \brackets*{1, 1 + \frac{4}{h^2}}.
            \]
            Also \(\lambda_{\text{min}} \ge 1, \lambda_{\text{max}} \le 1 + \frac{4}{h^2}\).
            \item Es gilt
            \[
                \norm*{A}_{L_2} = \sqrt{\lambda_{\text{max}}\parentheses*{A^T A}}.
            \]
            Da \(A\) symmetrisch ist, ist \(\lambda_{\text{max}}\parentheses*{A^T A} = \lambda_{\text{max}}\parentheses*{A}^2\) und somit folgt
            \[
                \norm*{A}_{L_2} = \sqrt{\lambda_{\text{max}}\parentheses*{A}^2} \le 1 + \frac{4}{h^2}.
            \]
            Für die inverse Matrix \(A^{-1}\) folgt analog
            \[
                \norm*{A^{-1}}_{L_2} = \sqrt{\lambda_{\text{max}}\parentheses*{A^{-T}A^{-1}}} = \sqrt{\lambda_{\text{max}}\parentheses*{A^{-1}}^2} = \absolute*{\lambda_{\text{max}}\parentheses*{A^{-1}}} = \frac{1}{\lambda_{\text{min}}\parentheses*{A}} \le 1.
            \]
            \item
            \[
                \kappa\parentheses*{A} = \frac{\lambda_{\text{max}}\parentheses*{A}}{\lambda_{\text{min}}\parentheses*{A}} \le 1 + \frac{4}{h^2}.
            \]
        \end{enumerate}
        \item
        \begin{enumerate}
            \item Die \(L_\infty\)-Norm einer Matrix ist die maximale Zeilensumme, d.h.
            \[
                \norm*{A}_{L_\infty} = \max_{i = 1, \ldots, n - 1}\sum_{j = 1}^{n - 1}\absolute*{a_{ij}} = 1 + \frac{2}{h^2} + \frac{1}{h^2} + \frac{1}{h^2} = 1 + \frac{4}{h^2}.
            \]
            \item Die \(L_1\)-Norm einer Matrix ist die maximale Spaltensumme, d.h.
            \[
                \norm*{A}_{L_1} = \max_{i = 1, \ldots, n - 1}\sum_{j = 1}^{n - 1}\absolute*{a_{ji}} = 1 + \frac{2}{h^2} + \frac{1}{h^2} + \frac{1}{h^2} = 1 + \frac{4}{h^2}.
            \]
        \end{enumerate}
    \end{enumerate}


    \section*{Aufgabe 8}
    
    \begin{problem}
        Gegeben sei das lineare Gleichungssystem \(Ax = b\) mit
        \[
            A = \begin{pmatrix}
                4 & -1 & 0\\
                -1 & 4 & 1\\
                0 & 1 & 4
            \end{pmatrix}, \quad b = \begin{pmatrix}
                0\\
                -4\\
                -3
            \end{pmatrix}.
        \]
        mit Lösung \(x = \parentheses*{-\frac{13}{56}, -\frac{13}{16}, -\frac{29}{56}}^T\).
        \begin{enumerate}
            \item Konvergieren das Jacobi- und Gauß-Seidel-Verfahren für jeden Startvektor \(x^0 \in \R^3\)?
            \item Führen Sie einen Schritt des Jacobi-Verfahrens durch.
            Verwenden Sie als Startvektor \(x^0 = \parentheses*{1, 0, 0}^T\).
            \item Zeigen Sie, dass die Voraussetzungen zur Anwendung des CG-Verfahrens gegeben sind und führen Sie einen Schritt mit dem CG-Verfahren durch.
            Verwenden Sie als Startvektor \(x^0 = \parentheses*{0, -1, -1}^T\).
            \item Welches Ergebnis erhält man nach drei Schritten mit dem CG-Verfahren?
        \end{enumerate}
    \end{problem}
    
    \subsection*{Lösung}
    \begin{enumerate}
        \item Da die Matrix \(A\) irreduzibel und (strikt) diagonaldominant ist, konvergieren sowohl das Jacobi- als auch das Gauß-Seidel-Verfahren.
        \item \(A\) lässt sich zerlegen in \(D - L - U\) mit \(A = D - L - U\) wie folgt
        \[
            D = \begin{pmatrix}
                4 & 0 & 0\\
                0 & 4 & 0\\
                0 & 0 & 4
            \end{pmatrix}, \quad L = \begin{pmatrix}
                0 & 0 & 0\\
                -1 & 0 & 0\\
                0 & 1 & 0
            \end{pmatrix}, \quad U = \begin{pmatrix}
                0 & -1 & 0\\
                0 & 0 & 1\\
                0 & 0 & 0
            \end{pmatrix}.
        \]
        Das Residuum für den Startvektor \(x^0\) beträgt \(r^0 = Ax^0 - b = \parentheses*{0, 3, 3}^T\).
        Wir können nun berechnen:
        \[
            x^1 = x^0 - D^{-1}\parentheses*{Ax^0 - b} = x^0 - D^{-1}r^0 = \begin{pmatrix}
                1\\
                0\\
                0
            \end{pmatrix} - \begin{pmatrix}
                \frac{1}{4} & 0 & 0\\
                0 & \frac{1}{4} & 0\\
                0 & 0 & \frac{1}{4}
            \end{pmatrix}\begin{pmatrix}
                0\\
                3\\
                3
            \end{pmatrix} = \begin{pmatrix}
                1\\
                -\frac{3}{4}\\
                -\frac{3}{4}
            \end{pmatrix}.
        \]
        \item Da die Matrix \(A\) symmetrisch  ist und positiv definit ist (die Hauptminoren \(15, 16, 15\) sind positiv), ist das CG-Verfahren wohldefiniert und konvergiert.
        Die Startwerte für das CG-Verfahrens sind \(x_0 = \begin{pmatrix}
            0\\
            -1\\
            -1
        \end{pmatrix}\) und \(d_0 = r_0 = b - Ax_0 = \begin{pmatrix}
            -1\\
            1\\
            2
        \end{pmatrix}\).
        Für den ersten Schritt erhalten wir somit
        \[
            \alpha_0 = \frac{r_0^T r_0}{d_0^T Ad_0} = \frac{1}{5}, \quad x_1 = x_0 + \alpha_0 d_0 = \begin{pmatrix}
                -\frac{1}{5}\\
                -\frac{4}{5}\\
                -\frac{3}{5}
            \end{pmatrix}.
        \]
        \item Da das CG-Verfahren endlich ist und für ein lineares Gleichungssystem der Dimension \(n\) spätestens nach \(n\) Schritten (bis auf Rundungsfehler) die exakte Lösung liefert, erhält man nach drei Schritten
        \[
            x^3 = \parentheses*{-\frac{13}{56}, -\frac{13}{16}, -\frac{29}{56}}^T.
        \]
    \end{enumerate}
\end{document}
