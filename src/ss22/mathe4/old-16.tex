\documentclass{exercise}

\institute{Applied and Computational Mathematics}
\title{Altklausur 16}
\author{Joshua Feld, 406718}
\course{Mathematische Grundlagen IV}
\professor{Torrilhon \& Berkels}
\semester{Sommersemester 2022}
\program{CES (Bachelor)}

\begin{document}
    \maketitle


    \section*{Aufgabe 1}
    
    \begin{problem}
        Gegeben sei die Gleichung
        \begin{align*}
            u_x\parentheses*{x, y} + 2yu_y\parentheses*{x, y} &= y, \quad x \in \R^+, y \in \R,\\
            u\parentheses*{0, y} &= y, \quad y \in \R.
        \end{align*}
        \begin{enumerate}
            \item Bestimmen Sie die Charakteristiken der Gleichung.
            \item Bestimmen Sie die Lösung der Gleichung.
            \begin{enumerate}
                \item Formulieren Sie dazu zunächst die Differentialgleichung, die die Lösung entlang der Charakteristiken erfüllt.
                \item Lösen Sie anschließend diese Differentialgleichung.
            \end{enumerate}
            \item Überprüfen Sie, ob die in b) bestimmte Lösung dei Gleichung erfüllt.
        \end{enumerate}
    \end{problem}
    
    \subsection*{Lösung}
    \begin{enumerate}
        \item Gesucht ist eine Charakteristik \(\gamma\parentheses*{s} = \parentheses*{\gamma_1\parentheses*{s}, \gamma_2\parentheses*{s}}^T\).
        Die Transportgleichung lässt sich als System schreiben mit
        \[
            \begin{pmatrix}
                \gamma_1' & \gamma_2'\\
                1 & 2y
            \end{pmatrix}\begin{pmatrix}
                u_x\\
                u_y
            \end{pmatrix} = \begin{pmatrix}
                \frac{\partial u}{\partial s}\\
                y
            \end{pmatrix}.
        \]
        Es handelt sich um eine Charakteristik, wenn
        \[
            \det\begin{pmatrix}
                \gamma_1' & \gamma_2'\\
                1 & 2y
            \end{pmatrix} = 2y\gamma_1' - \gamma_2' = 0
        \]
        gilt.
        Somit folgt für den charakteristischen Weg die DGL
        \[
            \begin{pmatrix}
                \gamma_1'\parentheses*{s}\\
                \gamma_2'\parentheses*{s}
            \end{pmatrix} = \begin{pmatrix}
                1\\
                2y
            \end{pmatrix} \quad \text{mit} \quad \begin{pmatrix}
                \gamma_1\parentheses*{0}\\
                \gamma_2\parentheses*{0}
            \end{pmatrix} = \begin{pmatrix}
                0\\
                y_0
            \end{pmatrix}.
        \]
        Als Lösung erhält man
        \[
            \gamma_1\parentheses*{s} = s + \alpha_1 \quad \text{und} \quad \gamma_2\parentheses*{s} = \alpha_2 e^{2s}.
        \]
        Aus den Anfangsbedingungen lassen sich die Konstanten bestimmen zu
        \[
            0 = \gamma_1\parentheses*{0} = \alpha_1 \implies \gamma_1\parentheses*{s} = s,
        \]
        sowie
        \[
            y_0 = \gamma_2\parentheses*{0} = \alpha_2 \implies \gamma_2\parentheses*{s} = y_0 e^{2s}.
        \]
    \end{enumerate}
\end{document}
