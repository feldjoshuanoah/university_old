\documentclass{exercise}

\institute{Applied and Computational Mathematics}
\title{Selbstrechenübung 1}
\author{Joshua Feld, 406718}
\course{Mathematische Grundlagen IV}
\professor{Torrilhon}
\semester{Sommersemester 2022}
\program{CES (Bachelor)}

\begin{document}
    \maketitle


    \section*{Aufgabe 1}

    \begin{problem}
        Man berechne für
        \[
            u: \R \times \parentheses*{-\frac{\pi}{2}, \frac{\pi}{2}} \times \R \to \R, u\parentheses*{x, y, z} = y\sin\parentheses*{x} + z\tan\parentheses*{y}
        \]
        sämtliche partielle Ableitungen \(D^\alpha u\) für \(\absolute*{\alpha} \le 2\).
    \end{problem}

    \subsection*{Lösung}


    \section*{Aufgabe 2}

    \begin{problem}
        Bestimmen Sie zu den Daten
        \begin{center}
            \begin{tabular}{rcccc}
                \toprule
                \(x_k\) & \(0\) & \(\frac{\pi}{2}\) & \(\pi\) & \(\frac{3\pi}{2}\)\\
                \midrule
                \(y_k\) & \(4\) & \(2\) & \(1\) & \(2\)\\
                \bottomrule
            \end{tabular}
        \end{center}
        ein reelles trigonometrisches Polynom der Form
        \[
            T_4\parentheses*{y; x} = \sum_{k = 0}^3 \hat{d}_k\parentheses*{y}\cos\parentheses*{kx},
        \]
        das die Daten interpoliert, also die Bedingung
        \[
            T_4\parentheses*{y; x_k} = y_k \quad \text{für} \quad k = 0, \ldots, 3
        \]
        erfüllt.
    \end{problem}

    \subsection*{Lösung}


    \section*{Aufgabe 3}

    \begin{problem}
        Es sei \(P\parentheses*{x} = \sum_{k = 0}^n a_k \cos\parentheses*{kx} + b_k \sin\parentheses*{kx}, a_k, b_k \in \R\) ein trigonometrisches Polynom vom Grad \(n\) und \(Q\parentheses*{x} = \sum_{k = 0}^m c_k \cos\parentheses*{kx} + d_k \sin\parentheses*{kx}, c_k, d_k \in \R\) eines vom Grad \(m\).
        Wir wollen beweisen, dass dann \(R\parentheses*{x} := P\parentheses*{x}Q\parentheses*{x}\) ein trigonometrisches Polynom vom Grad \(m + n\) ist.
        Gehen Sie dazu wie folgt vor:
        \begin{enumerate}
            \item Zeigen Sie
            \[
                \sin^2\parentheses*{x} = \frac{1 - \cos\parentheses*{2x}}{2}, \quad \cos^2\parentheses*{x} = \frac{1 + \cos\parentheses*{2x}}{2}, \quad \sin\parentheses*{x}\cos\parentheses*{x} = \frac{1}{2}\sin\parentheses*{2x}.
            \]
            \item Zeigen Sie
            \begin{align*}
                \cos\parentheses*{x}\cos\parentheses*{kx} &= \frac{1}{2}\parentheses*{\cos\parentheses*{\parentheses*{k + 1}x} + \cos\parentheses*{\parentheses*{k - 1}x}},\\
                \sin\parentheses*{x}\cos\parentheses*{kx} &= \frac{1}{2}\parentheses*{\sin\parentheses*{\parentheses*{k + 1}x} - \sin\parentheses*{\parentheses*{k - 1}x}},\\
                \cos\parentheses*{x}\sin\parentheses*{kx} &= \frac{1}{2}\parentheses*{\sin\parentheses*{\parentheses*{k + 1}x} + \sin\parentheses*{\parentheses*{k - 1}x}},\\
                \sin\parentheses*{x}\sin\parentheses*{kx} &= -\frac{1}{2}\parentheses*{\cos\parentheses*{\parentheses*{k + 1}x} - \cos\parentheses*{\parentheses*{k - 1}x}}.
            \end{align*}
            \item Beweisen Sie die Aussage für beliebige Paare \(n, m \in \N_0\) mittels vollständiger Induktion über \(m\).
        \end{enumerate}
    \end{problem}
\end{document}
