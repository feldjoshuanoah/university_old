\documentclass{exercise}

\institute{Applied and Computational Mathematics}
\title{Selbstrechenübung 1}
\author{Joshua Feld, 406718}
\course{Mathematische Grundlagen IV}
\professor{Torrilhon \& Berkels}
\semester{Sommersemester 2022}
\program{CES (Bachelor)}

\begin{document}
    \maketitle


    \section*{Aufgabe 1}

    \begin{problem}
        Man berechne für
        \[
            u: \R \times \parentheses*{-\frac{\pi}{2}, \frac{\pi}{2}} \times \R \to \R, u\parentheses*{x, y, z} = y\sin\parentheses*{x} + z\tan\parentheses*{y}
        \]
        sämtliche partielle Ableitungen \(D^\alpha u\) für \(\absolute*{\alpha} \le 2\).
    \end{problem}

    \subsection*{Lösung}
    Es ist
    \begin{align*}
        D^{\parentheses*{1, 0, 0}}u &= y\cos x, & D^{\parentheses*{0, 1, 0}}u &= \sin x + \frac{z}{\cos^2 y}, & D^{\parentheses*{0, 0, 1}}u &= \tan y,\\
        D^{\parentheses*{2, 0, 0}}u &= -y\sin x, & D^{\parentheses*{0, 2, 0}}u &= \frac{2z\sin y}{\cos^3 y}, & D^{\parentheses*{0, 0, 2}}u &= 0,\\
        D^{\parentheses*{1, 1, 0}}u &= \cos x, & D^{\parentheses*{1, 0, 1}}u &= 0, & D^{\parentheses*{0, 1, 1}}u &= \frac{1}{\cos^2 y}.
    \end{align*}


    \section*{Aufgabe 2}

    \begin{problem}
        Bestimmen Sie zu den Daten
        \begin{center}
            \begin{tabular}{rcccc}
                \toprule
                \(x_k\) & \(0\) & \(\frac{\pi}{2}\) & \(\pi\) & \(\frac{3\pi}{2}\)\\
                \midrule
                \(y_k\) & \(4\) & \(2\) & \(1\) & \(2\)\\
                \bottomrule
            \end{tabular}
        \end{center}
        ein reelles trigonometrisches Polynom der Form
        \[
            T_4\parentheses*{y; x} = \sum_{k = 0}^3 \hat{d}_k\parentheses*{y}\cos\parentheses*{kx},
        \]
        das die Daten interpoliert, also die Bedingung
        \[
            T_4\parentheses*{y; x_k} = y_k \quad \text{für} \quad k = 0, \ldots, 3
        \]
        erfüllt.
    \end{problem}

    \subsection*{Lösung}
    Ein komplexes trigonometrisches Polynom ist gegeben durch
    \[
        \tilde{T}_4\parentheses*{y; x} = \sum_{j = 0}^3 d_j\parentheses*{y}e^{ijx}
    \]
    mit
    \[
        d_j\parentheses*{y} = \frac{1}{4}\sum_{k = 0}^3 y_k e^{-ijx_k}
    \]
    und
    \[
        e^{-\frac{h\pi i}{2}} = \parentheses*{-i}^n.
    \]
    Es gilt
    \begin{align*}
        d_0 &= \frac{1}{4}\sum_{k = 0}^3 y_k = \frac{1}{4}\parentheses*{4 + 2 + 1 + 2} = \frac{9}{4},\\
        d_1 &= \frac{1}{4}\sum_{k = 0}^3 y_k e^{-ix_k} = \frac{1}{4}\parentheses*{4 - 2i + 1 + 2i} = \frac{3}{4},\\
        d_2 &= \frac{1}{4}\sum_{k = 0}^3 y_k e^{-2ix_k} = \frac{1}{4}\parentheses*{4 - 2 + 1 - 2} = \frac{1}{4},\\
        d_3 &= \frac{1}{4}\sum_{k = 0}^3 y_k e^{-3ix_k} = \frac{1}{4}\parentheses*{4 + 2i - 1 - 2i} = \frac{3}{4}.
    \end{align*}
    Damit gilt
    \begin{align*}
        \tilde{T}_4\parentheses*{y; x} &= \frac{1}{4}\parentheses*{9 + 3e^{ix} + e^{2ix} + 3e^{3ix}}\\
        &= \frac{1}{4}\parentheses*{9 + 3\parentheses*{\cos\parentheses*{x} + i\sin\parentheses*{x}} + \cos\parentheses*{2x} + i\sin\parentheses*{2x} + 3\parentheses*{\cos\parentheses*{3x} + i\sin\parentheses*{3x}}}\\
        &= \frac{1}{4}\parentheses*{9 + 3\cos\parentheses*{x} + \cos\parentheses*{2x} + 3\cos\parentheses*{3x}} + \frac{i}{4}\parentheses*{3\sin\parentheses*{x} + \sin\parentheses*{2x} + 3\sin\parentheses*{3x}},
    \end{align*}
    da nach dem Moivre'schen Satz \(e^{ix} = \cos\parentheses*{x} + i\sin\parentheses*{x}\) gilt.
    Nun gilt also \(\tilde{T}_4\parentheses*{y; x_k} = y_k\) und somit gilt dies auch für den Realteil \(\Re\parentheses*{\tilde{T}_4\parentheses*{y; x_k}} = \Re\parentheses*{y_k} = y_k\).
    Damit ist
    \[
        T_4\parentheses*{y; x} := \Re\parentheses*{\tilde{T}_4\parentheses*{y; x}} = \frac{1}{4}\parentheses*{9 + 3\cos\parentheses*{x} + \cos\parentheses*{2x} + 3\cos\parentheses*{3x}}
    \]
    ein reelles trigonometrisches Polynom, das die Daten interpoliert.


    \section*{Aufgabe 3}

    \begin{problem}
        Es sei \(P\parentheses*{x} = \sum_{k = 0}^n a_k \cos\parentheses*{kx} + b_k \sin\parentheses*{kx}, a_k, b_k \in \R\) ein trigonometrisches Polynom vom Grad \(n\) und \(Q\parentheses*{x} = \sum_{k = 0}^m c_k \cos\parentheses*{kx} + d_k \sin\parentheses*{kx}, c_k, d_k \in \R\) eines vom Grad \(m\).
        Wir wollen beweisen, dass dann \(R\parentheses*{x} := P\parentheses*{x}Q\parentheses*{x}\) ein trigonometrisches Polynom vom Grad \(m + n\) ist.
        Gehen Sie dazu wie folgt vor:
        \begin{enumerate}
            \item Zeigen Sie
            \[
                \sin^2\parentheses*{x} = \frac{1 - \cos\parentheses*{2x}}{2}, \quad \cos^2\parentheses*{x} = \frac{1 + \cos\parentheses*{2x}}{2}, \quad \sin\parentheses*{x}\cos\parentheses*{x} = \frac{1}{2}\sin\parentheses*{2x}.
            \]
            \item Zeigen Sie
            \begin{align*}
                \cos\parentheses*{x}\cos\parentheses*{kx} &= \frac{1}{2}\parentheses*{\cos\parentheses*{\parentheses*{k + 1}x} + \cos\parentheses*{\parentheses*{k - 1}x}},\\
                \sin\parentheses*{x}\cos\parentheses*{kx} &= \frac{1}{2}\parentheses*{\sin\parentheses*{\parentheses*{k + 1}x} - \sin\parentheses*{\parentheses*{k - 1}x}},\\
                \cos\parentheses*{x}\sin\parentheses*{kx} &= \frac{1}{2}\parentheses*{\sin\parentheses*{\parentheses*{k + 1}x} + \sin\parentheses*{\parentheses*{k - 1}x}},\\
                \sin\parentheses*{x}\sin\parentheses*{kx} &= -\frac{1}{2}\parentheses*{\cos\parentheses*{\parentheses*{k + 1}x} - \cos\parentheses*{\parentheses*{k - 1}x}}.
            \end{align*}
            \item Beweisen Sie die Aussage für beliebige Paare \(n, m \in \N_0\) mittels vollständiger Induktion über \(m\).
        \end{enumerate}
    \end{problem}

    \subsection*{Lösung}
    \begin{enumerate}
        \item Mit dem Additionstheorem für Kosinus und dem Satz von Pythagoras \(1 = \sin^2\parentheses*{x} + \cos^2\parentheses*{x}\) erhalten wir
        \[
            \sin^2\parentheses*{x} = \sin\parentheses*{x}\sin\parentheses*{x} = \cos^2\parentheses*{x} - \cos\parentheses*{2x} = 1 - \sin^2\parentheses*{x} - \cos\parentheses*{2x}
        \] 
        und somit die erste Aussage:
        \[
            \sin^2\parentheses*{x} = 1 - \sin^2\parentheses*{x} - \cos\parentheses*{2x} \iff 2\sin^2\parentheses*{x} = 1 - \cos\parentheses*{2x}.
        \]
        Ersetzen wir nun \(\sin^2\parentheses*{x} = 1 - \cos^2\parentheses*{x}\) erhalten wir die zweite Aussage:
        \[
            2\parentheses*{1 - \cos^2\parentheses*{x}} = 1 - \cos\parentheses*{2x} \iff 2\cos^2\parentheses*{x} = 1 + \cos\parentheses*{2x}.
        \]
        Zuletzt erhalten wir mit dem Additionstheorem für Sinus
        \[
            \sin\parentheses*{2x} = \sin\parentheses*{x + x} = 2\sin\parentheses*{x}\cos\parentheses*{x}.
        \]
        \item Wir zeigen den Beweis beispielhaft für den ersten Fall, die anderen Fälle laufen analog.
        Mit dem Additionstheorem für \(\cos\) und den Ergebnissen aus Aufgabenteil a) erhalten wir
        \begin{align*}
            \cos\parentheses*{x}\cos\parentheses*{kx} &= \cos\parentheses*{x}\cos\parentheses*{\parentheses*{k - 1}x + x}\\
            &= \underbrace{\cos^2\parentheses*{x}}_{\frac{1}{2}\parentheses*{1 + \cos\parentheses*{2x}}}\cos\parentheses*{\parentheses*{k - 1}x} - \underbrace{\cos\parentheses*{x}\sin\parentheses*{x}}_{\frac{1}{2}\sin\parentheses*{2x}}\sin\parentheses*{\parentheses*{k - 1}x}\\
            &= \frac{1}{2}\cos\parentheses*{\parentheses*{k - 1}x} + \frac{1}{2}\underbrace{\parentheses*{\cos\parentheses*{\parentheses*{k - 1}x}\cos\parentheses*{2x} - \sin\parentheses*{\parentheses*{k - 1}x}\sin\parentheses*{2x}}}_{\cos\parentheses*{\parentheses*{k - 1}x + 2x}}\\
            &= \frac{1}{2}\parentheses*{\cos\parentheses*{\parentheses*{k - 1}x} + \cos\parentheses*{\parentheses*{k + 1}x}}.
        \end{align*}
        \item
        \begin{itemize}
            \item \(m = 1\).
            Induktionsbeginn: Sei \(P\) ein trigonometrisches Polynom vom Grad \(n\) und \(Q\) eins vom Grad \(1\).
            Dann enthält man das Produkt \(P\parentheses*{x}Q\parentheses*{x}\) ausschließlich Terme der Form aus Aufgabenteil b).
            Somit ist das Produkt offensichtlich ein trigonometrisches Polynom vom Grad \(n + 1\).
            \item \(m \to m + 1\).
            Induktionsvoraussetzung: Für beliebige trigonometrische Polynome \(P\) und \(Q\) vom Grad \(n\) und \(m\) sei \(P\parentheses*{x}Q\parentheses*{x}\) ein trigonometrisches Polynom vom Grad \(n + m\).

            Sei \(\tilde{Q}\) ein beliebiges Polynom vom Grad \(m + 1\).
            Dann reicht es für das Produkt \(P\tilde{Q}\) die Terme höchster Ordnung zu betrachten, da gemäß Induktionsvoraussetzung die Terme \(\cos\parentheses*{kx}\cos\parentheses*{lx}\), \(\cos\parentheses*{kx}\sin\parentheses*{lx}\), \(\sin\parentheses*{kx}\cos\parentheses*{lx}\) und \(\sin\parentheses*{kx}\sin\parentheses*{lx}\) für \(1 \le k \le n\) und \(1 \le l \le m\) trigonometrische Polynome vom Grad höchstens \(n + m\) sind.

            Wir erhalten beispielsweise für \(\cos\parentheses*{nx}\cos\parentheses*{\parentheses*{m + 1}x}\):
            \begin{align*}
                \cos\parentheses*{nx}\cos\parentheses*{\parentheses*{m + 1}x} &= \cos\parentheses*{nx}\cos\parentheses*{mx + x}\\
                &= \cos\parentheses*{nx}\parentheses*{\cos\parentheses*{mx}\cos\parentheses*{x} - \sin\parentheses*{mx}\sin\parentheses*{x}}\\
                &= \cos\parentheses*{x}\underbrace{\cos\parentheses*{nx}\cos\parentheses*{mx}} - \sin\parentheses*{x}\underbrace{\cos\parentheses*{nx}\sin\parentheses*{mx}}.
            \end{align*}
            Die beiden unterklammerten Terme sind gemäß Induktionsvoraussetzung trigonometrische Polynome vom Grad \(n + m\).
            Multiplikation mit \(\cos\parentheses*{x}\) bzw. \(\sin\parentheses*{x}\) ergibt gemäß Aufgabenteil b) somit ein trigonometrisches Polynom vom Grad \(n + m + 1\).
        \end{itemize}
    \end{enumerate}
\end{document}
