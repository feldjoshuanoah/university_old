\documentclass{exercise}

\institute{Applied and Computational Mathematics}
\title{Selbstrechenübung 2}
\author{Joshua Feld, 406718}
\course{Mathematische Grundlagen IV}
\professor{Torrilhon \& Berkels}
\semester{Sommersemester 2022}
\program{CES (Bachelor)}

\begin{document}
    \maketitle


    \section*{Aufgabe 1}

    \begin{problem}
        Gegeben sei die folgende partielle Differentialgleichung 2. Ordnung:
        \[
            \partial_{tt} + b \cdot \partial_{tx} + \partial_{xx}, \quad b \in \R.
        \]
        Entscheiden Sie für welches \(b\) die Gleichung elliptisch, hyperbolisch oder parabolisch ist.
        Begründen Sie Ihre Antwort.
    \end{problem}

    \subsection*{Lösung}
    Wir stellen zunächst die Matrix
    \[
        A = \begin{pmatrix}
            1 & \frac{b}{2}\\
            \frac{b}{2} & 1
        \end{pmatrix}
    \]
    auf.
    Die Eigenwerte dieser Matrix sind offensichtlich \(\lambda_{1, 2} = \pm\frac{1}{2}b + 1\).
    Die partielle Differentialgleichung ist also
    \begin{itemize}
        \item elliptisch für \(b \in \parentheses*{-2, 2}\),
        \item hyperbolisch für \(\absolute*{b} > 2\),
        \item parabolisch für \(\absolute*{b} = 2\).
    \end{itemize}

    \section*{Aufgabe 2}

    \begin{problem}
        Lösen Sie die folgenden PDEs mittels Trennung der Variablen:
        \begin{enumerate}
            \item \(u_{xx} + u_{tt} + u_x + u = 0\),
            \item \(\parentheses*{u_x}^2 + \parentheses*{u_y}^2 = 1\). 
        \end{enumerate}
    \end{problem}

    \subsection*{Lösung}
    \begin{enumerate}
        \item Unser Ansatz ist
        \[
            u\parentheses*{x, t} = X\parentheses*{x} \cdot T\parentheses*{t}.
        \]
        Einsetzen liefert
        \[
            X''\parentheses*{x}T\parentheses*{t} + X\parentheses*{x}T''\parentheses*{t} + X'\parentheses*{x}T\parentheses*{t} + X\parentheses*{x}T\parentheses*{t} = 0.
        \]
        Separation der Variablen ergibt
        \[
            \frac{T''\parentheses*{t}}{T\parentheses*{t}} = -\frac{X''\parentheses*{x} + X'\parentheses*{x} + X\parentheses*{x}}{X\parentheses*{x}} \stackrel{!}{=} c^2.
        \]
        Damit folgt
        \[
            T''\parentheses*{t} = c^2 T \implies T\parentheses*{t} = Ce^{ct} + De^{-ct}
        \]
        und
        \begin{align*}
            X''\parentheses*{x} + X'\parentheses*{x} + X\parentheses*{x} = -c^2 X\parentheses*{x} &\implies X''\parentheses*{x} + X'\parentheses*{x} + \parentheses*{1 + c^2}X\parentheses*{x} = 0\\
            &\implies X\parentheses*{x} = Ae^{-\frac{1}{2} - \frac{1}{2}\sqrt{1 - 4\parentheses*{1 + c^2}}x} + Be^{-\frac{1}{2} - \frac{1}{2}\sqrt{1 - 4\parentheses*{1 + c^2}}x}
        \end{align*}
        Damit folgt
        \[
            u\parentheses*{x, t} = \parentheses*{Ae^{-\frac{1}{2} - \frac{1}{2}\sqrt{1 - 4\parentheses*{1 + c^2}}x} + Be^{-\frac{1}{2} - \frac{1}{2}\sqrt{1 - 4\parentheses*{1 + c^2}}x}}\parentheses*{Ce^{ct} + De^{-ct}}
        \]
        für beliebiges \(c\).
        \item Eine mögliche Trennung der Variablen ist
        \[
            X'\parentheses*{x}^2 = 1 - Y'\parentheses*{y}^2 \stackrel{!}{=} c.
        \]
        Damit erhalten wir 
        \begin{align*}
            X'\parentheses*{x} &= \sqrt{c}, & Y'\parentheses*{y} &= \sqrt{1 - c},\\
            X\parentheses*{x} &= \sqrt{c}x + A, & Y\parentheses*{y} &= \sqrt{1 - c}y + B.
        \end{align*}
        Somit erhalten wir als zusammengesetzte Lösung
        \[
            u\parentheses*{x, y} = \sqrt{c}x + \sqrt{1 - c}y + C
        \]
        für beliebiges \(c\).
    \end{enumerate}


    \section*{Aufgabe 3}

    \begin{problem}
        Gegeben ist die diskrete Fourier-Transformierte
        \[
            \hat{y} = \parentheses*{\frac{9}{4}, \frac{3}{4}, \frac{1}{4}, \frac{3}{4}}^T.
        \]
        Bestimmen Sie den zugehörigen Datenvektor \(y\), indem Sie zuerst die Matrixdarstellung der diskreten Fourier-Transformation \(\mathcal{F}: \C^4 \to \C^4\) mit \(y \mapsto \mathcal{F}\parentheses*{y} = \hat{y}\) bestimmen und anschließend die inverse Matrix verwenden um \(y\) zu berechnen.
    \end{problem}

    \subsection*{Lösung}
    Aus der Vorlesung wissen wir, dass die Matrixdarstellung zu \(\mathcal{F}\) mithilfe der Spaltenvektoren
    \[
        b_j = \parentheses*{e^{-\frac{2\pi ijk}{n}}}_{k = 0}^{n - 1} \in \C^n
    \]
    aufgestellt wird, wobei die Matrixdarstellung durch \(\frac{1}{n}B\) mit \(B = \parentheses*{b_0, \ldots, b_{n - 1}}\) gegeben ist.
    Hier ergibt sich somit die Darstellungsmatrix
    \[
        D := \frac{1}{4}\underbrace{\begin{pmatrix}
            1 & 1 & 1 & 1\\
            1 & -i & -1 & i\\
            1 & -1 & 1 & -1\\
            1 & i & -1 & -i
        \end{pmatrix}}_{= B}, \quad \hat{y} = Dy = \mathcal{F}\parentheses*{y}.
    \]
    Aus der Vorlesung wissen wir, dass für die Inverse \(D^{-1} = \overline{B}^T\) gilt, also
    \[
        D^{-1} = \overline{B}^T = \begin{pmatrix}
            1 & 1 & 1 & 1\\
            1 & i & -1 & -i\\
            1 & -1 & 1 & -1\\
            1 & -i & -1 & i
        \end{pmatrix}.
    \]
    Wir berechnen nun den gesuchten Datenvektor
    \[
        y = D^{-1}\hat{y} = \begin{pmatrix}
            1 & 1 & 1 & 1\\
            1 & i & -1 & -i\\
            1 & -1 & 1 & -1\\
            1 & -i & -1 & i
        \end{pmatrix}\parentheses*{\frac{1}{4}\begin{pmatrix}
            9\\3\\1\\3
        \end{pmatrix}} = \begin{pmatrix}
            4\\2\\1\\2
        \end{pmatrix}.
    \]
\end{document}
