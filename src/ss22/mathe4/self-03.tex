\documentclass{exercise}

\institute{Applied and Computational Mathematics}
\title{Selbstrechenübung 3}
\author{Joshua Feld, 406718}
\course{Mathematische Grundlagen IV}
\professor{Torrilhon \& Berkels}
\semester{Sommersemester 2022}
\program{CES (Bachelor)}

\begin{document}
    \maketitle


    \section*{Aufgabe 1}

    \begin{problem}
        Eine wichtige Voraussetzung um die FFT z.B. in der Bildbearbeitung zu verwenden ist, die FFT auf mehrere Dimensionen zu erweitern.
        Hier zeigen wir exemplarisch wie die bekannte eindimensionale FFT in die zweite Dimension erweitert werden kann.

        In Analogie zum eindimensionalen Fall wird auf \(\brackets*{0, 2\pi}^2\) das trigonometrische Polynom \(T_{n_1 n_2}\parentheses*{f; x, y}\) einer Funktion \(f: \brackets*{0, 2\pi}^2 \to \C\) definiert durch
        \[
            T_{n_1 n_2}\parentheses*{f; x, y} := \sum_{j_1 = 0}^{n_1 - 1}\sum_{j_2 = 0}^{n_2 - 1}d_{j_1, j_2}\parentheses*{f}e^{i\parentheses*{j_1 x + j_2 y}}.
        \]
        Die entsprechenden Fourierkoeffizienten von \(f\) ergeben sich zu
        \begin{align*}
            d_{j_1, j_2}\parentheses*{f} &:= \frac{1}{n_1 n_2}\sum_{l_1 = 0}^{n_1 - 1}\sum_{l_2 = 0}^{n_2 - 1}f\parentheses*{x_{l_1}, x_{l_2}}e^{-i\parentheses*{j_1 x_{l_1} + j_2 y_{l_2}}},\\
            x_{l_1} &:= \parentheses*{\frac{2\pi l_1}{n_1}}, \quad y_{l_2} := \parentheses*{\frac{2\pi l_2}{n_2}},
        \end{align*}
        für \(0 \le j_1 \le n_1 - 1\) und \(0 \le j_2 \le n_2 - 1\).
        \begin{enumerate}
            \item Sei \(\varepsilon_p^q := e^{-\frac{2\pi iq}{p}}\).
            Zeigen Sie:
            \begin{align*}
                \varepsilon_{2m}^{2kl} &= \varepsilon_m^{kl}, \quad k, l, m \in \Z,\\
                \varepsilon_m^{k\parentheses*{m + l}} &= \varepsilon_m^{kl}, \quad k, l, m \in \Z,\\
                \varepsilon_{2m}^{\parentheses*{2k + 1}\parentheses*{m + l}} &= \parentheses*{-1}\varepsilon_m^{kl}\varepsilon_{2m}^l, \quad k, l, m \in \Z.
            \end{align*}
            \item Nehmen wir an, dass die Seitenlängen \(n_1 = 2m_1\) und \(n_2 = 2m_2\) gerade sind.
            Verwenden Sie das Ergebnis aus Teilaufgabe a), um die 2D Fourier-Transformation der Längen \(n_1\) und \(n_2\) auf vier Fourier-Transformationen der in beiden Koordinatenrichtungen halbierten Längen mit den Koeffizienten
            \[
                d_{2k_1, 2k_2}, d_{2k_1 + 1, 2k_2}, d_{2k_1, 2k_2 + 1}\text{ und }d_{2k_1 + 1, 2k_2 + 1},
            \]
            für \(0 \le k_1 \le m_1 - 1\) bzw. \(0 \le k_2 \le m_2 - 1\) zurückzuführen.
        \end{enumerate}
    \end{problem}

    \subsection*{Lösung}

\end{document}
