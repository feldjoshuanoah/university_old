\documentclass{exercise}

\institute{Applied and Computational Mathematics}
\title{Selbstrechenübung 4}
\author{Joshua Feld, 406718}
\course{Mathematische Grundlagen IV}
\professor{Torrilhon \& Berkels}
\semester{Sommersemester 2022}
\program{CES (Bachelor)}

\begin{document}
    \maketitle


    \section*{Aufgabe 1}

    \begin{problem}
        Wir betrachten Polarkoordinaten \(\parentheses*{r, \varphi}\) mit
        \[
            x = r\cos\varphi, \quad y = r\sin\varphi, \quad r > 0, \varphi \in \left[0, 2\pi\right).
        \]
        Sei \(u\) eine harmonische Funktion in Polarkoordinaten, d.h. \(u = u\parentheses*{r, \varphi}\), in der Kreisscheibe
        \[
            B = \braces*{\parentheses*{r, \varphi} : r < 2}
        \]
        mit \(u\parentheses*{2, \varphi} = 3\sin\varphi + 1\).
        \begin{enumerate}
            \item Bestimmen Sie \(u\parentheses*{0, 0}\).
            \item Bestimmen Sie \(\max_{\parentheses*{r, \varphi} \in \bar{B}}u\parentheses*{r, \varphi}\) und \(\min_{\parentheses*{r, \varphi} \in \bar{B}}u\parentheses*{r, \varphi}\).
        \end{enumerate}
    \end{problem}

    \subsection*{Lösung}
    \begin{enumerate}
        \item Wir benutzen die Mittelwerteigenschaft:
        \begin{align*}
            u\parentheses*{0, 0} &= \frac{1}{\omega_n r^{n - 1}}\int_{\partial B_r\parentheses*{0}}u\parentheses*{2, \varphi}\d S_r\parentheses*{\varphi}\\
            &= \frac{1}{\omega_n}\int_{\partial B_1\parentheses*{0}}u\parentheses*{2, \varphi}\d\varphi\\
            &= \frac{1}{2\pi}\int_{\partial B_1\parentheses*{0}}\parentheses*{3\sin\parentheses*{\varphi} + 1}\d\varphi\\
            &= \frac{1}{2\pi} \cdot \parentheses*{0 + 2\pi}\\
            &= 1.
        \end{align*}
        \item Wir benutzen das Maximumprinzip:
        \[
            \max_{\parentheses*{r, \varphi} \in \bar{B}}u\parentheses*{r, \varphi} = \max_{\parentheses*{r, \varphi} \in \partial B}u\parentheses*{r, \varphi} = \max_{\varphi \in \left[0, 2\pi\right)}u\parentheses*{2, \varphi} = \max_{\varphi \in \left[0, 2\pi\right)}\parentheses*{3\sin\parentheses*{\varphi} + 1} = 4
        \]
        und
        \[
            \min_{\parentheses*{r, \varphi} \in \bar{B}}u\parentheses*{r, \varphi} = -\max_{\parentheses*{r, \varphi} \in \partial B}-u\parentheses*{r, \varphi} = -\max_{\varphi \in \left[0, 2\pi\right)}-u\parentheses*{2, \varphi} = -\max_{\varphi \in \left[0, 2\pi\right)}\parentheses*{-3\sin\parentheses*{\varphi} - 1} = -2.
        \]
    \end{enumerate}


    \section*{Aufgabe 2}

    \begin{problem}
        Lösen Sie das Randwertproblem
        \begin{align*}
            \Delta u\parentheses*{x, y} &= 0, \quad \text{in }\Omega = \parentheses*{0, 1}^2,\\
            u\parentheses*{0, y} &= 0, \quad y \in \parentheses*{0, 1},\\
            u_x\parentheses*{1, y} &= \sin\parentheses*{2\pi y}, \quad y \in \parentheses*{0, 1},\\
            u\parentheses*{x, 0} &= 0, \quad x \in \parentheses*{0, 1},\\
            u\parentheses*{x, 1} &= 0, \quad x \in \parentheses*{0, 1}.
        \end{align*}
    \end{problem}

    \subsection*{Lösung}
    Wir lösen dieses Randwertproblem mit dem Separationsansatz:
    \[
        u\parentheses*{x, y} = X\parentheses*{x}Y\parentheses*{y}, \quad X, Y \ne 0.
    \]
    Einsetzen in die DGL liefert
    \[
        Y\parentheses*{y}X''\parentheses*{x} + X\parentheses*{x}Y''\parentheses*{y} = 0 \iff \underbrace{\frac{X''\parentheses*{x}}{X\parentheses*{x}}}_{\text{nur von }x\text{ abh.}} + \underbrace{\frac{Y''\parentheses*{y}}{Y\parentheses*{y}}}_{\text{nur von }y\text{ abh.}} = 0
    \]
    Diese Gleichung kann nur erfüllt werden, wenn
    \begin{align*}
        \frac{X''}{X} = \lambda^2 &\iff X''\parentheses*{x} - \lambda^2 X\parentheses*{x} = 0,\\
        \frac{Y''}{Y} = -\lambda^2 &\iff Y''\parentheses*{y} + \lambda^2 Y\parentheses*{y} = 0
    \end{align*}
    mit \(\lambda \in \R^+\).
    Die Lösungen dieser ODEs sind dann
    \begin{align*}
        \chi_1\parentheses*{t} = t^2 - \lambda^2 = 0 \implies t_{1, 2} = \pm\lambda &\implies X\parentheses*{x} = \tilde{C}_1 e^{\lambda x} + \tilde{C}_2 e^{-\lambda x} = C_1 \sinh\parentheses*{\lambda x} + C_2 \cosh\parentheses*{\lambda x},\\
        \chi_2\parentheses*{t} = t^2 + \lambda^2 = 0 \implies t_{1, 2} = \pm i\lambda &\implies Y\parentheses*{y} = \tilde{C}_3 e^{i\lambda y} + \tilde{C}_4 e^{-i\lambda y} = C_3 \sin\parentheses*{\lambda y} + C_4 \cos\parentheses*{\lambda y}.
    \end{align*}
    Wir betrachten nun zunächst die erste Randbedingung:
    \[
        u\parentheses*{0, y} = X\parentheses*{0}Y\parentheses*{y} = \parentheses*{C_1 \sinh\parentheses*{0} + C_2 \cosh\parentheses*{0}}Y\parentheses*{y} = C_2 Y\parentheses*{y} = 0 \xRightarrow{Y \ne 0} C_2 = 0 \implies X\parentheses*{x} = C_1 \sinh\parentheses*{\lambda x}.
    \]
    Insbesondere gilt auch \(C_1 \ne 0\) weil \(X \ne 0\).
    Als nächstes betrachten wir
    \[
        u\parentheses*{x, 0} = X\parentheses*{x}Y\parentheses*{0} = X\parentheses*{x}\parentheses*{C_3 \sin\parentheses*{0} + C_4 \cos\parentheses*{0}} = X\parentheses*{x}C_4 = 0 \xRightarrow{C_1 \ne 0} C_4 = 0 \implies Y\parentheses*{y} = C_3 \sin\parentheses*{\lambda y}
    \]
    und auch hier folgt direkt \(C_3 \ne 0\) weil \(Y \ne 0\).
    Die folgende Randbedingung liefert
    \[
        u\parentheses*{x, 1} = X\parentheses*{x}Y\parentheses*{1} = C_1 \sinh\parentheses*{\lambda x}\parentheses*{C_3 \sin\parentheses*{\lambda}} = 0 \xRightarrow{C_1, C_3 \ne 0, \lambda > 0} \lambda_n = n\pi, n \in N.
    \]
    Einsetzen in die zweite Bedingung
    \[
        u_x\parentheses*{1, y} = \partial_x X\parentheses*{1}Y\parentheses*{y} = C_1 \lambda\cosh\parentheses*{\lambda}C_3\sin\parentheses*{\lambda y} = C_1 n\pi\cosh\parentheses*{n\pi}C_3 \sin\parentheses*{n\pi y} = \sin\parentheses*{2\pi y}
    \]
    liefert \(n = 2\), \(C_1 = \frac{1}{2\pi\cosh\parentheses*{2\pi}}\) und \(C_3 = 1\).
    Insgesamt ergibt sich als eindeutige Lösung
    \[
        u\parentheses*{x, y} = \frac{1}{2\pi\cosh\parentheses*{2\pi}}\sinh\parentheses*{2\pi x}\sin\parentheses*{2\pi y}.
    \]


    \section*{Aufgabe 3}

    \begin{problem}
        Diskretisieren Sie das Konvektionsproblem
        \begin{align*}
            u'\parentheses*{x} &= f\parentheses*{x}, \quad x \in \Omega = \parentheses*{0, 1}, u \in C^2\parentheses*{\Omega},\\
            u\parentheses*{0} &= 0,
        \end{align*}
        mit der einseitigen finiten Differenz
        \[
            u'\parentheses*{x} \approx \frac{u\parentheses*{x} - u\parentheses*{x - h}}{h}, \quad h = \frac{1}{n}, n \in \N.
        \]
        Approximieren Sie das Konvektionsproblem mittels dieser finiten Differenz in den \(n\) Gitterpunkten \(x_j = jh\).
        Leiten Sie ein lineares Gleichungssystem \(Au_h = b\) für den Vektor \(u_h\) her, der die Approximationen von \(u\parentheses*{x}\) an den Gitterpunkten enthält.
        Bestimmen Sie \(A^{-1}\) und \(\norm*{A^{-1}}_\infty\).
    \end{problem}

    \subsection*{Lösung}
    Sei \(h = \frac{1}{n}\), dann gibt es insgesamt \(n\) Gitterpunkte.
    Die Approximation von \(u_h\parentheses*{x}\) im Gitterpunkt \(x_j = jh\) sei \(u_j\).
    Dann gilt im Gitterpunkt \(x_1\)
    \[
        \frac{1}{h}u_1 = f_1 = f\parentheses*{x_1}
    \]
    und für \(j = 2, \ldots, n\)
    \[
        \frac{1}{h}\parentheses*{u_j - u_{j - 1}} = f\parentheses*{x_j} = f_j.
    \]
    Wir erhalten als das finite Differenzen Gleichungssystem
    \[
        Au_h = b
    \]
    mit
    \[
        A = \frac{1}{h}\begin{pmatrix}
            1 & 0 & \cdots & \cdots & 0\\
            -1 & \ddots & \ddots & & \vdots\\
            0 & \ddots & \ddots & \ddots & \vdots\\
            \vdots & \ddots & \ddots & \ddots & 0\\
            0 & \cdots & 0 & -1 & 1
        \end{pmatrix}, \quad u_h = \begin{pmatrix}
            u_1\\\vdots\\u_n
        \end{pmatrix}, \quad b = \begin{pmatrix}
            f_1\\\vdots\\f_n
        \end{pmatrix}.
    \]
    \(A^{-1}\) enthält in der Spalte \(j\) die Lösung des Gleichungssystems
    \[
        Ax = e_j,
    \]
    wobei \(e_j\) den \(j\)-ten Einheitsvektor bezeichne, d.h.
    \[
        Ax = e_j \iff \frac{1}{h}\begin{pmatrix}
            1 & 0 & \cdots & \cdots & 0\\
            -1 & \ddots & \ddots & & \vdots\\
            0 & \ddots & \ddots & \ddots & \vdots\\
            \vdots & \ddots & \ddots & \ddots & 0\\
            0 & \cdots & 0 & -1 & 1
        \end{pmatrix}\begin{pmatrix}
            x_1\\\vdots\\x_n
        \end{pmatrix} = \begin{pmatrix}
            0\\\vdots\\0\\1\\0\\\vdots\\0
        \end{pmatrix}.
    \]
    Aus der ersten Zeile folgt \(x_1 = 0\).
    Aus den Zeilen \(k < j\) folgt
    \[
        \frac{1}{h}x_k = \frac{1}{h}x_{k - 1} \iff x_k = x_{k - 1}.
    \]
    Es ist also \(x_k = 0\) für \(k < j\).
    Die \(j\)-te Zeile ergibt
    \[
        \frac{1}{h}x_j = 1 + \frac{1}{h}\underbrace{x_{j - 1}}_{= 0} \iff x_j = h.
    \]
    Aus den Zeilen \(l > j\) folgt
    \[
        \frac{1}{h}x_l = \frac{1}{h}x_{l - 1} \iff x_l = x_{l - 1}
    \]
    und es sind alle \(x_l = h\) für \(l > j\).
    Insgesamt ist dann
    \[
        x = \begin{pmatrix}
            0\\\vdots\\0\\h\\\vdots\\h
        \end{pmatrix}.
    \]
    Berechnet man so alle Spalten von \(A^{-1}\) erhält man
    \[
        A^{-1} = h\begin{pmatrix}
            1 & 0 & \cdots & 0\\
            \vdots & \ddots & \ddots & \vdots\\
            \vdots & & \ddots & 0\\
            1 & \cdots & \cdots & 1
        \end{pmatrix}
    \]
    Die \(\norm*{A^{-1}}_\infty\)-Norm ist die größte Zeilensumme, also
    \[
        \norm*{A^{-1}}_\infty = hn = 1.
    \]


    \section*{Aufgabe 4}

    \begin{problem}
        Gegeben ist die ODE
        \begin{align*}
            -u'' + q\parentheses*{x}u\parentheses*{x} &= f\parentheses*{x}, \quad u\parentheses*{x} \in \parentheses*{0, 1},\\
            u\parentheses*{0} = u\parentheses*{1} &= 0,
        \end{align*}
        mit \(q\parentheses*{x} = -32\) und \(f\parentheses*{x} = 32\parentheses*{x^3 - x^2} + 6x - 2\).
        Teilen Sie den Bereich \(0 < x < 1\) in \(n\) Bereiche der Länge \(h = \frac{1}{n}\), sodass \(x_i = ih\) für \(i = 1, \ldots, n - 1\) gilt.
        Diskretisieren Sie die Gleichung mit der folgenden finite Differenzen Regel
        \[
            \parentheses*{-\Delta_h u}\parentheses*{x} = \frac{-u\parentheses*{x - h} + 2u\parentheses*{x} - u\parentheses*{x + h}}{h^2}
        \]
        und berechnen Sie, falls die Systemmatrix invertierbar ist, die finite Differenzen Lösungen für die Gleichung für \(n = 3, 4, 5\).
        Skizzieren Sie die Lösungen.
        Andernfalls zeigen Sie, dass die Matrix nicht invertierbar ist.
    \end{problem}

    \subsection*{Lösung}
    \begin{itemize}
        \item Für \(n = 3\) gilt \(h = \frac{1}{3}\).
        Diskretisierte Gleichung:
        \begin{align*}
            3^2\begin{pmatrix}
                2 & -1\\
                -1 & 2
            \end{pmatrix}\begin{pmatrix}
                u_1\\u_2
            \end{pmatrix} + \begin{pmatrix}
                -32 & 0\\
                0 & -32
            \end{pmatrix}\begin{pmatrix}
                u_1\\u_2
            \end{pmatrix} &= \begin{pmatrix}
                -\frac{64}{27}\\-\frac{74}{27}
            \end{pmatrix}\\
            \iff \begin{pmatrix}
                -14 & -9\\
                -9 & -14
            \end{pmatrix}\begin{pmatrix}
                u_1\\u_2
            \end{pmatrix} &= -\frac{2}{27}\begin{pmatrix}
                -\frac{64}{27}\\-\frac{74}{27}
            \end{pmatrix}\\
            \iff \begin{pmatrix}
                u_1\\u_2
            \end{pmatrix} &= \frac{2}{27}\begin{pmatrix}
                1\\2
            \end{pmatrix}
        \end{align*}
        \item Für \(n = 4\) gilt \(h = \frac{1}{4}\).
        Diskretisierte Gleichung:
        \begin{align*}
            4^2\begin{pmatrix}
                2 & -1 & 0\\
                -1 & 2 & -1\\
                0 & -1 & 2
            \end{pmatrix}\begin{pmatrix}
                u_1\\u_2\\u_3
            \end{pmatrix} + \begin{pmatrix}
                -32 & 0 & 0\\
                0 & -32 & 0\\
                0 & 0 & -32
            \end{pmatrix}\begin{pmatrix}
                u_1\\u_2\\u_3
            \end{pmatrix} &= \begin{pmatrix}
                -2\\-3\\-2
            \end{pmatrix}\\
            \iff \begin{pmatrix}
                0 & -16 & 0\\
                -16 & 0 & -16\\
                0 & -16 & 0
            \end{pmatrix}\begin{pmatrix}
                u_1\\u_2\\u_3
            \end{pmatrix} &= \begin{pmatrix}
                -2\\-3\\-2
            \end{pmatrix}
        \end{align*}
        Die Matrix ist singulär und somit kann das System nicht gelöst werden.
        \item Für \(n = 5\) gilt \(h = \frac{1}{5}\).
        Diskretisierte Gleichung:
        \begin{align*}
            4^2\begin{pmatrix}
                2 & -1 & 0 & 0\\
                -1 & 2 & -1 & 0\\
                0 & -1 & 2 & -1\\
                0 & 0 & -1 & 2
            \end{pmatrix}\begin{pmatrix}
                u_1\\u_2\\u_3\\u_4
            \end{pmatrix} + \begin{pmatrix}
                -32 & 0 & 0 & 0\\
                0 & -32 & 0 & 0\\
                0 & 0 & -32 & 0\\
                0 & 0 & 0 & -32
            \end{pmatrix}\begin{pmatrix}
                u_1\\u_2\\u_3\\u_4
            \end{pmatrix} &= \begin{pmatrix}
                -\frac{228}{125}\\-\frac{334}{125}\\-\frac{376}{125}\\-\frac{162}{125}
            \end{pmatrix}\\
            \iff \begin{pmatrix}
                18 & -25 & 0 & 0\\
                -25 & 18 & -25 & 0\\
                0 & -25 & 18 & -25\\
                0 & 0 & -25 & 18
            \end{pmatrix}\begin{pmatrix}
                u_1\\u_2\\u_3\\u_4
            \end{pmatrix} &= \begin{pmatrix}
                -\frac{228}{125}\\-\frac{334}{125}\\-\frac{376}{125}\\-\frac{162}{125}
            \end{pmatrix}\\
            \iff \begin{pmatrix}
                u_1\\u_2\\u_3\\u_4
            \end{pmatrix} &= \frac{2}{125}\begin{pmatrix}
                2\\6\\9\\8
            \end{pmatrix}
        \end{align*}
    \end{itemize}
\end{document}
