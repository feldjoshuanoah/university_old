\documentclass{exercise}

\institute{Applied and Computational Mathematics}
\title{Selbstrechenübung 6}
\author{Joshua Feld, 406718}
\course{Mathematische Grundlagen IV}
\professor{Torrilhon \& Berkels}
\semester{Sommersemester 2022}
\program{CES (Bachelor)}

\begin{document}
    \maketitle


    \section*{Aufgabe 1}
    
    \begin{problem}
        Gegeben sei die Wellengleichung für \(x > 0, t > 0\)
        \[
            u_{tt} = c^2 u_{xx},
        \]
        mit den Rand- bzw. Anfangswertbedingungen
        \begin{align*}
            u\parentheses*{x, 0} &= g\parentheses*{x}, \quad x \ge 0,\\
            \frac{\partial u}{\partial t}\parentheses*{x, 0} &= h\parentheses*{x}, \quad x \ge 0,\\
            u\parentheses*{0, t} = 0, \quad t \ge 0.
        \end{align*}
        Dabei sind \(g\) und \(h\) glatte Funktionen auf \(\left[0, \infty\right)\) und \(g\parentheses*{0} = h\parentheses*{0} = 0\).
        Zeigen Sie, dass die d'Alembert'sche Formel
        \[
            u\parentheses*{t, x} = \frac{1}{2}\parentheses*{g\parentheses*{x + ct} + g\parentheses*{x - ct}} + \frac{1}{2c}\int_{x - ct}^{x + ct}h\parentheses*{\xi}\d\xi
        \]
        eine Lösung dieses Problems ist, wenn man \(g\) und \(h\) ungerade auf der negativen Halbachse fortsetzt (d.h. \(g\parentheses*{-x} = -g\parentheses*{x}\) und \(h\parentheses*{-x} = -h\parentheses*{x}\) für \(x > 0\)).
    \end{problem}
    
    \subsection*{Lösung}


    \section*{Aufgabe 2}
    
    \begin{problem}
        Lösen Sie mit der Methode der Charakteristiken die skalaren Cauchy-Probleme
        \begin{enumerate}
            \item
            \begin{align*}
                u_t + t^\alpha u_x &= 0, \quad \parentheses*{t, x} \in \R^+ \times \R, \alpha > 0,\\
                u\parentheses*{0, x} &= x^2 + 1, \quad x \in \R.
            \end{align*}
            \item
            \begin{align*}
                u_t + xu_x &= x^2, \quad \parentheses*{t, x} \in \R^+ \times \R,\\
                u\parentheses*{0, x} &= x, \quad x \in \R.
            \end{align*}
        \end{enumerate}
    \end{problem}
    
    \subsection*{Lösung}


    \section*{Aufgabe 3}
    
    \begin{problem}
        Wir verwenden folgende Definitionen:
        \begin{itemize}
            \item Eine Matrix \(A = \parentheses*{a_{ij}} \in \R^{n \times n}\) heißt \emph{\(L\)-Matrix}, falls gilt:
            \begin{align*}
                a_{ii} &> 0\text{ für alle }u,\\
                a_{ij} &\le 0\text{ für alle }i \ne j.
            \end{align*}
            \item Eine Matrix heißt \emph{diagonaldominant}, falls für alle \(i = 1, \ldots, n\) gilt:
            \[
                \sum_{j \ne i}\absolute*{a_{ij}} \le \absolute*{a_{ii}}.
            \]
            Eine diagonaldominante Matrix heißt \emph{strikt diagonaldominant}, falls für alle \(i = 1, \ldots, n\) gilt:
            \[
                \sum_{j \ne i}\absolute*{a_{ij}} < \absolute*{a_{ii}}.
            \]
            \item Eine Matrix \(A \in \R^{n \times n}\) heißt \emph{reduzibel}, falls eine Permutationsmatrix \(P \in \R^{n \times n}\) existiert, sodass \(A\) auf folgende Blockgestalt gebracht wird:
            \[
                PAP^T = \begin{pmatrix}
                    A_{11} & A_{12}\\
                    0 & A_{22}
                \end{pmatrix}, \quad A_{11} \in \R^{k \times k}, 1 \le k < n.
            \]
            \item Eine nicht reduzible Matrix nennt man \emph{irreduzibel}.
            \item Eine irreduzible Matrix heißt \emph{irreduzibel diagonaldominant}, falls sie diagonaldominant ist und für mindestens ein \(i = 1, \ldots, n\) gilt:
            \[
                \sum_{j \ne i}\absolute*{a_{ij}} < \absolute*{a_{ii}}.
            \]
            \item Eine invertierbare \(L\)-Matrix mit elementweise nicht-negativem Inversen ist eine \emph{\(M\)-Matrix}.

            \emph{Hinweis: Eine irreduzibel diagonaldominante \(L\)-Matrix ist eine \(M\)-Matrix.}
        \end{itemize}
        Entscheiden Sie mithilfe dieser Definitionen, welche der obigen Eigenschaften die folgenden Matrizen besitzen:
        \begin{align*}
            A_1 &= \begin{pmatrix}
                1 & 0 & 0 & 0\\
                0 & 2 & 0 & 0\\
                0 & 0 & 3 & 0\\
                0 & 0 & 0 & 4
            \end{pmatrix}, & A_2 &= \begin{pmatrix}
                3 & -1 & 1 & -1\\
                -1 & 3 & -1 & 1\\
                1 & -1 & 3 & -1\\
                -1 & 1 & -1 & 3
            \end{pmatrix}, & A_3 &= \begin{pmatrix}
                1 & 0 & 0 & -1\\
                -1 & 2 & -1 & -1\\
                -1 & -1 & 3 & -1\\
                -1 & 0 & 0 & 4
            \end{pmatrix},\\
            A_4 &= \begin{pmatrix}
                1 & 0 & 0 & 0\\
                1 & 1 & 0 & 0\\
                1 & 1 & 1 & 0\\
                1 & 1 & 1 & 1
            \end{pmatrix}, & A_5 &= \begin{pmatrix}
                0 & 0 & 0 & 0\\
                0 & 1 & -\frac{1}{2} & -\frac{1}{3}\\
                0 & -\frac{1}{2} & 2 & -1\\
                0 & -\frac{1}{3} & -1 & 3
            \end{pmatrix}, & A_6 &= \begin{pmatrix}
                2 & -1 & 0 & 0\\
                -1 & 2 & -1 & 0\\
                0 & -1 & 2 & -1\\
                0 & 0 & -1 & 2
            \end{pmatrix}.
        \end{align*}
    \end{problem}
    
    \subsection*{Lösung}
\end{document}
