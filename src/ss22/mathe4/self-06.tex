\documentclass{exercise}

\institute{Applied and Computational Mathematics}
\title{Selbstrechenübung 6}
\author{Joshua Feld, 406718}
\course{Mathematische Grundlagen IV}
\professor{Torrilhon \& Berkels}
\semester{Sommersemester 2022}
\program{CES (Bachelor)}

\begin{document}
    \maketitle


    \section*{Aufgabe 1}
    
    \begin{problem}
        Gegeben sei die Wellengleichung für \(x > 0, t > 0\)
        \[
            u_{tt} = c^2 u_{xx},
        \]
        mit den Rand- bzw. Anfangswertbedingungen
        \begin{align*}
            u\parentheses*{x, 0} &= g\parentheses*{x}, \quad x \ge 0,\\
            \frac{\partial u}{\partial t}\parentheses*{x, 0} &= h\parentheses*{x}, \quad x \ge 0,\\
            u\parentheses*{0, t} = 0, \quad t \ge 0.
        \end{align*}
        Dabei sind \(g\) und \(h\) glatte Funktionen auf \(\left[0, \infty\right)\) und \(g\parentheses*{0} = h\parentheses*{0} = 0\).
        
        Zeigen Sie, dass die d'Alembert'sche Formel
        \[
            u\parentheses*{t, x} = \frac{1}{2}\parentheses*{g\parentheses*{x + ct} + g\parentheses*{x - ct}} + \frac{1}{2c}\int_{x - ct}^{x + ct}h\parentheses*{\xi}\d\xi
        \]
        eine Lösung dieses Problems ist, wenn man \(g\) und \(h\) ungerade auf der negativen Halbachse fortsetzt (d.h. \(g\parentheses*{-x} = -g\parentheses*{x}\) und \(h\parentheses*{-x} = -h\parentheses*{x}\) für \(x > 0\)).
    \end{problem}
    
    \subsection*{Lösung}
    Aus der Vorlesung wissen wir, dass die Lösung der Wellengleichung \(u_{tt} = c^2 u_{xx}\) auf ganz \(\R\) mit den Anfangs- bzw. Randwertbedingungen
    \[
        u\parentheses*{x, 0} = G\parentheses*{x}, \quad \frac{\partial u}{\partial t}\parentheses*{x, 0} = H\parentheses*{x}
    \]
    für \(x \in \R\) über die d'Alembert'sche Formel
    \[
        u\parentheses*{t, x} = \frac{1}{2}\parentheses*{G\parentheses*{x + ct} + G\parentheses*{x - ct}} + \frac{1}{2c}\int_{x - ct}^{x + ct}H\parentheses*{\xi}\d\xi
    \]
    bestimmt werden kann.
    Dieses \(u\) ist also eine Lösung der Wellengleichung auf ganz \(\R\), also auch auf der positiven Halbebene.
    Damit \(u\) auch eine Lösung unseres Problems ist, bleibt zu zeigen, dass für
    \[
        G\parentheses*{x} = \begin{cases}
            g\parentheses*{x}, & \text{falls }x > 0,\\
            0, & \text{falls }x = 0,\\
            -g\parentheses*{-x}, & \text{falls }x < 0,
        \end{cases} \quad H\parentheses*{x} = \begin{cases}
            h\parentheses*{x}, & \text{falls }x > 0,\\
            0, & \text{falls }x = 0,\\
            -h\parentheses*{-x}, & \text{falls }x < 0,
        \end{cases}
    \]
    die Funktion \(u\) die Rand- und Anfangsbedingungen erfüllt.
    Für \(x \ge 0\) ist
    \[
        u\parentheses*{x, 0} = \frac{1}{2}\parentheses*{G\parentheses*{x} + G\parentheses*{x}} + \frac{1}{2c}\int_x^x H\parentheses*{\xi}\d\xi = G\parentheses*{x} = g\parentheses*{x}.
    \]
    Weiterhin ist für \(x \ge 0\) und \(t > 0\)
    \begin{align*}
        \frac{\partial}{\partial t}u\parentheses*{x, t} &= \frac{1}{2}\parentheses*{cG'\parentheses*{x + ct} - cG'\parentheses*{x - ct}} + \frac{1}{2c}\frac{\partial}{\partial t}\int_{x - ct}^{x + ct}H\parentheses*{\xi}\d\xi\\
        &= \frac{c}{2}\parentheses*{G'\parentheses*{x + ct} - G'\parentheses*{x - ct}} + \frac{1}{2}\parentheses*{H\parentheses*{x + ct} + H\parentheses*{x - ct}}
    \end{align*}
    bzw.
    \[
        \frac{\partial}{\partial t}u\parentheses*{x, 0} = \frac{c}{2}\parentheses*{G'\parentheses*{x} - G'\parentheses*{x}} + \frac{1}{2}\parentheses*{H\parentheses*{x} + H\parentheses*{x}} = H\parentheses*{x} = h\parentheses*{x},
    \]
    wobei wir benutzt haben, dass
    \[
        \frac{\d}{\d t}\int_a^{a + t}h\parentheses*{x}\d x = h\parentheses*{a + t}
    \]
    ist, was leicht mithilfe von Differenzenquotienten gezeigt werden kann.
    Für \(x = 0\) gilt
    \[
        u\parentheses*{0, t} = \frac{1}{2}\underbrace{\parentheses*{G\parentheses*{ct} + G\parentheses*{-ct}}}_{= 0\text{, da }G\text{ ungerade}} + \frac{1}{2c}\underbrace{\int_{-ct}^{ct}H\parentheses*{\xi}\d\xi}_{= 0\text{, da }H\text{ ungerade}} = 0.
    \]


    \section*{Aufgabe 2}
    
    \begin{problem}
        Lösen Sie mit der Methode der Charakteristiken die skalaren Cauchy-Probleme
        \begin{enumerate}
            \item
            \begin{align*}
                u_t + t^\alpha u_x &= 0, \quad \parentheses*{t, x} \in \R^+ \times \R, \alpha > 0,\\
                u\parentheses*{0, x} &= x^2 + 1, \quad x \in \R.
            \end{align*}
            \item
            \begin{align*}
                u_t + xu_x &= x^2, \quad \parentheses*{t, x} \in \R^+ \times \R,\\
                u\parentheses*{0, x} &= x, \quad x \in \R.
            \end{align*}
        \end{enumerate}
    \end{problem}
    
    \subsection*{Lösung}
    \begin{enumerate}
        \item Die Charakteristiken dieses Problems sind
        \[
            \parentheses*{t'\parentheses*{s}, x'\parentheses*{s}} = \parentheses*{1, t^\alpha\parentheses*{s}} \quad \text{und} \quad \parentheses*{t\parentheses*{0}, x\parentheses*{0}} = \parentheses*{0, x_0},
        \]
        also
        \[
            t\parentheses*{s} = s\text{ und }x'\parentheses*{s} = s^\alpha \implies x\parentheses*{s} = x_0 + \frac{s^{\alpha + 1}}{\alpha + 1}.
        \]
        Wir nutzen den Ansatz
        \[
            z\parentheses*{s} = u\parentheses*{t\parentheses*{s}, x\parentheses*{s}}
        \]
        mit \(z'\parentheses*{s} = 0\) und \(z\parentheses*{0} = x_0^2 + 1\).
        Daraus folgt dann
        \[
            z\parentheses*{s} = x_0^2 + 1.
        \]
        Mit \(t = s\) und \(x_0 = x - \frac{s^{\alpha + 1}}{\alpha + 1}\) erhalten wir die Lösung
        \[
            u\parentheses*{t, x} = 1 + \parentheses*{x - \frac{t^{\alpha + 1}}{\alpha + 1}}^2.
        \]
        \item Die Charakteristiken dieses Problems sind
        \[
            \parentheses*{t'\parentheses*{s}, x'\parentheses*{s}} = \parentheses*{1, x\parentheses*{s}} \quad \text{und} \quad \parentheses*{t\parentheses*{0}, x\parentheses*{0}} = \parentheses*{0, x_0},
        \]
        also
        \[
            \parentheses*{t\parentheses*{s}, x\parentheses*{s}} = \parentheses*{s, x_0 e^s}.
        \]
        Wir Lösen nun das Anfangswertproblem
        \[
            z'\parentheses*{s} = x\parentheses*{s}^2 = x_0^2 e^{2s}, \quad z\parentheses*{0} = x_0
        \]
        mithilfe von Trennung der Variablen:
        \[
            \underbrace{\int_{x_0}^z 1\d\xi}_{= z - x_0} = \int_0^s x_0^2 e^{2\xi}\d\xi = \frac{x_0^2}{2}\brackets*{e^{2\xi}}_0^s = \frac{x_0^2}{2}\parentheses*{e^{2s} - 1} \implies z\parentheses*{s} = \frac{x_0^2}{2}\parentheses*{e^{2s} - 1} + x_0.
        \]
        Mit \(x_0 = xe^{-s}\) und \(t = s\) folgt
        \[
            u\parentheses*{t, x} = \frac{x^2}{2}e^{-2t}\parentheses*{e^{2t} - 1} + xe^{-t} = xe^{-t} + \frac{x^2}{2}\parentheses*{1 - e^{-2t}}.
        \]
    \end{enumerate}


    \section*{Aufgabe 3}
    
    \begin{problem}
        \begin{itemize}
            \item Eine Matrix \(A = \parentheses*{a_{ij}} \in \R^{n \times n}\) heißt \emph{\(L\)-Matrix}, falls gilt:
            \begin{align*}
                a_{ii} &> 0\text{ für alle }u,\\
                a_{ij} &\le 0\text{ für alle }i \ne j.
            \end{align*}
            \item Eine Matrix heißt \emph{diagonaldominant}, falls für alle \(i = 1, \ldots, n\) gilt:
            \[
                \sum_{j \ne i}\absolute*{a_{ij}} \le \absolute*{a_{ii}}.
            \]
            Eine diagonaldominante Matrix heißt \emph{strikt diagonaldominant}, falls für alle \(i = 1, \ldots, n\) gilt:
            \[
                \sum_{j \ne i}\absolute*{a_{ij}} < \absolute*{a_{ii}}.
            \]
            \item Eine Matrix \(A \in \R^{n \times n}\) heißt \emph{reduzibel}, falls eine Permutationsmatrix \(P \in \R^{n \times n}\) existiert, sodass \(A\) auf folgende Blockgestalt gebracht wird:
            \[
                PAP^T = \begin{pmatrix}
                    A_{11} & A_{12}\\
                    0 & A_{22}
                \end{pmatrix}, \quad A_{11} \in \R^{k \times k}, 1 \le k < n.
            \]
            \item Eine nicht reduzible Matrix nennt man \emph{irreduzibel}.
            \item Eine irreduzible Matrix heißt \emph{irreduzibel diagonaldominant}, falls sie diagonaldominant ist und für mindestens ein \(i = 1, \ldots, n\) gilt:
            \[
                \sum_{j \ne i}\absolute*{a_{ij}} < \absolute*{a_{ii}}.
            \]
            \item Eine invertierbare \(L\)-Matrix mit elementweise nicht-negativem Inversen ist eine \emph{\(M\)-Matrix}.

            \emph{Hinweis: Eine irreduzibel diagonaldominante \(L\)-Matrix ist eine \(M\)-Matrix.}
        \end{itemize}
        Entscheiden Sie mithilfe dieser Definitionen, welche der obigen Eigenschaften die folgenden Matrizen besitzen:
        \begin{align*}
            A_1 &= \begin{pmatrix}
                1 & 0 & 0 & 0\\
                0 & 2 & 0 & 0\\
                0 & 0 & 3 & 0\\
                0 & 0 & 0 & 4
            \end{pmatrix}, & A_2 &= \begin{pmatrix}
                3 & -1 & 1 & -1\\
                -1 & 3 & -1 & 1\\
                1 & -1 & 3 & -1\\
                -1 & 1 & -1 & 3
            \end{pmatrix}, & A_3 &= \begin{pmatrix}
                1 & 0 & 0 & -1\\
                -1 & 2 & -1 & -1\\
                -1 & -1 & 3 & -1\\
                -1 & 0 & 0 & 4
            \end{pmatrix},\\
            A_4 &= \begin{pmatrix}
                1 & 0 & 0 & 0\\
                1 & 1 & 0 & 0\\
                1 & 1 & 1 & 0\\
                1 & 1 & 1 & 1
            \end{pmatrix}, & A_5 &= \begin{pmatrix}
                0 & 0 & 0 & 0\\
                0 & 1 & -\frac{1}{2} & -\frac{1}{3}\\
                0 & -\frac{1}{2} & 2 & -1\\
                0 & -\frac{1}{3} & -1 & 3
            \end{pmatrix}, & A_6 &= \begin{pmatrix}
                2 & -1 & 0 & 0\\
                -1 & 2 & -1 & 0\\
                0 & -1 & 2 & -1\\
                0 & 0 & -1 & 2
            \end{pmatrix}.
        \end{align*}
    \end{problem}
    
    \subsection*{Lösung}
    \begin{center}
        \begin{tabular}{lcccccc}
            \toprule
            & \(A_1\) & \(A_2\) & \(A_3\) & \(A_4\) & \(A_5\) & \(A_6\)\\
            \midrule
            \(L\)-Matrix & \(\times\) & \(\) & \(\times\) & \(\) & \(\) & \(\times\)\\
            diagonaldominant & \(\) & \(\) & \(\) & \(\) & \(\times\) & \(\times\)\\
            strikt diagonaldominant & \(\times\) & \(\) & \(\) & \(\) & \(\) & \(\)\\
            reduzibel & \(\times\) & \(\) & \(\times\) & \(\times\) & \(\times\) & \(\)\\
            irreduzibel & \(\) & \(\times\) & \(\) & \(\) & \(\) & \(\times\)\\
            irreduzibel diagonaldominant & \(\) & \(\) & \(\) & \(\) & \(\) & \(\times\)\\
            \(M\)-Matrix & \(\) & \(\) & \(\) & \(\) & \(\) & \(\times\)\\
            \bottomrule
        \end{tabular}
    \end{center}
\end{document}
