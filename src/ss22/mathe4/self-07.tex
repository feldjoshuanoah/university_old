\documentclass{exercise}

\institute{Applied and Computational Mathematics}
\title{Selbstrechenübung 7}
\author{Joshua Feld, 406718}
\course{Mathematische Grundlagen IV}
\professor{Torrilhon \& Berkels}
\semester{Sommersemester 2022}
\program{CES (Bachelor)}

\begin{document}
    \maketitle


    \section*{Aufgabe 1}
    
    \begin{problem}
        \emph{Laut ADAC bildeten sich auf den deutschen Autobahnen im Jahr 2009 pro Tag Staus mit einer durchschnittlichen Länge von \(1000\sis{\kilo\meter}\).
        Um ein Straßennetzwerk effizient auslasten zu können, muss das Verkehrsaufkommen auf den Straßen möglichst realistisch simuliert werden.
        Ein einfaches eindimensionales Verkehrsfluss-Modell wird in dieser Aufgabe vorgestellt.}

        Die Anzahl der Fahrzeuge pro Länge wird Verkehrsdichte \(\rho\parentheses*{\hat{t}, \hat{x}}\) genannt, die vom Ort \(\hat{x}\) und der Zeit \(\hat{t}\) abhängt.
        Bezeichne \(v\parentheses*{\hat{x}}\) die Geschwindigkeit des Verkehrsflusses an der Stelle \(\hat{x}\) zum Zeitpunkt \(\hat{t}\), so lässt sich aus der Erhaltung der Fahrzeuge (also ohne Zu- und Abfahrten) die Gleichung
        \begin{equation}\label{eq:1}
            \frac{\partial\rho}{\partial\hat{t}} + \frac{\partial\parentheses*{\rho v}}{\partial\hat{x}} = 0
        \end{equation}
        herleiten.
        Diese Gleichung hat allerdings immernoch zwei unbekannte Funktionen und ist daher in dieser Form nicht lösbar.
        \begin{enumerate}
            \item Bei dem Lighthill-Whitham Modell nimmt man den Zusammenhang
            \[
                v\parentheses*{\rho} = v_{\text{max}}\parentheses*{1 - \frac{\rho}{\rho_{\text{max}}}}
            \]
            an, wobei \(v_{\text{max}}\) die Maximalgeschwindigkeit und \(\rho_{\text{max}}\) die Maximaldichte bezeichnen.
            Zeigen Sie durch geeignete Substitutionen, dass \eqref{eq:1} auf die \emph{Burgers'sche Gleichung}
            \[
                \frac{\partial u}{\partial t} + u \cdot \frac{\partial u}{\partial x} = 0
            \]
            umgeformt werden kann.
            Dabei sind \(u\), \(t\) und \(x\) dimensionslose Größen.
            \item Erstellen Sie ein \(t\)-\(x\)-Diagramm für den Verlauf der Charakteristiken der Burgers'schen Gleichung auf \(\Omega = \brackets*{-5, 5} \times \brackets*{0, 2}\) mit dem Anfangswert
            \[
                u\parentheses*{x, 0} = 1 + e^{-x^2}.
            \]
            Was bedeutet en Schnittpunkt zweier Charakteristiken für die Lösung?
        \end{enumerate}
    \end{problem}

    \subsection*{Lösung}
    \begin{enumerate}
        \item
        \item
    \end{enumerate}


    \section*{Aufgabe 2}
    
    \begin{problem}
        Wir betrachten das Poisson-Problem:
        \begin{align*}
            -\Delta u\parentheses*{x, y} &= f\parentheses*{x, y}, \quad \parentheses*{x, y} \in \Omega = \parentheses*{0, 1}^2,\\
            u\parentheses*{x, y} &= g\parentheses*{x, y}, \quad \parentheses*{x, y} \in \partial\Omega.
        \end{align*}
        Bestimmen Sie, ob für diese Probleme die Konvergenzaussagen aus der Vorlesung erfüllt sind:
        \begin{enumerate}
            \item Problem 1:
            \begin{align*}
                u\parentheses*{x, y} &= \parentheses*{1 - \parentheses*{2x - 1}^4}\parentheses*{\parentheses*{2y - 1}^2 - 1},\\
                f\parentheses*{x, y} &= 8 \cdot \parentheses*{24 \cdot \parentheses*{1 - 2x}^2\parentheses*{y - 1}y + \parentheses*{1 - 2x}^4 - 1},\\
                g\parentheses*{x, y} &= u\parentheses*{x, y}.
            \end{align*}
            \item Problem 2:
            \begin{align*}
                u\parentheses*{x, y} &= y^4\parentheses*{\frac{1}{6}x^3\log\parentheses*{x} - \frac{11x^3}{36}},\\
                f\parentheses*{x, y} &= \frac{1}{3}xy^2 \cdot \parentheses*{-3 \cdot \parentheses*{2x^2 + y^2}\log\parentheses*{x} + 11x^2 + 3y^2},\\
                g\parentheses*{x, y} &= u\parentheses*{x, y}.
            \end{align*}
        \end{enumerate}
    \end{problem}
    
    \subsection*{Lösung}
    \begin{enumerate}
        \item
        \item 
    \end{enumerate}


    \section*{Aufgabe 3}
    
    \begin{problem}
        Diskretisieren Sie das Konvektionsproblem
        \begin{align*}
            u'\parentheses*{x} &= f\parentheses*{x}, \quad x \in \Omega = \parentheses*{0, 1}, u \in C^2\parentheses*{\Omega},\\
            u\parentheses*{0} &= 0
        \end{align*}
        mit der einseitigen finiten Differenz
        \[
            u'\parentheses*{x} \approx \frac{u\parentheses*{x} - u\parentheses*{x - h}}{h}, \quad h = \frac{1}{n}, n \in \N
        \]
        und stellen Sie das zugehörige Gleichungssystem \(Au_h = b\) auf.
        Bestimmen Sie \(A^{-1}\) und \(\norm*{A^{-1}}_\infty\).
    \end{problem}
    
    \subsection*{Lösung}
\end{document}
