\documentclass{exercise}

\institute{Applied and Computational Mathematics}
\title{Selbstrechenübung 8}
\author{Joshua Feld, 406718}
\course{Mathematische Grundlagen IV}
\professor{Torrilhon \& Berkels}
\semester{Sommersemester 2022}
\program{CES (Bachelor)}

\begin{document}
    \maketitle


    \section*{Aufgabe 1}
    
    \begin{problem}
        Gegeben sei das Gebiet \(\Omega = \parentheses*{0, L_x} \times \parentheses*{0, L_x} \subset \R^2\).
        Bestimmen Sie die Eigenwerte \(\lambda\) und die zugehörigen Eigenfunktionen \(u\parentheses*{x, y}\) des Laplace-Operators auf den Gebiet \(\Omega\) mit \(u\parentheses*{x, y} = 0\) auf \(\partial\Omega\), d.h. gesucht sind alle \(u \in C^2\parentheses*{\Omega} \ne 0\) und \(\lambda \in \R\) mit
        \begin{align}
            -\Delta u\parentheses*{x, y} &= \lambda u\parentheses*{x, y},\label{eq:1}\\
            u\parentheses*{x, y} &= 0, \quad \text{auf }\partial\Omega.\nonumber
        \end{align}
        Gehen Sie dabei wie folgt vor:
        \begin{enumerate}
            \item Lösen Sie zuerst das eindimensionale Eigenwertproblem: Bestimmen Sie alle \(\mu \in \R\) für die eine Lösung \(f \in C^2\parentheses*{\parentheses*{0, L}} \ne 0\) der Gleichung
            \begin{equation}\label{eq:2}
                -f''\parentheses*{x} = \mu f\parentheses*{x}
            \end{equation}
            mit \(f\parentheses*{0} = f\parentheses*{L} = 0\) existiert.
            Wie sieht die zu \(\mu\) gehörige Eigenfunktion \(f\) aus?
            \item Reduzieren Sie jetzt Gleichung \eqref{eq:1} mithilfe des Ansatzes
            \[
                u\parentheses*{x, y} = v\parentheses*{x}w\parentheses*{y}
            \]
            auf zwei Gleichungen vom Typ \eqref{eq:2}.
            Bestimmen Sie mithilfe der Ergebnisse aus a) die Werte \(\lambda \in \R\), für die eine Lösung \(u\parentheses*{x, y}\) der Gleichung \eqref{eq:1} existiert.
            Wie sehen diese Lösungen aus?
            Wie groß ist der kleinste Eigenwert?
        \end{enumerate}
    \end{problem}
    
    \subsection*{Lösung}
    \begin{enumerate}
        \item Das eindimensionale Problem
        \[
            -f''\parentheses*{x} = \mu f\parentheses*{x}, \quad f\parentheses*{0} = f\parentheses*{L} = 0
        \]
        behandeln wir im Folgenden mit einer Fallunterscheidung nach dem Vorzeichen von \(\mu\).
        \begin{itemize}
            \item Fall 1: \(\mu = 0\): Also ist \(f''\parentheses*{x} = 0\) und damit \(f\parentheses*{x} = ax + b\).
            Die Randbedingungen implizieren \(f\parentheses*{x} = 0\).
            Damit gibt es keine Eigenfunktionen.
            \item Fall 2: \(\mu < 0\): Sei \(\mu = -c^2\) mit \(c \ne 0\).
            Die Gleichung
            \[
                f''\parentheses*{x} = c^2 f\parentheses*{x}
            \]
            ist eine gewöhnliche DGL mit Lösung
            \[
                f\parentheses*{x} = ae^{cx} + be^{-cx}.
            \]
            Aus den Randbedingungen bestimmen wir die Koeffizienten \(a\) und \(b\):
            \begin{align*}
                0 = f\parentheses*{0} &= a + b,\\
                0 = f\parentheses*{L} &= ae^{cL} + be^{-cL}\\
                \implies a = b &= 0.
            \end{align*}
            Damit gibt es keine Eigenfunktion.
            \item Fall 3: \(\mu > 0\): Sei \(\mu = c^2\) mit \(c \ne 0\).
            Die Gleichung
            \[
                f''\parentheses*{x} = -c^2 f\parentheses*{x}
            \]
            ist eine gewöhnliche DGL mit Lösung
            \[
                f\parentheses*{x} = a\sin\parentheses*{cx} + b\cos\parentheses*{cx}.
            \]
            Aus den Randbedingungen bestimmen wir die Koeffizienten \(a\) und \(b\):
            \begin{align*}
                0 = f\parentheses*{0} &= a\sin\parentheses*{0} + b\cos\parentheses*{0} = b,\\
                0 = f\parentheses*{L} &= a\sin\parentheses*{cL}.
            \end{align*}
            Damit wir eine Eigenfunktion erhalten, muss \(a \ne 0\) sein.
            Um die Randbedingungen bei \(x = L\) erfüllen zu können, muss
            \[
                c = \frac{n\pi}{L}, \quad n \in \N
            \]
            gelten.
            Damit gilt für die Eigenwerte \(\mu_n = \parentheses*{\frac{n\pi}{L}}^2, n \in \N\) und für die zugehörigen Eigenfunktionen
            \[
                f_n\parentheses*{x} = k\sin\parentheses*{\frac{n\pi}{L}x}, \quad n \in \N, k \in \R \setminus \braces*{0}.
            \]
        \end{itemize}
        \item Für das zweidimensionale Problem
        \[
            -\Delta u\parentheses*{x, y} = \lambda u\parentheses*{x, y}
        \]
        mit dem Produktansatz \(u\parentheses*{x, y} = v\parentheses*{x}w\parentheses*{y}\) lautet die Gleichung
        \[
            -w\parentheses*{y}\partial_x^2 v\parentheses*{x} - v\parentheses*{x}\partial_y^2 w\parentheses*{y} = \lambda v\parentheses*{x}w\parentheses*{y}.
        \]
        Falls die rechte Seite von \(0\) verschieden ist, ist dies äquivalent zu
        \[
            -\frac{\partial_x^2 v\parentheses*{x}}{v\parentheses*{x}} - \frac{\partial_y^2 w\parentheses*{y}}{w\parentheses*{y}} = \lambda.
        \]
        Da die rechte Seite konstant ist und die Summanden jeweils einzeln nur von \(x\) bzw. \(y\) abhängen, ist die linke Seite unabhängig von \(x\) bzw. \(y\).
        Daraus folgt
        \[
            -\partial_x^2 v\parentheses*{x} = \lambda_x v\parentheses*{x}, \quad -\partial_y^2 w\parentheses*{y} = \lambda_y w\parentheses*{y}
        \]
        mit \(\lambda_x, \lambda_y \in \R\) konstant.
        Aus Teil a) wissen wir, dass mögliche Werte \(\lambda_x\) bzw. \(\lambda_y\) gegeben sind durch
        \[
            \lambda_x = \parentheses*{\frac{n\pi}{L_x}}^2, \quad \lambda_y = \parentheses*{\frac{m\pi}{L_y}}^2, \quad n, m \in \N
        \]
        und damit gilt für die zugehörigen Eigenfunktionen
        \[
            v_n\parentheses*{x} = a\sin\parentheses*{\frac{n\pi}{L_x}x}, \quad w_m\parentheses*{y} = b\sin\parentheses*{\frac{m\pi}{L_y}y}, \quad n, m \in \N, a, b \in \R \setminus \braces*{0}.
        \]
        Damit sind mögliche Eigenwerte gegeben zu
        \[
            \lambda = \lambda_x + \lambda_y = \parentheses*{\frac{n\pi}{L_x}}^2 + \parentheses*{\frac{m\pi}{L_y}}^2, \quad n, m \in \N
        \]
        mit Eigenfunktionen
        \[
            u_{nm}\parentheses*{x, y} = a\sin\parentheses*{\frac{n\pi}{L_x}x}\sin\parentheses*{\frac{m\pi}{L_y}y}, \quad n, m \in \N, a \in \R \setminus \braces*{0}.
        \]
        Den kleinsten Eigenwert erhält man für \(n = m = 1\), unabhängig davon, wie das Verhältnis \(\frac{L_x}{L_y}\) ist.
        Die Anordnung der übrigen Eigenwerte hängt von diesem Verhältnis ab.
    \end{enumerate}


    \section*{Aufgabe 2}
    
    \begin{problem}
        \begin{enumerate}
            \item Zeigen Sie, dass
            \[
                \phi_{j, k}\parentheses*{x, y} = \cos\parentheses*{j\pi x}\sin\parentheses*{k\pi y}, \quad j = 0, 1, 2, \ldots, k = 1, 2, \ldots
            \]
            die Eigenfunktionen des Laplace-Operators auf \(\Omega = \brackets*{0, 1}^2\) sind.
            Wie sind die Eigenwerte?
            \item Entwickeln Sie die Funktion
            \[
                f\parentheses*{x, y} = \frac{1}{2}\parentheses*{\sin\parentheses*{4\pi y} - \cos\parentheses*{4\pi x}\sin\parentheses*{10\pi y}}
            \]
            in den Eigenfunktionen \(\phi_{j, k}\).
            \item Geben Sie die Lösung des Anfangsrandwertproblems and
            \begin{align*}
                \partial_t u\parentheses*{t, x, y} &= \Delta u\parentheses*{t, x, y} + f\parentheses*{x, y}, \quad \text{in }\Omega,\\
                u\parentheses*{t, x, y} &= 0, \quad y \in \braces*{0, 1}, x \in \parentheses*{0, 1},\\
                \partial_x u\parentheses*{t, x, y} &= 0, \quad x \in \braces*{0, 1}, y \in \parentheses*{0, 1},\\
                u\parentheses*{0, x, y} &= 0.
            \end{align*}
        \end{enumerate}
    \end{problem}
    
    \subsection*{Lösung}
    \begin{enumerate}
        \item Beachte, dass \(\Delta\phi_{j, k} = -\pi^2 \parentheses*{j^2 + k^2}\phi_{j, k}\) gilt.
        Daher sind \(\phi_{j, k}\) die Eigenfunktionen von \(\Delta\).
        Die Eigenwerte sind hierbei \(\lambda_{j, k} = -\pi^2 \parentheses*{j^2 + k^2}\).
        \item Die Basisfunktionen \(\phi_{j, k}\) sind orthogonal und unabhängig auf \(\Omega\) aber nicht normiert.
        So können wir die Funktion \(f\parentheses*{x, y}\) als eine lineare Kombination von \(\phi_{j, k}\) eindeutig darstellen:
        \[
            f\parentheses*{x, y} = \frac{1}{2}\cos\parentheses*{0 \cdot \pi x}\sin\parentheses*{4\pi y} - \frac{1}{2}\cos\parentheses*{4\pi x}\sin\parentheses*{10\pi y} = \frac{1}{2}\phi_{0, 4} - \frac{1}{2}\phi_{4, 10} = \sum_{j, k}\alpha_{j, k}\phi_{j, k}
        \]
        mit
        \[
            \alpha_{j, k} = \begin{cases}
                \frac{1}{2}, & \text{falls }\parentheses*{j, k} = \parentheses*{0, 4},\\
                -\frac{1}{2}, & \text{falls }\parentheses*{j, k} = \parentheses*{4, 10},\\
                0, & \text{sonst}.
            \end{cases}
        \]
        \item Wir suchen nach einer Lösung in der Form
        \[
            u\parentheses*{t, x, y} = \sum_{j, k}\beta_{j, k}\parentheses*{t}\phi_{j, k}\parentheses*{x, y},
        \]
        sodass die PDE sich zu
        \[
            \sum_{k, j}\parentheses*{\beta_{j, k}'\parentheses*{t} - \lambda_{j, k}\beta_{j, k} - \alpha_{j, k}}\phi_{j, k} = 0
        \]
        vereinfacht, und für alle \(k, j\) erhalten wir
        \[
            \beta_{j, k}'\parentheses*{t} - \lambda_{j, k}\beta_{j, k} - \alpha_{j, k} = 0.
        \]
        Die Lösung davin ist bekannt:
        \[
            \beta_{j, k}\parentheses*{t} = C_{j, k}\exp\parentheses*{\lambda_{j, k}t} - \frac{\alpha_{j, k}}{\lambda_{j, k}}.
        \]
        Weil \(u\parentheses*{0, x, y} = 0\) gilt, folgt \(\beta_{j, k}\parentheses*{0} = 0\) oder \(C_{j, k} = \frac{\alpha_{j, k}}{\lambda_{j, k}}\).
        Daher gilt
        \[
            \beta_{j, k}\parentheses*{t} = \frac{\alpha_{j, k}}{\lambda_{j, k}}\parentheses*{\exp\parentheses*{\lambda_{j, k}t} - 1}.
        \]
        Zusammenfassend lautet die Lösung
        \[
            u\parentheses*{t, x, y} = -\frac{1}{32\pi^2}\parentheses*{e^{-16\pi^2 t} - 1}\sin\parentheses*{4\pi y} + \frac{1}{232\pi^2}\parentheses*{e^{-116\pi^2 t} - 1}\cos\parentheses*{4\pi x}\sin\parentheses*{10\pi y}.
        \]
    \end{enumerate}
\end{document}
