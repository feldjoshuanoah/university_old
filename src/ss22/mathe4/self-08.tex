\documentclass{exercise}

\institute{Applied and Computational Mathematics}
\title{Selbstrechenübung 8}
\author{Joshua Feld, 406718}
\course{Mathematische Grundlagen IV}
\professor{Torrilhon \& Berkels}
\semester{Sommersemester 2022}
\program{CES (Bachelor)}

\begin{document}
    \maketitle


    \section*{Aufgabe 1}
    
    \begin{problem}
        Gegeben sei das Gebiet \(\Omega = \parentheses*{0, L_x} \times \parentheses*{0, L_x} \subset \R^2\).
        Bestimmen Sie die Eigenwerte \(\lambda\) und die zugehörigen Eigenfunktionen \(u\parentheses*{x, y}\) des Laplace-Operators auf den Gebiet \(\Omega\) mit \(u\parentheses*{x, y} = 0\) auf \(\partial\Omega\), d.h. gesucht sind alle \(u \in C^2\parentheses*{\Omega} \ne 0\) und \(\lambda \in \R\) mit
        \begin{align}
            -\Delta u\parentheses*{x, y} &= \lambda u\parentheses*{x, y},\label{eq:1}\\
            u\parentheses*{x, y} &= 0, \quad \text{auf }\partial\Omega.\nonumber
        \end{align}
        Gehen Sie dabei wie folgt vor:
        \begin{enumerate}
            \item Lösen Sie zuerst das eindimensionale Eigenwertproblem: Bestimmen Sie alle \(\mu \in \R\) für die eine Lösung \(f \in C^2\parentheses*{\parentheses*{0, L}} \ne 0\) der Gleichung
            \begin{equation}\label{eq:2}
                -f''\parentheses*{x} = \mu f\parentheses*{x}
            \end{equation}
            mit \(f\parentheses*{0} = f\parentheses*{L} = 0\) existiert.
            Wie sieht die zu \(\mu\) gehörige Eigenfunktion \(f\) aus?
            \item Reduzieren Sie jetzt Gleichung \eqref{eq:1} mithilfe des Ansatzes
            \[
                u\parentheses*{x, y} = v\parentheses*{x}w\parentheses*{y}
            \]
            auf zwei Gleichungen vom Typ \eqref{eq:2}.
            Bestimmen Sie mithilfe der Ergebnisse aus a) die Werte \(\lambda \in \R\), für die eine Lösung \(u\parentheses*{x, y}\) der Gleichung \eqref{eq:1} existiert.
            Wie sehen diese Lösungen aus?
            Wie groß ist der kleinste Eigenwert?
        \end{enumerate}
    \end{problem}
    
    \subsection*{Lösung}
    \begin{enumerate}
        \item
        \item
    \end{enumerate}


    \section*{Aufgabe 2}
    
    \begin{problem}
        \begin{enumerate}
            \item Zeigen Sie, dass
            \[
                \phi_{j, k}\parentheses*{x, y} = \cos\parentheses*{j\pi x}\sin\parentheses*{k\pi y}, \quad j = 0, 1, 2, \ldots, k = 1, 2, \ldots
            \]
            die Eigenfunktionen des Laplace-Operators auf \(\Omega = \brackets*{0, 1}^2\) sind.
            Wie sind die Eigenwerte?
            \item Entwickeln Sie die Funktion
            \[
                f\parentheses*{x, y} = \frac{1}{2}\parentheses*{\sin\parentheses*{4\pi y} - \cos\parentheses*{4\pi x}\sin\parentheses*{10\pi y}}
            \]
            in den Eigenfunktionen \(\phi_{j, k}\).
            \item Geben Sie die Lösung des Anfangsrandwertproblems and
            \begin{align*}
                \partial_t u\parentheses*{t, x, y} &= \Delta u\parentheses*{t, x, y} + f\parentheses*{x, y}, \quad \text{in }\Omega,\\
                u\parentheses*{t, x, y} &= 0, \quad y \in \braces*{0, 1}, x \in \parentheses*{0, 1},\\
                \partial_x u\parentheses*{t, x, y} &= 0, \quad x \in \braces*{0, 1}, y \in \parentheses*{0, 1},\\
                u\parentheses*{0, x, y} &= 0.
            \end{align*}
        \end{enumerate}
    \end{problem}
    
    \subsection*{Lösung}
    \begin{enumerate}
        \item
        \item
        \item
    \end{enumerate}
\end{document}
