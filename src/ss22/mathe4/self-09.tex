\documentclass{exercise}

\institute{Applied and Computational Mathematics}
\title{Selbstrechenübung 9}
\author{Joshua Feld, 406718}
\course{Mathematische Grundlagen IV}
\professor{Torrilhon \& Berkels}
\semester{Sommersemester 2022}
\program{CES (Bachelor)}

\begin{document}
    \maketitle


    \section*{Aufgabe 1}
    
    \begin{problem}
        Gegeben ist das Konvektions-Diffusionsproblem: Gesucht ist \(u \in C^2\parentheses*{0, 1}\) mit
        \begin{align*}
            -u''\parentheses*{x} + 7u'\parentheses*{x} &= f\parentheses*{x}, \quad x \in \parentheses*{0, 1},\\
            u\parentheses*{0} = u\parentheses*{1} &= 0.
        \end{align*}
        Dieses Problem soll mithilfe einer finite Differenzen Methode auf einem regelmäßigen Gitter der Schrittweite \(h\) und den Gitterpunkten \(0 = x_0 < x_1 < \cdots < x_n = 1\) approximiert und in ein lineares Gleichungssystem der Form \(A_h u_h = b_h\) überführt werden.
        Dazu sollen die Differenzenquotienten
        \begin{align*}
            u'\parentheses*{x_i} &\approx \frac{u\parentheses*{x_{i + 1}} - u\parentheses*{x_i}}{h}\\
            u''\parentheses*{x_i} &\approx \frac{u\parentheses*{x_{i + 1}} - 2u\parentheses*{x_i} + u\parentheses*{x_{i - 1}}}{h^2}
        \end{align*}
        benutzt werden.
        \begin{enumerate}
            \item Bestimmen Sie \(A_h\) und \(b_h\).
            \item Geben Sie eine Bedingung an, unter der \(A_h\) diagonaldominant ist.
            \item Geben Sie eine mögliche finite Differenzen Diskretisierung für die gegebene Differentialgleichung an, sodass die resultierende Matrix \(A_h\) \emph{für alle} \(h > 0\) strikt diagonaldominant ist.
        \end{enumerate}
        \emph{Hinweis: Eine Matrix \(A\) ist diagonaldominant, wenn
        \[
            \sum_{i \ne j}\absolute*{a_{ij}} \le \absolute*{a_{ii}} \quad \forall i.
        \]
        Eine Matrix \(A\) ist strikt diagonaldominant, wenn sie diagonaldominant ist und für mindestens ein \(i = k\)
        \[
            \sum_{k \ne j}\absolute*{a_{kj}} < \absolute*{a_{kk}}.
        \]}
    \end{problem}
    
    \subsection*{Lösung}
    \begin{enumerate}
        \item Sei \(h = \frac{1}{n}\) und \(x_i = ih\) für \(i = 0, \ldots, n\).
        Weiter sei \(u_i = u\parentheses*{x_i}\).
        Dann gilt für \(i = 2, \ldots, n - 2\)
        \begin{align*}
            -\frac{u_{i + 1} - 2u_i + u_{i - 1}}{h^2} + 7 \cdot \frac{u_{i + 1} - u_i}{h} &= f\parentheses*{ih}\\
            \implies \frac{1}{h^2}\parentheses*{\parentheses*{-1 + 7h}u_{i + 1} + \parentheses*{2 - 7h}u_i - u_{i - 1}} &= f\parentheses*{ih}.
        \end{align*}
        Da \(u_0 = u\parentheses*{0} = 0\) und \(u_n = u\parentheses*{1} = 0\), gilt für \(i = 1\) bzw. \(i = n - 1\)
        \begin{align*}
            \frac{1}{h^2}\parentheses*{\parentheses*{-1 + 7h}u_2 + \parentheses*{2 - 7h}u_1} &= f\parentheses*{h},\\
            \frac{1}{h^2}\parentheses*{\parentheses*{2 - 7h}u_{n - 1} - u_{n - 2}} &= f\parentheses*{\parentheses*{n - 1}h}.
        \end{align*}
        Insgesamt erhalten wir das Gleichungssystem \(A_h u_h = b_h\) mit
        \[
            A_h = \frac{1}{h^2}\begin{pmatrix}
                2 - 7h & -1 + 7h\\
                -1 & 2 - 7h & -1 + 7h\\
                & \ddots & \ddots & \ddots\\
                & & -1 & 2 - 7h & -1 + 7h\\
                & & & -1 & 2 - 7h
            \end{pmatrix}, \quad b_h = \begin{pmatrix}
                f\parentheses*{h}\\
                f\parentheses*{2h}\\
                \vdots\\
                f\parentheses*{\parentheses*{n - 2}h}\\
                f\parentheses*{\parentheses*{n - 1}h}
            \end{pmatrix}, \quad u_h = \begin{pmatrix}
                u_1\\
                u_2\\
                \vdots\\
                u_{n - 2}\\
                u_{n - 1}
            \end{pmatrix}.
        \]
        \item Für Diagonaldominanz müssen die folgenden Bedingungen gelten:
        \begin{align}
            \absolute*{2 - 7h} &\ge \absolute*{1 - 7h},\label{eq:1}\\
            \absolute*{2 - 7h} &\ge 1 + \absolute*{1 - 7h},\label{eq:2}\\
            \absolute*{2 - 7h} &\ge 1.\label{eq:3}
        \end{align}
        Diese drei Ungleichungen können wir nun unabhängig voneinander lösen.
        
        Wir betrachten zunächst Ungleichung \eqref{eq:1}. Sei zunächst \(h \in \parentheses*{0, \frac{1}{7}}\), dann gilt
        \[
            2 - 7h \ge 1 - 7h \implies 2 \ge 1,
        \]
        was stets gilt.
        Sei nun \(h \in \parentheses*{\frac{1}{7}, \frac{2}{7}}\).
        Dies liefert
        \[
            2 - 7h \ge 7h - 1 \implies h \in \parentheses*{\frac{1}{7}, \frac{3}{14}}.
        \]
        Zuletzt sei \(h \in \parentheses*{\frac{2}{7}, \infty}\), womit folgt
        \[
            7h - 2 \ge 7h - 1 \implies -2 \ge -1,
        \]
        was niemals gelten kann.
        Somit ergibt sich für \eqref{eq:1}
        \[
            h \in \parentheses*{0, \frac{3}{14}}.
        \]

        Nun zu Ungleichung \eqref{eq:2}: Sei zunächst wieder \(h \in \parentheses*{0, \frac{1}{7}}\), dann gilt
        \[
            2 - 7h \ge 2 - 7h \implies h \in \parentheses*{0, \frac{1}{7}}.
        \]
        Sei nun \(h \in \parentheses*{\frac{1}{7}, \frac{2}{7}}\).
        Dies liefert
        \[
            2 - 7h \ge 7h \implies h \in \parentheses*{0, \frac{1}{7}},
        \]
        was nicht in der gegebenen Spanne liegt.
        Zuletzt sei \(h \in \parentheses*{\frac{2}{7}, \infty}\), womit folgt
        \[
            7h - 2 \ge 7h,
        \]
        was niemals gelten kann.
        Somit ergibt sich für \eqref{eq:2}
        \[
            h \in \parentheses*{0, \frac{1}{7}}.
        \]

        Führen wir die selben Schritte für \eqref{eq:3} durch, so erhalten wir
        \[
            h \in \parentheses*{0, \frac{1}{7}} \cup \parentheses*{\frac{3}{7}, \infty}.
        \]
        Das finale Ergebnis ist die Schnittmenge der Ergebnisse für \eqref{eq:1} - \eqref{eq:3}, also
        \[
            h \in \parentheses*{0, \frac{1}{7}}.
        \]
        \item Wird für die Approximation von \(u'\parentheses*{x_i}\) anstatt der Downwind-Diskretisierung, die Upwind-Diskretisierung
        \[
            u'\parentheses*{x_i} \approx \frac{u\parentheses*{x_i} - u\parentheses*{x_{i - 1}}}{h}
        \]
        benutzt, so erhalten wir die Matrix
        \[
            A_h = \frac{1}{h^2}\begin{pmatrix}
                2 + 7h & -1\\
                -1 - 7h & 2 + 7h & -1\\
                & \ddots & \ddots & \ddots\\
                & & -1 - 7h & 2 + 7h & -1\\
                & & & -1 - 7h & 2 + 7h
            \end{pmatrix}.
        \]
        Mit dieser Diskretisierung ist \(A_h\) eine schwach diagonaldominante Matrix, da
        \[
            \sum_{j \ne i}\absolute*{a_{i, j}} \le 2 + 7h = a_{i, i}, \quad i = 1, \ldots, n - 1.
        \]
        Für strikte Diagonaldominanz ergibt sich das Folgende für die erste Zeile der Matrix:
        \[
            \sum_{j \ne 1}\absolute*{a_{1, j}} = 1 < 2 + 7h.
        \]
    \end{enumerate}


    \section*{Aufgabe 2}
    
    \begin{problem}
        Gegeben seien die Funktionen \(h, f_\alpha \in \mathcal{D}\parentheses*{\R^n}, \alpha > 0\) definiert durch
        \[
            h\parentheses*{x} := \begin{cases}
                0, & \text{falls }\absolute*{x} \ge 1,\\
                \exp\parentheses*{-\frac{1}{1 - \absolute*{x}^2}}, & \text{falls }\absolute*{x} < 1
            \end{cases}
        \]
        und
        \[
            f_\alpha\parentheses*{x} := \frac{1}{M\alpha^n}h\parentheses*{\frac{x}{\alpha}}, \quad M := \int_{\R^n}h\parentheses*{x}\d x.
        \]
        Zeigen Sie, dass für die reguläre Distribution
        \[
            T_{f_\alpha}\parentheses*{\phi} := \int_{\R^n}f_\alpha\parentheses*{x}\phi\parentheses*{x}\d x, \quad \phi \in \mathcal{D}\parentheses*{\R^n},
        \]
        der Grenzwert \(\lim_{\alpha \to 0}T_{f_\alpha} = \delta\) im distributionellen Sinn erfüllt ist.
    \end{problem}
    
    \subsection*{Lösung}
    Es ist
    \[
        \lim_{\alpha \to 0}T_{f_\alpha}\parentheses*{\phi} = \delta\parentheses*{\phi} = \phi\parentheses*{0}
    \]
    für \(\phi \in \mathcal{D}\) zu zeigen.
    Es ist
    \begin{align*}
        \absolute*{\phi\parentheses*{0} - T_{f_\alpha}\parentheses*{\phi}} &= \absolute*{\phi\parentheses*{0} - \int_{\R^n}f_\alpha\parentheses*{x}\phi\parentheses*{x}\d x}\\
        &= \absolute*{\int_{\R^n}f_\alpha\parentheses*{x}\parentheses*{\phi\parentheses*{0} - \phi\parentheses*{x}}\d x}\\
        &\le \norm*{\phi\parentheses*{0} - \phi}_{\infty, B_\alpha\parentheses*{0}}\underbrace{\int_{\R^n}f_\alpha\parentheses*{x}\d x}_{= 1} \xrightarrow{\alpha \to 0^+} 0,
    \end{align*}
    da \(\phi\) stetig ist.
\end{document}
