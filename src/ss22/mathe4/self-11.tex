\documentclass{exercise}

\institute{Applied and Computational Mathematics}
\title{Selbstechenübung 10}
\author{Joshua Feld, 406718}
\course{Mathematische Grundlagen IV}
\professor{Torrilhon \& Berkels}
\semester{Sommersemester 2022}
\program{CES (Bachelor)}

\begin{document}
    \maketitle


    \section*{Aufgabe 1}
    
    \begin{problem}
        Bestimmen Sie die distributionelle Ableitung der folgenden Distributionen für \(\phi \in \mathcal{D}\parentheses*{\R}\):
        \begin{enumerate}
            \item \(T_f \phi = \angles*{f, \phi}\), wobei \(f\parentheses*{x} := \begin{cases}
                0, & \text{falls }x < 0,\\
                x + 1, & \text{falls }x \ge 0,
            \end{cases}\)
            \item \(T_g \phi = -\phi\parentheses*{-1} + 2\phi\parentheses*{0} - \phi\parentheses*{1}\).
        \end{enumerate}
    \end{problem}
    
    \subsection*{Lösung}
    \begin{enumerate}
        \item Für die distributionelle Ableitung \(T_f'\) gilt
        \begin{align*}
            T_f'\phi &= \angles*{f', \phi}\\
            &= -\angles*{f, \phi'}\\
            &= -T_f \phi'\\
            &= -\int_\R f\parentheses*{x}\phi'\parentheses*{x}\d x\\
            &= -\int_0^\infty \parentheses*{x + 1}\phi'\parentheses*{x}\d x\\
            &= -\int_0^\infty \phi'\parentheses*{x}\d x - \int_0^\infty x\phi'\parentheses*{x}\d x\\
            &= \brackets*{-\phi\parentheses*{x}}_0^\infty - \int_0^\infty x\phi'\parentheses*{x}\d x\\
            &= \phi\parentheses*{0} - \int_0^\infty x\phi'\parentheses*{x}\d x\\
            &= \phi\parentheses*{0} - \parentheses*{\brackets*{x\phi\parentheses*{x}}_0^\infty - \int_0^\infty \phi\parentheses*{x}\d x}\\
            &= \phi\parentheses*{0} - 0 + \int_\R H\parentheses*{x}\phi\parentheses*{x}\d x\\
            &= \angles*{\delta_0, \phi} + \angles*{H, \phi},
        \end{align*}
        mit der Heaviside-Funktion \(H\parentheses*{x}\).
        \item Für die distributionelle Ableitung \(T_g'\) gilt
        \[
            T_g \phi = -\phi\parentheses*{-1} + 2\phi\parentheses*{0} - \phi\parentheses*{1} \implies g\parentheses*{x} = -\delta_{-1} + 2\delta_0 - \delta_1
        \]
        und damit folgt
        \[
            T_g'\phi = -T_g \phi' = \phi'\parentheses*{-1} - 2\phi'\parentheses*{0} + \phi'\parentheses*{1} = \angles*{\delta_{-1}, \phi'} - 2\angles*{\delta_0, \phi'} + \angles*{\delta_1, \phi'} = \angles*{\delta_{-1} - 2\delta_0 + \delta_1, \phi'}.
        \]
    \end{enumerate}


    \section*{Aufgabe 2}
    
    \begin{problem}
        Zeigen Sie, dass für \(a \in \R\)
        \[
            E\parentheses*{x} := e^{-ax}H\parentheses*{x}, \quad x \in \R
        \]
        mit der Heaviside-Funktion
        \[
            H\parentheses*{x} := \begin{cases}
                1, & \text{falls }x > 0,\\
                0, & \text{falls }x \le 0
            \end{cases}
        \]
        eine Fundamentallösung für den Differentialoperator \(\mathcal{L}\) mit
        \[
            \mathcal{L}u := \frac{\d}{\d x}u + au
        \]
        ist.
    \end{problem}
    
    \subsection*{Lösung}
    Die Funktion \(E\) ist eine Fundamentallösung der Operator \(\mathcal{L}\), wenn
    \[
        \mathcal{L}E = \delta,
    \]
    wobei \(\delta\) die Dirac-Distribution ist, i.e. im Sinne von distributionelle Ableitung.
    Nun lautet die gegebene Funktion \(E\parentheses*{x}\) wie im Folgenden
    \[
        E\parentheses*{x} = e^{-ax}H\parentheses*{x}
    \]
    und sei \(\varphi\parentheses*{x} \in \mathcal{D}\parentheses*{\R}\) eine Testfunktion erhalten wir
    \begin{align*}
        T_E'\parentheses*{\varphi} &= \angles*{E', \varphi}\\
        &= -\angles*{E, \varphi'}\\
        &= -\int_{-\infty}^\infty e^{-ax}H\parentheses*{x}\varphi'\parentheses*{x}\d x\\
        &= -\int_0^\infty e^{-ax}\varphi'\parentheses*{x}\d x\\
        &= -\parentheses*{\brackets*{e^{-ax}\varphi\parentheses*{x}}_0^\infty + \int_0^\infty ae^{-ax}\varphi\parentheses*{x}\d x}\\
        &= -\parentheses*{\lim_{b \to \infty}e^{-b}\varphi\parentheses*{b} - \varphi\parentheses*{0} + \int_{-\infty}^\infty ae^{-ax}H\parentheses*{x}\varphi\parentheses*{x}\d x}\\
        &= \angles*{\delta_0, \varphi} - \angles*{ae^{-ax}H, \varphi}.
    \end{align*}
    Daher gilt
    \[
        E'\parentheses*{x} = \delta\parentheses*{x} - ae^{-ax}H\parentheses*{x}, \quad \text{in }\mathcal{D}''
    \]
    und
    \[
        \mathcal{L}E = \frac{\d}{\d x}E + aE = \delta\parentheses*{x} - ae^{-ax}H\parentheses*{x} + ae^{-ax}H\parentheses*{x} = \delta.
    \]


    \section*{Aufgabe 3}
    
    \begin{problem}
        Gegeben sei die eindimensionale Wärmeleitungsgleichung
        \begin{equation}\label{eq:1}
            \partial_t u\parentheses*{x, t} = \partial_x^2 u\parentheses*{x, t}, \quad t > 0, x \in \brackets*{0, 1}
        \end{equation}
        mit den Anfangswerten \(u\parentheses*{x, 0} = u_0\parentheses*{x}\) und den Randwerten \(u\parentheses*{0, t} = u\parentheses*{1, t} = 0\).
        Außerdem sei das Gitter \(0 = x_0, \ldots, x_n = 1\) mit \(x_{j + 1} - x_j = h_x, 0 \le j \le n - 1\) gegeben.
        \begin{enumerate}
            \item Diskretisieren Sie die Ortsableitung in Gleichung \eqref{eq:1} unter der Annahme \(y_j\parentheses*{t} \approx u\parentheses*{x_j, t}\), um ein System gewöhnlicher Differentialgleichungen \(y' = Ay\) zu erhalten.
            \item Formulieren Sie die Trapezregel und das implizite Euler-Verfahren (jeweils mit konstanter Schrittweite \(h_t\)) zur Lösung des Systems aus Teil a).
            Welches Verfahren würden Sie hinsichtlich der Konsistenzordnung bevorzugen?
        \end{enumerate}
    \end{problem}
    
    \subsection*{Lösung}
    \begin{enumerate}
        \item Wir nutzen zentrale Differenzen um die Ortsableitung zu approximieren:
        \[
            \partial_x^2 u\parentheses*{x_j, t} \approx \partial_x\parentheses*{\frac{u\parentheses*{x_j + \frac{h_x}{2}, t} - u\parentheses*{x_j - \frac{h_x}{2}, t}}{h_x}} \approx \frac{u\parentheses*{x_{j + 1}, t} - 2u\parentheses*{x_j, t} + u\parentheses*{x_{j - 1}, t}}{h_x^2}.
        \]
        Damit ergibt sich unter Ausnutzung der Annahme aus Teil a)
        \[
            y_j' = \frac{1}{h_x^2}\parentheses*{y_{j + 1} - 2y_j + y_{j - 1}}.
        \]
        Mit \(y := \parentheses*{y_1, \ldots, y_{n - 1}}\) schreiben wir \(y' = Ay\), wobei
        \[
            A = -\frac{1}{h_x^2}\begin{pmatrix}
                2 & -1 & 0 & \cdots & 0\\
                -1 & 2 & -1 & \ddots & \vdots\\
                0 & \ddots & \ddots & \ddots & 0\\
                \vdots & \ddots & -1 & 2 & -1\\
                0 & \cdots & 0 & -1 & 2
            \end{pmatrix}.
        \]
        \item Bezeichne \(y^i := y\parentheses*{t_i}\).
        Die Trapezregel hier angewandt ergibt
        \[
            y^{i + 1} = y^i + \frac{h_t}{2}\parentheses*{Ay^i + Ay^{i + 1}} \iff y^{i + 1} = y^i + \frac{h_t}{2}A\parentheses*{y^i + y^{i + 1}},
        \]
        also
        \[
            \parentheses*{I - \frac{h_t}{2}A}y^{i + 1} = \parentheses*{I + \frac{h_t}{2}A}y^i.
        \]
        Hingegen lautet das implizite Euler-Verfahren
        \[
            y^{i + 1} = y^i + h_t\parentheses*{Ay^{i + 1}},
        \]
        also
        \[
            \parentheses*{I - h_t A}y^{i + 1} = y^i.
        \]
        Da die Trapezregel Konsistenzordnung \(2\), das Euler-Verfahren aber nur Ordnung \(1\) besitzt, würden wir in diesem Falle die Trapezregel bevorzugen.
    \end{enumerate}
\end{document}
