\documentclass{exercise}

\institute{Applied and Computational Mathematics}
\title{Selbstrechenübung 12}
\author{Joshua Feld, 406718}
\course{Mathematische Grundlagen IV}
\professor{Torrilhon \& Berkels}
\semester{Sommersemester 2022}
\program{CES (Bachelor)}

\begin{document}
    \maketitle


    \section*{Aufgabe 1}
    
    \begin{problem}
        \begin{enumerate}
            \item Sei
            \[
                f: \R \to R, x \mapsto f\parentheses*{x} := \frac{1}{\sqrt{2\pi}}\exp\parentheses*{-\frac{1}{2}x^2}
            \]
            die Standardnormalverteilung.
            Bestimmen Sie die Fouriertransformierte \(\hat{f}\) von \(f\).
            \item Sie \(f \in L^1\parentheses*{\R^n}\) und \(c \in \R^n\).
            Zeigen Sie, dass
            \[
                \mathcal{F}\parentheses*{f\parentheses*{x + c}}\parentheses*{\xi} = e^{i\angles*{\xi, c}}\mathcal{F}\parentheses*{f}\parentheses*{\xi} \quad \forall\xi \in \R^n
            \]
            gilt, wobei \(\mathcal{F}\) die Fourier-Transformation bedeutet.
        \end{enumerate}
    \end{problem}
    
    \subsection*{Lösung}
    \begin{enumerate}
        \item Es handelt sich um eine Wahrscheinlichkeitsdichte wegen der Normierung
        \[
            \frac{1}{\sqrt{2\pi}}\int_\R \exp\parentheses*{-\frac{x^2}{2}}\d x = 1.
        \]
        Die standardisierte Normalverteilung geht bei Fourier-Transformation in sich über.
        Es gilt
        \begin{align*}
            \hat{f}\parentheses*{\xi} &= \int_{-\infty}^\infty \frac{1}{\sqrt{2\pi}}\exp\parentheses*{-\frac{x^2}{2}}\exp\parentheses*{-i\xi x}\d x\\
            &= \frac{1}{\sqrt{2\pi}}\int_{-\infty}^\infty \exp\parentheses*{-\frac{x^2}{2} - i\xi x}\d x\\
            &= \frac{1}{\sqrt{2\pi}}\int_{-\infty}^\infty \exp\parentheses*{-\frac{1}{2}\parentheses*{x^2 + 2i\xi x}}\d x\\
            &= \frac{1}{\sqrt{2\pi}}\int_{-\infty}^\infty \exp\parentheses*{-\frac{1}{2}\parentheses*{\parentheses*{x + i\xi}^2 + \xi^2}}\d x\\
            &= \frac{1}{\sqrt{2\pi}}\exp\parentheses*{-\frac{\xi^2}{2}}\int_{-\infty}^\infty \exp\parentheses*{-\frac{1}{2}\parentheses*{x + i\xi}^2}\d x\\
            &= \frac{1}{\sqrt{2\pi}}\exp\parentheses*{-\frac{\xi^2}{2}}\int_{-\infty}^\infty \exp\parentheses*{-\frac{u^2}{2}}\d u.
        \end{align*}
        Daher ist
        \[
            \hat{f}\parentheses*{\xi} = \exp\parentheses*{-\frac{\xi^2}{2}}.
        \]
        \item
        \begin{align*}
            \mathcal{F}\parentheses*{f\parentheses*{x + c}}\parentheses*{\xi} &= \int_{\R^n}f\parentheses*{x + c}\exp\parentheses*{-i\angles*{\xi, x}}\d x\\
            &= \int_{\R^n}f\parentheses*{y}\exp\parentheses*{-i\angles*{\xi, y - c}}\d y\\
            &= \int_{\R^n}f\parentheses*{y}\exp\parentheses*{i\angles*{\xi, c} - i\angles*{\xi, y}}\d y\\
            &= \exp\parentheses*{i\angles*{\xi, c}}\int_{\R^n}f\parentheses*{y}\exp\parentheses*{-i\angles*{\xi, y}}\d y\\
            &= \exp\parentheses*{i\angles*{\xi, c}}\mathcal{F}\parentheses*{f}\parentheses*{\xi}.
        \end{align*}
    \end{enumerate}


    \section*{Aufgabe 2}
    
    \begin{problem}
        Wir betrachten zum Beispiel das folgende Anfangs-Randwertproblem
        \[
            \frac{\partial u\parentheses*{t, x}}{\partial t} = \frac{\partial^2 u\parentheses*{t, x}}{\partial x^2}, \quad x \in \parentheses*{0, 1}, t \in \parentheses*{0, T},
        \]
        mit Anfangsbedingungen
        \[  
            u\parentheses*{0, x} = f\parentheses*{x} \quad \forall x \in \parentheses*{0, 1}
        \]
        und
        \[
            f\parentheses*{0} = 0 = f\parentheses*{1}
        \]
        mit Randbedingungen
        \[
            u\parentheses*{t, 0} = 0, \quad u\parentheses*{t, 1} = 0,
        \]
        für \(t \in \parentheses*{0, T}\).
        Erläutern Sie kurz (keine Berechnungen erforderlich):
        \begin{enumerate}
            \item Was ist der Unterschied zwischen der \emph{vertikalen} und der \emph{horizontalen} Linienmethode?
            \item Wie können Sie ein Verfahren mit höherer Genauigkeit als in der Vorlesung konstruieren (z.B. \(\mathcal{O}\parentheses*{\Delta t^3, \Delta x^3}\))?
            \item Welchen Vorteil hat das implizite Euler-Verfahren gegenüber dem expliziten Euler-Verfahren?
            \item Welchen Vorteil hat die Trapezregel gegenüber dem impliziten Euler-Verfahren?
        \end{enumerate}
    \end{problem}
    
    \subsection*{Lösung}
    \begin{enumerate}
        \item Bei der vertikalen Linienmethode wird zunächst nur die Ortsvariable diskretisiert, was zu einer semidiskretisierten Differentialgleichung in der Zeit führt.
        Bei der horizontalen Linienmethode wird umgekehrt zuerst die Zeit diskretisiert.
        \item Finite Differenzen-Verfahren höherer Ordnung (größerer Differenzenstern) kombiniert mit einem Runge-Kutta-Verfahren in der Zeit.
        \item Das explizite Euler-Verfahren benötigt eine parabilische CFL-Bedingung der Form (\(\Delta t \le \Delta x^2\)).
        Dies ist für das implizite Euler-Verfahren nicht notwendig.
        \item Die Trapezregel hat eine höhere Konsistenzordnung.
    \end{enumerate}


    \section*{Aufgabe 3}
    
    \begin{problem}
        Gegeben sei das lineare Gleichungssystem \(Ax = b\) mit
        \[
            A = \begin{pmatrix}
                2 & -1 & 1\\
                -1 & 2 & -1\\
                1 & -1 & 2
            \end{pmatrix}, \quad b = \begin{pmatrix}
                4\\
                -1\\
                7
            \end{pmatrix}.
        \]
        Das lineare Gleichungssystem hat die Lösung \(x = \parentheses*{1, 2, 4}^T\).
        \begin{enumerate}
            \item Führen Sie jeweils \emph{einen Schritt}
            \begin{enumerate}
                \item des Jacobi-Verfahrens,
                \item des Gauß-Seidel-Verfahrens,
            \end{enumerate}
            mit dem Startvektor \(x^0 = \parentheses*{1, 1, 1}^T\) durch.
            \item Gegeben sei die Richardson-Methode
            \[
                x^{k + 1} = x^k + \omega\parentheses*{b - Ax^k}.
            \]
            \begin{enumerate}
                \item Für welche Werte \(\omega\) konvergiert die Methode?
                \item Wie lautet der optimale Wert für \(\omega\)?
            \end{enumerate}
        \end{enumerate}
    \end{problem}
    
    \subsection*{Lösung}
    \[
        A = \begin{pmatrix}
            D & & & -U\\
            & D & &\\
            & & \ddots &\\
            -L & & & D
        \end{pmatrix}, \quad x^{n + 1} = Gx^n + \tilde{b}.
    \]
    \begin{enumerate}
        \item
        \begin{enumerate}
            \item
            \[
                G = D^{-1}\parentheses*{L + U} = \begin{pmatrix}
                    0 & 1 & -1\\
                    1 & 0 & 1\\
                    -1 & 1 & 0
                \end{pmatrix}, \quad \tilde{b} = D^{-1}b = \frac{1}{2}\begin{pmatrix}
                    4\\
                    -1\\
                    7
                \end{pmatrix}
            \]
            \[
                x^{n + 1} = \frac{1}{2}\begin{pmatrix}
                    0 & 1 & -1\\
                    1 & 0 & 1\\
                    -1 & 1 & 0
                \end{pmatrix}\begin{pmatrix}
                    1\\
                    1\\
                    1
                \end{pmatrix} + \frac{1}{2}\begin{pmatrix}
                    4\\
                    -1\\
                    7
                \end{pmatrix} = \frac{1}{2}\begin{pmatrix}
                    4\\
                    1\\
                    7
                \end{pmatrix}
            \]
            \item
            \[
                G = \parentheses*{D - L}^{-1}U, \quad \tilde{b} = \parentheses*{D - L}^{-1}b
            \]
            \[
                \parentheses*{D - L}x^{n + 1} = Ux^n + b \implies \begin{pmatrix}
                    2 & 0 & 0\\
                    -1 & 2 & 0\\
                    1 & -1 & 2
                \end{pmatrix}\begin{pmatrix}
                    x_1^{n + 1}\\
                    x_2^{n + 1}\\
                    x_3^{n + 1}
                \end{pmatrix} = \begin{pmatrix}
                    0 & -1 & 1\\
                    0 & 0 & -1\\
                    0 & 0 & 0
                \end{pmatrix}\begin{pmatrix}
                    1\\
                    1\\
                    1
                \end{pmatrix} + \begin{pmatrix}
                    4\\
                    -1\\
                    7
                \end{pmatrix}.
            \]
            Daher gilt
            \[
                x^{n + 1} = \begin{pmatrix}
                    2\\
                    1\\
                    3
                \end{pmatrix}.
            \]
        \end{enumerate}
        \item
        \begin{enumerate}
            \item Für die Matrix \(A\) gilt \(\lambda\parentheses*{A} = 1, 1, 4\).
            Das Verfahren konvergiert, wenn \(\absolute*{1 - \omega\lambda\parentheses*{A}} < 1\), deshalb
            \[
                0 < \omega < \frac{1}{2}.
            \]
            \item Der optimale Wert für \(\omega\) lautet
            \[
                \omega_{\text{opt}} = \frac{2}{\absolute*{\lambda\parentheses*{A}}_{\text{max}} + \absolute*{\lambda\parentheses*{A}}_{\text{min}}} = \frac{2}{4 + 1} = \frac{2}{5}.
            \]
        \end{enumerate}
    \end{enumerate}
\end{document}
