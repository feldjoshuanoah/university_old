\documentclass{exercise}

\institute{Lehr- und Forschungsgebiet Kontinuumsmechanik}
\title{Übung 1}
\author{Joshua Feld, 406718}
\course{Mechanik verformbarer Körper}
\professor{Itskov}
\semester{Sommersemester 2022}
\program{CES (Bachelor)}

\begin{document}
    \maketitle


    \section*{Aufgabe 1}

    \begin{problem}
        Einige Studenten nutzen das schöne Wetter und beschließen zu grillen.
        Die einen bringen Würstchen mit und für die Vegetarier gibt es Tofu-Bällchen.
        Leider bleibt das Grillgut zu lange auf dem Grill und sowohl Würstchen als auch Tofu patzen, da sich durch die Hitze ein zu hoher Innendruck entwickelt.
        Vergleichen Sie die Spannungen der beiden Grillgutarten und ermitteln Sie, an welcher Stelle ein Würstchen platzt.
    \end{problem}

    \subsection*{Lösung}
    Das Würstchen entspricht einem Zylinder mit zwei kugelförmigen Enden.
    Das Tofu-Bällchen ist eine Kugel.
    Beide werden als dünnwandige Behälter angesehen.

    Für das Würstchen müssen drei Spannungen verglichen werden:
    \begin{itemize}
        \item Tangentialspannung im Zylinder \(\sigma_{TZ}\),
        \item Radialspannung im Zylinder \(\sigma_R\) und
        \item Tangentialspannung in der Kugel \(\sigma_{TK}\).
    \end{itemize}
    Für \(\sigma_{TZ}\) folgt mit der Schnittfläche des Zylinders \(A_W\) und der Fläche an der der Druck angreifen kann \(A_i\)
    \[
        \sigma_{TZ} = \frac{F}{A_W} = \frac{pA_i}{A_W} = \frac{p \cdot 2rl}{2lt} = \frac{pr}{t}.
    \]
    Für \(\sigma_R\) folgt mit der Schnittfläche des Zylinders \(A_W\) und der Fläche \(A_i\) auf die der Druck wirkt
    \[
        \sigma_R = \frac{F}{A_W} = \frac{pA_i}{A_W} = \frac{p\pi r^2}{2\pi rt} = \frac{pr}{2t}.
    \]
    Die Fläche auf die der Druck wirkt wird vereinfachend als Multiplikation des Umfangs mit der Dicke angenommen.
    Diese Vereinfachung gilt nur für dünnwandige Zylinder.
    Somit ist im zweiten Fall \(A_W = 2\pi rt\).

    Die Tangentialspannung in einer Kugel entspricht der Radialspannung des Zylinders, da sowohl die Schnittfläche als auch die Fläche der Krafteinwirkung gleich sind.
    Es gilt daher
    \[
        \sigma_{TK} = \sigma_R = \frac{pr}{2t}.
    \]
    Hieraus folgt, dass \(\sigma_{TZ}\) doppelt so groß ist wie \(\sigma_R\) und \(\sigma_{TK}\).
    Damit wird das Würstchen in Längsrichtung und nicht in zwei Teile platzen.
    Die kugelförmigen Enden sind ebenfalls unkritisch.

    Die Tofu-Bällchen haben keine Vorzugsrichtung, da sie als dünnwandige Kugeln angenommen sind, die Spannung beträgt in jeder Richtung \(\sigma_{TK}\).
    Sie hält einem doppelt so hohen Innendruck wie das Würstchen stand.


    \section*{Aufgabe 2}

    \begin{problem}
        Ein dünnwandiger Zylinderkessel aus Stahl wird durch den Innendruck \(p\) belastet.
        Wie groß darf \(p\) höchstens sein, damit die Spannung in der Kesselwand die zulässige Spannung \(\sigma_{\text{zul}}\) nicht übersteigt?

        Gegeben: \(r = 1\sis{\meter}\), \(t = 1\sis{\centi\meter}\), \(\sigma_{\text{zul}} = 150\sis{\newton\per\milli\meter\squared}\)
    \end{problem}

    \subsection*{Lösung}
    Dünnwandig ist ein Synonym für den ebenen Spannungszustand.
    Das heißt, dass keine Schub- oder Normalspannungen in Richtung der Flächennormalen existieren und dass die Tangentialspannungen über die Dicke konstant sind.
    Es wird die Spannung in Längsrichtung des Zylinders berechnet, da diese betragsmäßig am größten ist (siehe Grillgutaufgabe).
    Diese Spannung ergibt sich als Kraft pro Fläche
    \[
        \sigma = \frac{F}{A_W} = \frac{pA_i}{A_W}
    \]
    mit der Schnittfläche des Zylinders \(A_W = 2lt\) und der Fläche an der der Druck angreifen kann \(A_i = 2lr\).
    Es gilt aufgrund der zulässigen Höchstspannung \(\sigma_{\text{zul}}\)
    \[
        \sigma_{\text{zul}} \ge \frac{pA_i}{A_W} = \frac{pr}{t}.
    \]
    Für den Grenzfall des maximal zulässigen Drucks \(p_{\text{max}}\) ergibt sich somit
    \[
        p_{\text{max}} = \frac{t}{r}\sigma_{\text{zul}} = 1,5\sis{\mega\pascal}.
    \]


    \section*{Aufgabe 3}

    \begin{problem}
        Eine dünnwandige Tauchkugel (Radius \(r\), Wandstärke \(t\)) befindet sich in einer Wassertiefe \(r\) (\(r \ll h\)).
        Im Inneren der Kugel herrscht Normaldruck.
        Die Dichte des Wassers wird mit \(\rho\) und die Erdbeschleunigung mit \(g\) angenommen.
        \begin{enumerate}
            \item Wie groß sind die auftretenden Spannungen in der Hülle der Tauchkugel?
            \item Wie groß müsste die Wandstärke sein, wenn die Tauchkugel in einer Wassertiefe \(1000\sis{\meter}\) tauchen soll und die zulässige Normalspannung in der Hülle das \(1,5\)-fache der in einer Wassertiefe von \(500\sis{\meter}\) auftretenden Spannung ist?
        \end{enumerate}
        Gegeben: \(r = 0,5\sis{\meter}\), \(t = 12,5\sis{\milli\meter}\), \(h = 500\sis{\meter}\), \(\rho = 1000\sis{\kilo\gram\per\meter\cubed}\), \(g = 9,81\sis{\meter\per\second\squared}\)
    \end{problem}
    
    \subsection*{Lösung}
    \begin{enumerate}
        \item Der hydrostatische Druck auf die Kugel wird vereinfachend als konstant über die Höhe der Kugel angenommen und berechnet sich zu
        \[
            p = \rho gh.
        \]
        Die Gleichgewichtsbedingung für einen beliebigen Schnitt in Äquatorialebene lautet
        \[
            \sum F = \sigma_t \cdot 2\pi rt + p\pi r^2 = 0.
        \]
        Damit ergibt sich die Tangentialspannung zu
        \[
            \sigma_t = -\frac{pr}{2t} = -98,1\sis{\mega\pascal}.
        \]
        Die Tangentialspannungen sind in alle Richtungen gleich \(\sigma_\phi = \sigma_\theta\), da der Schnitt durch die Kugel beliebig war.
        \item Der hydrostatische Druck \(p'\) auf \(1000\sis{\meter}\) berechnet sich zu
        \[
            p' = \rho gh.
        \]
        Die zulässige Spannung \(\sigma_{\text{zul}}\) ist
        \[
            \sigma_{\text{zul}} = \frac{3}{2}\sigma_t = -147,15\sis{\mega\pascal}.
        \]
        Die Wandstärke \(t\) berechnet sich folglich zu
        \[
            t = -\frac{p'r}{2\sigma_{\text{zul}}} \approx 16,67\sis{\milli\meter}.
        \]
    \end{enumerate}
    
\end{document}
