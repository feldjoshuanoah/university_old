\documentclass{exercise}

\institute{Lehr- und Forschungsgebiet Kontinuumsmechanik}
\title{Übung 2}
\author{Joshua Feld, 406718}
\course{Mechanik verformbarer Körper}
\professor{Itskov}
\semester{Sommersemester 2022}
\program{CES (Bachelor)}

\begin{document}
    \maketitle


    \section*{Aufgabe 1}

    \begin{problem}
        Bei Röntgenaufnahmen am Oberschenkelknochen eines Studenten wurde ein Anriss festgestellt, der unter einem Winkel \(\varphi\) gegenüber der Horizontalen geneigt ist.
        Um den Heilungsprozess nicht zu gefährden, darf im Anriss eine Normalspannung \(\sigma_{R, \text{zul}}\) nicht überschritten werden.
        Vereinfacht wird angenommen, dass der Knochen einen vollkreisförmigen Querschnitt hat, der Durchmesser in genügend großer Entfernung von den Enden konstant und der Spannungszustand einachsig ist.
        \begin{enumerate}
            \item Wie groß darf die Kraft \(F\) maximal werden, damit die zulässige Normalspannung in der Rissebene nicht überschritten wird?
            \item Wie groß ist die Schubspannung in der Rissebene für den Fall, dass der Knochen mit der maximal zulässigen Kraft belastet wird?
        \end{enumerate}
        Gegeben: \(d = 60\sis{\milli\meter}\), \(\sigma_{R, \text{zul}} = 0,1\sis{\newton\per\milli\meter\squared}\), \(\varphi = 25^\circ\)
    \end{problem}

    \subsection*{Lösung}
    \begin{center}
        \begin{tikzpicture}
            \draw (0,0) coordinate (o) -- (5,0) coordinate (x);
            \draw (o) -- (0,3) coordinate (y);
            \draw (x) -- (y);
            \draw[dotted] (0,0) -- (45/34,75/34) coordinate (xi);
            \pic[draw,->,"$\varphi$",angle eccentricity=1.5] {angle=x--o--xi};
            \pic[draw,->,"$\varphi$",angle eccentricity=1.5] {angle=o--y--x};
            \draw[->] (5.1,0) -- (5.6,0) node[above] {\(x\)};
            \draw[->] (0,3.1) -- (0,3.6) node[right] {\(y\)};
            \draw[->] (-.1,.06) -- (-.6,.46) node[below] {\(\eta\)};
            \draw[->] (-.1,2.25) -- (-.1,.75) node[left] {\(\tau_{yy}\)};
            \draw[->] (-.2,1.5) -- (-.7,1.5) node[below] {\(\sigma_x\)};
            \draw[->] (3.25,-.1) -- (1.75,-.1) node[below] {\(\tau_{xy}\)};
            \draw[->] (2.5,-.2) -- (2.5,-.7) node[right] {\(\sigma_y\)};
            \draw[->] (3.25,1.15) -- (1.75,2.05) node[above] {\(\tau_{\xi\eta}\)};
            \draw[->] (1.5,2.5) -- (1.8,3) node[above right] {\(\xi\)};
            \draw[->] (2.5,1.7) -- (2.8,2.2) node[left] {\(\sigma_\xi\)};
        \end{tikzpicture}
    \end{center}
    In der Abbildung ist ein dreieckiges Element mit angreifenden Spannungen dargestellt.
    Das Element befindet sich im Kräftegleichgewicht, das heißt, dass die Summe aller Kräfte null ergibt.
    Mit der infinitesimalen Schnittfläche \(\d A\) und unter der Annahme einer konstanten Dicke des Elements folgt für das Kräftegleichgewicht im \(\xi\)- beziehungsweise \(\eta\)-Richtung
    \begin{align*}
        \sum F_\xi = \sigma_\xi\d A - \sigma_x\cos\phi\d A\cos\phi - \tau_{xy}\sin\phi\d A\cos\phi - \sigma_y\sin\phi\d A\sin\phi - \tau_{yx}\cos\phi\d A\sin\phi &= 0,\\
        \sum F_\eta = \tau_{\xi\eta}\d A - \sigma_x\sin\phi\d A\cos\phi - \tau_{xy}\cos\phi\d A\cos\phi - \sigma_y\cos\phi\d A\sin\phi - \tau_{yx}\sin\phi\d A\sin\phi &= 0.
    \end{align*}
    Unter Verwendung der Gleichgewichtsbedingung \(\tau_{xy} = \tau_{yx}\) erhält man
    \[
        \sigma_\xi = \sigma_x\cos^2\phi + \sigma_y\sin^2\phi + 2\tau_{xy}\sin\phi\cos\phi
    \]
    und
    \[
        \tau_{\xi\eta} = -\parentheses*{\sigma_x - \sigma_y}\sin\phi\cos\phi + \tau_{xy}\parentheses*{\cos^2\phi - \sin^2\phi}.
    \]
    Mithilfe der Identitäten
    \begin{align*}
        \cos^2\phi &= \frac{1}{2}\parentheses*{1 + \cos\parentheses*{2\phi}}, & 2\sin\phi\cos\phi &= \sin\parentheses*{2\phi},\\
        \sin^2\phi &= \frac{1}{2}\parentheses*{1 - \cos\parentheses*{2\phi}}, & \cos^2\phi - \sin^2\phi &= \cos\parentheses*{2\phi}
    \end{align*}
    erhält man schließlich
    \begin{align*}
        \sigma_\xi &= \frac{1}{2}\parentheses*{\sigma_x + \sigma_y} + \frac{1}{2}\parentheses*{\sigma_x - \sigma_y}\cos\parentheses*{2\phi} + \tau_{xy}\sin\parentheses*{2\phi},\\
        \sigma_\eta &= \frac{1}{2}\parentheses*{\sigma_x + \sigma_y} - \frac{1}{2}\parentheses*{\sigma_x - \sigma_y}\cos\parentheses*{2\phi} - \tau_{xy}\sin\parentheses*{2\phi},\\
        \tau_{\xi\eta} &= -\frac{1}{2}\parentheses*{\sigma_x - \sigma_y}\sin\parentheses*{2\phi} + \tau_{xy}\cos\parentheses*{2\phi}.
    \end{align*}
    \begin{enumerate}
        \item Aufgrund des einachsigen Spannungszustand gilt \(\tau_{xy} = \sigma_x = 0\) und somit ergibt sich mit
        \[
            \sigma_\eta = \frac{\sigma_y}{2}\parentheses*{1 + \cos\parentheses*{2\phi}}, \quad \sigma_y = \frac{F}{A}, \quad A = \frac{\pi d^2}{4}
        \]
        die Beziehung zwischen \(\sigma_\eta\) und \(F\) als
        \[
            \sigma_\eta = \frac{F}{2A}\parentheses*{1 + \cos\parentheses*{2\phi}} = \frac{2F}{\pi d^2}\parentheses*{1 + \cos\parentheses*{2\phi}}.
        \]
        Für den Grenzfall ergibt sich mit der Beziehung \(\sigma_\eta \le \sigma_{R, \text{zul}}\) für die maximale Kraft
        \[
            F_{\text{max}} = \frac{\pi d^2\sigma_{R, \text{zul}}}{2\parentheses*{1 + \cos\parentheses*{2\phi}}} = 344,2\sis{\newton}.
        \]
        \item Die Schubspannung \(\tau_{\xi\eta}\) ergibt sich zu
        \[
            \tau_{\xi\eta} = \frac{\sigma_y}{2}\sin\parentheses*{2\phi} = \frac{F_{\text{max}}}{2A}\sin\parentheses*{2\phi} = \frac{2F_{\text{max}}}{\pi d^2}\sin\parentheses*{2\phi} = \frac{\sin\parentheses*{2\phi}}{1 + \cos\parentheses*{2\phi}}\sigma_{R, \text{zul}} = 0,0466\sis{\newton\per\milli\meter\squared}.
        \]
    \end{enumerate}

    
    \section*{Aufgabe 2}

    \begin{problem}
        Ein Blechstreifen mit schrägliegender Schweißnaht ist auf Zug beansprucht.
        \begin{enumerate}
            \item Berechnen Sie die zum Festigkeitsnachweis benötigten Normal- und Schubspannungen.
            \item Bei welchem Winkel nimmt die Schubspannung in der Schweißnaht einen Maximalwert an?
            Wie groß ist dann die Normalspannung?
        \end{enumerate}
    \end{problem}

    \subsection*{Lösung}
    \begin{enumerate}
        \item Aus der Aufgabenstellung ergibt sich \(\sigma_y = \sigma_0\), \(\sigma_x = 0\) und \(\tau_{xy} = 0\).
        Eine Koordinatentransformation wird durchgeführt.
        Es gelten die Formeln
        \[
            \sigma_\eta = \frac{\sigma_0}{2}\parentheses*{1 + \cos\parentheses*{2\phi}}, \quad \tau_{\xi\eta} = \frac{\sigma_0}{2}\sin\parentheses*{2\phi},
        \]
        mit \(\phi = -30^\circ\) folgt dann
        \[
            \sigma_\xi = \frac{1}{4}\sigma_0, \quad \sigma_\eta = \frac{3}{4}\sigma_0, \quad \tau_{\xi\eta} = -\frac{\sqrt{3}}{4}\sigma_0.
        \]
        \item Es gilt für die Abhängigkeit zwischen \(\tau_{\xi\eta}\) und \(\phi\) die Beziehung
        \[
            \tau_{\xi\eta}\parentheses*{\phi} = -\frac{\sigma_0}{2}\sin\parentheses*{2\phi}.
        \]
        Für die maximale \(\tau_{\xi\eta}\) lauten die Bedingungen
        \[
            \frac{\partial\tau_{\xi\eta}}{\partial\phi} = -\sigma_0\cos\parentheses*{2\phi} = 0, \quad \frac{\partial^2\tau_{\xi\eta}}{\partial\phi^2} = 2\sigma_0\sin\parentheses*{2\phi} < 0.
        \]
        Für \(\phi = -45^\circ\) wird die erste Bedingung erfüllt und somit die maximale Schubspannung erreicht.
        Da der Blechstreifen auf Zug belastet wurd ist \(\sigma_0 > 0\) und somit die zweite Bedingung erfüllt.
        Für die Normalspannung ergibt sich
        \[
            \sigma_\eta\parentheses*{-45^\circ} = \frac{1}{2}\sigma_0.
        \]
    \end{enumerate}


    \section*{Aufgabe 3}

    \begin{problem}
        Für einen ebenen Spannungszustand an einem quadratischen Element sind die Spannungen gegeben.
        Das \(\xi\)-\(\eta\)-Koordinatensystem ist gegenüber dem \(x\)-\(y\)-Koordinatensystem um den Winkel \(\alpha\) geneigt.
        Berechnen Sie
        \begin{enumerate}
            \item den Spannungszustand im \(x\)-\(y\)-System,
            \item die Hauptspannungen sowie eine Hauptspannungsrichtung,
            \item die Hauptschubspannungen sowie eine Hauptschubspannungsrichtung.
        \end{enumerate}
        Gegeben: \(\sigma_\xi = 76\sis{\newton\per\milli\meter\squared}\), \(\sigma_\eta = 64\sis{\newton\per\milli\meter\squared}\), \(\alpha = 30^\circ\), \(\tau_{\xi\eta} = -49,6\sis{\newton\per\milli\meter\squared}\)
    \end{problem}

    \subsection*{Lösung}
    \begin{enumerate}
        \item
        \begin{align*}
            \sigma_\xi &= \frac{1}{2}\parentheses*{\sigma_x + \sigma_y} + \frac{1}{2}\parentheses*{\sigma_x - \sigma_y}\cos\parentheses*{2\phi} + \tau_{xy}\sin\parentheses*{2\phi},\\
            \sigma_\eta &= \frac{1}{2}\parentheses*{\sigma_x + \sigma_y} - \frac{1}{2}\parentheses*{\sigma_x - \sigma_y}\cos\parentheses*{2\phi} - \tau_{xy}\sin\parentheses*{2\phi},\\
            \tau_{\xi\eta} &= -\frac{1}{2}\parentheses*{\sigma_x - \sigma_y}\sin\parentheses*{2\phi} - \tau_{xy}\cos\parentheses*{2\phi}.
        \end{align*}
        Setzen wir nun die gegebenen Werte für \(\sigma_\xi\), \(\sigma_\eta\) und \(\tau_{\xi\eta}\) in diese Gleichungen ein und formen um, so ergibt sich ein Gleichungssystem mit drei Unbekannten und drei Gleichungen.
        Es folgen die Werte
        \[
            \sigma_x = 115,95\sis{\newton\per\milli\meter\squared}, \quad \sigma_y = 24,05\sis{\newton\per\milli\meter\squared}, \quad \tau_{xy} = -19,6\sis{\newton\per\milli\meter\squared}.
        \]
        \item Die Hauptspannungen \(\sigma_{1, 2}\) berechnen sich mithilfe der Formel
        \[
            \sigma_{1, 2} = \frac{\sigma_x + \sigma_y}{2} \pm \sqrt{\parentheses*{\frac{\sigma_x - \sigma_y}{2}}^2 + \tau_{xy}^2}.
        \]
        Die Ergebnisse werden so sortiert, dass \(\sigma_1 \ge \sigma_2\) gilt, sodass sich \(\sigma_1 = 119,96\sis{\newton\per\milli\meter\squared}\) und \(\sigma_2 = 20,04\sis{\newton\per\milli\meter\squared}\) ergeben.
        Die Hauptspannungsrichtungen sind die Richtungen, in der die Normalspannungen maximal sind.
        Die Bedingung für ein Maximum ist
        \[
            \frac{\partial\sigma}{\partial\phi} = \parentheses*{\sigma_x - \sigma_y}\sin\parentheses*{2\phi} - 2\tau_{xy}\cos\parentheses*{2\phi} = 0
        \]
        und folglich gilt für eine Hauptspannungsrichtung \(\phi^*\)
        \[
            \tan\parentheses*{2\phi^*} = \frac{2\tau_{xy}}{\sigma_x - \sigma_y} \implies \phi^* = \frac{1}{2}\arctan\frac{2\tau_{xy}}{\sigma_x - \sigma_y} = -11,55^\circ.
        \]
        \item Die Hauptschubspannungen berechnen sich als
        \[
            \tau_{\text{max}} = \pm\sqrt{\parentheses*{\frac{\sigma_x - \sigma_y}{2}}^2 + \tau_{xy}^2} = \pm 49,96\sis{\newton\per\milli\meter\squared}.
        \]
        Die Richtung der Hauptschubspannungen \(\phi^{**}\) berechnet sich zu (Bedingung: \(\frac{\partial\tau}{\partial\phi} = 0\))
        \[
            \tan\parentheses*{2\phi^{**}} = -\frac{\sigma_x - \sigma_y}{2\tau_{xy}} \implies \phi^{**} = \frac{1}{2}\arctan\parentheses*{-\frac{\sigma_x - \sigma_y}{2\tau_{xy}}} = 33,45^\circ.
        \]
        Es wurde jeweils nur die Richtung einer Achse der Haupt(schub)spannungen angegeben, die jeweils andere ist um \(90^\circ\) im mathematisch positiven Sinn (gegen den Uhrzeigersinn) gedreht.
        Durch Einsetzen in die Transformationsgleichungen ist es zudem möglich Hauptspannungen ihren entsprechenden Richtungen zuzuordnen.
    \end{enumerate}

    
    \section*{Aufgabe 4}

    \begin{problem}
        In einem ebenen Bauteil herrschen die Hauptnormalspannungen \(\sigma_1\) und \(\sigma_2\).
        \begin{enumerate}
            \item Wie groß sind die Spannungen in Schnitten, die um \(60^\circ\) gegenüber den Hauptachsen geneigt sind?
            \item In welchem Schnitt \(\bar{\varphi}\) wird die Normalspannung zu null?
            Wie groß sind dann die Schubspannungen und die Normalspannung in einer zu \(\bar{\varphi}\) senkrechten Richtung?
            \item In welchen Schnitten treten die maximalen Schubspannungen auf und wie groß sind die zugehörigen Normalspannungen?
        \end{enumerate}
        Gegeben: \(\sigma_1 = \sigma_y = 90\sis{\mega\pascal}\), \(\sigma_2 = \sigma_x = -52\sis{\mega\pascal}\)
    \end{problem}

    \subsection*{Lösung}
    \begin{enumerate}
        \item Die Spannungen bei Schnitten, die um \(\varphi = 60^\circ\) gedreht sind, folgen direkt aus den Transformationsgleichungen, wobei \(\tau_{xy} = 0\) gilt, da es sich um einen Hauptspannungszustand handelt:
        \[
            \sigma_\xi = 54,5\sis{\mega\pascal}, \quad \sigma_\eta = -16,5\sis{\mega\pascal}, \quad \tau_{\xi\eta} = 61,49\sis{\mega\pascal}.
        \]
        \item Der gesuchte Zustand lässt sich mit \(\sigma_\xi = 0\), \(\sigma_\eta \ne 0\) und \(\tau_{\xi\eta} \ne 0\) zusammenfassen, wobei der dazugehörige Winkel \(\bar{\varphi}\) gesucht ist.
        Durch Umformen der Transformationsgleichung ergibt sich
        \[
            \bar{\varphi} = 37,24^\circ.
        \]
        Das Einsetzen dieses Winkels in die Transformationsgleichungen liefert den kompletten Spannungszustand:
        \[
            \sigma_\xi = 0\sis{\mega\pascal}, \quad \sigma_\eta = 38\sis{\mega\pascal}, \quad \tau_{\xi\eta} = 68,41\sis{\mega\pascal}.
        \]
        \item Die maximale Schubspannung tritt unter \(\pm 45^\circ\) zu den Hauptachsen auf.
        Für die Beziehung der Hauptspannungen und der maximalen Schubspannung ist folgendes bekannt:
        \[
            \tau_{\text{max}} = \pm\frac{1}{2}\parentheses*{\sigma_1 - \sigma_2} = \pm 71\sis{\mega\pascal}.
        \]
        Aus der Invariationsbeziehung folgt demnach für die Normalspannungen:
        \[
            \sigma_M = \frac{1}{2}\parentheses*{\sigma_x + \sigma_y} = \frac{1}{2}\parentheses*{\sigma_1 + \sigma_2} = 19\sis{\mega\pascal}.
        \]
    \end{enumerate}
\end{document}
