\documentclass{exercise}

\institute{Lehr- und Forschungsgebiet Kontinuumsmechanik}
\title{Übung 6}
\author{Joshua Feld, 406718}
\course{Mechanik verformbarer Körper}
\professor{Itskov}
\semester{Sommersemester 2022}
\program{CES (Bachelor)}

\begin{document}
    \maketitle
    
    
    \section*{Aufgabe 1}
    
    \begin{problem}
        Ein Wasserturm besteht aus einem Betonschaft mit der Dichte \(\rho_B\).
        Dieser trägt einen Wassertank mit dem Volumen \(V_W\).
        Am oberen Ende habe der Schaft die Fläche \(A_0\).
        Das Volumen des leeren Wassertanks sei vernachlässigbar.
        Berechnen Sie den Verlauf der Querschnittsfläche unter Berücksichtigung des Eigengewichtes des Betonschaftes, bei der in jedem Flächenelement die gleiche Normalspannung herrscht.
        
        Gegeben: \(\rho_B\), \(\rho_W\), \(A_0\), \(V_W\), \(g\)
    \end{problem}
    
    \subsection*{Lösung}
    Wir betrachten ein infinitesimales Scheibenelement des Betonschaftes.
    Die Änderung der Kraft, die durch diesen Teil hervorgerufen wird, folgt gänzlich aus dem Eigengewicht der Scheibe.
    Wir definieren die \(z\)-Koordinate von der Spitze des Schafts nach unten.
    Es gilt
    \[
        \frac{\d F}{\d z} = \rho_B gA.
    \]
    Aufgrund der Bedingung, dass die Spannung konstant bleibt, gilt
    \[
        \d F = \d\parentheses*{\sigma A} = \sigma_0 \d A
    \]
    mit der initialen Spannung \(\sigma_0\).
    Diese lässt sich direkt aus dem Gewicht des gefüllten Wassertanks berechnen, da dies die einzige Kraft ist, die an der Spitze des Wasserturms eine Spannung hervorruft:
    \[
        \sigma_0 = \frac{F_W}{A_0} = \frac{\rho_W V_W g}{A_0}.
    \]
    Das Kombinieren der ersten beiden Gleichungen nach der Methode der Trennung der Variablen liefert
    \[
        \frac{1}{A}\d A = \frac{\rho_B g}{\sigma_0}\d z \implies \int\frac{1}{A}\d A = \frac{\rho_B g}{\sigma_0}\int 1\d z.
    \]
    Durch Integrieren beider Seiten folgt mit einer Konstanten \(C\)
    \[
        \ln A\parentheses*{z} = \frac{\rho_B g}{\sigma_0}z + C
    \]
    und somit
    \[
        A\parentheses*{z} = \exp\parentheses*{\frac{\rho_B g}{\sigma_0}z + C} = \exp\parentheses*{\frac{\rho_B g}{\sigma_0}z} \cdot \exp\parentheses*{C}.
    \]
    Mit der Randbedingung \(A\parentheses*{0} = A_0\) folgt für die Konstante \(\exp\parentheses*{C} = A_0\) und daraus mit \(\sigma_0\)
    \[
        A\parentheses*{z} = A_0\exp\parentheses*{\frac{\rho_B A_0}{\rho_W V_W}z}.
    \]
    
    
    \section*{Aufgabe 2}
    
    \begin{problem}
        Ein konischer Stab der Länge \(l\) (E-Modul \(E\)) mit kreisförmigem Querschnitt wird mit einer axialen Kraft \(F\) auf Zug belastet.
        Der Radienverlauf ist in axialer Richtung linear (\(r\parentheses*{0} = r_1\), \(r\parentheses*{l} = r_2\)).
        Berechnen Sie die Normalspannung \(\sigma\parentheses*{x}\) sowie die Längenänderung \(\Delta l\).
        
        Gegeben: \(F\), \(E\), \(l\), \(r_1\), \(r_2\)
    \end{problem}
    
    \subsection*{Lösung}
    Mithilfe der Geradengleichung folgt für den Radius des Stabs
    \[
        r\parentheses*{x} = r_1 + \frac{r_2 - r_1}{l}x = r_1 + \frac{\Delta r}{l}x
    \]
    für die Fläche
    \[
        A\parentheses*{x} = \pi r^2\parentheses*{x} = \pi\parentheses*{r_1 + \frac{\Delta r}{l}x}^2
    \]
    und für die Spannung
    \[
        \sigma\parentheses*{x} = \frac{F}{A\parentheses*{x}} = \frac{F}{\pi\parentheses*{r_1 + \frac{\Delta r}{l}x}^2}.
    \]
    Das Hook'sche Gesetz liefert für die Dehnung
    \[
        \varepsilon\parentheses*{x} = \frac{\d u\parentheses*{x}}{\d x} = \frac{1}{E}\sigma\parentheses*{x} = \frac{F}{\pi E\parentheses*{r_1 + \frac{\Delta r}{l} x}^2}.
    \]
    Umformung zu
    \[
        \d u\parentheses*{x} = \frac{Fl^2}{E\pi\Delta r^2}\frac{1}{\parentheses*{x + \frac{r_1 l}{\Delta r}}^2}\d x
    \]
    und Integration liefert
    \[
        u\parentheses*{x} = \frac{Fl^2}{E\pi\Delta r^2}\int_0^x \frac{1}{\parentheses*{x + \frac{r_1 l}{\Delta r}}^2}\d x = \frac{Fl}{\pi E}\frac{x}{r_1 \Delta rx + r_1^2 l}.
    \]
    Für die Längenänderung ergibt sich durch Auswerten der Verschiebungsfunktion am Ende des Stabs
    \[
        \Delta l = u\parentheses*{l} = \frac{Fl}{\pi Er_1^2}\frac{1}{1 + \frac{\Delta r}{r_1}} = \frac{Fl}{\pi Er_1 r_2}.
    \]
    
    
    \section*{Aufgabe 3}
    
    \begin{problem}
        Ein Quader wird um eine Temperatur \(\Delta T\) erwärmt.
        Das kartesische Koordinatensystem \(x\)-\(y\)-\(z\) liege mit dem Ursprung in der Ecke des Quaders.
        Ermitteln Sie die Spannungen und Dehnungen bei folgenden Randbedingungen
        \begin{enumerate}
            \item allseitig freie Lagerung,
            \item Ausdehnung in \(z\)-Richtung verhindert,
            \item Ausdehnung in \(y\)- und \(z\)-Richtung verhindert,
            \item Ausdehnung allseitig verhindert.
        \end{enumerate}
    \end{problem}
    
    \subsection*{Lösung}
    Für diese Aufgabe wird das verallgemeinerte Hook'sche Gesetz benötigt:
    \begin{align*}
        \varepsilon_x &= \frac{1}{E}\parentheses*{\sigma_x - \nu\parentheses*{\sigma_y + \sigma_z}} + \alpha_T \Delta T,\\
        \varepsilon_y &= \frac{1}{E}\parentheses*{\sigma_y - \nu\parentheses*{\sigma_x + \sigma_z}} + \alpha_T \Delta T,\\
        \varepsilon_z &= \frac{1}{E}\parentheses*{\sigma_z - \nu\parentheses*{\sigma_x + \sigma_y}} + \alpha_T \Delta T.
    \end{align*}
    \begin{enumerate}
        \item Aus der allseitig freien Lagerung folgt
        \[
            \sigma_x = 0, \quad \sigma_y = 0, \quad \sigma_z = 0
        \]
        und somit
        \[
            \varepsilon_x = \alpha_T \Delta T, \quad \varepsilon_y = \alpha_T \Delta T, \quad \varepsilon_z = \alpha_T \Delta T.
        \]
        \item Aus der Lagerung, die nur die Ausdehnung in \(z\)-Richtung verhindert, folgt
        \[
            \sigma_x = 0, \quad \sigma_y = 0, \quad \varepsilon_z = 0
        \]
        und somit
        \[
            \varepsilon_x = -\frac{\nu}{E}\sigma_z + \alpha_T \Delta T = \parentheses*{\nu + 1}\alpha_T \Delta T, \quad \varepsilon_y = -\frac{\nu}{E}\sigma_z + \alpha_T \Delta T = \parentheses*{\nu + 1}\alpha_T \Delta T, \quad \sigma_z = -E\alpha_T \Delta T.
        \]
        \item Aus der Lagerung, die die Ausdehnung in \(y\)- und \(z\)-Richtung verhindert, folgt
        \[
            \sigma_x = 0, \quad \varepsilon_y = 0, \quad \varepsilon_z = 0
        \]
        und somit
        \[
            \varepsilon_x = \frac{1 + \nu}{1 - \nu}\alpha\Delta T, \quad \sigma_y = -\frac{E\alpha\Delta T}{1 - \nu}, \quad \sigma_z = -\frac{E\alpha\Delta T}{1 - \nu}.
        \]
        \item Aus der Lagerung, die die Dehnung in alle Richtungen verhindert, folgt
        \[
            \varepsilon_x = 0, \quad \varepsilon_y = 0, \quad \varepsilon_z = 0
        \]
        und somit
        \[
            \sigma_x = -\frac{E\alpha\Delta T}{1 - 2\nu}, \quad \sigma_y = -\frac{E\alpha\Delta T}{1 - 2\nu}, \quad \sigma_z = -\frac{E\alpha\Delta T}{1 - 2\nu}.
        \]
    \end{enumerate}


    \section*{Aufgabe 4}
    
    \begin{problem}
        Ein Gleitstück aus Stahl passt im spannungslosen Zustand genau spielfrei in ein vollkommen starres Führungsstück.
        Es wird angenommen, dass das Gleitstück reibungsfrei in dem Führungsstück gleiten kann.
        Wie groß sind die Spannungen und Dehnungen im Gleitstück in \(x\)-, \(y\)- und \(z\)-Richtung, wenn das Gleitstück um \(\Delta T\) erwärmt wird?
        
        Gegeben: \(E = 20,6 \cdot 10^4\sis{\newton\per\milli\meter\squared}\), \(\alpha = 12 \cdot 10^{-6}\sis{\per\kelvin}\), \(\Delta T = 25\sis{\kelvin}\), \(\nu = 0,3\)
    \end{problem}
    
    \subsection*{Lösung}
    Ausgehend vom verallgemeinerten Hook'schen Gesetz
    \begin{align*}
        \varepsilon_x &= \frac{1}{E}\parentheses*{\sigma_x - \nu\parentheses*{\sigma_y + \sigma_z}} + \alpha_T \Delta T,\\
        \varepsilon_y &= \frac{1}{E}\parentheses*{\sigma_y - \nu\parentheses*{\sigma_x + \sigma_z}} + \alpha_T \Delta T,\\
        \varepsilon_z &= \frac{1}{E}\parentheses*{\sigma_z - \nu\parentheses*{\sigma_x + \sigma_y}} + \alpha_T \Delta T
    \end{align*}
    und mit dem Führungsstück was Ausdehnung in zwei Richtungen blockiert (\(x\), \(y\)) und in der dritten Richtung (\(z\)) uneingeschränkt zulässt
    \[
        \varepsilon_x = 0, \quad \varepsilon_y = 0, \quad \sigma_z = 0
    \]
    folgt
    \begin{align*}
        \sigma_x = -\frac{E\alpha\Delta T}{1 - \nu} = -\frac{618}{7}\sis{\mega\pascal} = -88,286\sis{\mega\pascal},\\
        \sigma_y &= -\frac{E\alpha\Delta T}{1 - \nu} = -\frac{618}{7}\sis{\mega\pascal} = -88,286\sis{\mega\pascal},\\
        \varepsilon_z &= \frac{1 + \nu}{1 - \nu}\alpha\Delta T = \frac{39}{70000} = 5,571 \cdot 10^{-4}.
    \end{align*}
    
    
    \section*{Aufgabe 5}
    
    \begin{problem}
        In einem Blech seien die Spannungen \(\sigma_x\), \(\sigma_y\) und \(\tau_{xy}\) bekannt.
        Berechnen Sie die Vergleichsspannungen nach den Ihnen bekannten Vergleichsspannungshypothesen.
        
        Gegeben: \(\sigma_x = 20\sis{\kilo\pascal}\), \(\sigma_y = 30\sis{\kilo\pascal}\), \(\tau_{xy} = 10\sis{\kilo\pascal}\)
    \end{problem}
    
    \subsection*{Lösung}
    Die Hauptspannungen ergeben sich aus
    \[
        \sigma_{1, 2} = \frac{\sigma_x + \sigma_y}{2} \pm \sqrt{\parentheses*{\frac{\sigma_x - \sigma_y}{2}}^2 + \tau_{xy}^2}
    \]
    zu
    \[
        \sigma_1 = 36,18\sis{\kilo\pascal}, \quad \sigma_2 = 13,82\sis{\kilo\pascal}.
    \]
    \begin{itemize}
        \item Der Vergleichsspannungswert gemäß der Normalspannungshypothese ergibt sich aus der größten Hauptspannung:
        \[
            \sigma_{V, N} = \sigma_1 = 36,18\sis{\kilo\pascal}.
        \]
        \item Der Vergleichsspannungswert gemäß der Normalspannungshypothese ergibt sich aus dem doppelten der größten Hauptschubspannung (dies entspricht dem Durchmesser im Mohr'schen Spannungskreis):
        \[
            \sigma_{V, S} = 2\tau_{\text{max}} = \sqrt{\parentheses*{\sigma_x - \sigma_y}^2 + 4\tau_{xy}^2} = \sigma_1 - \sigma_2 ) 22,36\sis{\kilo\pascal}.
        \]
        \item Der Vergleichsspannungswert gemäß der Hypothese der Gestaltänderungsenergie ergibt sich aus dem elastischen Teil der Energie, der durch die Verformung zustande kommt:
        \[
            \sigma_{V, G} = \sqrt{\sigma_x^2 + \sigma_y^2 - \sigma_x \sigma_y + 3\tau_{xy}^2} = \sqrt{\sigma_1^2 + \sigma_2^2 - \sigma_1 \sigma_2} = 31,63\sis{\kilo\pascal}.
        \]
    \end{itemize}


    \section*{Aufgabe 6}

    \begin{problem}
        Für einen Stab aus einem Eisen-Gusswerkstoff mit Durchmesser \(d\) ist Versagen durch Bruch eingetreten.
        Das Aussehen der Bruchfläche lässt auf einen Sprödbruch schließen.
        Die gemittelte Neigung zwischen Bruchflächen-Normale und Stabachse ist \(\varphi\).
        Die bruchauslösenden statischen Beanspruchungen sind Zug und Torsion.
        An Proben des gleichen Materials wurde eine Zugfestigkeit \(R_m\) ermittelt.
        Welche Belastungen (\(F_N\), \(M_T\)) traten zum Zeitpunkt des Versagens auf?
        Zwischen Schubspannung und Torsionsmoment soll folgender Zusammenhang gelten:
        \[
            M_T = \frac{\pi d^3}{16}\tau_{xy}.
        \]
        Gegeben: \(\varphi = 28^\circ\), \(R_m = 250\sis{\newton\per\milli\meter\squared}\), \(d = 42\sis{\milli\meter}\)
    \end{problem}

    \subsection*{Lösung}
    Aus der Aufgabenstellung wird deutlich, dass es sich um einen Sprödbruch handelt, weshalb die Normalspannungshypothese Anwendung findet.
    Der Vergleichsspannungswert gemäß der Normalspannungshypothese ergibt sich zu
    \[
        \sigma_{V, N} = \sigma_1 = R_m.
    \]
    Auslöser des Bruches waren Zug und Torsion, sodass \(\sigma_y = 0\) ist und die folgenden Größen gesucht werden
    \[
        F_N = \sigma_x A, \quad M_T = \frac{\pi d^3}{16}\tau_{xy}.
    \]
    Der gegebene Winkel \(\varphi = 28^\circ\) kennzeichnet den Bruchwinkel und da der Bruch in der Hauptspannungsebene geschehen ist gibt \(\varphi\) den Winkel der Hauptspannung \(\varphi^*\) an.
    Aus
    \[
        \tan\parentheses*{2\varphi^*} = \frac{2\tau_{xy}}{\sigma_x - \sigma_y} = \frac{2\tau_{xy}}{\sigma_x} \implies \sigma_x = \frac{2\tau_{xy}}{\tan\parentheses*{2\varphi^*}}
    \]
    und
    \[
        \sigma_{1, 2} = \frac{\sigma_x + \sigma_y}{2} \pm \sqrt{\parentheses*{\frac{\sigma_x - \sigma_y}{2}}^2 + \tau_{xy}^2} \implies \sigma_1 = \frac{\sigma_x}{2} + \sqrt{\frac{\sigma_x^2}{4} + \tau_{xy}^2}
    \]
    folgt
    \[
        R_m = \sigma_1 = \frac{\frac{2\tau_{xy}}{\tan\parentheses*{2\varphi^*}}}{2} + \sqrt{\frac{\parentheses*{\frac{2\tau_{xy}}{\tan\parentheses*{2\varphi^*}}}^2}{4} + \tau_{xy}^2} = \tau_{xy}\parentheses*{\frac{1}{\tan\parentheses*{2\varphi^*}} + \sqrt{\frac{1}{\tan^2\parentheses*{2\varphi^*}}} + 1}
    \]
    und somit gilt für das Torsionsmoment
    \[
        M_T = \frac{\pi d^3}{16}\tau_{xy} = \frac{\pi d^3}{16}\frac{R_m}{\frac{1}{\tan\parentheses*{2\varphi^*}} + \sqrt{\frac{1}{\tan^2\parentheses*{2\varphi^*}} + 1}} = 1933,7\sis{\newton\meter}.
    \]
    Die entsprechende Zugkraft ergibt sich aus
    \[
        F_N = \sigma_x A = \frac{2\tau_{xy}}{\tan\parentheses*{2\varphi^*}}\frac{\pi d^2}{4} = \frac{8M_T}{d\tan\parentheses*{2\varphi^*}} = 248,4\sis{\kilo\newton}.
    \]
\end{document}