\documentclass{exercise}

\institute{Institut für Statistik und Wirtschaftsmathematik}
\title{Golbalübung 2}
\author{Joshua Feld, 406718}
\course{Statistik}
\professor{Cramer}
\semester{Sommersemester 2022}
\program{CES (Bachelor)}

\begin{document}
    \maketitle


    \section*{Aufgabe 1}

    \begin{problem}
        Gegeben sei die Situation aus Aufgabe 4 der ersten Globalübung.
        \begin{enumerate}
            \item Bestimmen Sie die empirische Verteilungsfunktion \(F_{15}\) zu den gegebenen Beobachtungswerten \(x_1, \ldots, x_{15}\) und stellen Sie diese graphisch dar.
            \item Kann man den bereits in Aufgabe 4 b) bestimmten Modalwert direkt aus dem Graphen der empirischen Verteilungsfunktion ablesen?
            \item Berechnen Sie mithilfe der empirischen Verteilungsfunktion den Anteil der Teilnehmer(innen), die
            \begin{enumerate}
                \item höchstens vier Fragen,
                \item mehr als zwei Fragen,
                \item mindestens drei, aber höchstens fünf Fragen
            \end{enumerate}
            richtig beantwortet haben.
            \item Bestimmen Sie den Median zu den gegebenen Beobachtungswerten \(x_1, \ldots, x_{15}\) und vergleichen Sie ihn mit dem Modalwert.
            \item Berechnen Sie das untere und obere Quartil sowie das zweite Dezentil zu den gegebenen Beobachtungswerten \(x_1, \ldots, x_{15}\).
        \end{enumerate}
    \end{problem}

    \subsection*{Lösung}
    \begin{enumerate}
        \item
        \item
        \item
        \begin{enumerate}
            \item
            \item 
            \item
        \end{enumerate}
        \item
        \item 
    \end{enumerate}


    \section*{Aufgabe 2}

    \begin{problem}
        Gegeben seien \(a, b \in \R\), ein metrischer Datensatz \(x_1, \ldots, x_n \in \R\) und es seien \(y_1, \ldots, y_n\) gegeben durch die (affin-)lineare Transformation
        \[
            y_i = a + bx_i, \quad i \in \braces*{1, \ldots, n}.
        \]
        Weiter bezeichnen \(\bar{x}, \bar{y}\) die arithmetischen Mittelwerte, \(s_x^2, s_y^2\) die empirischen Varianzen und \(s_x, s_y\) die empirischen Standardabweichungen zu \(x_1, \ldots, x_n\) bzw. \(y_1, \ldots, y_n\).
        Zeigen Sie:
        \begin{enumerate}
            \item \(\bar{y} = a + b\bar{x}\),
            \item \(s_y^2 = b^2 s_x^2\),
            \item \(s_y = \absolute*{b}s_x\). 
        \end{enumerate}
    \end{problem}

    \subsection*{Lösung}
    \begin{enumerate}
        \item
        \item
        \item
    \end{enumerate}


    \section*{Aufgabe 3}

    \begin{problem}
        Bei einem Test, der mit den 32 Schülerinnen und Schülern einer Klasse durchgeführt wurde, ergaben sich folgende Messwerte für die zum Lösen einer Rechenaufgabe benötigte Zeit (in Sekunden):
        \begin{align*}
            x_1 &= 7,2, & x_2 &= 5,6, & x_3 &= 2,9, & x_4 &= 4,1, & x_5 &= 3,9, & x_6 &= 9,1, & x_7 &= 5,2, & x_8 &= 5,2,\\
            x_9 &= 2,8, & x_{10} &= 5,1, & x_{11} &= 6,7, & x_{12} &= 8,1, & x_{13} &= 4,1, & x_{14} &= 4,8, & x_{15} &= 8,9, & x_{16} &= 3,0,\\
            x_{17} &= 7,7, & x_{18} &= 6,7, & x_{19} &= 8,0, & x_{20} &= 4,6, & x_{21} &= 4,8, & x_{22} &= 5,0, & x_{23} &= 7,0, & x_{24} &= 5,3,\\
            x_{25} &= 6,6, & x_{26} &= 8,8, & x_{27} &= 5,9, & x_{28} &= 6,8, & x_{29} &= 8,9, & x_{30} &= 6,1, & x_{31} &= 5,4, & x_{32} &= 4,7.
        \end{align*}
        \begin{enumerate}
            \item Berechnen Sie zu diesem Datensatz die zugehörigen Modalwerte, das \(5\%\)-Quantil, das \(90\%\)-Quantil, den arithmetischen Mittelwert, die Spannweite, den Quartilsabstand, die mittlere absolute Abweichung, die empirische Varianz, die empirische Standardabweichung und den Variationskoeffizienten.
            \item Erstellen Sie ein Box-Plot zu dem gegebenen Datensatz.
            \item Erstellen Sie ein Histogramm der gemessenen Zeiten zur folgenden Klasseneinteilung:
            \[
                K_1 = \brackets*{1,5, 2,5}, \quad K_2 = \left(2,5, 3,5\right], \quad K_3 = \left(3,5, 4,5\right], \quad K_4 = \left(4,5, 5,5\right], \quad K_5 = \left(5,5, 6,5\right],
            \]
            \[
                K_6 = \brackets*{6,5, 7}, \quad K_7 = \left(7, 7,5\right], \quad K_8 = \left(7,5, 8,5\right], \quad K_9 = \left(8,5, 9,5\right].
            \]
        \end{enumerate}
    \end{problem}

    \subsection*{Lösung}
    \begin{enumerate}
        \item
        \item
        \item 
    \end{enumerate}
\end{document}
