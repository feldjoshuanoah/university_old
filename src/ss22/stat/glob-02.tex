\documentclass{exercise}

\institute{Institut für Statistik und Wirtschaftsmathematik}
\title{Globalübung 2}
\author{Joshua Feld, 406718}
\course{Statistik}
\professor{Cramer}
\semester{Sommersemester 2022}
\program{CES (Bachelor)}

\begin{document}
    \maketitle


    \section*{Aufgabe 1}

    \begin{problem}
        Gegeben sei die Situation aus Aufgabe 4 der ersten Globalübung.
        \begin{enumerate}
            \item Bestimmen Sie die empirische Verteilungsfunktion \(F_{15}\) zu den gegebenen Beobachtungswerten \(x_1, \ldots, x_{15}\) und stellen Sie diese graphisch dar.
            \item Kann man den bereits in Aufgabe 4 b) bestimmten Modalwert direkt aus dem Graphen der empirischen Verteilungsfunktion ablesen?
            \item Berechnen Sie mithilfe der empirischen Verteilungsfunktion den Anteil der Teilnehmer(innen), die
            \begin{enumerate}
                \item höchstens vier Fragen,
                \item mehr als zwei Fragen,
                \item mindestens drei, aber höchstens fünf Fragen
            \end{enumerate}
            richtig beantwortet haben.
            \item Bestimmen Sie den Median zu den gegebenen Beobachtungswerten \(x_1, \ldots, x_{15}\) und vergleichen Sie ihn mit dem Modalwert.
            \item Berechnen Sie das untere und obere Quartil sowie das zweite Dezentil zu den gegebenen Beobachtungswerten \(x_1, \ldots, x_{15}\).
        \end{enumerate}
    \end{problem}

    \subsection*{Lösung}
    \begin{enumerate}
        \item Zur Bestimmung der empirischen Verteilungsfunktion \(F_{15}\) erweitern wir die Häufigkeitstabelle aus Aufgabe A b) um die zugehörigen kumulierten relativen Häufigkeiten:
        \begin{center}
            \begin{tabular}{lcccccccc}
                \toprule
                & \(j\) & \(1\) & \(2\) & \(3\) & \(4\) & \(5\) & \(6\) & \(7\)\\
                \midrule
                Beobachtungswert & \(u_{\parentheses*{j}}\) & \(2\) & \(3\) & \(4\) & \(5\) & \(6\) & \(7\) & \(8\)\\
                Absolute Häufigkeit & \(n_{\parentheses*{j}}\) & \(2\) & \(1\) & \(4\) & \(2\) & \(3\) & \(2\) & \(1\)\\
                Relative Häufigkeit & \(f_{\parentheses*{j}}\) & \(\frac{2}{15}\) & \(\frac{1}{15}\) & \(\frac{4}{15}\) & \(\frac{2}{15}\) & \(\frac{1}{5}\) & \(\frac{2}{15}\) & \(\frac{1}{15}\)\\
                \makecell[l]{Kumulierte relative\\Häufigkeit} & \(f_{\parentheses*{1}} + \cdots + f_{\parentheses*{j}}\) & \(\frac{2}{15}\) & \(\frac{1}{5}\) & \(\frac{7}{15}\) & \(\frac{3}{5}\) & \(\frac{4}{5}\) & \(\frac{14}{15}\) & \(1\)\\
                \bottomrule
            \end{tabular}
        \end{center}
        Dann kann man die Funktionswerte von \(F_{15}\) in der letzten Zeile dieser Tabelle ablesen (gemäß der Definition aus der Vorlesung).
        Wir erhalten dann
        \begin{equation}\label{eq:1}
            F_{15}\parentheses*{x} = \begin{cases}
                0, & \text{falls }x < 2,\\
                \frac{2}{15}, & \text{falls }2 \le x < 3,\\
                \frac{1}{5}, & \text{falls }3 \le x < 4,\\
                \frac{7}{15}, & \text{falls }4 \le x < 5,\\
                \frac{3}{5}, & \text{falls }5 \le x < 6,\\
                \frac{4}{5}, & \text{falls }6 \le x < 7,\\
                \frac{14}{15}, & \text{falls }7 \le x < 8,\\
                1, & \text{falls }8 \le x.\\
            \end{cases}
        \end{equation}
        \begin{center}
            \begin{tikzpicture}
                \draw[->] (0,0) -- (9.5,0) node[below right] {\(x\)};
                \draw[->] (0,0) -- (0,8) node[above left] {\(F_{15}\parentheses*{x}\)};
                \draw (0,0) -- (2,0);
                \draw (2,1) -- (3,1);
                \draw (3,1.5) -- (4,1.5);
                \draw (4,3.5) -- (5,3.5);
                \draw (5,4.5) -- (6,4.5);
                \draw (6,6) -- (7,6);
                \draw (7,7) -- (8,7);
                \draw (8,7.5) -- (9.5,7.5);
                \foreach \i in {0,...,9}
                {
                    \draw (\i,.1) -- (\i,-.1) node[below] {\(\i\)};
                }
                \foreach \i in {0,.5,...,7.5}
                {
                    \draw (.1,\i) -- (-.1,\i);
                }
                \node[anchor=east] at (-.1,0) {\(0\)};
                \node[anchor=east] at (-.1,1.5) {\(0,2\)};
                \node[anchor=east] at (-.1,3) {\(0,4\)};
                \node[anchor=east] at (-.1,4.5) {\(0,6\)};
                \node[anchor=east] at (-.1,6) {\(0,8\)};
                \node[anchor=east] at (-.1,7.5) {\(1\)};
                \draw[fill=black] (2,1) circle (.5mm);
                \draw[fill=black] (3,1.5) circle (.5mm);
                \draw[fill=black] (4,3.5) circle (.5mm);
                \draw[fill=black] (5,4.5) circle (.5mm);
                \draw[fill=black] (6,6) circle (.5mm);
                \draw[fill=black] (7,7) circle (.5mm);
                \draw[fill=black] (8,7.5) circle (.5mm);
                \draw[<->,dashed] (4,1.5) -- (4,3.5) node[midway,right,align=left] {Maximale Sprunghöhe:\\\(F_{15}\parentheses*{4} - F_{15}\parentheses*{3} = \frac{4}{15}\)};
            \end{tikzpicture}
        \end{center}
        \item Man kann Modalwerte direkt aus dem Graphen einer empirischen Verteilungsfunktion ablesen, denn gemäß Vorlesung gilt
        \begin{align*}
            \text{Modalwert} &\stackrel{\text{Def.}}{=} \text{Auspräung mit maximaler relativer Häufigkeit}\\
            &\stackrel{\text{Def.}}{=} \text{Ausprägung mit maximaler Sprunghöhe in }F_n\parentheses*{x}.
        \end{align*}
        Hier erhält man (vgl. Aufgabe 4 b)):
        \begin{align*}
            \max\braces*{f_{\parentheses*{1}}, \ldots, f_{\parentheses*{7}}} &\stackrel{\text{Def.}}{=} \max\braces*{F_{15}\parentheses*{u_{\parentheses*{1}}}, F_{15}\parentheses*{u_{\parentheses*{2}}} - F_{15}\parentheses*{u_{\parentheses*{1}}}, \ldots, F_{15}\parentheses*{u_{\parentheses*{7}}} - F_{15}\parentheses*{u_{\parentheses*{6}}}}\\
            &\stackrel{\text{Graph}}{=} \underbrace{F_{15}\parentheses*{4} - F_{15}\parentheses*{3}}_{= F_{15}\parentheses*{u_{\parentheses*{3}}} - F_{15}\parentheses*{u_{\parentheses*{2}}} = f_{\parentheses*{3}}}\\
            &\stackrel{\eqref{eq:1}}{=} \frac{7}{15} - \frac{1}{5} = \frac{4}{15}.
        \end{align*}
        Damit ist hier \(x_{\text{mod}} = u_{\parentheses*{3}} = 4\) der einzige Modalwert (wie bereits in Aufgabe 4 b) festgestellt).
        \item Gemäß Definition gibt \(F_{15}\parentheses*{x}\) für \(x \in \R\) den Anteil der Quiz-Teilnehmer(innen) an, die höchstens \(x\) Fragen (genauer \(\floor*{x}\) Fragen) richtig beantwortet haben.
        Damit beträgt der Anteil der Quiz-Teinehmer(innen), die
        \begin{enumerate}
            \item höchstens vier Fragen richtig beantwortet haben
            \[
                F_{15}\parentheses*{4} = \frac{7}{15}
            \]
            \item mehr als zwei Fragen richtig beantwortet haben
            \[
                1 - F_{15}\parentheses*{2} = 1 - \frac{2}{15} = \frac{13}{15}
            \]
            \item mindestens drei aber höchstens fünf Fragen richtig beantwortet haben
            \[
                F_{15}\parentheses*{5} - F_{15}\parentheses*{2} = \frac{3}{5} - \frac{2}{15} = \frac{7}{15}
            \]
            (da ``mindestens drei'' hier gleichbedeutend mit ``mehr als zwei'' ist).
        \end{enumerate}
    \end{enumerate}
    Zur Bestimmung des Medians und der Quantile wird (nochmals) die Rangwertreihe zu \(x_1, \ldots x_{15}\) benötigt, die bereits in Aufgabe 4 a) erstellt wurde.
    \[
        x_{\parentheses*{1}} = 2, x_{\parentheses*{2}} = 2, x_{\parentheses*{3}} = 3, x_{\parentheses*{4}} = 4, x_{\parentheses*{5}} = 4, x_{\parentheses*{6}} = 4, x_{\parentheses*{7}} = 4, x_{\parentheses*{8}} = 5,
    \]
    \[
        x_{\parentheses*{9}} = 5, x_{\parentheses*{10}} = 6, x_{\parentheses*{11}} = 6, x_{\parentheses*{12}} = 6, x_{\parentheses*{13}} = 7, x_{\parentheses*{14}} = 7, x_{\parentheses*{15}} = 8.
    \]
    \begin{enumerate}
        \item[d)]
        \[
            \tilde{x} = \tilde{x}_{0,5} = x_{\parentheses*{\frac{n + 1}{2}}} = x_{\parentheses*{8}} = 5 > 4 = x_{\text{mod}}.
        \]
        \item[e)] Unteres Quartil: \(\frac{n}{4} = \frac{15}{4} = 3,75 \not\in \N\), \(\frac{n}{4} = 3,75 < 4 < \frac{n}{4} + 1\), also
        \[
            \tilde{x}_{0,25} = x_{\parentheses*{4}} = 4.
        \]
        Oberes Quartil: \(\frac{3n}{4} = \frac{45}{4} = 11,25 \not\in \N\), \(\frac{3n}{4} = 11,25 < 12 < \frac{3n}{4} + 1\), also
        \[
            \tilde{x}_{0,75} = x_{\parentheses*{12}} = 6.
        \]
        Zweites Dezentil: \(\frac{2n}{10} = \frac{30}{10} = 3 \in \N\), also
        \[
            \tilde{x}_{0,2} = \frac{1}{2}\parentheses*{x_{\parentheses*{\frac{2n}{10}}} + x_{\parentheses*{\frac{2n}{10} + 1}}} = \frac{1}{2}\parentheses*{x_{\parentheses*{3}} + x_{\parentheses*{4}}} = \frac{1}{2}\parentheses*{3 + 4} = 3,5.
        \]
    \end{enumerate}


    \section*{Aufgabe 2}

    \begin{problem}
        Gegeben seien \(a, b \in \R\), ein metrischer Datensatz \(x_1, \ldots, x_n \in \R\) und es seien \(y_1, \ldots, y_n\) gegeben durch die (affin-)lineare Transformation
        \[
            y_i = a + bx_i, \quad i \in \braces*{1, \ldots, n}.
        \]
        Weiter bezeichnen \(\bar{x}, \bar{y}\) die arithmetischen Mittelwerte, \(s_x^2, s_y^2\) die empirischen Varianzen und \(s_x, s_y\) die empirischen Standardabweichungen zu \(x_1, \ldots, x_n\) bzw. \(y_1, \ldots, y_n\).
        Zeigen Sie:
        \begin{enumerate}
            \item \(\bar{y} = a + b\bar{x}\),
            \item \(s_y^2 = b^2 s_x^2\),
            \item \(s_y = \absolute*{b}s_x\). 
        \end{enumerate}
    \end{problem}

    \subsection*{Lösung}
    \begin{enumerate}
        \item Es gilt
        \[
            \bar{y} = \frac{1}{n}\sum_{i = 1}^n y_i = \frac{1}{n}\sum_{i = 1}^n \parentheses*{a + bx_i} = \frac{1}{n}\sum_{i = 1}^n a + b\underbrace{\frac{1}{n}\sum_{i = 1}^n x_i}_{= \bar{x}} = a + b\bar{x}.
        \]
        \item Mit der in a) hergeleiteten Darstellung für \(\bar{y}\) folgt
        \begin{align*}
            s_y^2 &= \frac{1}{n}\sum_{i = 1}^n \parentheses*{y_i - \bar{y}}^2 = \frac{1}{n}\sum_{i = 1}^n\parentheses*{a + bx_i - \parentheses*{a + b\bar{x}}}^2 = \frac{1}{n}\sum_{i = 1}^n\parentheses*{bx_i - b\bar{x}}^2 = b^2\frac{1}{n}\sum_{i = 1}^n\parentheses*{x_i - \bar{x}}^2 = b^2 s_x^2.
        \end{align*}
        \item Aus b) erhält man
        \[
            s_y = \sqrt{s_y^2} = \sqrt{b^2 s_x^2} = \sqrt{b^2}\sqrt{s_x^2} = \absolute*{b}s_x.
        \]
    \end{enumerate}


    \section*{Aufgabe 3}

    \begin{problem}
        Bei einem Test, der mit den 32 Schülerinnen und Schülern einer Klasse durchgeführt wurde, ergaben sich folgende Messwerte für die zum Lösen einer Rechenaufgabe benötigte Zeit (in Sekunden):
        \begin{align*}
            x_1 &= 7,2, & x_2 &= 5,6, & x_3 &= 2,9, & x_4 &= 4,1, & x_5 &= 3,9, & x_6 &= 9,1, & x_7 &= 5,2, & x_8 &= 5,2,\\
            x_9 &= 2,8, & x_{10} &= 5,1, & x_{11} &= 6,7, & x_{12} &= 8,1, & x_{13} &= 4,1, & x_{14} &= 4,8, & x_{15} &= 8,9, & x_{16} &= 3,0,\\
            x_{17} &= 7,7, & x_{18} &= 6,7, & x_{19} &= 8,0, & x_{20} &= 4,6, & x_{21} &= 4,8, & x_{22} &= 5,0, & x_{23} &= 7,0, & x_{24} &= 5,3,\\
            x_{25} &= 6,6, & x_{26} &= 8,8, & x_{27} &= 5,9, & x_{28} &= 6,8, & x_{29} &= 8,9, & x_{30} &= 6,1, & x_{31} &= 5,4, & x_{32} &= 4,7.
        \end{align*}
        \begin{enumerate}
            \item Berechnen Sie zu diesem Datensatz die zugehörigen Modalwerte, das \(5\%\)-Quantil, das \(90\%\)-Quantil, den arithmetischen Mittelwert, die Spannweite, den Quartilsabstand, die mittlere absolute Abweichung, die empirische Varianz, die empirische Standardabweichung und den Variationskoeffizienten.
            \item Erstellen Sie ein Box-Plot zu dem gegebenen Datensatz.
            \item Erstellen Sie ein Histogramm der gemessenen Zeiten zur folgenden Klasseneinteilung:
            \[
                K_1 = \brackets*{1,5, 2,5}, \quad K_2 = \left(2,5, 3,5\right], \quad K_3 = \left(3,5, 4,5\right], \quad K_4 = \left(4,5, 5,5\right], \quad K_5 = \left(5,5, 6,5\right],
            \]
            \[
                K_6 = \brackets*{6,5, 7}, \quad K_7 = \left(7, 7,5\right], \quad K_8 = \left(7,5, 8,5\right], \quad K_9 = \left(8,5, 9,5\right].
            \]
        \end{enumerate}
    \end{problem}

    \subsection*{Lösung}
    \begin{enumerate}
        \item Zunächst erstellen wir die zugehörige Rangwertreihe (benötigt zur Bestimmung von Minimum, Maximum und empirischen Quantilen):
        \begin{align*}
            x_{\parentheses*{1}} &= 2,8, & x_{\parentheses*{2}} &= 2,9, & x_{\parentheses*{3}} &= 3, & x_{\parentheses*{4}} &= 3,9, & x_{\parentheses*{5}} &= 4,1, & x_{\parentheses*{6}} &= 4,1,\\
            x_{\parentheses*{7}} &= 4,6, & x_{\parentheses*{8}} &= 4,7, & x_{\parentheses*{9}} &= 4,8, & x_{\parentheses*{10}} &= 4,8, & x_{\parentheses*{11}} &= 5, & x_{\parentheses*{12}} &= 5,1,\\
            x_{\parentheses*{13}} &= 5,2, & x_{\parentheses*{14}} &= 5,2, & x_{\parentheses*{15}} &= 5,3, & x_{\parentheses*{16}} &= 5,4, & x_{\parentheses*{17}} &= 5,6, & x_{\parentheses*{18}} &= 5,9,\\
            x_{\parentheses*{19}} &= 6,1, & x_{\parentheses*{20}} &= 6,6, & x_{\parentheses*{21}} &= 6,7, & x_{\parentheses*{22}} &= 6,7, & x_{\parentheses*{23}} &= 6,8, & x_{\parentheses*{24}} &= 7,\\
            x_{\parentheses*{25}} &= 7,2, & x_{\parentheses*{26}} &= 7,7, & x_{\parentheses*{27}} &= 8, & x_{\parentheses*{28}} &= 8,1, & x_{\parentheses*{29}} &= 8,8, & x_{\parentheses*{30}} &= 8,9,\\
            x_{\parentheses*{31}} &= 8,9, & x_{\parentheses*{32}} &= 9,1.
        \end{align*}
        Nun berechnen wir die einzelnen Kenngrößen.
        Hierbei beachte man, dass die gemessenen Zeiten Ausprägungen eines stetigen Merkmals sind.

        Modalwerte: An der Rangwertreihe liest man ab, dass folgende Werte jeweils doppelt vorkommen:
        \[
            4,1, \quad 4,8, \quad 5,2, \quad 6,7, \quad 8,9.
        \]
        Dies sind die Modalwerte, da die übrigen gemessenen Zeiten jeweils nur einmal vorkommen.
        (Insbesondere ist der Modalwert damit \emph{nicht} eindeutig bestimmt.)

        \(5\%\)-Quantil:
        \(\frac{5n}{100} = \frac{160}{100} = 1,6 \not\in \N\), \(\frac{5n}{100} = 1,6 < 2 < \frac{5n}{100} + 1\), also
        \[
            \tilde{x}_{0,05} = x_{\parentheses*{2}} = 2,9.
        \]

        \(90\%\)-Quantil:
        \(\frac{9n}{10} = \frac{288}{10} = 28,8 \not\in \N\), \(\frac{9n}{10} = 28,8 < 29 < \frac{9n}{10} + 1\), also
        \[
            \tilde{x}_{0,9} = x_{\parentheses*{29}} = 8,8.
        \]

        Arithmetischer Mittelwert:
        \[
            \bar{x} = \frac{1}{32}\sum_{i = 1}^{32}x_i = \frac{1}{32}\sum_{i = 1}^{32}x_{\parentheses*{i}} = \frac{189}{32} \approx 5,906
        \]

        Spannweite:
        \[
            R = x_{\parentheses*{32}} - x_{\parentheses*{1}} = 9,1 - 2,8 = 6,3.
        \]

        Quartilsabstand: \(\frac{n}{4} = \frac{32}{4} = 8 \in \N\), \(\frac{3n}{4} = \frac{96}{4} = 24 \in \N\), also
        \begin{align*}
            \tilde{x}_{0,25} &= \frac{1}{2}\parentheses*{x_{\parentheses*{\frac{n}{4}}} + x_{\parentheses*{\frac{n}{4} + 1}}} = \frac{1}{2}\parentheses*{x_{\parentheses*{8}} + x_{\parentheses*{9}}} = \frac{1}{2} \cdot \parentheses*{4,7 + 4,8} = 4,75,\\
            \tilde{x}_{0,75} &= \frac{1}{2}\parentheses*{x_{\parentheses*{\frac{3n}{4}}} + x_{\parentheses*{\frac{3n}{4} + 1}}} = \frac{1}{2}\parentheses*{x_{\parentheses*{24}} + x_{\parentheses*{25}}} = \frac{1}{2} \cdot \parentheses*{7 + 7,2} = 7,1
        \end{align*}
        und somit
        \[
            Q = \tilde{x}_{0,75} - \tilde{x}_{0,25} = 7,1 - 4,75 = 2,35.
        \]
        Mittlere absolute Abweichung: Hierzu müssen wir zunächst den Median berechnen:
        \[
            \tilde{x} = \frac{1}{2}\parentheses*{x_{\parentheses*{\frac{n}{2}}} + x_{\parentheses*{\frac{n}{2}} + 1}} = \frac{1}{2}\parentheses*{x_{\parentheses*{16}} + x_{\parentheses*{17}}} = \frac{1}{2} \cdot \parentheses*{5,4 + 5,6} = 5,5.
        \]
        Nun können wir die absoluten Abweichungen
        \[
            \absolute*{x_{\parentheses*{i}} - \tilde{x}}, \quad i \in \braces*{1, \ldots, 32}
        \]
        berechnen:
        \begin{center}
            \begin{tabular}{lcccccccccccccccc}
                \toprule
                \(i\) & \(1\) & \(2\) & \(3\) & \(4\) & \(5\) & \(6\) & \(7\) & \(8\) & \(9\) & \(10\) & \(11\) & \(12\) & \(13\) & \(14\) & \(15\) & \(16\)\\
                \midrule
                \(\absolute*{x_{\parentheses*{i}} - \tilde{x}}\) & \(2,7\) & \(2,6\) & \(2,5\) & \(1,6\) & \(1,4\) & \(1,4\) & \(0,9\) & \(0,8\) & \(0,7\) & \(0,7\) & \(0,5\) & \(0,4\) & \(0,3\) & \(0,3\) & \(0,2\) & \(0,1\)\\
                \bottomrule
            \end{tabular}
            \begin{tabular}{lcccccccccccccccc}
                \toprule
                \(i\) & \(17\) & \(18\) & \(19\) & \(20\) & \(21\) & \(22\) & \(23\) & \(24\) & \(25\) & \(26\) & \(27\) & \(28\) & \(29\) & \(30\) & \(31\) & \(32\)\\
                \midrule
                \(\absolute*{x_{\parentheses*{i}} - \tilde{x}}\) & \(0,1\) & \(0,4\) & \(0,6\) & \(1,1\) & \(1,2\) & \(1,2\) & \(1,3\) & \(1,5\) & \(1,7\) & \(2,2\) & \(2,5\) & \(2,6\) & \(3,3\) & \(3,4\) & \(3,4\) & \(3,6\)\\
                \bottomrule
            \end{tabular}
        \end{center}
        Damit erhält man
        \[
            d = \frac{1}{32}\sum_{i = 1}^{32}\absolute*{x_i - \tilde{x}} = \frac{1}{32}\sum_{i = 1}^{32}\absolute*{x_{\parentheses*{i}} - \tilde{x}} = \frac{47,2}{32} = 1,475.
        \]

        Empirische Varianz:
        \[
            s^2 = \frac{1}{32}\sum_{i = 1}^{32}\parentheses*{x_i - \bar{x}}^2 = \frac{1}{32}\sum_{i = 1}^{32}x_i^2 - \bar{x}^2 = \frac{1}{32}\sum_{i = 1}^{32}x_{\parentheses*{i}}^2 - \bar{x}^2 = \frac{1217,12}{32} - \parentheses*{\frac{189}{32}}^2 \approx 3,151.
        \]

        Empirische Standardabweichung:
        \[
            s = \sqrt{s^2} = \sqrt{3,151} \approx 1,775.
        \]

        Variationskoeffizient:
        \[
            V = \frac{s}{\bar{x}} = \frac{1,775 \cdot 32}{189} \approx 0,301.
        \]
        \item \,
        \begin{center}
            \begin{tikzpicture}
                \draw (0,2) -- (5,2);
                \draw (0,2) -- (0,10);
                \foreach \i in {2,2.1,...,10}
                {
                    \draw (.05,\i) -- (-.05,\i);
                }
                \foreach \i in {2,...,10}
                {
                    \draw (.1,\i) -- (-.1,\i) node[left] {\(\i\)};
                }
                \draw (1,2.8) -- (3,2.8);
                \draw (1,9.1) -- (3,9.1);
                \draw (.5,4.75) rectangle (3.5,7.1);
                \draw[thick] (.5,5.5) -- (3.5,5.5);
                \draw[dashed] (2,2.8) -- (2,4.75);
                \draw[dashed] (2,9.1) -- (2,7.1);
                \node[anchor=west] at (3.6,2.8) {\(x_{\parentheses*{1}}\)};
                \node[anchor=west] at (3.6,4.75) {\(\tilde{x}_{0,25}\)};
                \node[anchor=west] at (3.6,5.5) {\(\tilde{x}\)};
                \node[anchor=west] at (3.6,7.1) {\(\tilde{x}_{0,75}\)};
                \node[anchor=west] at (3.6,9.1) {\(x_{\parentheses*{32}}\)};
            \end{tikzpicture}
        \end{center}
        \item Wir führen die folgenden Schritte durch:
        \begin{enumerate}
            \item Auszählen der absoluten Klassenhäufigkeiten \(n\parentheses*{K_j}, j \in \braces*{1, \ldots, 9}\).
            \item Berechnung der relativen Klassenhäufigkeiten \(f\parentheses*{K_j} = \frac{n\parentheses*{K_j}}{n}, j \in \braces*{1, \ldots, 9}\) mit \(n = 32\).
            \item Berechnung der Rechteckhöhen des Histogramms gemäß
            \[
                h_j \cdot b_h = f\parentheses*{K_j} \iff h_j = \frac{f\parentheses*{K_j}}{b_j}, \quad j \in \braces*{1, \ldots, 9}.
            \]
        \end{enumerate}
        In der folgenden Tabelle sind die entsprechenden Werte dargestellt:
        \begin{center}
            \begin{tabular}{cccccc}
                \toprule
                Klassen-Nr. & Klasse & \makecell{Absolute\\Klassenhäufigkeit} & \makecell{Relative\\Häufigkeit} & Klassenbreite & Rechteckhöhe\\
                \(j\) & \(K_j\) & \(n\parentheses*{K_j}\) & \(f\parentheses*{K_j}\) & \(b_j\) & \(h_j\)\\
                \midrule
                \(1\) & \(\brackets*{1,5, 2,5}\) & \(0\) & \(0\) & \(1\) & \(0\)\\
                \(2\) & \(\left(2,5, 3,5\right]\) & \(3\) & \(\frac{3}{32}\) & \(1\) & \(\frac{3}{32}\)\\
                \(3\) & \(\left(3,5, 4,5\right]\) & \(3\) & \(\frac{3}{32}\) & \(1\) & \(\frac{3}{32}\)\\
                \(4\) & \(\left(4,5, 5,5\right]\) & \(10\) & \(\frac{5}{16}\) & \(1\) & \(\frac{5}{16}\)\\
                \(5\) & \(\left(5,5, 6,5\right]\) & \(3\) & \(\frac{3}{32}\) & \(1\) & \(\frac{3}{32}\)\\
                \(6\) & \(\left(6,5, 7\right]\) & \(5\) & \(\frac{5}{32}\) & \(1\) & \(\frac{5}{16}\)\\
                \(7\) & \(\left(7, 7,5\right]\) & \(1\) & \(\frac{1}{32}\) & \(1\) & \(\frac{1}{16}\)\\
                \(8\) & \(\left(7,5, 8,5\right]\) & \(3\) & \(\frac{3}{32}\) & \(1\) & \(\frac{3}{32}\)\\
                \(9\) & \(\left(8,5, 9,5\right]\) & \(4\) & \(\frac{1}{8}\) & \(1\) & \(\frac{1}{8}\)\\
                \bottomrule
            \end{tabular}
        \end{center}
        Damit können wir nun das Histogramm der benötigten Zeiten der 32 Schülerinnen und Schülern zum Lösen der Rechenaufgabe zeichnen.
        \begin{center}
            \begin{tikzpicture}
                \draw (1.5,0) -- (10,0);
                \draw (1.5,0) -- (1.5,10);

                \draw (1.5,0) -- (1.4,0) node[left] {\(0\)};
                \foreach \i in {1,...,10}
                {
                    \draw (1.6,\i) -- (1.4,\i) node[left] {\(\frac{\i}{32}\)};
                }
                \foreach \i in {1.5,2,...,10}
                {
                    \draw (\i,.05) -- (\i,-.05);
                }
                \foreach \i in {2,...,10}
                {
                    \draw (\i,.1) -- (\i,-.1) node[below] {\(\i\)};
                }
                \draw (2.5,0) rectangle (3.5,3);
                \draw (3.5,0) rectangle (4.5,3);
                \draw (4.5,0) rectangle (5.5,10);
                \draw (5.5,0) rectangle (6.5,3);
                \draw (6.5,0) rectangle (7,10);
                \draw (7,0) rectangle (7.5,2);
                \draw (7.5,0) rectangle (8.5,3);
                \draw (8.5,0) rectangle (9.5,4);
            \end{tikzpicture}
        \end{center}
    \end{enumerate}
\end{document}
