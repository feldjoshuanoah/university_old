\documentclass{exercise}

\institute{Institut für Statistik und Wirtschaftsmathematik}
\title{Globalübung 3}
\author{Joshua Feld, 406718}
\course{Statistik}
\professor{Cramer}
\semester{Sommersemester 2022}
\program{CES (Bachelor)}

\begin{document}
    \maketitle


    \section*{Aufgabe 1}

    \begin{problem}
        In einer medizinischen Untersuchung wird der gesundheitliche Zustand von Rauchern und Nichtrauchern verglichen.
        Dabei werden die an der Untersuchung teilnehmenden Personen in Männer und Frauen sowie in Raucher und Nichtraucher eingeteilt.
        Die Menge der Männer wird mit \(A\) bezeichnet, die Menge der Raucher mit \(B\).
        
        Welche Personen werden durch die folgenden Mengen beschrieben?
        \begin{enumerate}
            \item \(B^c\),
            \item \(A \cap B\),
            \item \(B \setminus A\),
            \item \(\parentheses*{A \cup B}^c\),
            \item \(\parentheses*{A \cup B^c} \cap \parentheses*{A^c \cup B}\).
        \end{enumerate}
    \end{problem}

    \subsection*{Lösung}
    Gemäß der Aufgabenstellung beschreiben
    \begin{enumerate}
        \item \(B^c\) die Menge der Nichtraucher,
        \item \(A \cap B\) die Menge der männlichen Raucher,
        \item \(B \setminus A = B \cap A^c\) die Menge der weiblichen Raucher,
        \item \(\parentheses*{A \cup B}^c = A^c \cap B^c\) die Menge der weiblichen Nichtraucher.
        \item Zunächst gilt
        \begin{align*}
            \parentheses*{A \cup B^c} \cap \parentheses*{A^c \cup B} &= \parentheses*{A \cap \parentheses*{A^c \cup B}} \cup \parentheses*{B^c \cap \parentheses*{A^c \cup B}}\\
            &= \parentheses*{\parentheses*{A \cap A^c} \cup \parentheses*{A \cap B}} \cup \parentheses*{\parentheses*{B^c \cap A^c} \cup \parentheses*{B^c \cap B}}\\
            &= \parentheses*{A \cap B} \cup \parentheses*{A^c \cap B^c}.
        \end{align*}
        Damit beschreibt \(\parentheses*{A \cup B^c} \cap \parentheses*{A^c \cup B}\) die Menge der männlichen Raucher und weiblichen Nichtraucher (vgl. b) und d)).
    \end{enumerate}


    \section*{Aufgabe 2}

    \begin{problem}
        In einer Stellenausschreibung werden nach Möglichkeit deutsche, englische und französische Sprachkenntnisse verlangt.
        Von den insgesamt 250 Personen, die sich bewerben, sprechen 60 nur Deutsch, 85 nur Englisch und 25 nur Französisch.
        20 sprechen Englisch und Deutsch, aber kein Französisch, 12 sprechen Deutsch und Französisch, aber kein Englisch und 8 sprechen Englisch und Französisch, aber kein Deutsch.
        (Jede Person, die sich bewirbt, spricht mindestens eine der drei Sprachen.)
        Berechnen Sie die Wahrscheinlichkeit dafür, dass eine zufällig aus den Bewerbungen ausgewählte Person folgende Sprachen spricht:
        \begin{enumerate}
            \item Deutsch,
            \item Englisch,
            \item Französisch,
            \item alle drei Sprachen.
        \end{enumerate}
    \end{problem}

    \subsection*{Lösung}
    Es seien
    \begin{align*}
        D &\hat{=} \text{Menge der Deutsch sprechenden Personen},\\
        E &\hat{=} \text{Menge der Englisch sprechenden Personen},\\
        F &\hat{=} \text{Menge der Französisch sprechenden Personen}.
    \end{align*}
    Mit diesen Bezeichnungen gilt
    \begin{align*}
        D &= \parentheses*{D \cap E \cap F} \cup \parentheses*{D \cap E \cap F^c} \cup \parentheses*{D \cap E^c \cap F} \cup \parentheses*{D \cap E^c \cap F^c},\\
        E &= \parentheses*{D \cap E \cap F} \cup \parentheses*{D^c \cap E \cap F} \cup \parentheses*{D \cap E \cap F^c} \cup \parentheses*{D^c \cap E \cap F^c},\\
        F &= \parentheses*{D \cap E \cap F} \cup \parentheses*{D^c \cap E \cap F} \cup \parentheses*{D \cap E^c \cap F} \cup \parentheses*{D^c \cap E^c \cap F},\\
    \end{align*}
    Da eine disjunkte Zerlegung der Menge \(D \cup E \cup F\) vorliegt, erhält man hieraus mit den gemäß Aufgabenstellung gegebenen Anzahlen
    \[
        250 = \absolute*{D \cup E \cup F} = \absolute*{D \cap E \cap F} + 60 + 85 + 25 + 20 + 20 + 12 + 8 = \absolute*{D \cap E \cap F} + 210.
    \]
    Es folgt
    \[
        \absolute*{D \cap E \cap F} = 250 - 210 = 40.
    \]
    Hiermit ergeben sich aus den oben angegebenen disjunkten Zerlegungen der Mengen \(D\), \(E\) und \(F\) die folgenden Anzahlen:
    \[
        \absolute*{D} = 40 + 20 + 12 + 60 = 132, \quad \absolute*{E} = 40 + 20 + 8 + 85 = 153, \quad \absolute*{F} = 40 + 8 + 12 + 25 = 85.
    \]
    Da die Auswahl aus den Bewerbungen zufällig erfolgt, kann die Situation mithilfe eines Laplace-Raums modelliert werden.
    Gemäß Aufgabenstellung gilt für den zugehörigen Ergebnisraum \(\Omega\) (mit den o.a. Bezeichnungen):
    \[
        \Omega = D \cup E \cup F
    \]
    (Jede Person spricht mindestens eine der drei Sprachen).
    Mit \(P\) als (diskreter) Gleichverteilung auf \(\Omega\) und \(\absolute*{\Omega} = 250\) folgt dann:
    \begin{enumerate}
        \item \(P\parentheses*{D} = \frac{\absolute*{D}}{\absolute*{\Omega}} = \frac{132}{250} = 0,528\),
        \item \(P\parentheses*{E} = \frac{\absolute*{E}}{\absolute*{\Omega}} = \frac{153}{250} = 0,612\),
        \item \(P\parentheses*{F} = \frac{\absolute*{F}}{\absolute*{\Omega}} = \frac{85}{250} = 0,34\),
        \item \(P\parentheses*{D \cap E \cap F} = \frac{\absolute*{D \cap E \cap F}}{\absolute*{\Omega}} = \frac{40}{250} = 0,16\),
    \end{enumerate}


    \section*{Aufgabe 3}

    \begin{problem}
        Eine homogene Münze (d.h. die Wahrscheinlichkeiten für das Auftreten von Kopf und Zahl sind identisch) werde sieben Mal hintereinander geworfen.
        \begin{enumerate}
            \item Geben Sie einen geeigneten Wahrscheinlichkeitsraum für dieses Zufallsexperiment an.
            \item Wie viele mögliche Versuchsausgänge gibt es?
            \item Beschreiben Sie die nachstehenden Ereignisse als Teilmengen von \(\Omega\), und berechnen Sie ihre Wahrscheinlichkeiten:
            \begin{enumerate}
                \item Genau sechs der sieben Münzwürfe ergeben Zahl.
                \item Mindestens sechs der sieben Münzwürfe ergeben Zahl.
                \item In zwei aufeinander folgenden Würfen fällt Kopf, oder in zwei aufeinander folgenden Würfen fällt Zahl.
            \end{enumerate}
        \end{enumerate}
    \end{problem}

    \subsection*{Lösung}
    \begin{enumerate}
        \item Eine geeignete Grundmenge ist
        \[
            \Omega = \braces*{\parentheses*{\omega_1, \ldots, \omega_7} : \omega_1, \ldots, \omega_7 \in \braces*{0, 1}},
        \]
        mit folgender Interpretation:
        Für \(i \in \braces*{1, \ldots, 7}\) bedeutet \(\omega_i = 0\) ``Kopf'' und \(\omega_i = 1\) ``Zahl'' im \(i\)-ten Wurf.
        Da die Münze homogen sein soll, kann die Situation durch einen Laplace-Raum \(\parentheses*{\Omega, P}\) beschrieben werden, d.h. es gilt
        \[
            P\parentheses*{E} = \frac{\absolute*{E}}{\absolute*{\Omega}}, \quad E \subseteq \Omega.
        \]
        \item Die Anzahl möglicher Versuchsausgänge ist gegeben durch
        \[
            \absolute*{\Omega} = 2^7 = 128.
        \]
        \item
        \begin{enumerate}
            \item Das gesuchte Ereignis kann beschrieben werden durch
            \[
                A := \braces*{\parentheses*{\omega_1, \ldots, \omega_7} \in \Omega : \sum_{i = 1}^7 \omega_i = 6} = \braces*{\parentheses*{0, 1, \ldots, 1}, \parentheses*{1, 0, 1, \ldots, 1}, \ldots, \parentheses*{1, \ldots, 1, 0}}.
            \]
            Die Wahrscheinlichkeit für \(A\) ist somit \(P\parentheses*{A} = \frac{\absolute*{A}}{\absolute*{\Omega}} = \frac{7}{128}\).
            \item Das gesuchte Ereignis kann beschrieben werden durch
            \[
                B := \braces*{\parentheses*{\omega_1, \ldots, \omega_7} \in \Omega : \sum_{i = 1}^7 \omega_i \ge 6} = A \cup \braces*{\parentheses*{1, \ldots, 1}}.
            \]
            Die Wahrscheinlichkeit für \(B\) ist somit \(P\parentheses*{B} = \frac{\absolute*{B}}{\absolute*{\Omega}} = \frac{1}{16}\).
            \item Es bezeichne \(C\) das gesuchte Ereignis.
            Dann ist
            \[
                C^c = \braces*{\parentheses*{1, 0, 1, 0, 1, 0, 1}, \parentheses*{0, 1, 0, 1, 0, 1, 0}}.
            \]
            Damit ist die Wahrscheinlichkeit für das Ereignis \(C\) gegeben durch
            \[
                P\parentheses*{C} = 1 - P\parentheses*{C^c} = 1 - \frac{\absolute*{C^c}}{\absolute*{\Omega}} = 1 - \frac{1}{64} = \frac{63}{64}.
            \]
        \end{enumerate}
    \end{enumerate}


    \section*{Aufgabe 4}

    \begin{problem}
        Eine Gruppe von \(k\) Personen sei hinsichtlich ihrer Geburtstage in einem Jahr (kein Schaltjahr, d.h. 365 Tage) zufällig zusammengesetzt.
        \begin{enumerate}
            \item Bestimmen Sie die Wahrscheinlichkeit dafür, dass mindestens zwei der \(k\) Personen am selben Tag des Jahres Geburtstag haben (allgemein für \(k \in \N\)).
            \item Geben Sie für \(k \in \braces*{10, 22, 23, 40}\) die zugehörigen Zahlenwerte der in a) bestimmten Wahrscheinlichkeit an.
        \end{enumerate}
    \end{problem}

    \subsection*{Lösung}
    \begin{enumerate}
        \item Sei \(k \in \N\) (beliebig).
        In unserem Modell denken wir uns die \(365\) verschiedenen Tage des Jahres der Reihe nach durchnummeriert.
        Die Verteilung der \(k\) Personen auf die \(365\) Tage des Jahres entspricht der \(k\)-fachen Ziehung aus einer Urne, in der sich \(365\) mit den Zahlen \(1, \ldots, 365\) durchnummerierte Kugeln befinden.
        Hierbei nehmen wir in dieser Aufgabe (der Einfachheit halber) an, dass alle Tage des Jahres als Geburtstage gleich wahrscheinlich sind.
        Es gilt:
        \begin{itemize}
            \item Die Ziehung erfolgt \emph{mit Zurücklegen}, da mehrere Personen am gleichen Tag Geburtstag haben können.
            \item Die einzelnen Personen werden unterschieden, d.h. die Ziehung erfolgt \emph{unter Berücksichtigung der Reihenfolge}.
        \end{itemize}
        Eine geeignete Ergebnismenge ist damit gegeben durch
        \[
            \Omega_k = \braces*{\parentheses*{\omega_1, \ldots, \omega_k} : \omega_1, \ldots, \omega_k \in \braces*{1, \ldots, 365}},
        \]
        mit der folgenden Interpretation: Für \(i \in \braces*{1, \ldots, k}\) und \(j \in \braces*{1, \ldots, 365}\) bedeutet \(\omega_i = j\), dass die \(i\)-te Person am \(j\)-ten Tag des Jahres Geburtstag hat.
        Es folgt (mit \(n = 365\)):
        \[
            \absolute*{\Omega_k} = 365^k.
        \]
        Gesucht ist die Wahrscheinlichkeit des Ereignisses
        \[
            E_k: \quad \text{``Mindestens zwei der }k\text{ Personen haben am selben Tag Geburtstag''}.
        \]
        Dann ist das Komplementärereignis
        \[
            E_k^c = \braces*{\parentheses*{\omega_1, \ldots, \omega_k} \in \Omega_k : \omega_i \ne \omega_j\text{ für }i \ne j}.
        \]
        Dies ist die Menge aller \(\parentheses*{n, k}\)-Permutationen \emph{ohne} Wiederholung mit \(n = 365\).
        Es folgt für \(k \le 365\)
        \[
            \absolute*{E_k^c} = 365 \cdot 364 \cdot \ldots \cdot \parentheses*{365 - k + 1} = \frac{365!}{\parentheses*{365 - k}!}.
        \]
        Mit \(P\) als (diskreter) Gleichverteilung auf \(\Omega_k\) folgt dann für \(k \le 365\)
        \begin{equation}\label{eq:1}
            P\parentheses*{E_k} = 1 - P\parentheses*{E_k^c} = 1 - \frac{\absolute*{E_k^c}}{\absolute*{\Omega_k}} = 1 - \frac{365!}{365^k\parentheses*{365 - k}!} = 1 - \frac{365 \cdot 364 \cdot \ldots \cdot \parentheses*{365 - k + 1}}{365^k}.
        \end{equation}
        Für \(k > 365\) ist \(\absolute*{E_k^c} = 0\) und damit
        \[
            P\parentheses*{E_k} = 1 - P\parentheses*{E_k^c} = 1 - 0 = 1.
        \]
        \item Aus \eqref{eq:1} ergeben sich für \(k \in \braces*{10, 22, 23, 40}\) die folgenden Wahrscheinlichkeiten für das betreffende Ereignis \(E_k\):
        \begin{center}
            \begin{tabular}{lcccc}
                \toprule
                \(k\) & \(10\) & \(22\) & \(23\) & \(40\)\\
                \midrule
                \(P\parentheses*{E_k}\) & \(0,117\) & \(0,476\) & \(0,507\) & \(0,891\)\\
                \bottomrule
            \end{tabular}
        \end{center}
        Bereits ab der Personenanzahl \(k = 23\) ist die Wahrscheinlichkeit dafür, dass mindestens zwei der \(k\) Personen am selben Tag des Jahres Geburtstag haben, größer als \(0,5\) und damit größer als die Wahrscheinlichkeit dafür, dass alle \(k\) Personen an verschiedenen Tagen Geburtstag haben.
    \end{enumerate}
\end{document}
