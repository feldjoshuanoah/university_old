\documentclass{exercise}

\institute{Institut für Statistik und Wirtschaftsmathematik}
\title{Globalübung 3}
\author{Joshua Feld, 406718}
\course{Statistik}
\professor{Cramer}
\semester{Sommersemester 2022}
\program{CES (Bachelor)}

\begin{document}
    \maketitle


    \section*{Aufgabe 1}

    \begin{problem}
        In einer medizinischen Untersuchung wird der gesundheitliche Zustand von Rauchern und Nichtrauchern verglichen.
        Dabei werden die an der Untersuchung teilnehmenden Personen in Männer und Frauen sowie in Raucher und Nichtraucher eingeteilt.
        Die Menge der Männer wird mit \(A\) bezeichnet, die Menge der Raucher mit \(B\).
        
        Welche Personen werden durch die folgenden Mengen beschrieben?
        \begin{enumerate}
            \item \(B^c\),
            \item \(A \cap B\),
            \item \(B \setminus A\),
            \item \(\parentheses*{A \cup B}^c\),
            \item \(\parentheses*{A \cup B^c} \cap \parentheses*{A^c \cup B}\).
        \end{enumerate}
    \end{problem}

    \subsection*{Lösung}
    \begin{enumerate}
        \item
        \item
        \item
        \item
        \item
    \end{enumerate}


    \section*{Aufgabe 2}

    \begin{problem}
        In einer Stellenausschreibung werden nach Möglichkeit deutsche, englische und französische Sprachkenntnisse verlangt.
        Von den insgesamt 250 Personen, die sich bewerben, sprechen 60 nur Deutsch, 85 nur Englisch und 25 nur Französisch.
        20 sprechen Englisch und Deutsch, aber kein Französisch, 12 sprechen Deutsch und Französisch, aber kein Englisch und 8 sprechen Englisch und Französisch, aber kein Deutsch.
        (Jede Person, die sich bewirbt, spricht mindestens eine der drei Sprachen.)
        Berechnen Sie die Wahrscheinlichkeit dafür, dass eine zufällig aus den Bewerbungen ausgewählte Person folgende Sprachen spricht:
        \begin{enumerate}
            \item Deutsch,
            \item Englisch,
            \item Französisch,
            \item alle drei Sprachen.
        \end{enumerate}
    \end{problem}

    \subsection*{Lösung}
    \begin{enumerate}
        \item
        \item
        \item
        \item 
    \end{enumerate}


    \section*{Aufgabe 3}

    \begin{problem}
        Eine homogene Münze (d.h. die Wahrscheinlichkeiten für das Auftreten von Kopf und Zahl sind identisch) werde sieben Mal hintereinander geworfen.
        \begin{enumerate}
            \item Geben Sie einen geeigneten Wahrscheinlichkeitsraum für dieses Zufallsexperiment an.
            \item Wie viele mögliche Versuchsausgänge gibt es?
            \item Beschreiben Sie die nachstehenden Ereignisse als Teilmengen von \(\Omega\), und berechnen Sie ihre Wahrscheinlichkeiten:
            \begin{enumerate}
                \item Genau sechs der sieben Münzwürfe ergeben Zahl.
                \item Mindestens sechs der sieben Münzwürfe ergeben Zahl.
                \item In zwei aufeinander folgenden Würfen fällt Kopf, oder in zwei aufeinander folgenden Würfen fällt Zahl.
            \end{enumerate}
        \end{enumerate}
    \end{problem}

    \subsection*{Lösung}
    \begin{enumerate}
        \item
        \item
        \item
        \begin{enumerate}
            \item
            \item
            \item
        \end{enumerate}
    \end{enumerate}


    \section*{Aufgabe 4}

    \begin{problem}
        Eine Gruppe von \(k\) Personen sei hinsichtlich ihrer Geburtstage in einem Jahr (kein Schaltjahr, d.h. 365 Tage) zufällig zusammengesetzt.
        \begin{enumerate}
            \item Bestimmen Sie die Wahrscheinlichkeit dafür, dass mindestens zwei der \(k\) Personen am selben Tag des Jahres Geburtstag haben (allgemein für \(k \in \N\)).
            \item Geben Sie für \(k \in \braces*{10, 22, 23, 40}\) die zugehörigen Zahlenwerte der in a) bestimmten Wahrscheinlichkeit an.
        \end{enumerate}
    \end{problem}

    \subsection*{Lösung}
    \begin{enumerate}
        \item
        \item
    \end{enumerate}
\end{document}
