\documentclass{exercise}

\institute{Institut für Statistik und Wirtschaftsmathematik}
\title{Globalübung 4}
\author{Joshua Feld, 406718}
\course{Statistik}
\professor{Cramer}
\semester{Sommersemester 2022}
\program{CES (Bachelor)}

\begin{document}
    \maketitle


    \section*{Aufgabe 1}

    \begin{problem}
        In einer Warensendung mit \(100\) Transistoren befinden sich \(5\) defekte Stücke.
        Der Empfänger entnimmt der Sendung zufällig \(10\) Transistoren und überprüft diese.
        Er verweigert die Annahme der Warensendung, wenn sich unter den überprüften Transistoren mindestens \(2\) als defekt erweisen.
        \begin{enumerate}
            \item Berechnen Sie die Wahrscheinlichkeit dafür, dass
            \begin{enumerate}
                \item von den \(10\) entnommenen Transistoren genau einer defekt ist,
                \item der Empfänger die Annahme der Warensendung verweigert
            \end{enumerate}
            unter der Voraussetzung, dass die überprüften Transistoren \emph{nicht} wieder in die Warensendung zurückgelegt werden.
            \item Berechnen Sie die entsprechenden Wahrscheinlichkeiten für die Ereignisse aus a) unter der Voraussetzung, dass jeder überprüfte Transistor anschließend wieder in die Warensendung zurückgelegt wird.
        \end{enumerate}
    \end{problem}

    \subsection*{Lösung}
    \begin{enumerate}
        \item
        \begin{enumerate}
            \item
            \item
        \end{enumerate}
        \item
    \end{enumerate}


    \section*{Aufgabe 2}

    \begin{problem}
        In einem Supermarkt soll mittels eines neu entwickelten Werbeplakats zum Kauf zweier Produkte \(P_1\) und \(P_2\) animiert werden.
        Es bezeichnen \(A\) bzw. \(B\) die Ereignisse, dass ein (beliebig ausgewählter) Kunde Produkt \(P_1\) bzw. \(P_2\) kauft, und \(P\) die zugrundeliegende Wahrscheinlichkeitsverteilung.
        Hierbei seien die folgenden Wahrscheinlichkeiten bekannt:
        \[
            P\parentheses*{A} = \frac{1}{2}, \quad P\parentheses*{B} = \frac{1}{5}, \quad P\parentheses*{A \cap B} = \frac{1}{10}.
        \]
        Berechnen Sie aus diesen Angaben die folgenden Wahrscheinlichkeiten:
        \begin{align*}
            &\text{a) }P\parentheses*{A \cup B}, & &\text{c) }P\parentheses*{A^c \cap B^c}, & &\text{e) }P\parentheses*{A \cap \parentheses*{A^c \cup B}},\\
            &\text{b) }P\parentheses*{\parentheses*{A \cap B}^c}, & &\text{d) }P\parentheses*{A \cup B^c}, & &\text{f) }P\parentheses*{\parentheses*{A^c \cap B^c} \cup \parentheses*{A \cap B}}.
        \end{align*}
        Beschreiben Sie zusätzlich die betrachteten Ereignisse jeweils verbal im Rahmen des gegebenen Zusammenhangs.
    \end{problem}

    \subsection*{Lösung}
    \begin{enumerate}
        \item \(A \cup B\): Der Kunde kauft \(P_1\) oder \(P_2\).
        \[
            P\parentheses*{A \cup B} = P\parentheses*{A} + P\parentheses*{B} - P\parentheses*{A \cap B} = \frac{1}{2} + \frac{1}{5} - \frac{1}{10} = \frac{3}{5}.
        \]
        \item \(\parentheses*{A \cap B}^c = A^c \cup B^c\): Der Kunde kauft \emph{nicht} beide Produkte.
        \[
            P\parentheses*{\parentheses*{A \cap B}^c} = 1 - P\parentheses*{A \cap B} = 1 - \frac{1}{10} = \frac{9}{10}.
        \]
        \item \(A^c \cap B^c = \parentheses*{A \cup B}^c\): Der Kunde kauft keins der beiden Produkte.
        \[
            P\parentheses*{A^c \cap B^c} = P\parentheses*{\parentheses*{A \cup B}^c} = 1 - P\parentheses*{A \cup B} = 1 - \frac{3}{5} = \frac{2}{5}
        \]
        \item \(A \cup B^c = \parentheses*{A^c \cap B}^c\): Der Kunde kauft \(P_1\) oder \emph{nicht} \(P_2\), d.h. der Kunde kauft nicht ausschließlich \(P_2\).
        \begin{align*}
            P\parentheses*{A \cup B^c} &= P\parentheses*{A} + P\parentheses*{B^c} - P\parentheses*{A \cap B^c}\\
            &= P\parentheses*{A} + 1 - P\parentheses*{B} - \parentheses*{P\parentheses*{A} - P\parentheses*{A \cap B}}\\
            &= 1 - P\parentheses*{B} + P\parentheses*{A \cap B}\\
            &= 1 - \frac{1}{5} + \frac{1}{10} = \frac{9}{10}
        \end{align*}
        \item \(A \cap \parentheses*{A^c \cup B} = \parentheses*{A \cap A^c} \cup \parentheses*{A \cap B} = A \cap B\): Der Kunde kauft \(P_1\) und \(P_2\).
        \[
            P\parentheses*{A \cap \parentheses*{A^c \cup B}} = P\parentheses*{A \cap B} = \frac{1}{10}
        \]
        \item \(\parentheses*{A^c \cap B^c} \cup \parentheses*{A \cap B}\): Der Kunde kauft kein Produkt oder beide.
        \begin{align*}
            P\parentheses*{\parentheses*{A^c \cap B^c} \cup \parentheses*{A \cap B}} &= P\parentheses*{A^c \cap B^c} + P\parentheses*{A \cap B} - P\parentheses*{\parentheses*{A^c \cap B^c} \cap \parentheses*{A \cap B}}\\
            &= P\parentheses*{A^c \cap B^c} + P\parentheses*{A \cap B} - \underbrace{P\parentheses*{\emptyset}}_{= 0}\\
            &= \frac{2}{5} + \frac{1}{10} = \frac{1}{2}
        \end{align*}
    \end{enumerate}


    \section*{Aufgabe 3}

    \begin{problem}
        Herr Planlos trifft zu einem zufälligen Zeitpunkt an einer Bushaltestelle ein, von der aus Busse in die gewünschte Richtung im Zehn-Minuten-Takt abfahren.
        Seine Wartezeit auf den nächsten Bus kann mithilfe der stetigen Gleichverteilung \(R\parentheses*{0, 10}\) auf dem Intervall \(\brackets*{0, 10}\) modelliert werden.

        Es bezeichne \(F\) die Verteilungsfunktion von \(R\parentheses*{0, 10}\).
        Dann gibt \(F\parentheses*{x}\) für \(x \in \R\) die Wahrscheinlichkeit dafür an, dass Herr Planlos höchstens \(x\) Minuten auf den nächsten Bus warten muss.

        Berechnen Sie die Wahrscheinlichkeit dafür, dass die Wartezeit von Herrn Planlos auf den nächsten Bus mehr als 7 Minuten beträgt.
    \end{problem}

    \subsection*{Lösung}


    \section*{Aufgabe 4}

    \begin{problem}
        Weisen Sie nach, dass durch die nachstehenden Funktionen eine Zähldichte bzw. eine Riemann-Dichte auf dem jeweils angegebenen Träger definiert ist.
        \begin{enumerate}
            \item \(p_k = p\parentheses*{1 - p}^k, k \in \N_0\), für ein \(p \in \parentheses*{0, 1}\) (Geometrische Verteilung \(\geo\parentheses*{p}\)),
            \item \(p_k = \frac{\lambda^k}{k!}e^{-\lambda}, k \in \N_0\), für ein \(\lambda > 0\) (Poisson-Verteilung \(\po\parentheses*{\lambda}\)),
            \item \(f\parentheses*{x} = \begin{cases}
                \lambda e^{-\lambda x}, & \text{falls }x > 0,\\
                0, & \text{falls }x \le 0,
            \end{cases}\) für ein \(\lambda > 0\) (Exponentialverteilung \(\Exp\parentheses*{\lambda}\)),
            \item \(f\parentheses*{x} = \begin{cases}
                \frac{\alpha}{x^{\alpha + 1}}, & \text{falls }x \ge 1,\\
                0, & \text{falls }x < 0,
            \end{cases}\) für ein \(\alpha > 0\) (Pareto-Verteilung \(\Par\parentheses*{\lambda}\)).
        \end{enumerate}
    \end{problem}

    \subsection*{Lösung}
\end{document}
