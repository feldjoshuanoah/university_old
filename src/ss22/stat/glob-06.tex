\documentclass{exercise}

\institute{Institut für Statistik und Wirtschaftsmathematik}
\title{Globalübung 6}
\author{Joshua Feld, 406718}
\course{Statistik}
\professor{Cramer}
\semester{Sommersemester 2022}
\program{CES (Bachelor)}

\begin{document}
    \maketitle


    \section*{Aufgabe 1}

    \begin{problem}
        J. Walker steht nach einem feuchtfröhlichen Abend reichlich mitgenommen vor seiner Haustür und bemüht sich, den passenden Schlüssel zu finden.
        Hierzu verwendet er folgende Strategie: Er greift zufällig einen der acht Schlussel in seiner Manteltasche heraus und probiert ihn aus.
        Passt der Schlüssel nicht, legt er ihn wieder in seine Manteltasche zurück und wiederholt diese Vorgehensweise (unabhängig von den vorigen Versuchen) so lange, bis er den passenden Schlüssel gefunden hat.
        
        Wie groß ist die Wahrscheinlichkeit dafür, dass Mr. Walker
        \begin{enumerate}
            \item den passenden Schlüssel im zweiten Versuch findet,
            \item midenstens fünf Versuche benötigt, den passenden Schlüssel zu finden,
            \item den passenden Schlüssel überhaupt nicht findet?
        \end{enumerate}
    \end{problem}

    \subsection*{Lösung}
    Es bezeiche \(X\) die Anzahl der Fehlversuche, die benötigt werden, bis der passende Schlüssel (erstmalig) gefunden ist (ohne den letzten erfolgreichen Versuch).
    Dann gilt, dass die (diskrete) Zufallsvariable \(X\) geometrisch mit Parameter \(p = \frac{1}{8}\) verteilt ist (kurz: \(X \sim \geo\parentheses*{\frac{1}{8}}\)).
    \begin{enumerate}
        \item \(P\parentheses*{X = 1} = \frac{1}{8} \cdot \frac{7}{8} \approx 0,109\).
        \item Das Ereignis \(\braces*{X \ge 4}\) lässt sich disjunkt in die Ereignisse
        \[
            \braces*{X = 4}, \braces*{X = 5}, \ldots
        \]
        zerlegen.
        Es gilt daher
        \begin{align*}
            P\parentheses*{X \ge 4} &= P\parentheses*{\bigcup_{k = 4}^\infty \braces*{X = k}}\\
            &= \sum_{k = 4}^\infty P\parentheses*{X = k}\\
            &= \frac{1}{8}\sum_{k = 4}^\infty \parentheses*{\frac{7}{8}}^k\\
            &= \frac{1}{8}\sum_{j = 0}^\infty \parentheses*{\frac{7}{8}}^{j + 4}\\
            &= \frac{1}{8} \cdot \parentheses*{\frac{7}{8}}^4 \cdot \sum_{j = 0}^\infty \parentheses*{\frac{7}{8}}^j\\
            &= \frac{1}{8} \cdot \parentheses*{\frac{7}{8}}^4 \cdot \frac{1}{1 - \frac{7}{8}}\\
            &= \parentheses*{\frac{7}{8}}^4 \approx 0,586.
        \end{align*}
        \item Wir betrachten hierzu das zugehörige Komplementärereignis \(\braces*{X \in \N_0}\), das sich analog zu b) in die Ereignisse
        \[
            \braces*{X = 0}, \braces*{X = 1}, \ldots
        \]
        zerlegen lässt.
        Es folgt
        \[
            P\parentheses*{X \in \N_0} = P\parentheses*{\bigcup_{k = 0}^\infty \braces*{X = k}} = \sum_{k = 0}^\infty P\parentheses*{X = k} = \frac{1}{8}\sum_{k = 0}^\infty \parentheses*{\frac{7}{8}}^k = \frac{1}{8} \cdot \frac{1}{1 - \frac{7}{8}} = 1.
        \]
        Hiermit erhält man für das gesuchte Ereignis
        \[
            P\parentheses*{X = \infty} = P\parentheses*{X \not\in \N_0} = 1 - P\parentheses*{X \in \N_0} = 0.
        \]
    \end{enumerate}


    \section*{Aufgabe 2}

    \begin{problem}
        Es sei \(X\) eine mit Parametern \(n \in \N\) und \(p \in \parentheses*{0, 1}\) binomialverteilte Zufallsvariable auf einem Wahrscheinlichkeitsraum \(\parentheses*{\Omega, \mathfrak{A}, P}\) (kurz: \(X \sim \bin\parentheses*{n, p}\)).
        Damit ist die Zähldichte \(p^X\) der (diskreten) Zufallsvariablen \(X\) nach der Vorlesung gegeben durch
        \[
            p^X\parentheses*{k} = P\parentheses*{X = k} = \binom{n}{k}p^k\parentheses*{1 - p}^{n - k}\text{ für } k \in \braces*{0, \ldots, n}.
        \]
        \begin{enumerate}
            \item Bestimmen und skizzieren Sie die Verteilungsfunktion \(F^X\) der Zufallsvariablen \(X\) für \(n = 8\) und \(p = 0,5\).
            \item Berechnen Sie für \(n = 8\) und \(p = 0,5\) die folgenden Wahrscheinlichkeiten:
            \[
                \text{(i) }P\parentheses*{X \le 3}, \quad \text{(ii) }P\parentheses*{X > 5}, \quad \text{(iii) }P\parentheses*{2 < X \le 4}, \quad \text{(iv) }P\parentheses*{2 \le X \le 4}. 
            \]
        \end{enumerate}
    \end{problem}

    \subsection*{Lösung}
    \begin{enumerate}
        \item Wir bestimmen zunächst die Verteilungsfunktion \(F^X\) der Zufallsvariablen \(X\) für \(n \in \N\) und \(p \in \parentheses*{0, 1}\) beliebig.
        Dann gilt
        \begin{align}
            F^X\parentheses*{x} &= P\parentheses*{X \le x}\nonumber\\
            &= \sum_{k = 0}^{\floor*{x}}\underbrace{p^X\parentheses*{k}}_{= P\parentheses*{X = k}}\nonumber\\
            &= \sum_{\substack{k = 0\\k \le x}}^n \binom{n}{k}p^k \parentheses*{1 - p}^{n - k}\nonumber\\
            &= \begin{cases}
                0, & \text{falls }x < 0,\\
                \sum_{k = 0}^m \binom{n}{k}p^k \parentheses*{1 - p}^{n - k}, & \text{falls }m \le x < m + 1, m \in \braces*{0, \ldots, n - 1},\\
                1, & \text{falls }x \ge n,
            \end{cases}\label{eq:1}
        \end{align}
        Hierbei wurde in der letzten Zeile benutzt, dass nach dem binomischen Lehrsatz
        \begin{equation}\label{eq:2}
            \sum_{k = 0}^n \binom{n}{k}p^k\parentheses*{1 - p}^{n - k} = \parentheses*{p + \parentheses*{1 + p}}^n = 1^n = 1.
        \end{equation}
        Aus \eqref{eq:1} und \eqref{eq:2} erhält man speziell
        \begin{equation}\label{eq:3}
            F^X\parentheses*{m} = \sum_{k = 0}^m \binom{n}{k}p^k \parentheses*{1 - p}^{n - k}, \quad m \in \braces*{0, \ldots, n}.
        \end{equation}
        Gemäß \eqref{eq:1} reicht es zur vollständigen Bestimmung der Verteilungsfunktion \(F^X\) aus, diese Funktionswerte \(F^X\parentheses*{0}, \ldots, F^X\parentheses*{n}\) in den Trägerpunkten \(0, \ldots, n\) zu berechnen.
        Seien nun \(n = 8\) und \(p = 0,5\).
        Dann gilt gemäß Aufgabenstellung für die Zähldichte
        \[
            p^X\parentheses*{k} = \binom{8}{k} \cdot \underbrace{0,5^k \cdot \parentheses*{1 - 0,5}^{8 - k}}_{= 0,5^k \cdot 0,5^{8 - k} = 0,5^8} = \binom{8}{k} \cdot 0,5^8, \quad k \in \braces*{0, \ldots, 8}.
        \]
        Hieraus folgt mit \eqref{eq:3}
        \[
            F^X\parentheses*{m} = 0,5^8 \cdot \sum_{k = 0}^m \binom{8}{k}, \quad m \in \braces*{0, \ldots, 8}.
        \]
        Für \(m \in \braces*{0, \ldots, 8}\) sind die Werte \(F^X\parentheses*{m}\) der Verteilungsfunktion zusammen mit den Werten \(p^X\parentheses*{m}\) der Zähldichte in der folgenden Tabelle angegeben:
        \begin{center}
            \begin{tabular}{lccccccccc}
                \toprule
                \(m\) & \(0\) & \(1\) & \(2\) & \(3\) & \(4\) & \(5\) & \(6\) & \(7\) & \(8\)\\
                \midrule
                \(p^X\parentheses*{m}\) & \(0,5^8\) & \(8 \cdot 0,5^8\) & \(28 \cdot 0,5^8\) & \(56 \cdot 0,5^8\) & \(70 \cdot 0,5^8\) & \(56 \cdot 0,5^8\) & \(28 \cdot 0,5^8\) & \(8 \cdot 0,5^8\) & \(0,5^8\)\\
                \(F^X\parentheses*{m}\) & \(0,5^8\) & \(9 \cdot 0,5^8\) & \(37 \cdot 0,5^8\) & \(93 \cdot 0,5^8\) & \(163 \cdot 0,5^8\) & \(219 \cdot 0,5^8\) & \(247 \cdot 0,5^8\) & \(255 \cdot 0,5^8\) & \(256 \cdot 0,5^8\)\\
                \bottomrule
            \end{tabular}
        \end{center}
        Hieraus erhält man mit \eqref{eq:1}
        \[
            F^X\parentheses*{x} = \begin{cases}
                0, & \text{falls }x < 0,\\
                0,0039, & \text{falls }0 \le x < 1,\\
                0,0352, & \text{falls }1 \le x < 2,\\
                0,1445, & \text{falls }2 \le x < 3,\\
                0,3633, & \text{falls }3 \le x < 4,\\
                0,6367, & \text{falls }4 \le x < 5,\\
                0,8555, & \text{falls }5 \le x < 6,\\
                0,9648, & \text{falls }6 \le x < 7,\\
                0,9961, & \text{falls }7 \le x < 8,\\
                1, & \text{falls }x \ge 8.
            \end{cases}
        \]
        \begin{center}
            \begin{tikzpicture}
                \draw[->] (-1.5,0) -- (9.5,0) node[above] {\(x\)};
                \foreach \i in {-1,0,...,9}
                {
                    \draw (\i,.1) -- (\i,-.1) node[below] {\(\i\)};
                }
                \draw[->] (0,0) -- (0,10.5) node[right] {\(F^X\parentheses*{x}\)};
                \foreach \i in {1,...,9}
                {
                    \draw (.1,\i) -- (-.1,\i) node[left] {\(0,\i\)};
                }
                \draw (.1,10) -- (-.1,10) node[left] {\(1\)};
                \draw (0,0.039) -- (1,0.039);
                \draw (1,0.352) -- (2,0.352);
                \draw (2,1.445) -- (3,1.445);
                \draw (3,3.633) -- (4,3.633);
                \draw (4,6.367) -- (5,6.367);
                \draw (5,8.555) -- (6,8.555);
                \draw (6,9.648) -- (7,9.648);
                \draw (7,9.961) -- (8,9.961);
                \draw (8,10) -- (9,10);
                \fill (0,0.039) circle (.5mm);
                \fill (1,0.352) circle (.5mm);
                \fill (2,1.445) circle (.5mm);
                \fill (3,3.633) circle (.5mm);
                \fill (4,6.367) circle (.5mm);
                \fill (5,8.555) circle (.5mm);
                \fill (6,9.648) circle (.5mm);
                \fill (7,9.961) circle (.5mm);
                \fill (8,10) circle (.5mm);
            \end{tikzpicture}
        \end{center}
        \item
        \begin{enumerate}
            \item \(P\parentheses*{X \le 3} = F^X\parentheses*{3} \approx 0,3633\).
            \item \(P\parentheses*{X > 5} = 1 - P\parentheses*{X \le 5} = 1 - F^X\parentheses*{5} \approx 1 - 0,8555 = 0,1445\).
            \item \(P\parentheses*{2 < X \le 4} = P\parentheses*{X \le 4} - P\parentheses*{X \le 2} = F^X\parentheses*{4} - F^X\parentheses*{2} \approx 0,6367 - 0,1445 = 0,4922\).
            \item
            \begin{align*}
                P\parentheses*{2 \le X \le 4} &= P\parentheses*{\braces*{\omega \in \Omega : X\parentheses*{\omega} = 2} \cup \braces*{\omega \in \Omega : 2 < X\parentheses*{\omega} \le 4}}\\
                &= P\parentheses*{\braces*{\omega \in \Omega : X\parentheses*{\omega} = 2}} + P\parentheses*{\braces*{\omega \in \Omega : 2 < X\parentheses*{\omega} \le 4}}\\
                &= P\parentheses*{X = 2} + P\parentheses*{2 < X \le 4}\\
                &= F^X\parentheses*{2} - \lim_{x \to 2^-}F^X\parentheses*{x} + F^{X}\parentheses*{4} - F^X\parentheses*{2}\\
                &= F^X\parentheses*{4} - \underbrace{\lim_{x \to 2^-}F^X\parentheses*{x}}_{= F^X\parentheses*{1}}\\
                &= F^X\parentheses*{4} - F^X\parentheses*{1}\\
                &\approx 0,6367 - 0,0352 = 0,6015.
            \end{align*}
        \end{enumerate} 
    \end{enumerate}


    \section*{Aufgabe 3}

    \begin{problem}
        Es seien \(X\) und \(Y\) zwei stochastisch unabhängige Zufallsvariablen auf einem Wahrscheinlichkeitsraum \(\parentheses*{\Omega, \mathfrak{A}, P}\), die jeweils gleichverteilt auf dem Intervall \(\brackets*{0, 2}\) sind (kurz: \(X \sim R\parentheses*{0, 2}\) und \(Y \sim R\parentheses*{0, 2}\)).
        Zeigen Sie, dass durch
        \[
            f^Z\parentheses*{z} = \begin{cases}
                0, & \text{falls }z < 0,\\
                \frac{z}{4}, & \text{falls }0 \le z < 2,\\
                1 - \frac{z}{4}, & \text{falls }2 \le z \le 4,\\
                0, & \text{falls }z > 4
            \end{cases}
        \]
        eine Riemann-Dichte der (stetigen) Zufallsvariablen \(Z = X + Y\) gegeben ist.
    \end{problem}

    \subsection*{Lösung}
    Gemäß Voraussetzung und Vorlesung ist eine Riemann-Dichte \(f\) von \(X\) bzw. \(Y\) gegeben durch
    \[
        f\parentheses*{x} = \begin{cases}
            \frac{1}{2}, & \text{falls }x \in \brackets*{0, 2},\\
            0, & \text{falls }x \in \R \setminus \brackets*{0, 2}.
        \end{cases}
    \]
    Weiter sind \(X\) und \(Y\) stochastisch unabhängig gemäß Voraussetzung.
    Eine Dichtefunktion \(f^Z\) von \(Z = X + Y\) ist dann gemäß der Faltungsformel für stetige Zufallsvariablen gegeben durch
    \[
        f^Z\parentheses*{z} = \int_{-\infty}^\infty f\parentheses*{z - y}\underbrace{f\parentheses*{y}}_{\substack{= 0\text{ für}\\y < 0\text{ oder }y > 2}}\d y = \frac{1}{2}\int_0^2 \underbrace{f\parentheses*{z - y}}_{\substack{\ne 0\\\iff 0 \le z - y \le 2\\\iff z - 2 \le y \le z}}\d y.
    \]
    \begin{itemize}
        \item Fall 1: Sei \(z < 0\). Dann folgt
        \begin{equation}\label{eq:4}
            f^Z\parentheses*{z} = \frac{1}{2}\int_0^2 \underbrace{f\parentheses*{z - y}}_{\substack{= 0\text{, da }z - y < 0\\\text{für }y \in \brackets*{0, 2}}}\d y = 0.
        \end{equation}
        \item Fall 2: Sei \(0 \le z < 2\). Dann folgt
        \begin{equation}
            f^Z\parentheses*{z} = \frac{1}{2}\int_0^2 \underbrace{f\parentheses*{z - y}}_{\substack{= 0\text{ für }z - y < 0\\\iff z < y}}\d y = \frac{1}{2}\int_0^z \underbrace{f\parentheses*{z - y}}_{\substack{= \frac{1}{2}\text{, da}\\0 \le z - y \le 2}}\d y = \frac{1}{4}\int_0^z \d y = \frac{1}{4}z.
        \end{equation}
        \item Fall 3: Sei \(2 \le z \le 4\). Dann folgt
        \begin{equation}
            f^Z\parentheses*{z} = \frac{1}{2}\int_0^2 \underbrace{f\parentheses*{z - y}}_{\substack{= 0\text{ für }z - y > 2\\\iff y < z - 2}}\d y = \frac{1}{2}\int_{z - 2}^2 \underbrace{f\parentheses*{z - y}}_{\substack{= \frac{1}{2}\text{, da}\\0 \le z - y \le 2}}\d y = \frac{1}{4}\int_{z - 2}^2 \d y = \frac{1}{4}\parentheses*{2 - \parentheses*{z - 2}} = \frac{1}{4}\parentheses*{4 - z}.
        \end{equation}
        \item Fall 4: Sei \(z > 4\). Dann folgt
        \begin{equation}\label{eq:7}
            f^Z\parentheses*{z} = \frac{1}{2}\int_0^2 \underbrace{f\parentheses*{z - y}}_{\substack{= 0\text{, da}\\z - y > 4 - 2 = 2}}\d y = 0.
        \end{equation}
    \end{itemize}
    Mit \eqref{eq:4} - \eqref{eq:7} hat damit die Riemann-Dichte \(f^Z\) von \(Z\) insgesamt die in der Aufgabenstellung angegebene Darstellung.
\end{document}
