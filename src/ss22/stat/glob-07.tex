\documentclass{exercise}

\institute{Institut für Statistik und Wirtschaftsmathematik}
\title{Globalübung 7}
\author{Joshua Feld, 406718}
\course{Statistik}
\professor{Cramer}
\semester{Sommersemester 2022}
\program{CES (Bachelor)}

\begin{document}
    \maketitle


    \section*{Aufgabe 1}
    
    \begin{problem}
        \(X\) und \(Y\) seien zwei diskrete Zufallsvariablen auf einem Wahrscheinlichkeitsraum \(\parentheses*{\Omega, \mathfrak{U}, P}\), wobei \(X\) die Werte \(-1\), \(0\) und \(1\) und \(Y\) die Werte \(1\), \(2\) und \(3\) annimmt.
        Die zugehörigen Wahrscheinlichkeiten \(P\parentheses*{X = i, Y = j}\) für \(i \in \braces*{-1, 0, 1}, j \in \braces*{1, 2, 3}\) sind in der folgenden Tabelle angegeben:
        \begin{center}
            \begin{tabular}{ccccc}
                \toprule
                \multicolumn{2}{c}{\multirow{2}{*}{\(P\parentheses*{X = i, Y = j}\)}} & \multicolumn{3}{c}{\(j\)}\\
                \multicolumn{2}{c}{\multirow{2}{*}{}} & \(1\) & \(2\) & \(3\)\\
                \midrule
                \multirow{3}{*}{\(i\)} & \(-1\) & \(\frac{1}{20}\) & \(\frac{1}{5}\) & \(0\)\\
                & \(0\) & \(\frac{1}{5}\) & \(\frac{1}{5}\) & \(\frac{1}{10}\)\\
                & \(1\) & \(\frac{1}{10}\) & \(\frac{1}{10}\) & \(\frac{1}{20}\)\\
                \bottomrule
            \end{tabular}
        \end{center}
        \begin{enumerate}
            \item Bestimmen Sie die zugehörigen Rand-Zähldichten \(p^X\) von \(X\) und \(p^Y\) von \(Y\), für die in der gegebenen Situation gilt:
            \begin{align*}
                p^X\parentheses*{i} &= P\parentheses*{X = i} = \sum_{j \in \braces*{1, 2, 3}}P\parentheses*{X = i, Y = j}, \quad i \in \braces*{-1, 0, 1},\\
                p^Y\parentheses*{j} &= P\parentheses*{Y = j} = \sum_{i \in \braces*{-1, 0, 1}}P\parentheses*{X = i, Y = j}, \quad j \in \braces*{1, 2, 3}.
            \end{align*}
            Entscheiden Sie, ob die Zufallsvariablen \(X\) und \(Y\) stochastisch unabhängig sind.
            \item Berechnen Sie die bedingte Wahrscheinlichkeit \(P\parentheses*{X = 0 \mid Y \ge 2}\).
            \item Bestimmen Sie die Verteilung der Zufallsvariablen \(Z = X + Y\), deren Zähldichte \(p^Z\) in der vorliegenden Situation gegeben ist durch:
            \[
                p^Z\parentheses*{k} = \sum_{j \in \braces*{1, 2, 3}}P\parentheses*{X = k - j, Y = j} = \sum_{i \in \braces*{-1, 0, 1}}P\parentheses*{X = i, Y = k - i}, \quad k \in \braces*{0, 1, 2, 3, 4}.
            \]
        \end{enumerate}
    \end{problem}
    
    \subsection*{Lösung}
    \begin{enumerate}
        \item Gemäß Aufgabenstellung ergeben sich folgende Werte für die Rand-Zähldichten \(p^X\) von \(X\) bzw. \(p^Y\) von \(Y\):
        \begin{align*}
            p^X\parentheses*{-1} &= P\parentheses*{X = -1} = \sum_{j = 1}^3 P\parentheses*{X = -1, Y = j} = \frac{1}{20} + \frac{1}{5} + 0 = \frac{1}{4},\\
            p^X\parentheses*{0} &= P\parentheses*{X = 0} = \sum_{j = 1}^3 P\parentheses*{X = 0, Y = j} = \frac{1}{5} + \frac{1}{5} + \frac{1}{10} = \frac{1}{2},\\
            p^X\parentheses*{1} &= P\parentheses*{X = 1} = \sum_{j = 1}^3 P\parentheses*{X = 1, Y = j} = \frac{1}{10} + \frac{1}{10} + \frac{1}{20} = \frac{1}{4},\\
            p^Y\parentheses*{1} &= P\parentheses*{Y = 1} = \sum_{i = -1}^1 P\parentheses*{X = i, Y = 1} = \frac{1}{20} + \frac{1}{5} + \frac{1}{10} = \frac{7}{20},\\
            p^Y\parentheses*{2} &= P\parentheses*{Y = 2} = \sum_{i = -1}^1 P\parentheses*{X = i, Y = 2} = \frac{1}{5} + \frac{1}{5} + \frac{1}{10} = \frac{1}{2},\\
            p^Y\parentheses*{3} &= P\parentheses*{Y = 3} = \sum_{i = -1}^1 P\parentheses*{X = i, Y = 3} = 0 + \frac{1}{10} + \frac{1}{20} = \frac{3}{20}.
        \end{align*}
        Hiermit erhält man die folgende, um die Rand-Zähldichten ergänzte Tabelle zur Verteilung des (diskreten) zweidimensionalen Zufallsvektors \(\parentheses*{X, Y}\):
        \begin{center}
            \begin{tabular}{cccccc}
                \toprule
                \multicolumn{2}{c}{\multirow{2}{*}{\(P\parentheses*{X = i, Y = j}\)}} & \multicolumn{3}{c}{\(j\)} & \multirow{2}{*}{\(P\parentheses*{X = i}\)}\\
                \multicolumn{2}{c}{\multirow{2}{*}{}} & \(1\) & \(2\) & \(3\) & \multirow{2}{*}{}\\
                \midrule
                \multirow{3}{*}{\(i\)} & \(-1\) & \(\frac{1}{20}\) & \(\frac{1}{5}\) & \(0\) & \(\frac{1}{4}\)\\
                & \(0\) & \(\frac{1}{5}\) & \(\frac{1}{5}\) & \(\frac{1}{10}\) & \(\frac{1}{2}\)\\
                & \(1\) & \(\frac{1}{10}\) & \(\frac{1}{10}\) & \(\frac{1}{20}\) & \(\frac{1}{4}\)\\
                \multicolumn{2}{c}{\(P\parentheses*{Y = j}\)} & \(\frac{7}{20}\) & \(\frac{1}{2}\) & \(\frac{3}{20}\) &\\
                \bottomrule
            \end{tabular}
        \end{center}
        In der vorliegenden Situation sind \(X\) und \(Y\) stochastisch unabhängig, falls \emph{für alle} \(i \in \braces*{-1, 0, 1}, j \in \braces*{1, 2, 3}\) gilt:
        \[
            P\parentheses*{X = i, Y = j} = P\parentheses*{X = i}P\parentheses*{Y = j}.
        \]
        Hier gilt jedoch z.B.:
        \[
            P\parentheses*{X = -1, Y = 3} = 0 \ne \frac{1}{4} \cdot \frac{3}{20} = P\parentheses*{X = -1} \cdot P\parentheses*{Y = 3}.
        \]
        Deshalb sind \(X\) und \(Y\) \emph{nicht} stochastisch unabhängig.
        \item Gemäß der Definition bedingter Wahrscheinlichkeiten in B5.2 und laut Aufgabenstellung folgt:
        \[
            P\parentheses*{X = 0 \mid Y \ge 2} = \frac{P\parentheses*{X = 0, Y \ge 2}}{P\parentheses*{Y \ge 2}} = \frac{P\parentheses*{X = 0, Y = 2} + P\parentheses*{X = 0, Y = 3}}{P\parentheses*{Y = 2} + P\parentheses*{Y = 3}} = \frac{\frac{1}{5} + \frac{1}{10}}{\frac{1}{2} + \frac{3}{20}} = \frac{\frac{3}{10}}{\frac{13}{20}} = \frac{6}{13}.
        \]
        Im zweiten Schritt ging ein, dass sich laut Aufgabenstellung das Ereignis \(\braces*{Y \ge 2}\) disjunkt in die Ereignisse \(\braces*{Y = 2}\) und \(\braces*{Y= 3}\) und das Ereignis \(\braces*{X = 0} \cap \braces*{Y \ge 2}\) disjunkt in die Ereignisse \(\braces*{X = 0} \cap \braces*{Y = 2}\) und \(\braces*{X = 0} \cap \braces*{Y = 3}\) zerlegen lässt.
        \item Gemäß Aufgabenstellung besitzt die Zufallsvariable \(Z = X + Y\) den folgenden Wertebereich:
        \[
            \mathcal{Z} = \braces*{i + j : i \in \braces*{-1, 0, 1}, j \in \braces*{1, 2, 3}} = \braces*{0, 1, 2, 3, 4}.
        \]
        Für die zugehörige Zähldichte \(p^Z\) gilt (analog zur Situation für unabhängige Zufallsvariablen; vgl. Aufgabe 4 aus der sechsten Präsenzübung):
        \[
            p^Z\parentheses*{k} = \sum_{j \in \braces*{1, 2, 3}}P\parentheses*{X = k - j, Y = j} = \sum_{i \in \braces*{-1, 0, 1}}P\parentheses*{X = i, Y = k - i}, \quad k \in \mathcal{Z}.
        \]
        Hiermit und mit den laut Aufgabenstellung gegebenen Wahrscheinlichkeiten erhält man:
        \begin{align*}
            p^Z\parentheses*{0} &= P\parentheses*{Z = 0} = P\parentheses*{X = -1, Y = 1} = \frac{1}{20},\\
            p^Z\parentheses*{1} &= P\parentheses*{Z = 1} = P\parentheses*{X = -1, Y = 2} + P\parentheses*{X = 0, Y = 1} = \frac{1}{5} + \frac{1}{5} = \frac{2}{5},\\
            p^Z\parentheses*{2} &= P\parentheses*{Z = 2} = P\parentheses*{X = -1, Y = 3} + P\parentheses*{X = 0, Y = 2} + P\parentheses*{X = 1, Y = 1} = 0 + \frac{1}{5} + \frac{1}{10} = \frac{3}{10},\\
            p^Z\parentheses*{3} &= P\parentheses*{Z = 3} = P\parentheses*{X = 0, Y = 3} + P\parentheses*{X = 1, Y = 2} = \frac{1}{10} + \frac{1}{10} = \frac{1}{5},\\
            p^Z\parentheses*{4} &= P\parentheses*{Z = 4} = P\parentheses*{X = 1, Y = 3} = \frac{1}{20}.
        \end{align*}
    \end{enumerate}
    
    
    \section*{Aufgabe 2}
    
    \begin{problem}
        Die Riemann-Dichte \(f^{\parentheses*{X, Y}}: \R^2 \to \R\) des zweidimensionalen stetigen Zufallsvektors \(\parentheses*{X, Y}\) sei gegeben durch:
        \[
            f^{\parentheses*{X, Y}}\parentheses*{x, y} = \begin{cases}
                2, & \text{falls }x, y \ge 0\text{ und }x + y \le 1,\\
                0, & \text{sonst}.
            \end{cases}
        \]
        Bestimmen Sie die zugehörigen Randdichten \(f^X\) und \(f^Y\) von \(X\) bzw. \(Y\).
    \end{problem}
    
    \subsection*{Lösung}
    Zunächst gilt (da gemäß Aufgabenstellung die Bedingungen zur Festlegung der Dichtefunktion symmetrisch in den beiden Variablen \(x\) und \(y\) sind):
    \[
        f^{\parentheses*{X, Y}}\parentheses*{x, y} = \begin{cases}
            2, & \text{falls }x, y \ge 0\text{ und }x + y \le 1,\\
            0, & \text{sonst}
        \end{cases} = f^{\parentheses*{X, Y}}\parentheses*{y, x}
    \]
    für \(x, y \in \R\).
    D.h.: Die Dichtefunktion \(f^{\parentheses*{X, Y}}\) ist symmetrisch bzgl. ihrer beiden Argumente.
    Wir bestimmen zunächst die Randdichte \(f^X\) von \(X\).
    \begin{itemize}
        \item Fall 1: \(x \in \R \setminus \brackets*{0, 1}\).
        Dann ist \(x < 0\) oder \(x > 1\) und damit \(x + y > 1\) für \(y \ge 0\).
        Es folgt daher:
        \[
            f^X\parentheses*{x} = \int_{-\infty}^\infty \underbrace{f^{\parentheses*{X, Y}}\parentheses*{x, y}}_{= 0\text{ für }y < 0}\d y = \int_0^\infty\underbrace{f^{\parentheses*{X, Y}}\parentheses*{x, y}}_{\substack{= 0\text{, da}\\x < 0\text{ oder }x + y > 1}}\d y = 0
        \]
        \item Fall 2: \(x \in \brackets*{0, 1}\).
        Dann ist \(x \ge 0\) und \(x + y \le 1 \iff y \le 1 - x\) für \(y \in \R\).
        Hiermit folgt
        \[
            f^X\parentheses*{x} = \int_{-\infty}^\infty \underbrace{f^{\parentheses*{X, Y}}\parentheses*{x, y}}_{= 0\text{ für }y < 0}\d y = \int_0^\infty \underbrace{f^{\parentheses*{X, Y}}\parentheses*{x, y}}_{\substack{= 0\text{ für}\\x + y > 1 \iff y > 1 - x}}\d y = \int_0^{1 - x}2\d y = 2 \cdot \parentheses*{1 - x}.
        \]
    \end{itemize}
    Man erhält somit
    \[
        f^X\parentheses*{x} = \begin{cases}
            2 \cdot \parentheses*{1 - x}, & \text{falls }x \in \brackets*{0, 1},\\
            0, & \text{sonst}.
        \end{cases}
    \]
    Aufgrund der anfangs festgestellten Symmetrie gilt
    \[
        f^Y\parentheses*{y} = \int_{-\infty}^\infty f^{\parentheses*{X, Y}}\parentheses*{x, y}\d x = \int_{-\infty}^\infty f^{\parentheses*{X, Y}}\parentheses*{y, x}\d x = f^X\parentheses*{y}, \quad y \in \R.
    \]
    Insgesamt erhält man
    \[
        f^Y\parentheses*{t} = f^X\parentheses*{t} = \begin{cases}
            2 \cdot \parentheses*{1 - t}, & \text{falls }t \in \brackets*{0, 1},\\
            0, & \text{sonst}.
        \end{cases}
    \]
    
    \section*{Aufgabe 3}
    
    \begin{problem}
        In einem \(220\)-Volt-Stromkreis werde die Spannung \(U\) (in Volt) gemessen.
        Hierbei kann die Messabweichung \(X\) (in Volt) als standardnormalverteilt angenommen werden, d.h. für die gemessene Spannung \(U\) gelte:
        \[
            U = 220 + X \quad \text{mit} \quad X \sim N\parentheses*{0, 1}.
        \]
        \begin{enumerate}
            \item Wie groß ist die Wahrscheinlichkeit, dass bei einer Messung höchstens \(221\) Volt gemessen werden?
            \item Wie groß ist die Wahrscheinlichkeit, dass bei drei unabhängig voneinander erfolgenden Messungen alle drei Messwerte um weniger als \(1,1\) Volt vom Sollwert \(220\) Volt abweichen?
            \item Ab wie vielen voneinander unabhänggen Messungen ist die Wahrscheinlichkeit für das Auftreten mindestens einer Abweichung von über \(2\) Volt vom Sollwert \(220\) Volt größer als \(50\%\)?
        \end{enumerate}
    \end{problem}
    
    \subsection*{Lösung}
    Gemäß Aufgabenstellung gilt die folgende Modellgleichung:
    \begin{equation}\label{eq:1}
        \underbrace{U}_{\substack{\text{gemessene}\\\text{Spannung}}} = \underbrace{220}_{\text{Sollwert}} + \underbrace{X}_{\substack{\text{Abweichung}\\\text{vom Sollwert}}} \quad \text{mit} \quad X \sim N\parentheses*{0, 1}.
    \end{equation}
    Es bezeichnen \(\parentheses*{\Omega, \mathfrak{U}, P}\) den zugrundeliegenden Wahrscheinlichkeitsraum und \(\Phi\) die Verteilungsfunktion der Standard-Normalverteilung \(N\parentheses*{0, 1}\) (d.h. \(\Phi\) ist die Verteilungsfunktion der Zufallsvariablen \(X\)).
    Zugehörige, ausgewählte Funktionswerte \(\Phi\) sind in der Tabelle des siebten Übungsblatts zur Globalübung angegeben.
    \begin{enumerate}
        \item Man erhält mit \eqref{eq:1}
        \[
            P\parentheses*{U \le 221} = P\parentheses*{X + 220 \le 221} = P\parentheses*{X \le 1} = \Phi\parentheses*{1} \approx 0,841.
        \]
        \item Es bezeichnen \(U_1, U_2, U_3\) die hierbei gemessenen Spannungen und \(X_1, X_2, X_3\) die zugehörigen Abweichungen gemäß \eqref{eq:1}.
        Es gilt somit:
        \begin{equation}\label{eq:2}
            U_i = 220 + X_i \quad \text{mit} \quad X_i \sim N\parentheses*{0, 1} \quad \text{für} \quad i \in \braces*{1, 2, 3}.
        \end{equation}
        Hierbei sind \(U_1, U_2, U_3\) bzw. \(X_1, X_2, X_3\) stochastisch unabhängig.
        Es folgt
        \begin{align*}
            P\parentheses*{\absolute*{X_1} < 1,1, \absolute*{X_2} < 1,1, \absolute*{X_3} < 1,1} &= P\parentheses*{\absolute*{X_1} < 1,1} \cdot P\parentheses*{\absolute*{X_2} < 1,1} \cdot P\parentheses*{\absolute*{X_3} < 1,1}\\
            &= \parentheses*{P\parentheses*{\absolute*{X} < 1,1}}^3\\
            &= \parentheses*{P\parentheses*{-1,1 < X < 1,1}}^3\\
            &= P\parentheses*{P\parentheses*{-1,1 < X \le 1,1}}^3\\
            &= \parentheses*{\Phi\parentheses*{1,1} - \Phi\parentheses*{-1,1}}^3\\
            &= \parentheses*{2\Phi\parentheses*{1,1} - 1}^3\\
            &\approx \parentheses*{2 \cdot 0,864334 - 1}^3 \approx 0,387.
        \end{align*}
        Hierbei ging im zweiten Schritt ein, dass \(X_1, X_2, X_3\) und \(X\) gemäß \eqref{eq:1} und \eqref{eq:2} die gleiche Verteilung (nämlich \(N\parentheses*{0, 1}\)) besitzen und demnach gilt:
        \[
            P\parentheses*{X_i \in A} = P\parentheses*{X \in A}
        \]
        für \(i \in \braces*{1, 2, 3}\) und alle Ereignisse \(A \subseteq \R\).
        Weiter ging im dritten Schritt ein, dass die \emph{stetige} Zufallsvariable \(X\) gemäß Vorlesung jeden Wert \(x \in \R\) mit Wahrscheinlichkeit \(0\) annimmt, und im letzten Schritt wurde die Symmetrie der Standardnormalverteilung und die hieraus resultierende Gleichheit \(\Phi\parentheses*{-x} = 1 - \Phi\parentheses*{x}\) für \(x \in \R\) verwendet.
        \item Analog zu b) bezeichnen \(U_1, \ldots, U_n\) die hierbei gemessenen Spannungen und \(X_1, \ldots, X_n\) die zugehörigen Abweichungen gemäß \eqref{eq:1}.
        Es gilt somit
        \[
            U_i = 220 + X_i \quad \text{mit} \quad X_i \sim N\parentheses*{0, 1} \quad \text{für} \quad i \in \braces*{1, \ldots, n}.
        \]
        Hierbei sind \(U_1, U_2, U_3\) bzw. \(X_1, X_2, X_3\) stochastisch unabhängig.
        Analog zu b) erhält man
        \begin{align*}
            P\parentheses*{\absolute*{X_1} \le 2, \ldots, \absolute*{X_n} \le 2} &= P\parentheses*{\absolute*{X_1} \le 2} \cdot \ldots \cdot P\parentheses*{\absolute*{X_n} \le 2}\\
            &= \parentheses*{P\parentheses*{\absolute*{X} \le 2}}^n\\
            &= \parentheses*{P\parentheses*{-2 \le X \le 2}}^n\\
            &= \parentheses*{P\parentheses*{-2 < X \le 2}}^n\\
            &= \parentheses*{\Phi\parentheses*{2} - \Phi\parentheses*{-2}}^n\\
            &= \parentheses*{2\Phi\parentheses*{2} - 1}^n
        \end{align*}
        Nun folgt
        \begin{align*}
            \parentheses*{2\Phi\parentheses*{2} - 1}^n \stackrel{!}{<} 0,5 &\iff \ln\parentheses*{\parentheses*{2\Phi\parentheses*{2} - 1}^n} < \ln\parentheses*{0,5}\\
            &\iff n \cdot \underbrace{\ln(\underbrace{2\Phi\parentheses*{2} - 1}_{< 1})}_{< 0} < \ln\parentheses*{0,5}\\
            &\iff n > \frac{\ln\parentheses*{0,5}}{\ln\parentheses*{2\Phi\parentheses*{2} - 1}} \approx \frac{\ln\parentheses*{0,5}}{\ln\parentheses*{2 \cdot 0,97725 - 1}} \approx 14,88.
        \end{align*}
        Da \(n \in \N\) sein muss, ist \(n = 15\) die gesuchte Lösung, d.h.: Ab \(15\) voneinander unabhängigen Messungen ist die Wahrscheinlichkeit für das Auftreten mindestens einer Abweichung von über \(2\) Volt vom Sollwert \(220\) Volt größer als \(50\%\).
    \end{enumerate}
\end{document}
