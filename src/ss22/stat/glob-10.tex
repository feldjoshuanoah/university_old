\documentclass{exercise}

\usepackage{multirow}
\DeclareMathOperator{\Var}{Var}

\institute{Institut für Statistik und Wirtschaftsmathematik}
\title{Globalübung 9}
\author{Joshua Feld, 406718}
\course{Statistik}
\professor{Cramer}
\semester{Sommersemester 2022}
\program{CES (Bachelor)}

\begin{document}
    \maketitle


    \section*{Aufgabe 1}

    \begin{problem}
        Die Zufallsvariablen \(X_1, \ldots, X_n\) seien stochastisch unabhängig und jeweils Pareto-verteilt mit (unbekanntem) Parameter \(\alpha > 0\).
        Die zugehörige Dichtefunktion \(f_\alpha\) und die zugehörige Verteilungsfunktion \(F_\alpha\) der Zufallsvariablen \(X_i\) für \(i \in \braces*{1, \ldots, n}\) in Abhängigkeit vom Parameter \(\alpha\) sind dann gemäß Definition 11 der siebten Vorlesung gegeben durch
        \[
            f_\alpha\parentheses*{x} = \begin{cases}
                \frac{\alpha}{x^{\alpha + 1}}, \quad \text{falls }x \ge 1,\\
                0, & \text{falls }x < 1,
            \end{cases} \quad \text{bzw.} \quad F_\alpha\parentheses*{x} = \begin{cases}
                1 - \frac{1}{x^\alpha}, & \text{falls }x \ge 1,\\
                0, & \text{falls }x < 1.
            \end{cases}
        \]
        \begin{enumerate}
            \item Bestimmen Sie zu gegebenen Realisationen \(x_1, \ldots, x_n \in \parentheses*{1, \infty}\) von \(X_1, \ldots, X_n\) eine Maximum-Likelihood-Schätzung \(\hat{\alpha}\) für den Parameter \(\alpha\).
            
            \emph{Hinweis: Betrachten Sie die zugehörige Log-Likelihood-Funktion.}
            \item Berechnen Sie den aus a) resultierenden Schätzwert zu folgenden Daten:
            \[
                1,65, \quad 2,97, \quad 3,52, \quad 3,07, \quad 2,06, \quad 2,41, \quad 3,84, \quad 5,48, \quad 1,86, \quad 4,9, \quad 7,36, \quad 1,58.
            \]
            \item Approximieren Sie die Wahrscheinlichkeit dafür, dass \(X_1\) den Wert \(x = 2\) übersteigt, indem Sie den unbekannten Parameter \(\alpha\) durch den in b) berechneten Schätzwert ersetzen.
        \end{enumerate}
    \end{problem}
    
    \subsection*{Lösung}
    \begin{enumerate}
        \item
        \item
        \item
    \end{enumerate}
    
    
    \section*{Aufgabe 2}
    
    \begin{problem}
        Die Zufallsvariablen \(X_1, \ldots, X_n\) seien stochastisch unabhängig und jeweils geometrisch verteilt mit (unbekanntem) Parameter \(p \in \parentheses*{0, 1}\).
        Bestimmen Sie zu gegebenen Realisationen \(x_1, \ldots, x_n \in \N_0\) von \(X_1, \ldots, X_n\) mit \(\bar{x} > 0\) eine Maximum-Likelihood-Schätzung \(\hat{p}\) für den Parameter \(p\).
    \end{problem}
    
    \subsection*{Lösung}
    
    
    \section*{Aufgabe 3}
    
    \begin{problem}
        Eine Fluggesellschaft möchte wissen, wie hoch der Anteil \(p\) der Passagiere ist, die ihren Flug nicht antreten.
        Hierzu soll ein Konfidenzintervall für \(p\) bestimmt werden.
        Die Überprüfung von \(1000\) zufällig ausgewählten Passagieren ergibt, dass \(74\) von ihnen den Flug nicht angetreten haben.
        Bestimmen Sie anhand dieses Ergebnisses ein approximatives zweiseitiges Konfidenzintervall für \(p\) zum Konfidenzniveau \(90\%\).
    \end{problem}
    
    \subsection*{Lösung}
\end{document}
