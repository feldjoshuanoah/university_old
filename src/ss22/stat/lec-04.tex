\documentclass{lecture}

\institute{Institut für Statistik und Wirtschaftsmathematik}
\title{Vorlesung 4}
\author{Joshua Feld, 406718}
\course{Statistik}
\professor{Cramer}
\semester{Sommersemester 2022}
\program{CES (Bachelor)}

\begin{document}
    \maketitle


    \section*{Mittlere absolute Abweichung}

    Die bisher vorgestellten Streuungsmaße messen die Streuung in Relation zum arithmetischen Mittel der zu Grunde liegenden Daten.
    Die mittlere absolute Abweichung ist eine Kenngröße, die die Abweichungen der Beobachtungsdaten von deren Median zur Messung der Streuung innerhalb eines Datensatzes verwendet.
    Hierzu werden zunächst die Differenzen zwischen jedem Beobachtungswert und dem Median berechnet.
    Danach werden die Beträge dieser Differenzen, die absoluten Abweichungen, gebildet.

    \begin{definition}
        Für einen metrischen Datensatz \(x_1, \ldots, x_n\) mit zugehörigem Median \(\tilde{x}\) heißt
        \[
            d = \frac{1}{n}\sum_{i = 1}^n \absolute*{x_i - \tilde{x}}
        \]
        \emph{mittlere absolute Abweichung} \(d\) vom Median (der Daten \(x_1, \ldots, x_n\)).
    \end{definition}

    \begin{calcrule}
        Liegen die Daten in Form einer Häufigkeitsverteilung \(f_1, \ldots, f_m\) mit verschiedenen Merkmalsausprägungen \(u_1, \ldots, u_m\) des betrachteten Merkmals vor, so kann die mittlere absolute Abweichung berechnet werden als
        \[
            d = \sum_{j = 1}^m f_j\absolute*{u_j - \tilde{x}}.
        \]
    \end{calcrule}

    Werden die mittlere absolute Abweichung und die empirische Standardabweichung für den selben Datensatz ausgewertet, so liefern beide Streuungsmaße Werte in der selben Einheit.
    Die Streuungsmaße können daher direkt miteinander verglichen werden.
    In diesem Zusammenhang ist die folgende Ordnungsbeziehung gültig.

    \begin{calcrule}
        Für die mittlere absolute Abweichung \(d\) und die empirische Standardabweichung \(s\) eines Datensatzes gilt \(d \le s\).
    \end{calcrule}


    \section*{Lage- und Streuungsmaße bei linearer Transformation}

    Eine wichtige Transformation von Daten ist die lineare Transformation.

    \begin{definition}
        Für Zahlen \(a, b \in \R\) heißt die Vorschrift
        \[
            y = a + bx, \quad x \in \R,
        \]
        \emph{lineare Transformation}.
        Die Anwendung einer linearen Transformation \(y = a + bx\) auf den metrischskalierten Datensatz \(x_1, \ldots, x_n\) liefert den \emph{linear transformierten Datensatz} \(y_1, \ldots, y_n\) mit
        \[
            y_i = a + bx_i, \quad i \in \braces*{1, \ldots, n}.
        \]
    \end{definition}

    Einige der in den vorhergehenden Abschnitten vorgestellten Lage- und Streuungsmaße zeigen bzgl. linearer Transformation ein nützliches Verhalten, das in der folgenden Regel zusammengefasst wird.

    \begin{calcrule}
        Seien \(a, b \in \R\) und \(y_1, \ldots, y_n\) ein linear transformierter Datensatz von \(x_1, \ldots, x_n\):
        \[
            y_i = a + bx_i, \quad i \in \braces*{1, \ldots, n}.
        \]
        Dann gilt
        \begin{enumerate}
            \item \(\tilde{y} = a + b\tilde{x}\),
            \item \(\bar{y} = a + b\bar{x}\),
            \item \(s_y^2 = b^2 s_x^2\),
            \item \(s_y = \absolute*{b}s_x\),
            \item \(d_y = \absolute*{b}d_x\),
        \end{enumerate}
        wobei \(s_x^2, s_y^2, s_x, s_y, d_x, d_y\) die zum jeweiligen Datensatz gehörigen Streuungsmaße bezeichnen.
    \end{calcrule}

    Eine einfache Methode, Abweichungen der Beobachtungswerte zu beschreiben, ist die Zentrierung der Daten am arithmetischen Mittel.

    \begin{definition}
        Für Beobachtungswerte \(x_1, \ldots, x_n\) eines metrischen Merkmals heißt die lineare Transformation
        \[
            y_i = x_i - \bar{x}, \quad i \in \braces*{1, \ldots, n},
        \]
        \emph{Zentrierung}.
        Die transformierten Daten \(y_1, \ldots, y_n\) werden als \emph{zentriert} (oder als \emph{Residuen}) bezeichnet.
    \end{definition}

    \begin{calcrule}
        Ist \(y_1, \ldots, y_n\) der zum Datensatz \(x_1, \ldots, x_n\) gehörende zentrierte Datensatz, so gilt für das zugehörige arithmetische Mittel \(\bar{y} = 0\).
    \end{calcrule}

    Sollen Beobachtungswerte aus verschiedenen Messreihen direkt miteinander verglichen werden, so ist es sinnvoll, zusätzliche Informationen über Lage und Streuung der jeweiligen Daten zu berücksichtigen.
    Die Verwendung standardisierter Daten bietet sich an.

    \begin{definition}
        Seien \(x_1, \ldots, x_n\) Beobachtungswerte mit positiver empirischer Standardabweichung \(s_x > 0\) und arithmetischem Mittel \(\bar{x}\).
        Die lineare Transformation
        \[
            z_i = \frac{x_i - \bar{x}}{s_x}, \quad i \in \braces*{1, \ldots, n},
        \]
        der Daten heißt \emph{Standardisierung}.
        Die transformierten Daten \(z_1, \ldots, z_n\) werden als \emph{standardisiert} bezeichnet.
    \end{definition}

    Durch eine Standardisierung können unterschiedliche Datensätze so transformiert werden, dass die arithmetischen Mittelwerte und die Standardabweichungen in allen Datensätzen gleich sind.

    \begin{calcrule}
        Für standardisierte Beobachtungswerte \(z_1, \ldots, z_n\) gilt \(\bar{z} = 0\) und \(s_z = 1\).
    \end{calcrule}


    \section*{Grundlagen der Wahrscheinlichkeitsrechnung}

    In diesem Abschnitt werden grundlegende Begriffe und Bezeichnungen eingeführt.

    \begin{definition}
        Die Menge aller möglichen Ergebnisse eines Zufallsvorgangs (Zufallsexperiments) wird \emph{Grundraum} (\emph{Grundmenge}, \emph{Ergebnisraum}) genannt und meistens mit dem griechischen Buchstaben \(\Omega\) bezeichnet:
        \[
            \Omega = \braces*{\omega : \omega\text{ ist mögliches Ergebnis eines zufallsabhängigen Vorgangs}}.
        \]
        Ein Element \(\omega\) von \(\Omega\) heißt \emph{Ergebnis}.
        Eine Menge von Ergebnissen heißt \emph{Ereignis}.
        Ereignisse werden meist mit großen lateinischen Buchstaben \(A, B, C, \ldots\) bezeichnet.
        Ein Ereignis, das genau ein Element besitzt, heißt \emph{Elementarereignis}.
    \end{definition}

    \begin{example}
        Das Zufallsexperiment eines einfachen Würfelwurfs wird betrachtet.
        Die möglichen Ergebnisse sind die Ziffern \(1\), \(2\), \(3\), \(4\), \(5\) und \(6\), d.h. der Grundraum \(\Omega = \braces*{1, 2, 3, 4, 5, 6}\).
        Die Elementarereignisse sind \(\braces*{1}, \braces*{2}, \braces*{3}, \braces*{4}, \braces*{5}, \braces*{6}\).
        Andere Ereignisse sind etwa:
        \begin{itemize}
            \item ``Es fällt eine gerade Ziffer'': \(A = \braces*{2, 4, 6}\)
            \item ``Es fällt eine Ziffer kleiner als \(3\)'': \(B = \braces*{1, 2}\)
        \end{itemize}
        Kombinationen von Ereignissen sind von besonderem Interesse.
        Dazu werden folgende Bezeichnungen vereinbart.
    \end{example}

    \begin{remark}
        Seien \(I\) eine Indexmenge sowie \(A\), \(B\) und \(A_i, i \in I\) Ereignisse in einem Grundraum \(\Omega\).
        \begin{itemize}
            \item \(A \cap B = \braces*{\omega \in \Omega : \omega \in A\text{ und }\omega \in B}\) heißt Schnittereignis der Ereignisse \(A\) und \(B\).
            \item \(\bigcap_{i \in I}A_i = \braces*{\omega \in \Omega : \omega \in A_i\text{ für jedes }i \in I}\) heißt Schnittereignis der Ereignisse \(A_i, i \in I\).
            \item Die Ereignisse \(A\) und \(B\) heißen disjunkt, falls \(A \cap B = \emptyset\).
            \item Die Ereignisse \(A_i, i \in I\) heißen paarweise disjunkt, falls für jede Auswahl zweier verschiedener Indizes \(i, j \in I\) gilt: \(A_i \cap A_j = \emptyset\).
            \item \(A \cup B = \braces*{\omega \in \Omega : \omega \in A\text{ oder }\omega \in B}\) heißt Vereinigungsereignis der Ereignisse \(A\) und \(B\).
            \item \(\bigcup_{i \in I}A_i = \braces*{\omega \in \Omega : \text{es gibt ein }i \in I\text{ mit }\omega \in A_i}\) heißt Vereinigungsereignis der Ereignisse \(A_i, i \in I\).
            \item Gilt \(A \subseteq B\), d.h. für jedes \(\omega \in A\) gilt \(\omega \in B\), so heißt \(A\) Teilereignis von \(B\).
            \item Die Menge \(A^c = \Omega \setminus A = \braces*{\omega \in \Omega : \omega \not\in A}\) ist das Komplementärereignis von \(A\); \(A^c\) heißt Komplement von \(A\) (in \(\Omega\)).
            \item Die Menge \(B \setminus A = \braces*{\omega \in \Omega : \omega \in B\text{ und }\omega \not\in A} = B \cap A^c\) ist das Differenzereignis von \(B\) und \(A\); \(B \setminus A\) heißt Komplement von \(A\) in \(B\).
        \end{itemize}
    \end{remark}

    \begin{definition}
        Seien \(\Omega = \braces*{\omega_1, \omega_2, \omega_3, \ldots}\) ein endlicher oder abzählbar unendlicher Grundraum und \(\mathfrak{A} = \Pot\parentheses*{\Omega}\) die Potenzmenge von \(\Omega\) (also die Menge aller Ereignisse über \(\Omega\)).
        Ferner sei \(p: \Omega \to \brackets*{0, 1}\) eine Abbildung mit \(\sum_{\omega \in \Omega}p\parentheses*{\omega} = 1\).
        Die durch
        \[
            P\parentheses*{A} = \sum_{\omega \in A}p\parentheses*{\omega}, \quad A \in \mathfrak{A},
        \]
        definierte Abbildung \(P: \mathfrak{A} \to \brackets*{0, 1}, A \mapsto P\parentheses*{A}\), die jedem Ereignis \(A\) eine Wahrscheinlichkeit \(P\parentheses*{A}\) zuordnet, heißt \emph{diskretes Wahrscheinlichkeitsmaß} oder \emph{diskrete Wahrscheinlichkeitsverteilung} auf \(\mathfrak{A}\) (oder über \(\Omega\)).
        Die Abbildung \(p\) heißt \emph{Zähldichte}.

        Für \(\omega \in \Omega\) heißt \(p\parentheses*{\omega} = P\parentheses*{\braces*{\omega}}\) \emph{Elementarwahrscheinlichkeit} des Elementarereignisses \(\braces*{\omega}\); als Kurzschreibweise wird \(P\parentheses*{\omega} = P\parentheses*{\braces*{\omega}}\) verwendet.
    \end{definition}

    \begin{remark}
        Eine diskrete Wahrscheinlichkeitsverteilung \(P\) besitzt folgende Eigenschaften:
        \begin{enumerate}
            \item \(0 \le P\parentheses*{A} \le 1\) für jedes Ereignis \(A \in \mathfrak{A}\),
            \item \(P\parentheses*{\Omega} = 1\),
            \item \(P\) ist \(\sigma\)-additiv, d.h. für alle paarweise disjunkten Ereignisse \(A_i, i \in \N\) gilt
            \[
                P\parentheses*{\bigcup_{i = 1}^\infty A_i} = \sum_{i = 1}^\infty P\parentheses*{A_i}.
            \]
            Insbesondere ist \(P\parentheses*{\bigcup_{i = 1}^n A_i} = \sum_{i = 1}^n P\parentheses*{A_i}\) für alle \(n \in \N\).
        \end{enumerate}
    \end{remark}

    Diese Eigenschaften heißen auch Kolmogorov-Axiome.

    \begin{definition}
        Seien \(\Omega = \braces*{\omega_1, \omega_2, \ldots}\) ein endlicher oder abzählbar unendlicher Grundraum, \(\mathfrak{A} = \Pot\parentheses*{\Omega}\) und \(P\) ein diskretes Wahrscheinlichkeitsmaß auf \(\mathfrak{A}\).
        Das Paar \(\parentheses*{\Omega, P}\) wird \emph{diskreter Wahrscheinlichkeitsraum} genannt.
        Ist der Grundraum endlich, d.h. \(\Omega = \braces*{\omega_1, \ldots, \omega_n}\), so wird \(\parentheses*{\Omega, P}\) als \emph{endlicher diskreter Wahrscheinlichkeitsraum} bezeichnet.
    \end{definition}

    \begin{example}
        Ist \(\Omega = \braces*{\omega_1, \ldots, \omega_n}, n \in \N\) eine endliche Menge bestehend aus \(n\) Elementen, dann wird durch die Vorschrift
        \[
            P\parentheses*{A} = \frac{\absolute*{A}}{\absolute*{\Omega}} = \frac{\absolute*{A}}{n}, \quad A \subseteq \Omega,
        \]
        ein Wahrscheinlichkeitsmaß über \(\Pot\parentheses*{\Omega}\) definiert.
        Dabei bezeichnet \(\absolute*{A}\) die Anzahl der Elemente des Ereignisses \(A\).
        Das auf diese Weise definierte Wahrscheinlichkeitsmaß \(P\) wird als Laplace-Verteilung oder auch als diskrete Gleichverteilung auf \(\Omega\) bezeichnet.
        \(\parentheses*{\Omega, P}\) heißt Laplace-Raum über \(\Omega\).
    \end{example}

    \begin{example}
        Der einfache Würfelwurf wird modelliert durch die Grundmenge \(\Omega = \braces*{1, \ldots, 6}\) und die Laplace-Verteilung auf \(\Omega\).
        Die Wahrscheinlichkeit eines beliebigen Ereignisses \(A \in \text{Pot}\parentheses*{\Omega}\) wird berechnet gemäß \(P\parentheses*{A} = \frac{\absolute*{A}}{6}\).
        Beispielsweise wird das Ereignis \(A\) ``Augenzahl ungerade'' beschrieben durch \(A = \braces*{1, 3, 5}\), so dass seine Wahrscheinlichkeit im Laplace-Modell durch
        \[
            P\parentheses*{A} = P\parentheses*{\braces*{1, 3, 5}} = \frac{\absolute*{\braces*{1, 3, 5}}}{6} = \frac{3}{6} = \frac{1}{2}
        \]
        gegeben ist.
    \end{example}

    In einem Laplace-Raum \(\parentheses*{\Omega, P}\) über einer Menge \(\Omega = \braces*{\omega_1, \ldots, \omega_n}, n \in \N\) ist die Berechnung von Wahrscheinlichkeiten besonders einfach.
    Für jedes Elementarereignis gilt
    \[
        P\parentheses*{\braces*{\omega_i}} = \frac{1}{\absolute*{\Omega}} = \frac{1}{n}, \quad \omega_i \in \Omega,
    \]
    d.h. die Wahrscheinlichkeit eines jeden Elementarereignisses ist gleich \(\frac{1}{n}\).
    Damit ist die Zähldichte gegeben durch \(p\parentheses*{\omega} = \frac{1}{n}, \omega \in \Omega\).
    Die Wahrscheinlichkeit eines beliebigen Ereignisses berechnet sich aus der Anzahl der Elemente des Ereignisses.
    Bezeichnet man diese Ereignisse als günstige Fälle, so erhält man die folgende Merkregel:
    \[
        P\parentheses*{A} = \frac{\absolute*{A}}{\absolute*{\Omega}} = \frac{\text{Anzahl günstiger Fälle}}{\text{Anzahl möglicher Fälle}}.
    \]

    \begin{example}
        Ein Problem des Chevalier de Méré aus dem 17. Jahrhundert lautet: Was ist bei drei Würfelwürfen wahrscheinlicher: Augensumme gleich \(11\) oder Augensumme gleich \(12\)?

        Die Situation wird modelliert durch den Grundraum aller Tripel mit einer der Ziffern \(1, \ldots, 6\) in den Komponenten: \(\Omega = \braces*{\omega = \parentheses*{\omega_1, \omega_2, \omega_3} : \omega_i \in \braces*{1, \ldots, 6}, i \in \braces*{1, 2, 3}}\) sowie der Zähldichte \(p\parentheses*{\omega} = \frac{1}{\absolute*{\Omega}} = \frac{1}{6^3}\) für alle \(\omega \in \Omega\) (aus Symmetriegründen).

        Von Interesse sind die Ereignisse \(A = \braces*{\omega \in \Omega : \omega_1 + \omega_2 + \omega_3 = 11}\) und \(B = \braces*{\omega \in \Omega : \omega_1 + \omega_2 + \omega_3 = 12}\) im Laplace-Raum \(\parentheses*{\Omega, P}\).
        Informell lassen sich diese Mengen tabellarisch darstellen:
        \begin{center}
            \begin{tabular}{cccc}
                \toprule
                & \(A\) & & \\
                \midrule
                \(6\) & \(4\) & \(1\) & \(\parentheses*{6}\)\\
                \(6\) & \(3\) & \(2\) & \(\parentheses*{6}\)\\
                \(5\) & \(5\) & \(1\) & \(\parentheses*{3}\)\\
                \(5\) & \(4\) & \(2\) & \(\parentheses*{6}\)\\
                \(5\) & \(3\) & \(3\) & \(\parentheses*{3}\)\\
                \(4\) & \(4\) & \(3\) & \(\parentheses*{3}\)\\
                \midrule
                \(\sum\) & \(11\) & & \(\parentheses*{27}\)\\
                \bottomrule
            \end{tabular}
            \quad
            \begin{tabular}{cccc}
                \toprule
                & \(B\) & & \\
                \midrule
                \(6\) & \(5\) & \(1\) & \(\parentheses*{6}\)\\
                \(6\) & \(4\) & \(2\) & \(\parentheses*{6}\)\\
                \(6\) & \(3\) & \(3\) & \(\parentheses*{3}\)\\
                \(5\) & \(5\) & \(2\) & \(\parentheses*{3}\)\\
                \(5\) & \(4\) & \(3\) & \(\parentheses*{6}\)\\
                \(4\) & \(4\) & \(4\) & \(\parentheses*{1}\)\\
                \midrule
                \(\sum\) & \(12\) & & \(\parentheses*{25}\)\\
                \bottomrule
            \end{tabular}
        \end{center}
        Es gibt zwar jeweils sechs Fälle von ``Zahlenkombinationen'', aber \(\Omega\) ist eine Menge von Tripeln, d.h. die Würfel werden als unterscheidbar modelliert.
        Beispielsweise gilt \(\parentheses*{6, 4, 1} \ne \parentheses*{6, 1, 4} \ne \parentheses*{1, 6, 4}\), usw.
        Abzählen der Fälle liefert \(\absolute*{A} = 27\) und \(\absolute*{B} = 25\), und somit ergibt sich
        \[
            P\parentheses*{A} = \frac{27}{216} \ge \frac{25}{216} = P\parentheses*{B}.
        \]
    \end{example}

    Liegt ein Laplace-Raum vor, so reduziert sich die Berechnung von Wahrscheinlichkeiten also auf das Abzählen von Elementen eines Ereignisses.
    Mit solchen Fragestellungen beschäftigt sich die Kombinatorik.
    Ehe die Grundmodelle der Kombinatorik eingeführt werden, wird noch der relevante Bereich einer Wahrscheinlichkeitsverteilung, der Träger, eingeführt.

    \begin{definition}
        Sei \(\parentheses*{\Omega, P}\) ein diskreter Wahrscheinlichkeitsraum.
        Die Menge \(T = \braces*{\omega \in \Omega : P\parentheses*{\omega} > 0}\) heißt \emph{Träger} von \(P\).
    \end{definition}
\end{document}
