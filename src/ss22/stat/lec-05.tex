\documentclass{lecture}

\institute{Institut für Statistik und Wirtschaftsmathematik}
\title{Vorlesung 5}
\author{Joshua Feld, 406718}
\course{Statistik}
\professor{Cramer}
\semester{Sommersemester 2022}
\program{CES (Bachelor)}

\begin{document}
    \maketitle


    \section*{Urnenmodelle}

    Zur Veranschaulichung einfacher Stichprobenverfahren und damit der Bestimmung der Mächtigkeit endlicher Mengen werden Urnenmodelle verwendet.
    Eine Urne enthalte dazu \(n\) nummerierte Kugeln (mit den Nummern \(1, \ldots, n\)), die die Grundgesamtheit oder den Grundraum bilden.
    
    Das Ziehen einer Kugel aus der Urne entspricht der (zufälligen) Auswahl eines Objektes aus der Grundgesamtheit.
    Die Erhebung einer Stichprobe vom Umfang \(k\) aus einer Grundgesamtheit von \(n\) Objekten entspricht daher dem Ziehen von \(k\) Kugeln aus einer Urne mit \(n\) Kugeln.
    Die Urne wird im Folgenden als die Menge der Zahlen \(1, \ldots, n\) verstanden: \(\mathcal{U}_n = \braces*{1, \ldots, n}\), wobei die Zahl \(i\) der \(i\)-ten Kugel entspricht.
    Resultat einer Ziehung von k Kugeln ist ein geordnetes Tupel \(\parentheses*{\omega_1, \ldots, \omega_k}\), wobei \(\omega_i\) die im \(i\)-ten Zug entnommene Kugel repräsentiert (z.B. durch deren Nummer).
    Jede Kugel werde jeweils mit derselben Wahrscheinlichkeit gezogen, d.h. als Ausgangspunkt wird ein Laplace-Modell gewählt.

    Im Folgenden werden insgesamt vier Urnenmodelle nach dem Ziehungsablauf und der Notation der Ziehung unterschieden:
    \begin{enumerate}[label=(\roman*)]
        \item Ziehungsablauf
        \begin{enumerate}[label=\alph*)]
            \item Die gezogene Kugel wird nach Feststellung ihrer Nummer in die Urne zurückgelegt.
            \item Die gezogene Kugel wird nach Feststellung ihrer Nummer nicht in die Urne zurückgelegt.
        \end{enumerate}
        \item Notation der Ziehung
        \begin{enumerate}[label=\alph*)]
            \item Die Reihenfolge der Ziehungen wird berücksichtigt.
            \item Die Reihenfolge der Ziehungen wird nicht berücksichtigt.
        \end{enumerate}
    \end{enumerate}
    Für die Urnenmodelle werden folgende Bezeichnungen verwendet.
    \begin{center}
        \begin{tabular}{l|cc}
            \toprule
            \makecell[l]{Ziehen von \(k\) Kugeln\\aus \(n\) Kugeln} & mit Zurücklegen & ohne Zurücklegen\\
            \midrule
            \makecell[l]{mit Berücksichtigung\\der Reihenfolge} & \makecell{\(\parentheses*{n, k}\)-Permutationen\\mit Wiederholung} & \makecell{\(\parentheses*{n, k}\)-Permutationen\\ohne Wiederholung}\\
            \makecell[l]{ohne Berücksichtigung\\der Reihenfolge} & \makecell{\(\parentheses*{n, k}\)-Kombinationen\\mit Wiederholung} & \makecell{\(\parentheses*{n, k}\)-Kombinationen\\ohne Wiederholung}\\
            \bottomrule
        \end{tabular}
    \end{center}

    \begin{definition}
        Die Menge aller \(\parentheses*{n, k}\)-Permutationen mit Wiederholung ist die Menge aller Ergebnisse, die im Urnenmodell mit Zurücklegen und mit Berücksichtigung der Reihenfolge auftreten können (\(n\) Kugeln in der Urne, \(k\) Ziehungen).
        Ist \(\mathcal{U}_n = \braces*{1, \ldots, n}\) die Menge der in der Urne enthaltenen Kugeln, so beschreibt
        \[
            \Omega_{\text{PmW}} = \braces*{\parentheses*{\omega_1, \ldots, \omega_k} : \omega_i \in \mathcal{U}_n, 1 \le i \le k}
        \]
        die Menge aller \(\parentheses*{n, k}\)-Permutationen mit Wiederholung über \(\mathcal{U}_n\).
        Ein Element \(\parentheses*{\omega_1, \ldots, \omega_k}\) von \(\Omega_{\text{PmW}}\) heißt \emph{\(\parentheses*{n, k}\)-Permutation mit Wiederholung} über \(\mathcal{U}_n\).
    \end{definition}

    Durch \(\parentheses*{\Omega_{\text{PmW}}, P}\) mit
    \[
        P\parentheses*{A} = \frac{\absolute*{A}}{\absolute*{\Omega_{\text{PmW}}}} = \frac{\absolute*{A}}{n^k}, \quad A \subseteq \Omega_{\text{PmW}},
    \]
    wird ein Laplace-Raum auf \(\Omega_{\text{PmW}}\) definiert.
    Die Mächtigkeit von \(\Omega_{\text{PmW}}\) ist durch die Zahl
    \[
        \Per_{\text{mW}}\parentheses*{n, k} = \absolute*{\Omega_{\text{PmW}}} = \underbrace{n \cdot n \cdot \ldots \cdot n}_{k\text{-mal}} = n^k
    \]
    gegeben, d.h. es gibt \(n^k\) Möglichkeiten, \(k\) Kugeln aus einer Urne mit \(n\) Kugeln mit Zurücklegen und mit Beachtung der Zugreihenfolge zu entnehmen.

    \begin{example}
        Eine Urne enthält vier Kugeln, die mit \(1\), \(2\), \(3\) und \(4\) nummeriert sind.
        Drei Mal hintereinander wird aus dieser Urne eine Kugel entnommen, ihre Zahl notiert und danach wieder zurückgelegt.
        Gesucht ist die Anzahl der \(\parentheses*{4, 3}\)-Permutationen mit Wiederholung.

        Dazu wird zunächst die Menge \(\Omega_{\text{PmW}}\) aller \(\parentheses*{4, 3}\)-Permutationen mit Wiederholung explizit angegeben:
        \begin{align*}
            \Omega_{\text{PmW}} = \{&\parentheses*{1, 1, 1}, \parentheses*{1, 1, 2}, \parentheses*{1, 1, 3}, \parentheses*{1, 1, 4}, \parentheses*{1, 2, 1}, \parentheses*{1, 2, 2}, \parentheses*{1, 2, 3}, \parentheses*{1, 2, 4},\\
            &\parentheses*{1, 3, 1}, \parentheses*{1, 3, 2}, \parentheses*{1, 3, 3}, \parentheses*{1, 3, 4}, \parentheses*{1, 4, 1}, \parentheses*{1, 4, 2}, \parentheses*{1, 4, 3}, \parentheses*{1, 4, 4},\\
            &\parentheses*{2, 1, 1}, \parentheses*{2, 1, 2}, \parentheses*{2, 1, 3}, \parentheses*{2, 1, 4}, \parentheses*{2, 2, 1}, \parentheses*{2, 2, 2}, \parentheses*{2, 2, 3}, \parentheses*{2, 2, 4},\\
            &\parentheses*{2, 3, 1}, \parentheses*{2, 3, 2}, \parentheses*{2, 3, 3}, \parentheses*{2, 3, 4}, \parentheses*{2, 4, 1}, \parentheses*{2, 4, 2}, \parentheses*{2, 4, 3}, \parentheses*{2, 4, 4},\\
            &\parentheses*{3, 1, 1}, \parentheses*{3, 1, 2}, \parentheses*{3, 1, 3}, \parentheses*{3, 1, 4}, \parentheses*{3, 2, 1}, \parentheses*{3, 2, 2}, \parentheses*{3, 2, 3}, \parentheses*{3, 2, 4},\\
            &\parentheses*{3, 3, 1}, \parentheses*{3, 3, 2}, \parentheses*{3, 3, 3}, \parentheses*{3, 3, 4}, \parentheses*{3, 4, 1}, \parentheses*{3, 4, 2}, \parentheses*{3, 4, 3}, \parentheses*{3, 4, 4},\\
            &\parentheses*{4, 1, 1}, \parentheses*{4, 1, 2}, \parentheses*{4, 1, 3}, \parentheses*{4, 1, 4}, \parentheses*{4, 2, 1}, \parentheses*{4, 2, 2}, \parentheses*{4, 2, 3}, \parentheses*{4, 2, 4},\\
            &\parentheses*{4, 3, 1}, \parentheses*{4, 3, 2}, \parentheses*{4, 3, 3}, \parentheses*{4, 3, 4}, \parentheses*{4, 4, 1}, \parentheses*{4, 4, 2}, \parentheses*{4, 4, 3}, \parentheses*{4, 4, 4}\}.
        \end{align*}
        Abzählen ergibt \(64\) verschiedene \(\parentheses*{4, 3}\)-Permutationen mit Wiederholung.
        Mit Anwendung der allgemeinen Formel berechnet sich die Anzahl der \(\parentheses*{4, 3}\)-Permutationen mit Wiederholung gemäß
        \[
            \absolute*{\Omega_{\text{PmW}}} = 4^3 = 64.
        \]
    \end{example}

    \begin{definition}
        Die Menge aller \(\parentheses*{n, k}\)-Permutationen ohne Wiederholung ist die Menge aller Ergebnisse, die im Urnenmodell ohne Zurücklegen und mit Berücksichtigung der Reihenfolge auftreten können (\(n\) Kugeln in der Urne, \(k\) Ziehungen).
        Ist \(\mathcal{U}_n = \braces*{1, \ldots, n}\) die Menge der in der Urne enthaltenen Kugeln, so beschreibt
        \[
            \Omega_{\text{PoW}} = \braces*{\parentheses*{\omega_1, \ldots, \omega_k} : \omega_i \in \mathcal{U}_n, 1 \le i \le k\text{ und }\omega_i \ne \omega_j\text{ für }1 \le i \ne j \le k}
        \]
        die Menge aller \(\parentheses*{n, k}\)-Permutationen ohne Wiederholung über \(\mathcal{U}_n\).
        Bei diesem Urnenmodell ist die Anzahl der Ziehungen \(k\) notwendig kleiner oder gleich der Anzahl \(n\) von Kugeln in der Urne, d.h. \(1 \le k \le n\).
        Ein Element von \(\Omega_{\text{PoW}}\) heißt \emph{\(\parentheses*{n, k}\)-Permutation mit Wiederholung} über \(\mathcal{U}_n\).
    \end{definition}

    Durch \(\parentheses*{\Omega_{\text{PoW}}, P}\) mit
    \[
        P\parentheses*{A} = \frac{\absolute*{A}}{\absolute*{\Omega_{\text{PoW}}}} = \frac{\absolute*{A}}{\frac{n!}{\parentheses*{n - k}!}} = \frac{\parentheses*{n - k}!}{n!}\absolute*{A}, \quad A \subseteq \Omega_{\text{PoW}},
    \]
    wird ein Laplace-Raum auf \(\Omega_{\text{PoW}}\) definiert.
    Die Mächtigkeit von \(\Omega_{\text{PoW}}\) ist durch die Zahl
    \[
        \Per_{\text{oW}}\parentheses*{n, k} = \absolute*{\Omega_{\text{PoW}}} = n \cdot \parentheses*{n - 1} \cdot \ldots \cdot \parentheses*{n - k + 1} = \frac{n!}{\parentheses*{n - k}!}
    \]
    gegeben, d.h. es gibt \(\frac{n!}{\parentheses*{n - k}!}\) Möglichkeiten, \(k\) Kugeln aus einer Urne mit \(n\) Kugeln ohne Zurücklegen und mit Beachtung der Zugreihenfolge zu entnehmen.
    Speziell für \(n = k\) gilt \(\absolute*{\Omega_{\text{PoW}}} = n!\) und \(\Omega_{\text{PoW}}\) ist die Menge aller Permutationen der Zahlen von \(1\) bis \(n\).

    \begin{example}
        Eine Urne enthält vier Kugeln, die mit \(1\), \(2\), \(3\) und \(4\) nummeriert sind.
        Drei Mal hintereinander wird aus dieser Urne eine Kugel entnommen, ihre Zahl notiert und danach zur Seite gelegt.
        Es werden also \(3\)-Tupel mit den Zahlen \(1\), \(2\), \(3\) und \(4\) notiert, wobei jede Zahl höchstens ein Mal vorkommen darf und die Zugreihenfolge berücksichtigt wird.
        Gesucht ist die Anzahl der \(\parentheses*{4, 3}\)-Permutationen ohne Wiederholung.

        Dazu wird zunächst die Menge \(\Omega_{\text{PoW}}\) aller \(\parentheses*{4, 3}\)-Permutationen ohne Wiederholung explizit angegeben:
        \begin{align*}
            \Omega_{\text{PoW}} = \{&\parentheses*{1, 2, 3}, \parentheses*{1, 2, 4}, \parentheses*{1, 3, 2}, \parentheses*{1, 3, 4}, \parentheses*{1, 4, 2}, \parentheses*{1, 4, 3},\\
            &\parentheses*{2, 1, 3}, \parentheses*{2, 1, 4}, \parentheses*{2, 3, 1}, \parentheses*{2, 3, 4}, \parentheses*{2, 4, 1}, \parentheses*{2, 4, 3},\\
            &\parentheses*{3, 1, 2}, \parentheses*{3, 1, 4}, \parentheses*{3, 2, 1}, \parentheses*{3, 2, 4}, \parentheses*{3, 4, 1}, \parentheses*{3, 4, 2},\\
            &\parentheses*{4, 1, 2}, \parentheses*{4, 1, 3}, \parentheses*{4, 2, 1}, \parentheses*{4, 2, 3}, \parentheses*{4, 3, 1}, \parentheses*{4, 3, 2}\}.
        \end{align*}
        Durch Abzählen erhält man, dass es \(24\) verschiedene \(\parentheses*{4, 3}\)-Permutationen ohne Wiederholung gibt.
        Mit Anwendung der allgemeinen Formel berechnet sich die Anzahl der \(\parentheses*{4, 3}\)-Permutationen ohne Wiederholung gemäß
        \[
            \absolute*{\Omega_{\text{PoW}}} = \frac{4!}{\parentheses*{4 - 3}!} = 24.
        \]
    \end{example}

    \begin{definition}
        Die Menge aller \(\parentheses*{n, k}\)-Kombinationen ohne Wiederholung ist die Menge aller Ergebnisse, die im Urnenmodell ohne Zurücklegen und ohne Berücksichtigung der Reihenfolge auftreten können (\(n\) Kugeln in der Urne, \(k\) Ziehungen).
        Ist die Menge der in der Urne enthaltenen Kugeln gegeben durch \(\mathcal{U}_n = \braces*{1, \ldots, n}\), so beschreiben
        \[
            \Omega_{\text{KoW}} = \braces*{\parentheses*{\omega_1, \ldots, \omega_k} : \omega_i \in \mathcal{U}_n, \omega_1 < \cdots < \omega_k}
        \]
        oder alternativ
        \[
            \Omega_{\text{KoW}}' = \braces*{A \subseteq \mathcal{U}_n : \absolute*{A} = k}
        \]
        die Menge aller \(\parentheses*{n, k}\)-Kombinationen ohne Wiederholung über \(\mathcal{U}_n\).
        Ein Element von \(\Omega_{\text{KoW}}\) heißt \emph{\(\parentheses*{n, k}\)-Kombination ohne Wiederholung} über \(\mathcal{U}_n = \braces*{1, \ldots, n}\).
    \end{definition}

    Da die Reihenfolge bei \(\parentheses*{n, k}\)-Kombinationen ohne Bedeutung ist, werden die Einträge \(\omega_i\) des \(k\)-Tupels aufsteigend geordnet.
    Durch \(\parentheses*{\Omega_{\text{KoW}}, P}\) mit
    \[
        P\parentheses*{A} = \frac{\absolute*{A}}{\absolute*{\Omega_{\text{KoW}}}} = \frac{\absolute*{A}}{\binom{n}{k}}, \quad A \subseteq \Omega_{\text{KoW}},
    \]
    wird ein Laplace-Raum auf \(\Omega_{\text{KoW}}\) definiert.
    Die Mächtigkeit von \(\Omega_{\text{KoW}}\) ist durch die Zahl
    \[
        \Kom_{\text{oW}}\parentheses*{n, k} = \absolute*{\Omega_{\text{KoW}}} = \frac{n!}{\parentheses*{n - k}! \cdot k!} = \binom{n}{k}
    \]
    gegeben, d.h. es gibt \(\binom{n}{k}\) Möglichkeiten, \(k\) Kugeln aus einer Urne mit \(n\) Kugeln ohne Zurücklegen und ohne Beachtung der Zugreihenfolge zu entnehmen.

    \begin{example}
        Eine Urne enthält vier Kugeln, die mit \(1\), \(2\), \(3\) und \(4\) nummeriert sind.
        Drei Kugeln werden nacheinander der Urne entnommen, ohne dass eine zurückgelegt wird.
        Anschließend werden die Kugeln gemäß ihrer Nummer aufsteigend sortiert.
        Alternativ kann die Ziehung auch so durchgeführt werden, dass die drei Kugeln auf einmal aus dieser Urne entnommen werden und die Zahlen aufsteigend notiert werden.
        Es werden also \(3\)-Tupel mit den Zahlen \(1\), \(2\), \(3\) und \(4\) notiert, wobei jede Zahl höchstens ein Mal vorkommen darf und die Zugreihenfolge nicht berücksichtigt wird.
        Gesucht ist die Anzahl der \(\parentheses*{4, 3}\)-Kombinationen ohne Wiederholung.

        Dazu wird zunächst die Menge \(\Omega_{\text{KoW}}\) aller \(\parentheses*{4, 3}\)-Kombinationen ohne Wiederholung explizit angegeben:
        \begin{align*}
            \Omega_{\text{KoW}} = \{&\parentheses*{1, 2, 3}, \parentheses*{1, 2, 4}, \parentheses*{1, 3, 2}, \parentheses*{1, 3, 4}, \parentheses*{1, 4, 2}, \parentheses*{1, 4, 3},\\
            &\parentheses*{2, 1, 3}, \parentheses*{2, 1, 4}, \parentheses*{2, 3, 1}, \parentheses*{2, 3, 4}, \parentheses*{2, 4, 1}, \parentheses*{2, 4, 3},\\
            &\parentheses*{3, 1, 2}, \parentheses*{3, 1, 4}, \parentheses*{3, 2, 1}, \parentheses*{3, 2, 4}, \parentheses*{3, 4, 1}, \parentheses*{3, 4, 2},\\
            &\parentheses*{4, 1, 2}, \parentheses*{4, 1, 3}, \parentheses*{4, 2, 1}, \parentheses*{4, 2, 3}, \parentheses*{4, 3, 1}, \parentheses*{4, 3, 2}\}.
        \end{align*}
        Durch Abzählen erhält man, dass es \(24\) verschiedene \(\parentheses*{4, 3}\)-Permutationen ohne Wiederholung gibt.
        Mit Anwendung der allgemeinen Formel berechnet sich die Anzahl der \(\parentheses*{4, 3}\)-Permutationen ohne Wiederholung gemäß
        \[
            \absolute*{\Omega_{\text{KoW}}} = \binom{4}{3} = \frac{4!}{3! \cdot \parentheses*{4 - 3}!} = \frac{24}{6} = 4.
        \]
    \end{example}

    Das folgende, vierte Grundmodell der Kombinatorik führt nicht auf einen Laplace-Raum.

    \begin{definition}
        Die Menge aller \(\parentheses*{n, k}\)-Kombinationen mit Wiederholung ist die Menge aller Ergebnisse, die im Urnenmodell mit Zurücklegen und ohne Berücksichtigung der Reihenfolge auftreten können (\(n\) Kugeln in der Urne, \(k\) Ziehungen).
        Ist die Menge der in der Urne enthaltenen Kugeln gegeben durch \(\mathcal{U}_n = \braces*{1, \ldots, n}\), so beschreiben
        \[
            \Omega_{\text{KmW}} = \braces*{\parentheses*{\omega_1, \ldots, \omega_k} : \omega_i \in \mathcal{U}_n, \omega_1 \le \cdots \le \omega_k}
        \]
        oder alternativ
        \[
            \Omega_{\text{KoW}}' = \braces*{A \subseteq \mathcal{U}_n : \absolute*{A} = k}
        \]
        die Menge aller \(\parentheses*{n, k}\)-Kombinationen mit Wiederholung über \(\mathcal{U}_n\).
        Ein Element von \(\Omega_{\text{KoW}}\) heißt \emph{\(\parentheses*{n, k}\)-Kombination mit Wiederholung} über \(\mathcal{U}_n = \braces*{1, \ldots, n}\).
    \end{definition}

    Da die Reihenfolge bei \(\parentheses*{n, k}\)-Kombinationen ohne Bedeutung ist, werden die Einträge \(\omega_i\) des \(n\)-Tupels aufsteigend geordnet.
    Hierbei ist zu beachten, dass Einträge mehrfach auftreten können.

    Die Mächtigkeit von \(\Omega_{\text{KmW}}\) ist durch die Zahl
    \[
        \Kom_{\text{mW}}\parentheses*{n, k} = \binom{n + k - 1}{k} = \frac{\parentheses*{n + k - 1}!}{\parentheses*{n - 1}! \cdot k!}
    \]
    gegeben, d.h. es gibt \(\binom{n + k - 1}{k}\) Möglichkeiten, \(k\) Kugeln aus einer Urne mit \(n\) Kugeln mit Zurücklegen und ohne Beachtung der Zugreihenfolge zu entnehmen.

    Dieses Urnenmodell kann nicht zur Definition eines Laplace-Raums verwendet werden, da die Konsistenz zur Modellierung des Experiments als Laplace-Modell mit Beachtung der Zugreihenfolge nicht gegeben wäre.
    Dies wird im folgenden Beispiel illustriert.

    \begin{example}
        Aus einer Urne mit \(n = 4\) mit den Ziffern \(1, 2, 3, 4\) nummerierten Kugeln wird \(k = 4\) mal gezogen.
        Im Folgenden werden die Wahrscheinlichkeiten der Ereignisse
        \begin{enumerate}
            \item es wird viermal eine Eins gezogen,
            \item es werden nur verschiedene Ziffern gezogen,
        \end{enumerate}
        betrachtet, wobei die Zugreihenfolge keine Rolle spielt.
        Dazu werden die Grundräume \(\Omega_{\text{PmW}}\) und \(\Omega_{\text{KmW}}\) herangezogen.

        Zunächst wird ein Laplace-Modell \(\parentheses*{\Omega_{\text{PmW}}, P_{\text{PmW}}}\) zur Modellierung verwendet, d.h. die Zugreihenfolge wird im Modell berücksichtigt.
        Dann resultieren mit \(A = \braces*{\parentheses*{1, 1, 1, 1}}\) und
        \[
            B = \braces*{\parentheses*{1, 2, 3, 4}, \parentheses*{1, 2, 4, 3}, \parentheses*{1, 3, 2, 4}, \ldots, \parentheses*{4, 3, 2, 1}}
        \]
        die Wahrscheinlichkeiten
        \[
            P_{\text{PmW}}\parentheses*{A} = \frac{1}{4^4}, \quad P_{\text{PmW}}\parentheses*{B} = \frac{4!}{4^4}.
        \]
        Im Grundraum \(\Omega_{\text{KmW}}\) entsprechen den Ereignissen \(A\) und \(B\) die Ereignisse \(A^* = \braces*{\parentheses*{1, 1, 1, 1}}\) und \(B^* = \braces*{\parentheses*{1, 2, 3, 4}}\).
        Eine Laplace-Annahme auf der Grundmenge \(\Omega_{\text{KmW}}\) würde den Ereignissen \(A^*\) und \(B^*\) jedoch dieselbe Wahrscheinlichkeit \(\frac{1}{\binom{7}{4}}\) zuordnen, so dass diese Festlegung den Wahrscheinlichkeiten \(P_{\text{PmW}}\parentheses*{A}\) und \(P_{\text{PmW}}\parentheses*{B}\) im Modell \(\parentheses*{\Omega_{\text{PmW}}, P_{\text{PmW}}}\) widersprechen würde.
        Daher eignet sich der Raum \(\Omega_{\text{KmW}}\) nicht zur Definition eines Laplace-Raums.
    \end{example}

    \begin{example}
        Eine Urne enthält vier Kugeln, die mit \(1\), \(2\), \(3\) und \(4\) nummeriert sind.
        Drei Mal wird aus dieser Urne eine Kugel entnommen, ihre Zahl notiert und danach wieder zurückgelegt.
        Da die Reihenfolge der gezogenen Zahlen keine Rolle spielen soll, werden sie aufsteigend geordnet.
        Es werden also aufsteigend geordnete \(3\)-Tupel mit den Zahlen \(1\), \(2\), \(3\) und \(4\) notiert, wobei jede Zahl auch mehrmals auftreten kann.
        Gesucht ist die Anzahl der \(\parentheses*{4, 3}\)-Kombinationen mit Wiederholung.

        Dazu wird zunächst die Menge \(\Omega_{\text{KmW}}\) aller \(\parentheses*{4, 3}\)-Kombinationen mit Wiederholung explizit angegeben:
        \begin{align*}
            \Omega_{\text{KmW}} = \{&\parentheses*{1, 1, 1}, \parentheses*{1, 1, 2}, \parentheses*{1, 1, 3}, \parentheses*{1, 1, 4}, \parentheses*{1, 2, 2},\\
            &\parentheses*{1, 2, 3}, \parentheses*{1, 2, 4}, \parentheses*{1, 3, 3}, \parentheses*{1, 3, 4}, \parentheses*{1, 4, 4},\\
            &\parentheses*{2, 2, 2}, \parentheses*{2, 2, 3}, \parentheses*{2, 2, 4}, \parentheses*{2, 3, 3}, \parentheses*{2, 3, 4},\\
            &\parentheses*{2, 4, 4}, \parentheses*{3, 3, 3}, \parentheses*{3, 3, 4}, \parentheses*{3, 4, 4}, \parentheses*{4, 4, 4}\}.
        \end{align*}
        Durch Abzählen erhält man, dass es \(20\) verschiedene \(\parentheses*{4, 3}\)-Kombinationen mit Wiederholung gibt.
        Durch Anwendung der allgemeinen Formel berechnet sich die Anzahl der \(\parentheses*{4, 3}\)-Kombinationen mit Wiederholung gemäß
        \[
            \absolute*{\Omega_{\text{KmW}}} = \binom{4 + 3 - 1}{3} = \binom{6}{3} = \frac{6!}{3! \cdot \parentheses*{6 - 3}!} = \frac{720}{36} = 20.
        \]
    \end{example}

    \begin{remark}
        Alternativ zu den Urnenmodellen können zur Veranschaulichung der Grundmodelle der Kombinatorik auch ``Murmelmodelle'' verwendet werden.
        Anstelle von \(n\) Kugeln und \(k\) Ziehungen werden dabei \(n\) Zellen betrachtet, auf die \(k\) Murmeln verteilt werden.

        Wie in den Urnenmodellen werden vier Situationen unterschieden:
        \begin{enumerate}[label=(\roman*)]
            \item Belegungsmodus
            \begin{enumerate}[label=\alph*)]
                \item Es ist möglich, eine Zelle mit mehreren Murmeln zu belegen.
                \item Jede Zelle darf nur mit einer Murmel belegt werden.
            \end{enumerate}
            \item Notation der Murmeln
            \begin{enumerate}[label=\alph*)]
                \item Die Murmeln sind nummeriert und unterscheidbar.
                \item Die Murmeln sind nicht unterscheidbar.
            \end{enumerate}
        \end{enumerate}
    \end{remark}


    \section*{Diskrete Wahrscheinlichkeitsverteilungen}

    Eine Vielzahl diskreter Wahrscheinlichkeitsverteilungen wird in Anwendungen eingesetzt.
    Zur Festlegung der Verteilung wird jedem Element einer Menge von Trägerpunkten \(T = \braces*{x_1, x_2, \ldots}\) eine Wahrscheinlichkeit \(p_k = P\parentheses*{\braces*{x_k}} \in \left(0, 1\right], k = 1, 2, \ldots\) zugeordnet, wobei die Zahlen \(p_k\) die Summationsbedingung \(\sum_k p_k = 1\) erfüllen müssen.
    Für \(x \in \Omega \setminus T\) gilt stets \(p\parentheses*{x} = 0\).

    \begin{definition}
        Die \emph{Einpunktverteilung} \(\delta_x\) in einem Punkt \(x \in \R\) ist definiert durch die Zähldichte
        \[
            p\parentheses*{x} = 1,
        \]
        d.h. die Einpunktverteilung (oder Dirac-Verteilung oder Punktmaß) hat den Träger \(T = \braces*{x}\).
    \end{definition}

    \begin{definition}
        Die \emph{diskrete Gleichverteilung} auf den Punkten \(x_1 < \cdots < x_n\) ist definiert durch die Zähldichte
        \[
            p\parentheses*{x_k} = p_k = \frac{1}{n}, \quad k \in \braces*{1, \ldots, n},\text{ für ein }n \in \N,
        \]
        d.h. jedem der \(n\) Trägerpunkte wird dieselbe Wahrscheinlichkeit zugeordnet.
    \end{definition}

    \begin{definition}
        Die \emph{hypergeometrische Verteilung} \(\hyp\parentheses*{n, r, s}\) ist definiert durch die Zähldichte
        \[
            p_k = \frac{\binom{r}{k}\binom{s}{n - k}}{\binom{r + s}{n}}, \quad n - s \le k \le \min\parentheses*{r, n}, k \in \N_0,
        \]
        für \(n, r, s \in \N\) mit \(n \le r + s\).
    \end{definition}

    In einem Urnenmodell wird \(p_k\) erzeugt als die Wahrscheinlichkeit, bei \(n\)-maligen Ziehen ohne Zurücklegen aus einer Urne mit insgesamt \(r\) roten und \(s\) schwarzen Kugeln genau \(k\) rote und \(n - k\) schwarze Kugeln zu erhalten.

    Anwendung findet die hypergeometrische Verteilung z.B. bei der sogenannten Gut-Schlecht-Prüfung im Rahmen der Qualitätskontrolle durch ein Warenstichprobe.
    Einer Lieferung von \(r + s\) Teilen, die \(r\) defekte und \(s\) intakte Teile enthält, wird eine Stichprobe vom Umfang \(n\) (ohne Zurücklegen) entnommen; \(p_k\) ist dann die Wahrscheinlichkeit, dass genau \(k\) defekte Teile in der Stichprobe enthalten sind.
    Mit einer analogen Argumentation ist die Wahrscheinlichkeit für ``\(4\) Richtige'' beim Zahlenlotto ``\(6\) aus \(49\)'' durch \(p_4 = \frac{\binom{6}{4}\binom{43}{2}}{\binom{49}{6}}\) gegeben.
\end{document}
