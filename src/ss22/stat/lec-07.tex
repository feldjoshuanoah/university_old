\documentclass{lecture}

\institute{Institut für Statistik und Wirtschaftsmathematik}
\title{Vorlesung 7}
\author{Joshua Feld, 406718}
\course{Statistik}
\professor{Cramer}
\semester{Sommersemester 2022}
\program{CES (Bachelor)}

\begin{document}
    \maketitle


    \section*{Wahrscheinlichkeitsmaße mit Riemann-Dichten}

    Diskrete Wahrscheinlichkeitsmaße werden über Zähldichten eingeführt, die Elementen einer (höchstens) abzählbaren Menge Wahrscheinlichkeiten zuweisen.
    In der Praxis benötigt man insbesondere Modelle, in denen reelle Zahlen zur Beschreibung eines Versuchsergebnisses verwendet werden (etwa die reellen Zahlen im Intervall \(\brackets*{0, 1}\) zur Beschreibung einer prozentualen Steigerung).
    Aus mathematischer Sicht besteht der qualitative Unterschied darin, dass statt abzählbarer Grundräume nun Mengen mit überabzählbar vielen Elementen betrachtet werden.
    Hier muss eine andere Vorgehensweise zur Definition einer Wahrscheinlichkeitsverteilung gewählt werden.
    Nicht mehr jedem Elementarereignis, sondern jedem Intervall wird eine Wahrscheinlichkeit zugewiesen.

    Ist der Grundraum \(\Omega\) eine überabzählbare Teilmenge der reellen Zahlen, dann muss aus mathematischen Gründen auch die Potenzmenge als ``Vorrat'' von Ereignissen durch ein anderes Mengensystem ersetzt werden; die geeignete Struktur ist die sogenannte \(\sigma\)-Algebra.

    \begin{definition}
        Seien \(\Omega \ne \emptyset\) und \(\mathfrak{U} \subseteq \Pot\parentheses*{\Omega}\) ein System von Teilmengen von \(\Omega\).
        \(\mathfrak{U}\) heißt \emph{\(\sigma\)-Algebra} von Ereignissen über \(\Omega\), falls gilt:
        \begin{enumerate}
            \item \(\Omega \in \mathfrak{U}\),
            \item \(A \in \mathfrak{U} \implies A^c \in \mathfrak{U}\) für jedes \(A \in \mathfrak{U}\),
            \item für jede Folge \(A_1, A_2, \ldots\) von Mengen aus \(\mathfrak{U}\) gilt: \(\bigcup_{n = 1}^\infty A_n \in \mathfrak{U}\).
        \end{enumerate}
    \end{definition}

    Eine \(\sigma\)-Algebra ist ein System von Teilmengen von \(\Omega\), das abgeschlossen ist gegenüber der Bildung von Komplementen und abzählbaren Vereinigungen.
    Als Elementarereignis bezeichnet man in diesem Zusammenhang eine Menge aus \(\mathfrak{U}\), die keine echte Vereinigung anderer Ereignisse (Mengen aus \(\mathfrak{U}\)) ist.
    Aus der Definition einer \(\sigma\)-Algebra folgt sofort, dass \(\emptyset \in \mathfrak{U}\) und für jede Folge \(A_1, A_2, \ldots\) von Mengen aus \(\mathfrak{U}\) gilt: \(\bigcap_{n = 1}^\infty A_n \in \mathfrak{U}\).
    Die Potenzmenge \(\Pot\parentheses*{\Omega}\) einer nicht-leeren Menge \(\Omega\) ist stets eine \(\sigma\)-Algebra.
    Basierend auf einer \(\sigma\)-Algebra wird der Begriff eines allgemeinen Wahrscheinlichkeitsraums eingeführt.

    \begin{definition}
        Sei \(\mathfrak{U}\) eine \(\sigma\)-Algebra über \(\Omega \ne \emptyset\).
        Eine Abbildung \(P: \mathfrak{U} \to \brackets*{0, 1}\) mit
        \begin{enumerate}
            \item \(P\parentheses*{A} \ge 0 \quad \forall A \in \mathfrak{U}\),
            \item \(P\parentheses*{\Omega} = 1\) und
            \item \(P\parentheses*{\bigcup_{n = 1}^\infty A_n} = \sum_{n = 1}^\infty P\parentheses*{A_n}\) für jede Wahl paarweise disjunkter Mengen aus \(\mathfrak{U}\) (\(\sigma\)-Additivität)
        \end{enumerate}
        heißt \emph{Wahrscheinlichkeitsverteilung} oder \emph{Wahrscheinlichkeitsmaß} auf \(\Omega\) bzw. \(\parentheses*{\Omega, \mathfrak{U}}\).
        \(\parentheses*{\Omega, \mathfrak{U}, P}\) heißt \emph{Wahrscheinlichkeitsraum}, \(\parentheses*{\Omega, \mathfrak{U}}\) heißt \emph{messbarer Raum} oder \emph{Messraum}.
    \end{definition}

    Aus den Eigenschaften (ii) und (iii) in Definition 2 folgt unmittelbar \(P\parentheses*{\emptyset} = 0\).
    Dazu wähle man in (iii) die Folge paarweiser disjunkter Mengen gemäß \(A_1 = \emptyset\), \(A_2 = \Omega\) und \(A_j = \emptyset, j \ge 3\).
    Dann gilt
    \[
        P\parentheses*{\Omega} = P\parentheses*{\bigcup_{n = 1}^\infty A_n} = \sum_{n = 1}^\infty P\parentheses*{A_n} = P\parentheses*{A_1} + P\parentheses*{\bigcup_{n = 2}^\infty A_n} = P\parentheses*{A_1} + P\parentheses*{\Omega}.
    \]
    Daraus folgt direkt \(P\parentheses*{A_1} = P\parentheses*{\emptyset} = 0\).

    \begin{definition}
        Werden Intervalle \(\brackets*{a, b}\) oder \(\parentheses*{a, b}\) (auch \(\left[0, \infty\right)\), \(\R\)) als Grundraum \(\Omega \subseteq \R\) in einem Modell angesetzt, so wählt man jeweils die kleinstmögliche \(\sigma\)-Algebra, die \emph{alle} Teilmengen \(\left(c, d\right] \subseteq \brackets*{a, b}\) enthält.
        Diese \(\sigma\)-Algebra bezeichnet man als \emph{Borelsche \(\sigma\)-Algebra} \(\mathcal{B}\) über \(\brackets*{a, b}\) bzw. \(\parentheses*{a, b}\); sie ist eine echte Teilmenge der Potenzmenge von \(\brackets*{a, b}\) bzw. \(\parentheses*{a, b}\).
        Analog geht man in höheren Dimensionen vor.

        Für \(\Omega = \R^n\) wählt man als Menge von Ereignissen die Borelsche \(\sigma\)-Algebra \(\mathcal{B}^n\), die als kleinstmögliche \(\sigma\)-Algebra definiert ist mit der Eigenschaft, alle \(n\)-dimensionalen Intervalle
        \[
            \left(a, b\right] = \braces*{x = \parentheses*{x_1, \ldots, x_n} \in \R^n : a_i < x_i \le b_i, 1 \le i \le n}
        \]
        für \(a = \parentheses*{a_1, \ldots, a_n} \in \R^n\) und \(b = \parentheses*{b_1, \ldots, b_n} \in \R^n\) zu enthalten.

        Betrachtet man einen Grundraum \(\tilde{\Omega} \subseteq \R^n\), so ist die geeignete Borelsche \(\sigma\)-Algebra gegeben durch \(\tilde{\mathcal{B}}^n = \braces*{B \cap \tilde{\Omega} : B \in \mathcal{B}^n}\) (die sogenannte \emph{Spur-\(\sigma\)-Algebra}).
    \end{definition}

    Zunächst wird nun der Fall \(n = 1\) betrachtet, d.h. es wird ein Grundraum \(\Omega \subseteq \R\) zugrunde gelegt.
    Insbesondere sei angenommen, das \(\Omega\) ein Intervall ist.

    \begin{definition}
        Eine integrierbare Funktion \(f: \R \to \R\) mit \(f\parentheses*{x} \ge 0, x \in \R\) und \(\int_{-\infty}^\infty f\parentheses*{x}\d x = 1\) heißt \emph{Riemann-Dichte} oder \emph{Riemann-Dichtefunktion}.
        Über die Festlegung von Wahrscheinlichkeiten mittels
        \[
            F\parentheses*{x} = P\parentheses*{\left(-\infty, x\right]} = \int_{-\infty}^x f\parentheses*{y}\d y, \quad x \in \R,
        \]
        wird stets eindeutig ein Wahrscheinlichkeitsmaß definiert, die Funktion \(F: \R \to \brackets*{0, 1}\) wird als \emph{Verteilungsfunktion} bezeichnet.
    \end{definition}

    Die Wahrscheinlichkeit für ein Intervall \(\left(a, b\right] \subseteq \Omega\) ist dann gegeben durch
    \[
        P\parentheses*{\left(a, b\right]} = \int_a^b f\parentheses*{x}\d x.
    \]
    Mit dieser Setzung ist klar, dass einem einzelnen Punkt (im Gegensatz zu diskreten Wahrscheinlichkeitsmaßen) stets die Wahrscheinlichkeit \(0\) zugewiesen wird; \(P\parentheses*{\braces*{x}} = 0\) für alle \(x \in \Omega\).
    Das hat auch zur Konsequenz, dass alle Intervalle mit Grenzen \(a, b \in \Omega, a < b\), dieselbe Wahrscheinlichkeit haben:
    \[
        P\parentheses*{\parentheses*{a, b}} = P\parentheses*{\brackets*{a, b}} = P\parentheses*{\left(a, b\right]} = P\parentheses*{\left[a, b\right)}.
    \]
    Im Folgenden werden wichtige Wahrscheinlichkeitsverteilungen mit Riemann-Dichten vorgestellt, die abkürzend auch als stetige Wahrscheinlichkeitsverteilungen bezeichnet werden.

    \begin{remark}
        Im weiteren Verlauf kann stets davon ausgegangen werden, das \(\Omega = \R\) gilt.
        Ist \(\Omega \subsetneq \R\), so kann das Modell auf \(\R\) erweitert werden, indem die Riemann-Dichte für \(x \in \R \setminus \Omega\) zu Null definiert wird.
    \end{remark}

    \begin{definition}
        Die \emph{Rechteckverteilung} \(R\parentheses*{a, b}\) ist definiert durch die Dichtefunktion
        \[
            f\parentheses*{x} = \frac{1}{b - a}\mathbb{I}_{\brackets*{a, b}}\parentheses*{x} = \begin{cases}
                \frac{1}{b - a}, & \text{falls }x \in \brackets*{a, b},\\
                0, & \text{falls }x \not\in \brackets*{a, b}
            \end{cases}
        \]
        für Parameter \(a, b \in \R\) mit \(a < b\).
        Die Verteilungsfunktion ist gegeben durch
        \[
            F\parentheses*{x} = \begin{cases}
                0, & \text{falls }x < a,\\
                \frac{x - a}{b - a}, & \text{falls }x \in \brackets*{a, b},\\
                1, & \text{falls }x > b.
            \end{cases}
        \]
    \end{definition}

    Die Rechteckverteilung \(R\parentheses*{a, b}\) besitzt die Eigenschaft, dass die Wahrscheinlichkeit eines in \(\brackets*{a, b}\) enthaltenen Intervalls nur von der Länge des Intervalls nicht aber von dessen Lage abhängt.

    \begin{definition}
        Die \emph{Exponentialverteilung} \(\Exp\parentheses*{\lambda}\) ist definiert durch die Dichtefunktion
        \[
            f\parentheses*{x} = \lambda e^{-\lambda x}\mathbb{I}_{\parentheses*{0, \infty}}\parentheses*{x} = \begin{cases}
                \lambda e^{-\lambda x}, & \text{falls }x > 0,\\
                0, & \text{falls }y \le 0
            \end{cases}
        \]
        für einen Parameter \(\lambda > 0\).
        Die Verteilungsfunktion ist gegeben durch
        \[
            F\parentheses*{x} = \begin{cases}
                1 - e^{-\lambda x}, & \text{falls }x > 0,\\
                0, & \text{falls }x \le 0.
            \end{cases}
        \]
    \end{definition}

    Die Exponentialverteilung wird vielfältig und sehr häufig in stochastischen Modellen verwendet (etwa in der Beschreibung von Wartezeiten oder Lebensdauern).
    Gelegentlich wird eine Parametrisierung \(\Exp\parentheses*{\theta^{-1}}\) mit \(\theta > 0\) gewählt, so dass z.B.
    \[
        F\parentheses*{x} = \begin{cases}
            1 - e^{-\frac{x}{\theta}}, & \text{falls }x > 0,\\
            0, & \text{falls }x \le 0
        \end{cases}
    \]
    gilt.
    Die beiden nächsten Verteilungen sind Verallgemeinerungen der Exponentialverteilung und können aufgrund eines weiteren Parameters reale Gegebenheiten oft besser modellieren.
    Sie werden häufig in technischen Anwendungen verwendet.

    \begin{definition}
        Die \emph{Weibull-Verteilung} \(\Wei\parentheses*{\alpha, \beta}\) ist definiert durch die Dichtefunktion
        \[
            f\parentheses*{x} = \begin{cases}
                \alpha\beta x^{\beta - 1}e^{-\alpha x^\beta}, & \text{falls }x > 0,\\
                0, & \text{falls }x \le 0
            \end{cases}
        \]
        für Parameter \(\alpha > 0\) und \(\beta > 0\).
        Die Verteilungsfunktion ist gegeben durch
        \[
            F\parentheses*{x} = \begin{cases}
                1 - e^{-\alpha x^\beta}, & \text{falls }x > 0,\\
                0, & \text{falls }x \le 0.
            \end{cases}
        \]
    \end{definition}

    \begin{definition}
        Die \emph{Gammaverteilung} \(\Gamma\parentheses*{\alpha, \beta}\) ist definiert durch die Dichtefunktion
        \[
            f\parentheses*{x} = \begin{cases}
                \frac{\alpha^\beta}{\Gamma\parentheses*{\beta}}x^{\beta - 1}e^{-\alpha x}, & \text{falls }x > 0,\\
                0, & \text{falls }x \le 0
            \end{cases}
        \]
        für Parameter \(\alpha > 0\) und \(\beta > 0\).
        Dabei bezeichnet \(\Gamma\parentheses*{\cdot}\) die durch \(\Gamma\parentheses*{z} = \int_0^\infty t^{z - 1}e^{-t}\d t, z > 0\) definierte Gammafunktion.
        Eine geschlossene Darstellung der Verteilungsfunktion existiert nur für \(\beta \in \N\); die Verteilung wird dann auch als \emph{Erlang-Verteilung} \(\Erl\parentheses*{\beta}\) bezeichnet.
        Für \(\beta \in \N\) ist die Verteilungsfunktion gegeben durch
        \[
            F\parentheses*{x} = \begin{cases}
                1 - e^{-\alpha x}\parentheses*{\sum_{j = 0}^{\beta - 1}\frac{\parentheses*{\alpha x}^j}{j!}}, & \text{falls }x > 0,\\
                0, & \text{falls }x \le 0.
            \end{cases}
        \]
        Für \(\alpha = \lambda\) und \(\beta = 1\) ergibt sich die Exponentialverteilung.
    \end{definition}

    Die nächste Verteilung hat als eine spezielle Gammaverteilung eine besondere Bedeutung in der schließenden Statistik.

    \begin{definition}
        Die \emph{\(\chi^2\)-Verteilung} \(\chi^2\parentheses*{n}\) mit \(n\) Freiheitsgraden ist definiert durch die Dichtefunktion
        \[
            f\parentheses*{x} = \begin{cases}
                \frac{1}{2^{\frac{n}{2}}\Gamma\parentheses*{\frac{n}{2}}}x^{\frac{n}{2} - 1}e^{-\frac{x}{2}}, & \text{falls }x > 0,\\
                0, & \text{falls }x \le 0
            \end{cases}
        \]
        mit \(n \in \N\).
        Sie stimmt mit der \(\Gamma\parentheses*{\frac{1}{2}, \frac{n}{2}}\)-Verteilung überein.
    \end{definition}

    Neben der Rechteckverteilung ist eine zweite Verteilung über einem endlichen Intervall von Bedeutung.

    \begin{definition}
        Die \emph{Betaverteilung} \(\betadist\parentheses*{\alpha, \beta}\) ist definiert durch die Dichtefunktion
        \[
            f\parentheses*{x} = \begin{cases}
                \frac{\Gamma\parentheses*{\alpha + \beta}}{\Gamma\parentheses*{\alpha}\Gamma\parentheses*{\beta}}x^{\alpha - 1}\parentheses*{1 - x}^{\beta - 1}, & \text{falls }x \in \parentheses*{0, 1},\\
                0, & \text{falls }x \not\in \parentheses*{0, 1}
            \end{cases}
        \]
        für Parameter \(\alpha > 0\) und \(\beta > 0\).
        Die Verteilungsfunktion ist im Allgemeinen nicht geschlossen darstellbar.
    \end{definition}

    Die speziellen Betaverteilungen \(\betadist\parentheses*{\alpha, 1}\) heißen auch Potenzverteilungen und besitzen die Verteilungsfunktion
    \[
        F\parentheses*{x} = \begin{cases}
            0, & \text{falls }x < 0,\\
            x^\alpha, & \text{falls }0 \le x < 1,\\
            1, & \text{falls }x \ge 1.
        \end{cases}
    \]
    Die Rechteckverteilung \(R\parentheses*{0, 1}\) ist ein Spezialfall mit \(\alpha = 1\).

    In ökonomischen Anwendungen finden Pareto-Verteilungen Verwendung.

    \begin{definition}
        Die \emph{Pareto-Verteilung} \(\Par\parentheses*{\alpha}\) ist definiert durch die Dichtefunktion
        \[
            f\parentheses*{x} = \begin{cases}
                \frac{\alpha}{x^{\alpha + 1}}, & \text{falls }x \ge 1,\\
                0, & \text{falls }x < 1
            \end{cases}
        \]
        für einen Parameter \(\alpha > 0\).
        Die Verteilungsfunktion ist gegeben durch
        \[
            F\parentheses*{x} = \begin{cases}
                1 - x^{-\alpha}, & \text{falls }x \ge 1,\\
                0, & \text{falls }x < 1
            \end{cases}
        \]
    \end{definition}

    Die in der Stochastik wichtigste Verteilung ist die Normalverteilung.

    \begin{definition}
        Die \emph{Normalverteilung} \(N\parentheses*{\mu, \sigma^2}\) ist definiert durch die Dichtefunktion
        \[
            f\parentheses*{x} = \varphi_{\mu, \sigma^2}\parentheses*{x} = \frac{1}{\sqrt{2\pi}\sigma}\exp\parentheses*{-\frac{\parentheses*{x - \mu}^2}{2\sigma^2}}, \quad x \in \R,
        \]
        mit den Parametern \(\mu \in \R\) und \(\sigma > 0\).
        Die Verteilungsfunktion ist nicht geschlossen darstellbar.
        Speziell für \(\mu = 0\) und \(\sigma^2 = 1\) heißt \(N\parentheses*{0, 1}\) \emph{Standardnormalverteilung}; die zugehörige Verteilungsfunktion wird mit \(\Phi\), die Riemann-Dichte mit \(\varphi = \varphi_{0, 1}\) bezeichnet.
        Die Verteilungsfunktion \(\Phi_{\mu, \sigma^2}\) von \(N\parentheses*{\mu, \sigma^2}\) lässt sich durch die Identität
        \[
            \Phi_{\mu, \sigma^2}\parentheses*{x} = \Phi\parentheses*{\frac{x - \mu}{\sigma}}, \quad x \in \R
        \]
        darstellen.
        Da \(\varphi\) achsensymmetrisch ist, d.h. \(\varphi\parentheses*{x} = \varphi\parentheses*{-x}, x \in \R\), gilt die Identität
        \[
            \Phi\parentheses*{x} = 1 - \Phi\parentheses*{-x}, \quad x \in \R.
        \]
    \end{definition}

    Schließlich werden noch zwei weitere Verteilungen mit Riemann-Dichten eingeführt, die in der schließenden Statistik verwendet werden.

    \begin{definition}
        Die \emph{\(t\)-Verteilung} \(t\parentheses*{n}\) mit \(n\) Freiheitsgraden ist definiert durch die Dichtefunktion
        \[
            f\parentheses*{x} = \frac{\Gamma\parentheses*{\frac{n + 1}{2}}}{\sqrt{n\pi}\Gamma\parentheses*{\frac{n}{2}}}\parentheses*{1 + \frac{x^2}{n}}^{-\frac{n + 1}{2}}, \quad x \in \R,
        \]
        mit \(n \in \N\).
        Die Verteilungsfunktion ist nicht geschlossen darstellbar.
    \end{definition}

    \begin{definition}
        Die \emph{\(F\)-Verteilung} \(F\parentheses*{n, m}\) mit \(n\) und \(m\) Freiheitsgraden ist definiert durch die Dichtefunktion
        \[
            f\parentheses*{x} = \begin{cases}
                \frac{\Gamma\parentheses*{\frac{n + m}{2}}}{\Gamma\parentheses*{\frac{n}{2}}\Gamma\parentheses*{\frac{m}{2}}}\parentheses*{\frac{n}{m}}^{\frac{n}{2}}\frac{x^{\frac{n}{2} - 1}}{\parentheses*{1 + \frac{n}{m}x}^{\frac{n + m}{2}}}, & \text{falls }x > 0,\\
                0, & \text{falls }x \le 0.
            \end{cases}
        \]
        mit \(n \in \N\) und \(m \in \N\). Die Verteilungsfunktion ist nicht geschlossen darstellbar.
    \end{definition}
\end{document}
