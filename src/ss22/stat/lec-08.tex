\documentclass{lecture}

\institute{Institut für Statistik und Wirtschaftsmathematik}
\title{Vorlesung 8}
\author{Joshua Feld, 406718}
\course{Statistik}
\professor{Cramer}
\semester{Sommersemester 2022}
\program{CES (Bachelor)}

\begin{document}
    \maketitle


    \begin{definition}
        Sei \(f\) eine Riemann-Dichtefunktion.
        Die Menge \(\braces*{x \in \R : f\parentheses*{x} > 0}\) heißt \emph{Träger} der zugehörigen Wahrscheinlichkeitsverteilung \(P\) (oder von \(f\) bzw. der zugehörigen Verteilungsfunktion \(F\)).
        Sie wird auch mit \(\supp\parentheses*{P}\), \(\supp\parentheses*{f}\) oder \(\supp\parentheses*{F}\) bezeichnet.
        Ist der Träger ein Intervall, so spricht man von einem \emph{Trägerintervall}.
    \end{definition}

    Grundsätzlich können Verteilungen durch eine Lage-/Skalentransformation modifiziert und möglicherweise um neue Parameter ergänzt werden, um bessere Modellanpassungen zu erreichen.
    Dazu betrachtet man zu einer Verteilungsfunktion \(F\) die Verteilungsfunktion
    \[
        F_{a, b}\parentheses*{x} = F\parentheses*{\frac{x - a}{b}}
    \]
    mit Parametern \(a \in \R\) und \(b > 0\).
    \(a\) wird als Lageparameter, \(b\) als Skalenparameter bezeichnet.
    Ist \(\parentheses*{\alpha, \omega}\) das Trägerintervall von \(F\), so ist \(\parentheses*{a + b\alpha, a + b\omega}\) das Trägerintervall von \(F_{a, b}\).
    Für die zugehörigen Riemann-Dichten gilt:
    \[
        f_{a, b}\parentheses*{x} = \frac{1}{b}f\parentheses*{\frac{x - a}{b}}, \quad x \in \R.
    \]
    Die resultierende Familie von Verteilungen
    \[
        \mathcal{P} = \braces*{P_{a, b} : P_{a, b}\text{ hat die Riemann-Dichte }f_{a, b}, a \in \R, b > 0}
    \]
    heißt Lokations-Skalenfamilie mit Standardvertreter \(f = f_{0, 1}\) (bzw. \(P_{0, 1}\)).

    \begin{example}
        \begin{enumerate}
            \item Die obige Transformation führt bei einer Betaverteilung \(\betadist\parentheses*{\alpha, \beta}\) zum Trägerintervall \(\parentheses*{a, a + b}\), so dass beliebige endliche Intervalle als Träger gewählt werden können.
            Die Dichtefunktion \(f_{a, b}\) entsteht mit \(f\) durch
            \begin{align*}
                f_{a, b}\parentheses*{x} &= \frac{1}{b}f\parentheses*{\frac{x - a}{b}}\\
                &= \begin{cases}
                    \frac{\Gamma\parentheses*{\alpha + \beta}}{b^{\alpha + \beta - 1}\Gamma\parentheses*{\alpha}\Gamma\parentheses*{\beta}}\parentheses*{x - a}^{\alpha - 1}\parentheses*{a + b - x}^{\beta - 1}, & \text{falls }x \in \parentheses*{a, a + b},\\
                    0, & \text{falls }x \not\in \parentheses*{a, a + b}.
                \end{cases}
            \end{align*}
            \item Bei einer Standardexponentialverteilung \(\Exp\parentheses*{1}\) resultiert mit Lokationsparameter \(\mu\) und Skalenparameter \(\theta\) das Trägerintervall \(\parentheses*{\mu, \infty}\) und die Dichtefunktion \(f_{\mu, \theta}\) gegeben durch
            \[
                f_{\mu, \theta}\parentheses*{x} = \frac{1}{\theta}f\parentheses*{\frac{x - \mu}{\theta}} = \begin{cases}
                    \frac{1}{\theta}e^{-\frac{x - \mu}{\theta}}, & \text{falls }x > \mu,\\
                    0, & \text{falls }x \le \mu.
                \end{cases}
            \]
            Die durch diese Dichte festgelegte Verteilung heißt zweiparametrige Exponentialverteilung mit Lageparameter \(\mu\) und Skalenparameter \(\theta\).
            \item Eine Lage-/Skalentransformation führt nicht immer zu einer Verteilungsfamilie mit (zwei) neuen Parametern.
            Wird beispielsweise eine Exponentialverteilung \(\Exp\parentheses*{\lambda}\) mit einem Parameter \(\lambda > 0\) und der Dichtefunktion \(f\) mit \(f\parentheses*{x} = \lambda e^{-\lambda x}\mathbb{I}_{\parentheses*{0, \infty}}\parentheses*{x}, x \in \R\), als Ausgangspunkt gewählt, so ist die transformierte Dichtefunktion \(f_{a, b}\) gegeben durch
            \[
                f_{a, b}\parentheses*{x} = \frac{1}{b}f\parentheses*{\frac{x - a}{b}} = \frac{\lambda}{b}\exp\parentheses*{-\frac{\lambda}{b}\parentheses*{x - a}}\mathbb{I}_{\parentheses*{a, \infty}}\parentheses*{x}, \quad x \in \R.
            \]
            Mit der Setzung \(\mu = a\) und \(\theta = \frac{b}{\lambda} (> 0)\) resultiert eine zweiparametrige Exponentialverteilung mit Lageparameter \(\mu\) und Skalenparameter \(\theta\).
            Der Skalenparameter \(\lambda\) bildet zusammen mit dem eingeführten Skalenparameter \(b\) den neuen Skalenparameter \(\theta = \frac{b}{\lambda}\).
            \item Die Familie der Normalverteilungen
            \[
                \mathcal{P} = \braces*{N\parentheses*{\mu, \sigma^2} : \mu \in \R, \sigma^2 > 0}
            \]
            bildet eine Lokations-Skalenfamilie mit dem Lokationsparameter \(\mu \in \R\) und dem Skalenparameter \(\sigma > 0\) sowie dem Standardvertreter \(\varphi_{0, 1} = \varphi\) (bzw. \(N\parentheses*{0, 1}\)).
        \end{enumerate}
    \end{example}


    \section*{Eigenschaften von Wahrscheinlichkeitsmaßen}

    Aus der Definition eines Wahrscheinlichkeitsmaßes folgen wichtige Eigenschaften für das Rechnen mit Wahrscheinlichkeiten.
    Diese Regeln basieren nur auf den Kolmogorov-Axiomen und gelten damit in allgemeinen Wahrscheinlichkeitsräumen, insbesondere also für diskrete und stetige Wahrscheinlichkeitsverteilungen, die über Zähldichten bzw. Riemann-Dichten eingeführt werden.

    \begin{lemma}
        Sei \(\parentheses*{\Omega, \mathfrak{A}, P}\) ein Wahrscheinlichkeitsraum.
        Für \(A, B \in \mathfrak{A}\) gilt:
        \begin{enumerate}
            \item \(P\parentheses*{A \cup B} = P\parentheses*{A} + P\parentheses*{B}\), falls \(A \cap B = \emptyset\),
            \item \(P\parentheses*{A \setminus B} = P\parentheses*{A} - P\parentheses*{B}\), falls \(A \subseteq B\) (Subtraktivität von \(P\)),
            \item \(P\parentheses*{A^c} = 1 - P\parentheses*{A}\),
            \item \(A \subseteq B \implies P\parentheses*{A} \le P\parentheses*{B}\) (Monotonie von \(P\)),
            \item \(P\parentheses*{A \cup B} = P\parentheses*{A} + P\parentheses*{B} - P\parentheses*{A \cap B}\).
        \end{enumerate}
    \end{lemma}

    \begin{proof}
        \begin{enumerate}
            \item Folgt aus der \(\sigma\)-Additivität mit der Setzung \(A_1 = A\), \(A_2 = B\) und \(A_j = \emptyset, j \ge 3\).
            \item Für \(A \subseteq B\) ist \(A \cap B = A\).
            Weiter ist \(B \setminus A = B \cap A^c\).
            Aus der Disjunktheitvon \(A = A \cap B\) und \(B \cap A^c\) folgt:
            \begin{align*}
                P\parentheses*{B} &= P\parentheses*{B \cap \Omega}\\
                &= P\parentheses*{B \cap \parentheses*{A \cup A^c}}\\
                &= P\parentheses*{\parentheses*{B \cap A} \cup \parentheses*{B \cap A^c}}\\
                &= P\parentheses*{A} + P\parentheses*{B \cap A^c}\\
                &= P\parentheses*{A} + P\parentheses*{B \setminus A}.
            \end{align*}
            \item Ergibt sich aus (ii) mit \(B = \Omega\).
            \item Mit \(A \subseteq B\) gilt wegen (ii): \(P\parentheses*{A} = P\parentheses*{B} - P\parentheses*{B \setminus A} \le P\parentheses*{B}\).
            \item Wegen (i) gilt (vgl. auch Beweis von (ii)):
            \[
                P\parentheses*{A^c \cap B} + P\parentheses*{A \cap B} = P\parentheses*{B}\text{ und }P\parentheses*{B^c \cap A} + P\parentheses*{A \cap B} = P\parentheses*{A}.
            \]
            Da \(A \cup B = \parentheses*{A \cap B^c} \cup \parentheses*{A \cap B} \cup \parentheses*{A^c \cap B}\) Vereinigung paarweise disjunkter Mengen ist, folgt
            \begin{align*}
                P\parentheses*{A \cup B} &= P\parentheses*{A \cap B^c} + P\parentheses*{A \cap B} + P\parentheses*{A^c \cap B}\\
                &= P\parentheses*{A} - P\parentheses*{A \cap B} + P\parentheses*{A \cap B} + P\parentheses*{B} - P\parentheses*{A \cap B}\\
                &= P\parentheses*{A} + P\parentheses*{B} - P\parentheses*{A \cap B}.
            \end{align*}
        \end{enumerate}
    \end{proof}
\end{document}
