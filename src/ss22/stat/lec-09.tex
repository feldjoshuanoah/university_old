\documentclass{lecture}

\institute{Institut für Statistik und Wirtschaftsmathematik}
\title{Vorlesung 9}
\author{Joshua Feld, 406718}
\course{Statistik}
\professor{Cramer}
\semester{Sommersemester 2022}
\program{CES (Bachelor)}

\begin{document}
    \maketitle


    In stochastischen Modellen benötigt man häufig Folgen von Ereignissen und deren Eigenschaften.

    \begin{definition}
        Seien \(\mathfrak{U}\) eine \(\sigma\)-Algebra über \(\Omega \ne \emptyset\) und \(\parentheses*{A_n}_{n \in N} \subseteq \mathfrak{U}\).
        \[
            \parentheses*{A_n}_n\text{ heißt}\begin{cases}
                \text{\emph{isoton} (monoton wachsend)}, & \text{falls }A_n \subseteq A_{n + 1} \forall n \in \N,\\
                \text{\emph{antiton} (monoton fallend)}, & \text{falls }A_n \supseteq A_{n + 1} \forall n \in \N.
            \end{cases}
        \]
        Für isotone bzw. antitone Ereignisfolgen heißt jeweils
        \[
            \lim_{n \to \infty}A_n = \bigcup_{n = 1}^\infty A_n\text{ bzw. }\lim_{n \to \infty}A_n = \bigcap_{n = 1}^\infty A_n
        \]
        der \emph{Grenzwert} (Limes) von \(\parentheses*{A_n}_{n \in \N}\).

        Für eine beliebige Ereignisfolge \(\parentheses*{A_n}_n\) heißen
        \[
            \limsup_{n \to \infty}A_n = \lim_{n \to \infty}\parentheses*{\bigcup_{k = n}^\infty A_k} = \bigcap_{n = 1}^\infty\bigcup_{k = n}^\infty A_k
        \]
        der \emph{limes superior} und
        \[
            \liminf_{n \to \infty}A_n = \lim_{n \to \infty}\parentheses*{\bigcap_{k = n}^\infty A_k} = \bigcup_{n = 1}^\infty\bigcap_{k = n}^\infty A_k
        \]
        der \emph{limes inferior} der Folge \(\parentheses*{A_n}_{n \in \N}\).
    \end{definition}

    \begin{remark}
        Mit \(\parentheses*{A_n}_n \subseteq \mathfrak{U}\) sind der limes superior und der limes inferior selbst wieder Ereignisse aus \(\mathfrak{U}\), d.h.
        \[
            \limsup_{n \to \infty}A_n, \liminf_{n \to \infty}A_n \in \mathfrak{U}
        \]
        und es gilt
        \begin{align*}
            \limsup_{n \to \infty}A_n &= \braces*{\omega \in \Omega : \omega\text{ liegt in unendlich vielen der }A_i},\\
            \liminf_{n \to \infty}A_n &= \braces*{\omega \in \Omega : \omega\text{ liegt in allen }A_i\text{ bis auf endlich viele}}.
        \end{align*}
        Der limes superior beschreibt also das Ereignis, dass unendlich viele der \(A_i\) entreten, der limes inferior das Ereignis, dass alle bis auf endlich viele der \(A_i\) eintreten.
    \end{remark}

    Für Ereignisfolgen erhält man folgende Eigenschaften.

    \begin{lemma}
        Seien \(\parentheses*{\Omega, \mathfrak{U}, P}\) ein Wahrscheinlichkeitsraum und \(\parentheses*{A_n}_n \subseteq \mathfrak{U}\).
        Dann gilt:
        \begin{enumerate}
            \item \(P\parentheses*{\bigcup_{n = 1}^\infty A_n} = P\parentheses*{\lim_{n \to \infty}A_n} = \lim_{n \to \infty}P\parentheses*{A_n}\), falls die Ereignisfolge \(\parentheses*{A_n}_n\) isoton ist (Stetigkeit von \(P\) von unten),
            \item \(P\parentheses*{\bigcap_{n = 1}^\infty A_n} = P\parentheses*{\lim_{n \to \infty}A_n} = \lim_{n \to \infty}P\parentheses*{A_n}\), falls \(\parentheses*{A_n}_n\) antiton ist (Stetigkeit von \(P\) von oben),
            \item \(P\parentheses*{\limsup_{n \to \infty}A_n} = \lim_{n \to \infty}P\parentheses*{\bigcup_{k = n}^\infty A_k}\), \(P\parentheses*{\liminf_{n \to \infty}A_n} = \lim_{n \to \infty}P\parentheses*{\bigcap_{k = n}^\infty A_k}\).
        \end{enumerate}
    \end{lemma}

    \begin{proof}
        \begin{enumerate}
            \item Seien \(B_1 = A_1\), \(B_{n + 1} = A_{n + 1} \cap A_n^c\) und \(A_n \subseteq A_{n + 1}, n \in \N\).
            Damit sind die Mengen \(B_1, B_2, \ldots\) paarweise disjunkt und es gilt
            \[
                \bigcup_{n = 1}^\infty B_n = B_1 \cup \bigcup_{n = 2}^\infty\underbrace{\parentheses*{A_n \cap A_{n - 1}^c}}_{\subseteq A_n} \subseteq \bigcup_{n = 1}^\infty A_n.
            \]
            Weiterhin folgt
            \[
                \omega \in \bigcup_{n = 1}^\infty A_n \implies \exists i: \omega \in A_i \land \omega \not\in A_j \forall j < i \implies \omega \in B_i \implies \omega \in \bigcup_{n = 1}^\infty A_n.
            \]
            Damit gilt unter Anwendung der \(\sigma\)-Additivität von \(P\):
            \begin{align*}
                P\parentheses*{\bigcup_{n = 1}^\infty A_n} &= P\parentheses*{\bigcup_{n = 1}^\infty B_n} = \sum_{n = 1}^\infty P\parentheses*{B_n}\\
                &\stackrel{A_1 = B_1}{=} P\parentheses*{A_1} + \sum_{n = 2}^\infty P\parentheses*{B_n}\\
                &= P\parentheses*{A_1} + \lim_{m \to \infty}\sum_{n = 1}^m\underbrace{P\parentheses*{B_{n + 1}}}_{= P\parentheses*{A_{n + 1}} - P\parentheses*{A_n}}\\
                &= P\parentheses*{A_1} + \lim_{m \to \infty}\parentheses*{P\parentheses*{A_{m + 1}} - P\parentheses*{A_1}} = \lim_{m \to \infty}P\parentheses*{A_m}.
            \end{align*}
            \item Mit der de Morganschen Regel folgt aus (i): \(\bigcap_{n = 1}^\infty A_n = \parentheses*{\bigcup_{n = 1}^\infty A_n^c}^c\), wobei \(\parentheses*{A_n^c}_n\) eine isotone Mengenfolge ist.
            Damit gilt:
            \[
                P\parentheses*{\bigcap_{n = 1}^\infty A_n} = 1 - P\parentheses*{\bigcup_{n = 1}^\infty A_n^c} = 1 - \lim_{m \to \infty}\underbrace{P\parentheses*{A_m^c}}_{= 1 - P\parentheses*{A_m}} = \lim_{m \to \infty}P\parentheses*{A_m}.
            \]
            \item Die Anwendung von (i) und (ii) liefert
            \[
                P\parentheses*{\limsup_{n \to \infty}A_n} = P\parentheses*{\lim_{n \to \infty}\bigcup_{k = n}^\infty A_k} = \lim_{n \to \infty}P\parentheses*{\bigcup_{k = n}^\infty A_k},
            \]
            denn \(\parentheses*{\bigcup_{k = n}^\infty A_k}_n\) ist eine antitone Folge und
            \[
                P\parentheses*{\liminf_{n \to \infty}A_n} = P\parentheses*{\lim_{n \to \infty}\bigcap_{k = n}^\infty A_k} = \lim_{n \to \infty}P\parentheses*{\bigcap_{k = n}^\infty A_k},
            \]
            denn \(\parentheses*{\bigcap_{k = n}^\infty A_k}_n\) ist eine isotone Folge.
        \end{enumerate}
    \end{proof}

    Die Wahrscheinlichkeit für ein Vereinigungsereignis ist die Summe der Einzelwahrscheinlichkeiten, falls die Ereignisse paarweise disjunkt sind.
    Ist dies nicht der Fall, lässt sich die Wahrscheinlichkeit auf die Wahrscheinlichkeiten aller Schnittereignisse zurückführen.

    \begin{lemma}
        Für Ereignisse \(\parentheses*{A_n}_{n \in \N}\) in einem Wahrscheinlichkeitsraum \(\parentheses*{\Omega, \mathfrak{U}, P}\) gilt:
        \begin{align*}
            P\parentheses*{\bigcup_{k = 1}^n A_k} = &\sum_{k = 1}^n P\parentheses*{A_k} - \sum_{1 \le i_1 < i_2 \le n} P\parentheses*{A_{i_1} \cap A_{i_2}}\\
            &+ \sum_{1 \le i_1 < i_2 < i_3 \le n}P\parentheses*{A_{i_1} \cap A_{i_2} \cap A_{i_3}}\\
            &\pm \cdots + \parentheses*{-1}^{n + 1}P\parentheses*{\bigcap_{k = 1}^n A_k}.
        \end{align*}
    \end{lemma}
    In den Fällen \(n = 2\) und \(n = 3\) gelten somit speziell:
    \begin{align*}
        P\parentheses*{A_1 \cup A_2} =\ &P\parentheses*{A_1} + P\parentheses*{A_2} - P\parentheses*{A_1 \cap A_2},\\
        P\parentheses*{A_1 \cup A_2 \cup A_3} =\ &P\parentheses*{A_1} + P\parentheses*{A_2} - P\parentheses*{A_1 \cap A_2}\\
        &- P\parentheses*{A_1 \cap A_2} - P\parentheses*{A_1 \cap A_3} - P\parentheses*{A_2 \cap A_3}\\
        &+ P\parentheses*{A_1 \cap A_2 \cap A_3}.
    \end{align*}

    \begin{remark}
        Aus der Sylvester-Poincaré-Siebformel werden sogenannte Bonferroni-Ungleichungen gewonnen.
        Seien \(A_1, \ldots, A_n\) Ereignisse aus \(\mathfrak{U}\) im Wahrscheinlichkeitsraum \(\parentheses*{\Omega, \mathfrak{U}, P}\).
        Dann gilt
        \[
            \sum_{k = 1}^n P\parentheses*{A_k} - \sum_{1 \le i_1 < i_2 \le n}P\parentheses*{A_{i_1} \cap A_{i_2}} \le P\parentheses*{\bigcup_{k = 1}^n A_k} \le \sum_{k = 1}^n P\parentheses*{A_k}.
        \]
        Weitere Ungleichungen entstehen durch Abbruch der Siebformel nach Termen gerader bzw. ungerader Ordnung.
    \end{remark}


    \section*{Bedingte Wahrscheinlichkeiten}

    Das Konzept der bedingten Wahrscheinlichkeit dient der Beschreibung des Einflusses von Vor- oder Zusatzinformationen bzw. deren Einbeziehung in ein stochastisches Modell.

    Für \(A \in \mathfrak{U}\) ist \(P\parentheses*{A}\) die Wahrscheinlichkeit des Eintretens von Ereignis \(A\).
    Nun sei bekannt oder gefordert, dass Ereignis \(B \in \mathfrak{U}\) eintritt.
    Welchen Einfluss hat diese Information auf die Wahrscheinlichkeit des Eintretens von \(A\)?

    \begin{center}
        \begin{tabular}{cl}
            \toprule
            Mathematisches Objekt & Interpretation\\
            \midrule
            \(\Omega\) & Grundraum, Ergebnisraum\\
            \(\omega\) & (mögliches) Ergebnis\\
            \(A \in \mathfrak{U}\) & Ereignis\\
            \(\mathfrak{U}\) & Menge der (möglichen) Ereignisse\\
            \(\Omega\) & sicheres Ereignis\\
            \(\emptyset\) & unmögliches Ereignis\\
            \(\omega \in A\) & Ereignis \(A\) tritt ein\\
            \(\omega \in A^c\) & Ereignis \(A\) tritt nicht ein\\
            \(\omega \in A \cup B\) & Ereignis \(A\) oder Ereignis \(B\) tritt ein\\
            \(\omega \in A \cap B\) & Ereignis \(A\) und Ereignis \(B\) treten ein\\
            \(A \subseteq B\) & Eintreten von Ereignis \(A\) impliziert Eintreten von Ereignis \(B\)\\
            \(A \cap B = \emptyset\) & Ereignisse \(A\) und \(B\) schließen einander aus, \(A\) und \(B\) sind disjunkt\\
            \(\omega \in \bigcup_{i \in I}A_i\) & mindestens ein Ereignis \(A_i, i \in I\) tritt ein\\
            \(\omega \in \bigcap_{i \in I}A_i\) & alle Ereignisse \(A_i, i \in I\) treten ein\\
            \(\omega \in \limsup_{i \in I}A_i\) & unendlich viele Ereignisse \(A_i, i \in I\) treten ein\\
            \(\omega \in \liminf_{i \in I}A_i\) & alle bis auf endlich viele Ereignisse \(A_i, i \in I\) treten ein\\
            \bottomrule
        \end{tabular}
    \end{center}

    \begin{example}
        \begin{enumerate}
            \item Würfelwurf: Wie groß ist die Wahrscheinlichkeit für das Auftreten der Zwei unter der Bedingung, dass eine gerade Zahl auftritt?
            Die intuitive Antwort ist \(\frac{1}{3}\).
            In Gedanken schränkt man den Grundraum auf die Menge \(\braces*{2, 4, 6}\) ein.
            \item Urnenmodell: Aus einer Urne mit zwei weißen und drei schwarzen Kugeln werden zwei Kugeln ohne Zurücklegen gezogen.
            Dann erscheint klar:
            \[
                P\binom{\text{Die zweite Kugel ist schwarz bedingt unter der Kenntnis,}}{\text{dass die erste Kugel weiß ist}} = \frac{3}{4}.
            \]
            Die Ziehung der Kugeln wird als Laplace-Experiment mit dem Grundraum
            \[
                \Omega = \braces*{\parentheses*{i, j} : i, j \in \braces*{1, \ldots, 5}, i \ne j}
            \]
            mit \(\absolute*{\Omega} = 5 \cdot 4\) modelliert.
            Die Interpretation von \(\Omega\) ist gegeben durch: Die weißen Kugeln haben die Nummern \(1\) und \(2\), die schwarzen Kugeln die Nummern \(3\), \(4\) und \(5\).
            Mit der Definition
            \begin{quote}
                Ereignis \(A\): ``Zweite Kugel ist schwarz'' und

                Ereignis \(B\): ``Erste Kugel ist weiß''
            \end{quote}
            gilt dann
            \begin{align*}
                A \cap B &= \braces*{\parentheses*{1, 3}, \parentheses*{1, 4}, \parentheses*{1, 5}, \parentheses*{2, 3}, \parentheses*{2, 4}, \parentheses*{2, 5}}, \absolute*{A \cap B} = 6, P\parentheses*{A \cap B} = \frac{3}{10},\\
                B &= \braces*{\parentheses*{1, 2}, \parentheses*{1, 2}, \parentheses*{1, 2}, \parentheses*{1, 2}, \parentheses*{1, 2}, \parentheses*{1, 2}, \parentheses*{1, 2}, \parentheses*{1, 2}}, \absolute*{B} = 8, P\parentheses*{B} = \frac{2}{5}.
            \end{align*}
            Damit ist
            \[
                \frac{P\parentheses*{A \cap B}}{P\parentheses*{B}} = \frac{\frac{3}{10}}{\frac{2}{5}} = \frac{3}{4}.
            \]
            Dieser Quotient wird allgemein als bedingte Wahrscheinlichkeit definiert.
        \end{enumerate}
    \end{example}

    \begin{definition}
        Sei \(\parentheses*{\Omega, \mathfrak{U}, P}\) ein Wahrscheinlichkeitsraum.
        Für jedes \(B \in \mathfrak{U}\) mit \(P\parentheses*{B} > 0\) wird durch
        \[
            P\parentheses*{A \mid B} = \frac{P\parentheses*{A \cap B}}{P\parentheses*{B}}, \quad A \in \mathfrak{U},
        \]
        eine Wahrscheinlichkeitsverteilung \(P\parentheses*{\cdot \mid B}\) auf \(\mathfrak{U}\) definiert, die sogenannte \emph{bedingte Verteilung} unter (der Hypothese) \(B\).
        \(P\parentheses*{A \mid B}\) heißt \emph{(elementar) bedingte Wahrscheinlichkeit} von \(A\) unter \(B\). 
    \end{definition}

    Als Funktion von \(A\) bildet \(P\parentheses*{A \mid B}\) wiederum eine Wahrscheinlichkeitsverteilung und \(\parentheses*{\Omega, \mathfrak{U}, P\parentheses*{\cdot \mid B}}\) ist ein Wahrscheinlichkeitsraum.
    Wegen \(P\parentheses*{A \mid B} = P\parentheses*{A \cap B \mid B}\) ist auch \(\parentheses*{B, \braces*{A \cap B : A \in \mathfrak{U}}, P\parentheses*{\cdot \mid B}}\) ein Wahrscheinlichkeitsraum über dem (eingeschränkten) Grundraum \(B \subseteq \Omega\).
    Die Menge \(\braces*{A \cap B : A \in \mathfrak{U}}\) ist eine \(\sigma\)-Algebra über \(B\).

    Für das Arbeiten mit bedingten Wahrscheinlichkeiten sind die nachfolgenden Eigenschaften und Begriffe wesentlich.

    \begin{lemma}
        Seien \(\parentheses*{\Omega, \mathfrak{U}, P}\) ein Wahrscheinlichkeitsraum und \(A, B \in \mathfrak{U}\) mit \(P\parentheses*{A} > 0, P\parentheses*{B} > 0\) sowie \(A_1, \ldots, A_n \in \mathfrak{U}\) mit \(P\parentheses*{\bigcap_{i = 1}^{n - 1}A_i} > 0\).
        Dann gilt:
        \begin{enumerate}
            \item \(P\parentheses*{A \mid B} = P\parentheses*{B \mid A} \cdot \frac{P\parentheses*{A}}{P\parentheses*{B}}\),
            \item \(P\parentheses*{\bigcap_{i = 1}^n A_i} = P\parentheses*{A_1} \cdot P\parentheses*{A_2 \mid A_1} \cdot P\parentheses*{A_3 \mid A_1 \cap A_2} \cdot \ldots \cdot P\parentheses*{A_n \mid \bigcap_{i = 1}^{n - 1}A_i}\).
        \end{enumerate}
    \end{lemma}

    \begin{proof}
        \begin{enumerate}
            \item \(P\parentheses*{A \mid B}\frac{P\parentheses*{A}}{P\parentheses*{B}} = \frac{P\parentheses*{B \cap A}}{P\parentheses*{A}}\frac{P\parentheses*{A}}{P\parentheses*{B}} = P\parentheses*{A \mid B}\).
            \item
            \begin{align*}
                &P\parentheses*{A_1} \cdot P\parentheses*{A_2 \mid A_1} \cdot P\parentheses*{A_3 \mid A_1 \cap A_2} \cdot \ldots \cdot P\parentheses*{A_n \mid \bigcap_{i = 1}^{n - 1}A_i}\\
                &= P\parentheses*{A_1} \cdot \frac{P\parentheses*{A_1 \cap A_2}}{P\parentheses*{A_1}} \cdot \frac{P\parentheses*{A_1 \cap A_2 \cap A_3}}{P\parentheses*{A_1 \cap A_2}} \cdot \ldots \cdot \frac{P\parentheses*{\bigcap_{i = 1}^n A_i}}{P\parentheses*{\bigcap_{i = 1}^{n - 1} A_i}} = P\parentheses*{\bigcap_{i = 1}^n A_i}.
            \end{align*}
        \end{enumerate}
    \end{proof}

    \begin{lemma}
        Seien \(\parentheses*{\Omega, \mathfrak{U}, P}\) ein Wahrscheinlichkeitsraum und \(A \in \mathfrak{U}, \parentheses*{B_n}_n \subseteq \mathfrak{U}\), \(B_n\) paarweise disjunkt mit \(A \subseteq \bigcup_{n = 1}^\infty B_n\).
        Dann gilt:
        \[
            P\parentheses*{A} = \sum_{n = 1}^\infty P\parentheses*{A \cap B_n} = \sum_{n = 1}^\infty P\parentheses*{A \mid B_n} \cdot P\parentheses*{B_n}.
        \]
        Ist \(P\parentheses*{B_n} = 0\), so ist \(P\parentheses*{A \mid B_n}\) nicht definiert.
        Man setzt in diesem Fall \(P\parentheses*{B_n} \cdot P\parentheses*{A \mid B_n} = 0\) mit \(P\parentheses*{A \mid B_n} \in \brackets*{0, 1}\) beliebig.
    \end{lemma}

    \begin{proof}
        Wegen \(A \subseteq \bigcup_{n = 1}^\infty B_n\) ist \(A = \bigcup_{n = 1}^\infty \underbrace{\parentheses*{A \cap B_n}}_{\text{paarweise disjunkt}}\).
        Damit folgt
        \[
            P\parentheses*{A} = P\parentheses*{\bigcup_{n = 1}^\infty\parentheses*{A \cap B_n}} = \sum_{n = 1}^\infty P\parentheses*{A \cap B_n} = \sum_{n = 1}^\infty P\parentheses*{A \mid B_n} \cdot P\parentheses*{B_n}.
        \]
    \end{proof}

    \begin{lemma}
        Seien \(\parentheses*{\Omega, \mathfrak{U}, P}\) ein Wahrscheinlichkeitsraum und \(A \in \mathfrak{U}, \parentheses*{B_n}_n \subseteq \mathfrak{U}, B_1, B_2, \ldots\) paarweise disjunkt, \(A \subseteq \cup_{n = 1}^\infty B_n\) mit \(P\parentheses*{A} > 0\).
        Dann gilt für jedes \(k \in \N\)
        \[
            P\parentheses*{B_k \mid A} = \frac{P\parentheses*{B_k} \cdot P\parentheses*{A \mid B_k}}{\sum_{n = 1}^\infty P\parentheses*{A \mid B_n} \cdot P\parentheses*{B_n}}.
        \]
    \end{lemma}

    \begin{proof}
        Wegen \(P\parentheses*{B_k \mid A} = \frac{P\parentheses*{B_k} \cdot P\parentheses*{A \mid B_k}}{P\parentheses*{A}}\) folgt nach der Formel von der totalen Wahrscheinlichkeit die Behauptung.
    \end{proof}

    Bei der Bayesschen Formel wird von der ``Wirkung'' \(A\) auf eine ``Ursache'' \(B_k\) geschlossen.
    In einem medizinischen Kontext kann obige Formel wie folgt interpretiert werden: Ein Arzt stellt ein Symptom \(A\) fest, das von verschiedenen Krankheiten \(B_1, \ldots, B_n\) herrühren kann.
    \begin{enumerate}[label=(\roman*)]
        \item Die relative Häufigkeit einer jeden Krankheit sei bekannt.
        Diese ist eine ``Schätzung'' für \(P\parentheses*{B_k}, k = 1, 2, \ldots\).
        \item Wenn die Krankheit \(B_k\) vorliegt, dann seien die relativen Häufigkeiten für das Auftreten von Symptom \(A\) bekannt.
        Diese liefern Werte für \(P\parentheses*{A \mid B_k}, k = 1, 2, \ldots\).
    \end{enumerate}
    Gesucht ist die Wahrscheinlichkeit für die Krankheit Bk , wenn Symptom \(A\) auftritt.
    \(P\parentheses*{B_k}\) heißt auch a priori Wahrscheinlichkeit, \(P\parentheses*{B_k \mid A}\) a posteriori Wahrscheinlichkeit.


    \section*{Stochastische Unabhängigkeit von Ereignissen}

    Die stochastische Unabhängigkeit ist ein zentraler Begriff in der Stochastik.
    Zunächst wird die stochastische Unabhängigkeit von zwei oder mehr Ereignissen behandelt.
    Intuitiv würden Ereignisse \(A\) und \(B\) als unabhängig betrachtet werden, wenn die Wahrscheinlichkeit von \(A\) nicht davon abhängt, ob \(B\) eingetreten ist oder nicht, d.h. die bedingte Wahrscheinlichkeit \(P\parentheses*{A \mid B}\) ist unabhängig von \(B\): \(P\parentheses*{A \mid B} = P\parentheses*{A}\).
    Dabei wird jedoch \(P\parentheses*{B} > 0\) vorausgesetzt, da sonst die bedingte Wahrscheinlichkeit nicht definiert ist.
    Unter Berücksichtigung der Definition von \(P\parentheses*{A \mid B}\) folgt äquivalent die Gleichung
    \[
        P\parentheses*{A \mid B} = P\parentheses*{A} \iff \frac{P\parentheses*{A \cap B}}{P\parentheses*{B}} = P\parentheses*{A} \iff P\parentheses*{A \cap B} = P\parentheses*{A} \cdot P\parentheses*{B}.
    \]
    Diese intuitive Vorgehensweise impliziert also, dass die Wahrscheinlichkeit des Schnittereignisses \(A \cap B\) als Produkt der Wahrscheinlichkeiten der Ereignisse \(A\) und \(B\) bestimmt werden kann.
    Da diese Beziehung auch für den Fall \(P\parentheses*{B} = 0\) verwendet werden kann, wird die stochastische Unabhängigkeit in dieser Weise definiert.

    \begin{definition}
        Seien \(\parentheses*{\Omega, \mathfrak{U}, P}\) ein Wahrscheinlichkeitsraum und \(A, B \in \mathfrak{U}\).
        Die Ereignisse \(A\) und \(B\) heißen \emph{stochastisch unabhängig}, falls gilt:
        \[
            P\parentheses*{A \cap B} = P\parentheses*{A} \cdot P\parentheses*{B}.
        \]
    \end{definition}

    \begin{example}
        Seien \(\Omega = \braces*{1, 2, 3, 4}\) und \(P\) die Laplace-Verteilung auf \(\Omega\).
        Die Ereignisse \(A = \braces*{1, 2}\) und \(B = \braces*{1, 3}\) sind stochastisch unabhängig, da \(P\parentheses*{A} = P\parentheses*{B} = \frac{1}{2}\) und
        \[
            P\parentheses*{A \cap B} = P\parentheses*{\braces*{1}} = \frac{1}{4} = \frac{1}{2} \cdot \frac{1}{2} = P\parentheses*{A} \cdot P\parentheses*{B}.
        \]
    \end{example}

    \begin{lemma}
        Für Ereignisse \(A\) und \(B\) in einem Wahrscheinlichkeitsraum \(\parentheses*{\Omega, \mathfrak{U}, P}\) gelten folgende Eigenschaften:
        \begin{enumerate}
            \item Sind \(A\) und \(B\) stochastisch unabhängig, dann sind auch die Ereignisse
            \[
                A\text{ und }B^c, \quad A^c\text{ und }B \quad \text{sowie }A^c\text{ und }B^c
            \]
            jeweils stochastisch unabhängig.
            \item Ist \(P\parentheses*{B} > 0\), so gilt:
            \[
                A\text{ und }B\text{ sind stochastisch unabhängig} \iff P\parentheses*{A \mid B} = P\parentheses*{A}.
            \]
            \item Ist \(P\parentheses*{A} \in \braces*{0, 1}\), so gilt für alle Ereignisse \(B \in \mathfrak{U}\): \(A\) und \(B\) sind stochastisch unabhängig.
        \end{enumerate}
    \end{lemma}

    Als Erweiterung des obigen Unabhängigkeitsbegriffs wird die stochastische Unabhängigkeit einer Familie von Ereignissen definiert.

    \begin{definition}
        Seien \(I\) eine beliebige Indexmenge und \(A_i, i \in I\) Ereignisse in einem Wahrscheinlichkeitsraum \(\parentheses*{\Omega, \mathfrak{U}, P}\).
        Dann heißen diese Ereignisse
        \begin{enumerate}
            \item \emph{paarweise stochastisch unabhängig}, falls
            \[
                P\parentheses*{A_i \cap A_j} = P\parentheses*{A_i} \cdot P\parentheses*{A_j} \quad \forall i, j \in I, i \ne j.
            \]
            \item \emph{(gemeinsam) stochastisch unabhängig}, falls für jede endliche Auswahl von Indizes \(\braces*{i_1, \ldots, i_s} \subseteq I\) gilt:
            \[
                P\parentheses*{A_{i_1} \cap \cdots \cap A_{i_s}} = P\parentheses*{A_{i_1} \cdot \ldots \cdot P\parentheses*{A_{i_s}}}.
            \]
        \end{enumerate}
    \end{definition}

    Für \(n = 3\) lassen sich die Forderungen der paarweisen bzw. gemeinsamen stochastischen Unabhängigkeit folgendermaßen formulieren:
    \begin{enumerate}[label=(\roman*)]
        \item Die Ereignisse \(A_1, A_2, A_3\) sind paarweise stochastisch unabhängig, wenn gilt:
        \[
            P\parentheses*{A_1, \cap A_2} = P\parentheses*{A_1}P\parentheses*{A_2}, \quad P\parentheses*{A_1, \cap A_3} = P\parentheses*{A_1}P\parentheses*{A_3}, \quad P\parentheses*{A_2, \cap A_3} = P\parentheses*{A_2}P\parentheses*{A_3}.
        \]
        \item \(A_1, A_2, A_3\) sind gemeinsam stochastisch unabhängig, wenn gilt:
        \[
            P\parentheses*{A_1, \cap A_2} = P\parentheses*{A_1}P\parentheses*{A_2}, \quad P\parentheses*{A_1, \cap A_3} = P\parentheses*{A_1}P\parentheses*{A_3}, \quad P\parentheses*{A_2, \cap A_3} = P\parentheses*{A_2}P\parentheses*{A_3},
        \]
        \[
            P\parentheses*{A_1 \cap A_2 \cap A_3} = P\parentheses*{A_1}P\parentheses*{A_2}P\parentheses*{A_3}.
        \]
    \end{enumerate}
    Im Vergleich zur paarweise stochastischen Unabhängigkeit kommt bei der gemeinsamen stochastischen Unabhängigkeit im zweiten Fall (\(n = 3\)) eine zusätzliche Forderung hinzu.
    Aus der gemeinsamen stochastischen Unabhängigkeit von Ereignissen folgt deren paarweise stochastische Unabhängigkeit.
    Die Umkehrung ist aber im Allgemeinen nicht richtig.
\end{document}
