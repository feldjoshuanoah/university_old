\documentclass{lecture}

\institute{Institut für Statistik und Wirtschaftsmathematik}
\title{Vorlesung 10}
\author{Joshua Feld, 406718}
\course{Statistik}
\professor{Cramer}
\semester{Sommersemester 2022}
\program{CES (Bachelor)}

\begin{document}
    \maketitle


    \begin{example}
        Seien \(\Omega = \braces*{1, 2, 3, 4}\) und \(P\) die Laplace-Verteilung auf \(\Omega\).
        Die Ereignisse \(A = \braces*{1, 2}, B = \braces*{1, 3}, C = \braces*{2, 3}\) sind wegen
        \[
            P\parentheses*{A} = P\parentheses*{B} = P\parentheses*{C} = \frac{1}{2}, \quad P\parentheses*{A \cap B} = P\parentheses*{A \cap C} = P\parentheses*{B \cap C} = \frac{1}{4}
        \]
        paarweise stochastisch unabhängig.
        Wegen \(A \cap B \cap C = \emptyset\) gilt jedoch
        \[
            P\parentheses*{A \cap B \cap C} = 0 \ne \frac{1}{8} = P\parentheses*{A} \cdot P\parentheses*{B} \cdot P\parentheses* {C},
        \]
        d.h. \(A, B, C\) sind nicht gemeinsam stochastisch unabhängig.
    \end{example}

    \begin{lemma}
        Seien \(\parentheses*{\Omega, \mathfrak{U}, P}\) ein Wahrscheinlichkeitsraum, \(I\) eine Indexmenge und \(A_i \in \mathfrak{U}, i \in I\).
        \begin{enumerate}
            \item Jede Teilmenge stochastisch unabhängiger Ereignisse ist eine Menge stochastisch unabhängiger Ereignisse.
            \item Sind die Ereignisse \(A_i, i \in I\) stochastisch unabhängig und \(B_i \in \braces*{A_i, A_i^c, \emptyset, \Omega}, i \in I\), so sind auch die Ereignisse \(B_i, i \in I\) stochastisch unabhängig.
            \item Sind die Ereignisse \(A_1, \ldots, A_n\) stochastisch unabhängig, dann gilt:
            \[
                P\parentheses*{\bigcup_{i = 1}^n A_i} = 1 - P\parentheses*{\bigcap_{i = 1}^n A_i^c} = 1 - \prod_{i = 1}^n\parentheses*{1 - P\parentheses*{A_i}}.
            \]
        \end{enumerate}
    \end{lemma}

    Die Modellierung der ``unabhängigen Versuchswiederholung'' wird am folgenden Beispiel verdeutlicht.

    \begin{example}
        Ein Experiment liefere mit Wahrscheinlichkeit \(p \in \brackets*{0, 1}\) das Ergebnis \(1\) und mit Wahrscheinlichkeit \(1 - p\) das Ergebnis \(0\) (z.B. ein Münzwurf).
        Dieses Experiment werde \(n\)-mal ``unabhängig'' ausgeführt.
        Das geeignete stochastische Modell ist durch den Grundraum
        \[
            \Omega = \braces*{\omega = \parentheses*{\omega_1, \ldots, \omega_n} : \omega_i \in \braces*{0, 1}, 1 \le i \le n}
        \]
        und die Zähldichte \(f: \Omega \to \brackets*{0, 1}\) definiert durch
        \[
            f\parentheses*{\omega_1, \ldots, \omega_n} = \prod_{j = 1}^n p^{\omega_j}\parentheses*{1 - p}^{1 - \omega_j} = p^{\sum\omega_j}\parentheses*{1 - p}^{n - \sum\omega_j} = p^k\parentheses*{1 - p}^{n - k},
        \]
        falls \(\omega = \parentheses*{\omega_1, \ldots, \omega_n}\) genau \(k\) Komponenten mit Wert \(1\) hat, spezifiziert (es gilt \(\sum_{\omega \in \Omega}f\parentheses*{\omega} = \sum_{k = 0}^n\binom{n}{k}p^k\parentheses*{1 - p}^{n - k} = 1\)).
        Dieses Modell heißt Bernoulli-Modell.
        Der Zusammenhang zur stochastischen Unabhängigkeit ist wie folgt gegeben.
        Sei \(A_i\) das Ereignis für das Ergebnis \(1\) im \(i\)-ten Versuch, d.h.
        \[
            A_i = \braces*{\omega \in \Omega : \omega_i = 1}.
        \]
        Dann kann gezeigt werden: \(P\parentheses*{A_i} = p\), \(P\parentheses*{A_i \cap A_j} = p^2, i \ne j\), \(P\parentheses*{A_i \cap A_j \cap A_k} = p^3, i, j, k\) alle verschieden, usw.
    \end{example}

    \begin{lemma}
        Sei \(\parentheses*{A_n}_{n \in \N}\) eine Folge von Ereignissen in einem Wahrscheinlichkeitsraum \(\parentheses*{\Omega, \mathfrak{U}, P}\).
        Dann gilt:
        \begin{enumerate}
            \item \(\sum_{n = 1}^\infty P\parentheses*{A_n} < \infty \implies P\parentheses*{\limsup_{n \to \infty}A_n} = 0\).
            \item Für stochastisch unabhängige Ereignisse \(A_n, n \in \N\) gilt:
            \[
                \sum_{n = 1}^\infty P\parentheses*{A_n} = \infty \implies P\parentheses*{\limsup_{n \to \infty}A_n} = 1.
            \]
        \end{enumerate}
    \end{lemma}

    \begin{remark}
        \begin{enumerate}[label=(\roman*)]
            \item Analog zu den Aussagen des Lemmas von Borel-Cantelli gilt:
            \begin{enumerate}
                \item \(\sum_{n = 1}^\infty P\parentheses*{A_n^c} < \infty \implies P\parentheses*{\liminf_{n \to \infty}A_n} = 1\).
                \item Für stochastisch unabhängige Ereignisse \(A_n, n \in \N\) gilt:
                \[
                    \sum_{n = 1}^\infty P\parentheses*{A_n^c} = \infty \implies P\parentheses*{\liminf_{n \to \infty}A_n} = 0.
                \]
            \end{enumerate}
            \item Für eine Folge stochastisch unabhängiger Ereignisse gilt stets:
            \[
                P\parentheses*{\limsup_{n \to \infty}A_n} \in \braces*{0, 1}\text{ und }P\parentheses*{\liminf_{n \to \infty}A_n} \in \braces*{0, 1}.
            \]
        \end{enumerate}
    \end{remark}
\end{document}
