\documentclass{lecture}

\institute{Institut für Statistik und Wirtschaftsmathematik}
\title{Vorlesung 13}
\author{Joshua Feld, 406718}
\course{Statistik}
\professor{Cramer}
\semester{Sommersemester 2022}
\program{CES (Bachelor)}

\begin{document}
    \maketitle


    \begin{example}
        \begin{enumerate}
            \item Die Zähldichte der \(\bin\parentheses*{5, \frac{1}{2}}\)-Verteilung ist bestimmt durch
            \[
                p^X\parentheses*{x} = \begin{cases}
                    \binom{5}{x}\parentheses*{\frac{1}{2}}^x \parentheses*{1 - \frac{1}{2}}^{5 - x} = \binom{5}{x}\parentheses*{\frac{1}{2}}^5, & \text{falls }x \in \braces*{0, \ldots, 5},\\
                    0, & \text{sonst}
                \end{cases}
            \]
            mit \(\supp\parentheses*{P^X} = \braces*{0, \ldots, 5}\).
            Daher ist die Verteilungsfunktion gegeben durch
            \[
                F^X\parentheses*{x} = \begin{cases}
                    0, & \text{falls }x < 0,\\
                    \frac{1}{2^5}\sum_{i = 0}^{\floor*{x}}\binom{5}{i}, & \text{falls }0 \le x \le 5,\\
                    1, & \text{falls }x > 5,
                \end{cases}
            \]
            wobei \(\floor*{x} = \max\braces*{z \in \Z : z \le x}\) die sogenannte untere Gauß-Klammer von \(x \in \R\) bezeichnet.
            \item Die Verteilungsfunktion der Exponentialverteilung mit Parameter \(\lambda > 0\) ist gegeben durch
            \[
                F\parentheses*{x} = \begin{cases}
                    0, & \text{falls }x < 0,\\
                    1 - e^{-\lambda x}, & \text{falls }x \ge 0
                \end{cases}
            \]
            und damit stetig auf \(\R\).
            Die Verteilungsfunktion ergibt sich durch Integration der Riemann-Dichte wie folgt: Da für die Riemann-Dichte \(f\parentheses*{t} = 0\) gilt, falls \(t < 0\) ist, folgt für \(x < 0\)
            \[
                F\parentheses*{x} = \int_{-\infty}^x f\parentheses*{t}\d t = \int_{-\infty}^x 0\d t = 0.
            \]
            Ist \(x \ge 0\), so liefert dies mit der Darstellung der Riemann-Dichte für positive Argumente
            \[
                F\parentheses*{x} = \int_{-\infty}^x f\parentheses*{t}\d t = \int_0^x \lambda e^{-\lambda t} = \brackets*{-e^{-\lambda t}}_0^x = 1 - e^{-\lambda x}.
            \]
            \item In statistischen Normalverteilungsmodellen werden häufig die \(\sigma\)-, \(2\sigma\)- und \(3\sigma\)-Bereiche herangezogen, womit die Intervalle
            \[
                \parentheses*{\mu - \sigma, \mu + \sigma}, \quad \parentheses*{\mu - 2\sigma, \mu + 2\sigma}, \quad \parentheses*{\mu - 3\sigma, \mu + 3\sigma}
            \]
            gemeint sind.
            Im \(3\sigma\)-Bereich liegt nahezu die gesamte ``Wahrscheinlichkeitsmasse'' einer \(N\parentheses*{\mu, \sigma^2}\)-Verteilung.
            Mit der Bezeichnung \(\Phi\) für die Verteilungsfunktion der Standardnormalverteilung \(N\parentheses*{0, 1}\) und einer \(N\parentheses*{\mu, \sigma^2}\)-verteilten Zufallsvariablen \(X\) gilt:
            \begin{align*}
                P\parentheses*{X \in \parentheses*{\mu - k\sigma, \mu + k\sigma}} &= P\parentheses*{\mu - k\sigma \le X \le \mu + k\sigma}\\
                &= P\parentheses*{-k \le \frac{X - \mu}{\sigma} \le k}\\
                &= \Phi\parentheses*{k} - \Phi\parentheses*{-k}\\
                &= 2\Phi\parentheses*{k} - 1\\
                &\approx \begin{cases}
                    0,6827, & \text{falls }k = 1,\\
                    0,9545, & \text{falls }k = 2,\\
                    0,9973, & \text{falls }k = 3.
                \end{cases}
            \end{align*}
        \end{enumerate}
    \end{example}

    Ist die Verteilungsfunktion streng monoton wachsend, so kann die Umkehrfunktion gebildet werden.
    In einer allgemeineren Definition werden auch Verteilungsfunktionen zugelassen, die stückweise konstant sind.

    \begin{definition}
        Sei \(F\) eine Verteilungsfunktion.
        Dann heißt die durch
        \[
            F^{-1}\parentheses*{y} = \inf\braces*{x \in \R : F\parentheses*{x} \ge y}, \quad y \in \parentheses*{0, 1}
        \]
        definierte Funktion \(F^{-1}: \parentheses*{0, 1} \to \R\) \emph{Quantilfunktion} oder \emph{Pseudoinverse} von \(F\).
    \end{definition}

    Eigenschaften der Quantilfunktion sind im folgenden Lemma zusammengestellt.

    \begin{lemma}
        Sei \(F^{-1}\) die Quantilfunktion der Verteilungsfunktion \(F\).
        Dann gilt:
        \begin{enumerate}
            \item \(F^{-1}\) ist monoton wachsend und linksseitig stetig.
            \item Für alle \(x \in \R\) und \(y \in \parentheses*{0, 1}\) gilt:
            \begin{enumerate}
                \item \(F\parentheses*{x} \ge y \iff x \ge F^{-1}\parentheses*{y}\),
                \item \(F\parentheses*{x-} \le y \iff x \le F^{-1}\parentheses*{y+}\),
                \item \(F\parentheses*{F^{-1}\parentheses*{y}-} \le y \le F\parentheses*{F^{-1}\parentheses*{y}}\),
                \item \(F^{-1}\parentheses*{F\parentheses*{x}} \le x \le F^{-1}\parentheses*{F\parentheses*{x}+}\).
            \end{enumerate}
            Dabei bezeichnen \(g\parentheses*{t+}\) und \(g\parentheses*{t-}\) den rechtsseitigen bzw. linksseitigen Grenzwert der Funktion \(g\) an der Stelle \(t \in \R\).
            \item Sei \(X\) eine Zufallsvariable mit stetiger Verteilungsfunktion \(F\).
            Dann ist \(F\parentheses*{X} \sim R\parentheses*{0, 1}\).
            \item Sei \(Y \sim R\parentheses*{0, 1}\).
            Dann ist \(F^{-1}\parentheses*{Y} \sim F\).
        \end{enumerate}
    \end{lemma}

    In Lemma 1, (ii), c) und d) gilt offenbar Gleichheit, wenn \(F\) im Punkt \(F^{-1}\parentheses*{y}\) bzw. wenn \(F^{-1}\) im Punkt \(F\parentheses*{x}\) stetig ist.
    Die Aussage in Lemma 1, (iv) findet u.a. in der Simulation Verwendung und zeigt auf, wie eine rechteckverteilte Zufallsvariable transformiert wird, um eine Zufallsvariable mit Verteilungsfunktion \(F\) zu erhalten.

    Größte Werte \(x \in \R\), für die \(F\parentheses*{x} \le p\) oder kleinste Werte \(x \in \R\), für die \(F\parentheses*{x} > 1 - q\) gilt, spielen in der schließenden Statistik (bei statistischen Tests) eine wichtige Rolle und werden Quantile genannt.

    \begin{definition}
        Für \(p \in \parentheses*{0, 1}\) heißt \(Q_p = F^{-1}\parentheses*{p}\) das \emph{\(p\)-Quantil} von \(F\) bzw. der zu \(F\) gehörigen Wahrscheinlichkeitsverteilung \(P\).
    \end{definition}

    \begin{theorem}
        Sei \(p \in \parentheses*{0, 1}\).
        Ist \(F\) streng monoton wachsend und stetig, so ist \(Q_p\) die eindeutige Lösung der Gleichung \(F\parentheses*{x} = p\) in \(x \in \R\).
    \end{theorem}


    \section*{Mehrdimensionale Zufallsvariablen und Verteilungen}

    Zufallsvektoren, die auch als mehrdimensionale Zufallsvariablen bezeichnet werden, werden in Definition C 1.3 als Funktionen von \(\Omega\) nach \(\R^n\) eingeführt.
    Analog zu Definition C 1.5 spricht man von der Verteilung \(P^X\) von \(X = \parentheses*{X_1, \ldots, X_n}\).
    Hierbei sind die Mengen \(A\), die in \(P^X\) eingesetzt werden können, aus einer geeigneten \(\sigma\)-Algebra zu wählen.
    In dieser Vorlesung werden zur Vereinfachung nur Zufallsvektoren betrachtet, die entweder in allen Komponenten diskret sind oder Riemann-Dichten besitzen.

    Die gemeinsame Verteilung der Zufallsvariablen \(X_1, \ldots, X_n\) ist durch die Angabe aller Wahrscheinlichkeiten \(P\parentheses*{X_1 \in A_1, \ldots, X_n \in A_n}\), für alle \(A_i \in \mathfrak{U}_i, i \in \braces*{1, \ldots, n}\) bestimmt.
    Im diskreten Fall kann man sich dabei auf einelementige Mengen beschränken; im Fall von Riemann-Dichten genügt es, für \(A_i\) alle halboffenen Intervalle im \(R_n\) zu betrachten.
    
    Wie im eindimensionalen Fall wird die Verteilungsfunktion zur Beschreibung der Wahrscheinlichkeitsverteilung \(P^X\) eingeführt.

    \begin{definition}
        Sei \(X = \parentheses*{X_1, \ldots, X_n}\) ein Zufallsvektor mit Wahrscheinlichkeitsverteilung \(P^X\).
        Die durch
        \begin{align*}
            F^X\parentheses*{x} &= P^X\parentheses*{\left(-\infty, x_1\right] \times \cdots \times \left(-\infty, x_n\right]}\\
            &= P\parentheses*{X_1 \in \left(-\infty, x_1\right], \ldots, X_n \in \left(-\infty, x_n\right]}\\
            &= P\parentheses*{X_1 \le x_1, \ldots, X_n \le x_n}, \quad \parentheses*{x_1, \ldots, x_n} \in \R^n
        \end{align*}
        definierte Funktion heißt \emph{multivariate} (oder \emph{mehrdimensionale}) \emph{Verteilungsfunktion}.
    \end{definition}

    Führt man für einen diskreten \(n\)-dimensionalen Zufallsvektor \(X = \parentheses*{X_1, \ldots, X_n}\) mit Verteilung \(P^X\) analog zum eindimensionalen Fall den Träger
    \[
        T = \supp\parentheses*{P^{X_1, \ldots, X_n}} = \braces*{\parentheses*{x_1, \ldots, x_n} \in \R^n : P\parentheses*{X_1 = x_1, \ldots, X_n = x_n} > 0}
    \]
    ein, so heißt die Funktion \(p^X: \R^n \to \brackets*{0, 1}\) mit
    \[
        p^X\parentheses*{x} = \begin{cases}
            P^X\parentheses*{\braces*{x}} = P\parentheses*{X = x} = P\parentheses*{X_1 = x_1, \ldots, X_n = x_n}, & \text{falls }x \in T,\\
            0, & \text{falls }x \not\in T
        \end{cases}
    \]
    Zähldichte von \(P^X\) (bzw. von \(X\)) mit Träger \(T\).
    Die Verteilungsfunktion \(F^X\) lässt sich schreiben als
    \[
        F^X\parentheses*{x_1, \ldots, x_n} = \sum_{\substack{\parentheses*{t_1, \ldots, t_n} \in T,\\t_i \le x_i, 1 \le i \le n}}p^X\parentheses*{t_1, \ldots, t_n}, \quad \parentheses*{x_1, \ldots, x_n} \in \R^n.
    \]
    In vielen Anwendungen sind die gemeinsame Verteilung einer Auswahl von Komponenten des Vektors \(X\) und die Verteilungen der Komponenten \(X_i\) des Vektorsvon Interesse.
    Diese heißen Rand- oder Marginalverteilungen.

    \begin{definition}
        Sei \(X = \parentheses*{X_1, \ldots, X_n}\) ein Zufallsvektor mit Verteilung \(P^X\).
        Die Verteilung von \(\parentheses*{X_{i_1}, \ldots, X_{i_m}}\) für \(m \parentheses*{< n}\) Indizes mit \(1 \le i_1 < \cdots < i_m \le n\) heißt \emph{\(m\)-dimensionale Rand-} oder \emph{Marginalverteilung} zu \(\parentheses*{i_1, \ldots, i_m}\).
        Dier Verteilung von \(X_i\) heißt \emph{\(i\)-te Rand-} oder \emph{Marginalverteilung}.
    \end{definition}

    Sei \(X = \parentheses*{X_1, \ldots, X_n}\) ein Zufallsvektor mit Verteilung \(P^X\).
    Die Randverteilung von \(\parentheses*{X_{i_1}, \ldots, X_{i_m}}\) wird bestimmt, indem in den ``nicht benötigten Komponenten'' \(\R\) als Menge eingesetzt wird:
    Sei \(B = \bigtimes_{j = 1}^m \brackets*{a_j, b_j}\) das kartesische Produkt von Intervallen \(\brackets*{a_j, b_j}, j \in \braces*{1, \ldots, m}\).
    Dann ist
    \begin{align*}
        P^{\parentheses*{X_{i_1}, \ldots, X_{i_m}}}\parentheses*{B} &= P\parentheses*{X_{i_1} \in \brackets*{a_1, b_1}, \ldots, X_{i_m} \in \brackets*{a_m, b_m}}\\
        &= P\parentheses*{X_{i_j} \in \brackets*{a_j, b_j}, j \in \braces*{1, \ldots, m}\text{, und }X_j \in \R, j \in \braces*{1, \ldots, n} \setminus \braces*{i_1, \ldots, i_m}}.
    \end{align*}
    Insbesondere ist die Verteilungsfunktion \(F^{X_i}\) der \(i\)-ten Randverteilung bestimmt durch
    \[
        F^{X_i}\parentheses*{x} = P^X(\R \times \cdots \times \R \times \underbrace{\left(-\infty, x\right]}_{i\text{-te Komponente}} \times \R \times \cdots \times \R).
    \]
    Dieser Zusammenhang kann wie folgt ausgenutzt werden.
    Ist die Verteilungsfunktion \(F^{X_1, \ldots, X_n}\) eines Zufallsvektors \(\parentheses*{X_1, \ldots, X_n}\) gegeben, so kann die Verteilungsfunktion eines Teilvektors \(\parentheses*{X_{i_1}, \ldots, X_{i_m}}\) durch Grenzwertbildung ermittelt werden.
    Es gilt der Zusammenhang
    \[
        F^{X_{i_1}, \ldots, X_{i_m}}\parentheses*{x_{i_1}, \ldots, x_{i_m}} = \lim_{x_j \to \infty, j \not\in \braces*{i_1, \ldots, i_m}}F^{X_1, \ldots, X_n}\parentheses*{x_1, \ldots, x_n}.
    \]
    Als Spezialfall werde der Fall \(n = 3\) betrachtet.
    Dann gilt etwa
    \[
        F^{X_1}\parentheses*{x_1} = \lim_{x_2, x_3 \to \infty}F^{X_1, X_2, X_3}\parentheses*{x_1, x_2, x_3}, \quad F^{X_2, X_3}\parentheses*{x_2, x_3} = \lim_{x_1 \to \infty}F^{X_1, X_2, X_3}\parentheses*{x_1, x_2, x_3}.
    \]
    Die eindimensionalen Randverteilungen legen die gemeinsame Verteilung nicht eindeutig fest, wie das folgende Beispiel zeigt.

    \begin{example}
        Seien \(X = \parentheses*{X_1, X_2}, Y = \parentheses*{Y_1, Y_2}\) Zufallsvektoren auf dem Wahrscheinlichkeitsraum \(\parentheses*{\Omega, \mathfrak{U}, P}\) mit
        \[
            X\parentheses*{\Omega} = Y\parentheses*{\Omega} = \braces*{0, 1}^2 = \braces*{\parentheses*{0, 0}, \parentheses*{1, 0}, \parentheses*{0, 1}, \parentheses*{1, 1}},
        \]
        \begin{align*}
            P^X\parentheses*{\parentheses*{0, 1}} = P^X\parentheses*{\parentheses*{1, 0}} &= \frac{1}{2}, & P^X\parentheses*{\parentheses*{0, 0}} = P^X\parentheses*{\parentheses*{1, 1}} &= 0,\\
            P^Y\parentheses*{\parentheses*{0, 1}} = P^Y\parentheses*{\parentheses*{1, 0}} &= 0, & P^Y\parentheses*{\parentheses*{0, 0}} = P^Y\parentheses*{\parentheses*{1, 1}} &= \frac{1}{2}.
        \end{align*}
        Damit ist offensichtlich \(P^X \ne P^Y\).
        Für die Randverteilungen gilt jedoch:
        \begin{align*}
            P^{X_1}\parentheses*{j} &= P^X\parentheses*{\braces*{j} \times \R}\\
            &= P\parentheses*{X_1 = j, X_2 \in \braces*{0, 1}}\\
            &= P\parentheses*{X_1 = j, X_2 = 0} + P\parentheses*{X_1 = j, X_2 = 1}\\
            &= P^X\parentheses*{\parentheses*{j, 0}} + P^X\parentheses*{\parentheses*{j, 1}} = \frac{1}{2}, \quad j \in \braces*{0, 1},
        \end{align*}
        und analog
        \[
            P^{Y_1}\parentheses*{j} = P^Y\parentheses*{\parentheses*{j, 0}} + P^Y\parentheses*{\parentheses*{j, 1}} = \frac{1}{2}, \quad j \in \braces*{0, 1}.
        \]
        Also ist \(P^{X_1} = P^{Y_1}\).
        Ebenso zeigt man \(P^{X_2} = P^{Y_2}\).
    \end{example}

    \begin{example}
        Diskrete endliche mehrdimensionale Verteilungen werden oft in Form einer sogenannten Wahrscheinlichkeitstafel oder Kontingenztafel notiert.
        Die Wahrscheinlichkeitsverteilungen \(P^{\parentheses*{X_1, X_2}}\) und \(P^{\parentheses*{Y_1, Y_2}}\) aus vorherigem Beispiel werden dann geschrieben als (\(p_{ij}^{\parentheses*{X_1, X_2}} = P\parentheses*{X_1 = i, X_2 = j}\) etc.)
        \begin{center}
            \begin{tabular}{c|cc|c}
                \(p_{ij}^{\parentheses*{X_1, X_2}}\) & \(0\) & \(1\) & \(P\parentheses*{X_1 = i}\)\\
                \midrule
                \(0\) & \(0\) & \(\frac{1}{2}\) & \(\frac{1}{2}\)\\
                \(1\) & \(\frac{1}{2}\) & \(0\) & \(\frac{1}{2}\)\\
                \midrule
                \(P\parentheses*{X_2 = j}\) & \(\frac{1}{2}\) & \(\frac{1}{2}\) & \(1\)
            \end{tabular} \quad \begin{tabular}{c|cc|c}
                \(p_{ij}^{\parentheses*{Y_1, Y_2}}\) & \(0\) & \(1\) & \(P\parentheses*{Y_1 = i}\)\\
                \midrule
                \(0\) & \(\frac{1}{2}\) & \(0\) & \(\frac{1}{2}\)\\
                \(1\) & \(0\) & \(\frac{1}{2}\) & \(\frac{1}{2}\)\\
                \midrule
                \(P\parentheses*{Y_2 = j}\) & \(\frac{1}{2}\) & \(\frac{1}{2}\) & \(1\)
            \end{tabular}
        \end{center}
        Die Zeilen- und Spaltensummen führen zu den Randverteilungen.
        Die Aussage aus dem vorherigen Beispiel ist anhand der beiden Tafeln offensichtlich.
    \end{example}

    Ein Beispiel für eine multivariate diskrete Wahrscheinlichkeitsverteilung wurde in Definition 3 der sechsten Vorlesung mit der Polynomialverteilung bereits gegeben.
    Diese wird hier in der Notation mit Zufallsvariablen wiederholt und interpretiert.

    \begin{example}
        Ein Zufallsexperiment liefere eines von \(m \ge 2\) möglichen Ergebnissen \(A_i, 1 \le i \le m\) die als Mengen beschrieben seien.
        Seien \(A_1, \ldots, A_m\) paarweise disjunkt mit \(P\parentheses*{A_j} = p_j, 1 \le j \le m\) und \(\sum_{j = 1}^m p_j = 1\).
        Nun betrachtet man die \(n\)-malige unabhängige Versuchswiederholung und beschreibt dieses Experiment über dem Grundraum \(\Omega = \braces*{1, \ldots, m}^n\).
        Interessiert man sich für die Verteilung der Ergebnisse und beschreibt mit der Zufallsvariablen \(X_j\) die Anzahl des Auftretens von Ereignis \(A_j\) bei \(n\) Versuchen, \(1 \le j \le m\), so kann man für \(k_j \in \N_0, 1 \le j \le m\) mit \(\sum_{j = 1}^m k_j = n\) zeigen:
        \begin{align*}
            P\parentheses*{X_1 = k_1, \ldots, X_m = k_m} &= P\parentheses*{\braces*{\omega \in \Omega : X_1\parentheses*{\omega} = k_1, \ldots, X_m\parentheses*{\omega} = k_m}}\\
            &= P^{\parentheses*{X_1, \ldots, X_m}}\parentheses*{\braces*{\parentheses*{k_1, \ldots, k_m}}}\\
            &= \frac{n!}{k_1! \cdot \ldots \cdot k_m!}\prod_{i = 1}^m p_i^{k_i}.
        \end{align*}
        Die Randverteilung von \(X_1\) ist gegeben durch die Zähldichte (\(k \in \braces*{0, \ldots, n}\))
        \[
            P^{X_1}\parentheses*{\braces*{k}} = P^{\parentheses*{X_1, \ldots, X_m}}\parentheses*{\braces*{k} \times \R \times \cdots \times \R} = \cdots = \binom{n}{k}p_1^k\parentheses*{1 - p_1}^{n - k},
        \]
        und damit eine \(\bin\parentheses*{n, p_1}\)-Verteilung.
        Dies gilt entsprechend für die anderen eindimensionalen Randverteilungen mit \(p_j\) an Stelle von \(p_1\).
    \end{example}

    Bei stochastisch unabhängigen Zufallsvariablen ist die gemeinsame Verteilungsfunktion eindeutig durch die Verteilungsfunktionen der eindimensionalen Randverteilungen bestimmt.

    \begin{theorem}
        \(X_1, \ldots, X_n\) sind stochastisch unabhängige Zufallsvariablen mit Verteilungsfunktionen \(F^{X_1}, \ldots, F^{X_n}\) genau dann, wenn
        \[
            F^{\parentheses*{X_1, \ldots, X_n}}\parentheses*{x_1, \ldots, x_n} = F^{X_1}\parentheses*{x_1} \cdot \ldots \cdot F^{X_n}\parentheses*{x_n} \quad \forall\parentheses*{x_1, \ldots, x_n} \in \R^n.
        \]
    \end{theorem}

    Der Begriff der Riemann-Dichte zur Beschreibung einer Wahrscheinlichkeitsverteilung wird auf den \(n\)-dimensionalen Fall übertragen.

    \begin{definition}
        Eine Riemann-integrierbare Funktion \(f: \R^n \to \R\) heißt \emph{Riemann-Dichtefunktion} (oder Riemann-Dichte oder kurz Dichte) über \(\R^n\), falls \(f\parentheses*{x} \ge 0, x \in \R^n\), und
        \[
            \int_{-\infty}^\infty \cdots \int_{-\infty}^\infty f\parentheses*{x_1, \ldots, x_n}\d x_1 \ldots \d x_n = 1
        \]
        gilt.
    \end{definition}

    Eine Riemann-Dichte legt eindeutig ein Wahrscheinlichkeitsmaß über \(\R^n\) fest.

    \begin{remark}
        Ist \(f\) eine Dichte über \(\R^n\), so ist die zugehörige Verteilungsfunktion \(F\) stetig und gegeben durch
        \[
            F\parentheses*{x_1, \ldots, x_n} = \int_{-\infty}^{x_n} \cdots \int_{-\infty}^{x_1}f\parentheses*{y_1, \ldots, y_n}\d y_1 \ldots \d y_n, \quad \parentheses*{x_1, \ldots, x_n} \in \R^n.
        \]
        Wahrscheinlichkeiten für das zugehörige Wahrscheinlichkeitsmaß \(P\) bestimmt man über die Integraldarstellung
        \[
            P\parentheses*{\bigtimes_{i = 1}^n \brackets*{a_i, b_i}} = \int_{a_n}^{b_n} \cdots \int_{a_1}^{b_1}f\parentheses*{y_1, \ldots, y_n}\d y_1 \ldots \d y_n
        \]
        für alle \(a = \parentheses*{a_1, \ldots, a_n} \in \R^n\) und \(b = \parentheses*{b_1, \ldots, b_n} \in \R^n\) mit \(a_i \le b_i, 1 \le i \le n\).
        Die eindimensionalen Intervalle \(\brackets*{a_i, b_i}, 1 \le i \le n\) können dabei auch (ohne Änderung des Wertes der Wahrscheinlichkeit) durch halboffene oder offene Intervalle ersetzt werden.
    \end{remark}

    \begin{example}
        Eine zweidimensionale Dichtefunktion ist gegeben durch
        \[
            f\parentheses*{x, y} = \begin{cases}
                2e^{-\parentheses*{2x + y}}, & \text{falls }x, y \ge 0,\\
                0, & \text{sonst}.
            \end{cases}
        \]
        Die zugehörige Verteilungsfunktion berechnet sich zu
        \[
            F\parentheses*{x, y} = \begin{cases}
                \parentheses*{1 - e^{-2x}}\parentheses*{1 - e^{-y}}, & \text{falls }x, y \ge 0,\\
                0, & \text{sonst}.
            \end{cases}
        \]
    \end{example}
\end{document}
