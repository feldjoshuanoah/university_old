\documentclass{lecture}

\institute{Institut für Statistik und Wirtschaftsmathematik}
\title{Vorlesung 14}
\author{Joshua Feld, 406718}
\course{Statistik}
\professor{Cramer}
\semester{Sommersemester 2022}
\program{CES (Bachelor)}

\begin{document}
    \maketitle


    \begin{example}
        Die bivariate Normalverteilung eines Zufallsvektors \(\parentheses*{X_1, X_2}\) ist definiert durch die Dichtefunktion (für \(x_1, x_2 \in \R\))
        \begin{equation}\label{eq:1}
            f^{X_1, X_2}\parentheses*{x_1, x_2} = \frac{1}{2\pi\sigma_1 \sigma_2 \sqrt{1 - \rho^2}}\exp\parentheses*{-\frac{1}{2 \cdot \parentheses*{1 - \rho^2}}\parentheses*{\frac{\parentheses*{x_1 - \mu_1}^2}{\sigma_1^2} - 2\rho\frac{\parentheses*{x_1 - \mu_1}\parentheses*{x_2 - \mu_2}}{\sigma_1 \sigma_2} + \frac{\parentheses*{x_2 - \mu_2}^2}{\sigma_2^2}}}
        \end{equation}
        mit den fünf Parametern \(\mu_1, \mu_2 \in \R\), \(\sigma_1^2, \sigma_2^2 > 0\) und \(\rho \in \parentheses*{-1, 1}\).
        Sie wird bezeichnet mit \(\parentheses*{X_1, X_2} \sim N_2\parentheses*{\mu_1, \mu_2, \sigma_1^2, \sigma_2^2, \rho}\).
        Die Dichtefunktion wird direkt aus der in Definition x aus Vorlesung 13 gegebenen Dichte der multivariaten Normalverteilung hergeleitet.
        Als Matrix \(\Sigma\) wird
        \[
            \Sigma = \begin{pmatrix}
                \sigma_1^2 & \rho\sigma_1 \sigma_2\\
                \rho\sigma_1 \sigma_2 & \sigma_2^2
            \end{pmatrix} \in \R^{2 \times 2}
        \]
        gewählt, deren Determinante \(\det\Sigma = \sigma_1^2 \sigma_2^2\parentheses*{1 - \rho^2}\) ist.
        Dies impliziert insbesondere nach dem Minorantenkriterium, dass \(\Sigma\) nur für \(\rho \in \parentheses*{-1, 1}\) positiv definit ist.
        Der Vektor \(\mu\) ist gegeben durch \(\mu = \parentheses*{\mu_1, \mu_2} \in \R^2\).
        Ausmultiplizieren des Arguments der Exponentialfunktion in Beispiel x aus Vorlesung 13 mit
        \[
            \Sigma^{-1} = \frac{1}{1 - \rho^2}\begin{pmatrix}
                \frac{1}{\sigma_1^2} & -\frac{\rho}{\sigma_1 \sigma_2}\\
                -\frac{\rho}{\sigma_1 \sigma_2} & \frac{1}{\sigma_2^2}
            \end{pmatrix}
        \]
        liefert die Darstellung in \eqref{eq:1}.
    \end{example}

    Die \(i\)-te Randdichte einer multivariaten Verteilung, die durch ihre \(n\)-dimensionale Dichte bestimmt ist, erhält man durch Integration.

    \begin{remark}
        Ist \(X = \parentheses*{X_1, \ldots, X_n}\) ein Zufallsvektor mit Dichte \(f^X\), so gilt für die \(i\)-te Randdichte (die Dichtefunktion der \(i\)-ten Randverteilung) mit \(t \in \R\):
        \[
            f^{X_i}\parentheses*{t} = \int_{-\infty}^\infty \cdots \int_{-\infty}^\infty f^X\parentheses*{x_1, \ldots, x_{i - 1}, t, x_{i + 1}, \ldots, x_n}\d x_1 \ldots \d x_{i - 1} \d x_{i + 1} \ldots \d x_n.
        \]
    \end{remark}

    Die multivariate Normalverteilung hat die besondere Eigenschaft, dass die Randverteilungen wiederum Normalverteilungen sind.
    Für die bivariate Normalverteilung ist diese Aussage im folgenden Satz enthalten.

    \begin{theorem}
        Sei \(\parentheses*{X_1, X_2} \sim N_2\parentheses*{\mu_1, \mu_2, \sigma_1^2, \sigma_2^2, \rho}\) mit \(\mu_1, \mu_2 \in \R\), \(\sigma_1^2, \sigma_2^2 > 0\) sowie \(\rho \in \parentheses*{-1, 1}\).
        Dann gilt:
        \[
            X_1 \sim N\parentheses*{\mu_1, \sigma_1^2}, \quad X_2 \sim N\parentheses*{\mu_2, \sigma_2^2}.
        \]
    \end{theorem}

    \begin{proof}
        \(\parentheses*{X_1, X_2}\) hat die in \eqref{eq:1} gegebene Dichte.
        Das Argument der Exponentialfunktion kann durch eine geeignete quadratische Ergänzung als Summe zweier Quadrate geschrieben werden:
        \begin{align*}
            &\frac{\parentheses*{x_1 - \mu_1}^2}{\sigma_1^2} - 2\rho\frac{\parentheses*{x_1 - \mu_1}\parentheses*{x_2 - \mu_2}}{\sigma_1 \sigma_2} + \frac{\parentheses*{x_2 - \mu_2}^2}{\sigma_2^2}\\
            &= \frac{\parentheses*{x_1 - \mu_1}^2}{\sigma_1^2} - 2\rho\frac{\parentheses*{x_1 - \mu_1}\parentheses*{x_2 - \mu_2}}{\sigma_1 \sigma_2} + \frac{\rho^2\parentheses*{x_2 - \mu_2}^2}{\sigma_2^2} + \frac{\parentheses*{1 - \rho^2}\parentheses*{x_2 - \mu_2}^2}{\sigma_2^2}\\
            &= \parentheses*{\frac{x_1 - \mu_1}{\sigma_1} - \frac{\rho\parentheses*{x_2 - \mu_2}}{\sigma_2}}^2 + \frac{\parentheses*{1 - \rho^2}\parentheses*{x_2 - \mu_2}^2}{\sigma_2^2}\\
            &= \frac{\parentheses*{x_1 - \mu_1 - \frac{\sigma_1 \rho\parentheses*{x_2 - \mu_2}}{\sigma_2}}^2}{\sigma_1^2} + \frac{\parentheses*{1 - \rho^2}\parentheses*{x_2 - \mu_2}^2}{\sigma_2^2}.
        \end{align*}
        Daraus ergibt sich für die Dichte in \eqref{eq:1} die Faktorisierung
        \[
            f^{X_1, X_2}\parentheses*{x_1, x_2} = g\parentheses*{x_1, x_2} \cdot h\parentheses*{x_2},
        \]
        wobei
        \[
            g\parentheses*{x_1, x_2} = \frac{1}{\sqrt{2\pi}\sigma_1\sqrt{1 - \rho^2}}\exp\parentheses*{-\frac{1}{2\parentheses*{1 - \rho^2}\sigma_1^2}\parentheses*{x_1 - \mu_1 - \frac{\sigma_1 \rho\parentheses*{x_2 - \mu_2}}{\sigma_2}}^2}
        \]
        und
        \[
            h\parentheses*{x_2} = \frac{1}{\sqrt{2\pi}\sigma_2}\exp\parentheses*{-\frac{\parentheses*{x_2 - \mu_2}^2}{2\sigma_2^2}}.
        \]
        Dabei ist \(g\parentheses+{\cdot, x_2}\) für jedes feste \(x_2 \in \R\) die Dichte einer \(N\parentheses*{\mu_1 + \frac{\sigma_1 \rho\parentheses*{x_2 - \mu_2}}{\sigma_2}, \parentheses*{1 - \rho^2}\sigma_1^2}\)-Verteilung, \(h\) ist Dichte einer \(N\parentheses*{\mu_2, \sigma_2^2}\)-Verteilung.
        Die Integration bzgl. \(x_1\) liefert nun
        \[
            f^{X_2}\parentheses*{x_2} = \int_{-\infty}^\infty f^{X_1, X_2}\parentheses*{x_1, x_2}\d x_1 = h\parentheses*{x_2}\underbrace{\int_{-\infty}^\infty g\parentheses*{x_1, x_2}\d x_1}_{= 1} = h\parentheses*{x_2},
        \]
        da das Integral über eine Dichtefunktion stets gleich Eins ist.
        Damit ist \(f^{X_2}\) die Dichte einer \(N\parentheses*{\mu_2, \sigma_2^2}\)-Verteilung.
        Eine analoge Argumentation liefert die Randverteilung von \(X_1\).
    \end{proof}

    Für Dichten multivariater Verteilungen gilt ferner allgemein der folgende Zusammenhang zur stochastischen Unabhängigkeit.

    \begin{theorem}\label{thrm:2}
        \(X_1, \ldots, X_n\) sind stochastisch unabhängige Zufallsvariablen mit Dichten \(f^{X_1}, \ldots, f^{X_n}\) genau dann, wenn
        \[
            f^{\parentheses*{X_1, \ldots, X_n}}\parentheses*{x_1, \ldots, x_n} = \prod_{i = 1}^n f^{X_i}\parentheses*{x_i} \quad \forall\parentheses*{x_1, \ldots, x_n} \in \R^n.
        \]
    \end{theorem}

    Im Fall der stochastischen Unabhängigkeit ist also die gemeinsame Dichte gerade durch das Produkt der Randdichten gegeben.
    Für die bivariate Normalverteilung liefert Theorem \ref{thrm:2} eine einfache Charakterisierung der stochastischen Unabhängigkeit.

    \begin{theorem}
        Sei \(\parentheses*{X_1, X_2} \sim N_2\parentheses*{\mu_1, \mu_2, \sigma_1^2, \sigma_2^2, \rho}\) mit \(\mu_1, \mu_2 \in \R\), \(\sigma_1^2, \sigma_2^2 > 0\) sowie \(\rho \in \parentheses*{-1, 1}\).
        Dann gilt:
        \[
            X_1, X_2\text{ stochastisch unabhängig} \iff \rho = 0.
        \]
    \end{theorem}

    \begin{proof}
        Ist \(\rho = 0\), so resultiert direkt die geforderte Produktdarstellung aus \eqref{eq:1} mit den Eigenschaften der Exponentialfunktion.
        Sei umgekehrt die Gleichung \(f^{X_1, X_2}\parentheses*{x_1, x_2} = f^{X_1}\parentheses*{x_1} \cdot f^{X_2}\parentheses*{x_2}\) für alle \(x_1, x_2 \in \R\) gegeben.
        Dann gilt speziell für \(x_1 = \mu_1\) und \(x_2 = \mu_2\)
        \[
            f^{X_1, X_2}\parentheses*{\mu_1, \mu_2} = f^{X_1}\parentheses*{\mu_1} \cdot f^{X_2}\parentheses*{\mu_2} \iff \frac{1}{2\pi\sigma_1 \sigma_2 \sqrt{1 - \rho^2}} = \frac{1}{\sqrt{2\pi}\sigma_1} \cdot \frac{1}{\sqrt{2\pi}\sigma_2}.
        \]
        Die letzte Gleichung ist äquivalent zu \(\sqrt{1 - \rho^2} = 1\) bzw. \(\rho^2 = 0\).
    \end{proof}


    \section*{Transformationen von Zufallsvariablen}

    Ausgehend von einer Wahrscheinlichkeitsverteilung wird in der Analyse stochastischer Modelle häufg die Verteilung transformierter Zufallsvariablen oder Zufallsvektoren benötigt.
    Als Hilfsmittel zur Berechnung der Verteilungen derartiger Transformationen werden nachfolgend für Zufallsvariablen und Zufallsvektoren mit Riemann-Dichten sogenannte Transformationsformeln betrachtet.
    Im Eindimensionalen wird meist über die Verteilungsfunktion argumentiert.

    \begin{example}
        \begin{enumerate}
            \item Die Normalverteilung wurde in Definition 12 der siebten Vorlesung eingeführt.
            Dort wurde bereits die folgende Eigenschaft erwähnt: Ist \(X\) standardnormalverteilt und sind \(\mu \in \R, \sigma > 0\) Parameter, so gilt
            \[
                Y = \sigma X + \mu \sim N\parentheses*{\mu, \sigma^2},
            \]
            d.h. die lineare Transformation \(Y\) einer normalverteilten Zufallsvariablen \(X\) ist wiederum normalverteilt.
            Für \(y \in \R\) gilt nämlich:
            \begin{align*}
                F^Y\parentheses*{y} &= P\parentheses*{Y \le y}\\
                &= P\parentheses*{X \le \frac{y - \mu}{\sigma}}\\
                &= \int_{-\infty}^{\frac{y - \mu}{\sigma}}\frac{1}{\sqrt{2\pi}}\exp\braces*{-\frac{x^2}{2}}\d x\\
                &= \frac{1}{\sqrt{2\pi}\sigma}\int_{-\infty}^y \exp\braces*{-\frac{\parentheses*{x - \mu}^2}{2\sigma^2}}\d x\\
                &= \Phi_{\mu, \sigma^2}\parentheses*{y},
            \end{align*}
            da \(f\parentheses*{x} = \frac{1}{\sqrt{2\pi}\sigma}\exp\braces*{-\frac{\parentheses*{x - \mu}^2}{2\sigma^2}}, x \in \R\) die Dichte der \(N\parentheses*{\mu, \sigma^2}\)-Verteilung ist.

            Andererseits folgt mit derselben Argumentation: Ist \(Y \sim N\parentheses*{\mu, \sigma^2}\), so gilt \(X = \frac{Y - \mu}{\sigma} \sim N\parentheses*{0, 1}\).
            Da also jede Normalverteilung (durch eine lineare Transformation) auf eine Standardnormalverteilung transformiert werden kann, müssen nur für diese numerische Werte vorliegen; diese findet man in Tabellen zusammengefasst in vielen Büchern zur Statistik.
            Der Zusammenhang wird in der schließenden Statistik bei zugrundegelegten Normalverteilungen oft verwendet.
            \item Seien \(X \sim N\parentheses*{0, 1}\) und \(Y = X^2\).
            Für die Verteilung des Quadrats einer standardnormalverteilten Zufallsvariablen gilt aufgrund der Beziehung \(\Phi\parentheses*{x} = 1 - \Phi\parentheses*{-x}\):
            \[
                F^Y\parentheses*{y} = P\parentheses*{X^2 \le y} = \begin{cases}
                    0, & \text{falls }y \le 0,\\
                    P\parentheses*{-\sqrt{y} \le X \le \sqrt{y}} = \Phi\parentheses*{\sqrt{y}} - \Phi\parentheses*{-\sqrt{y}} = 2\Phi\parentheses*{\sqrt{y}} - 1, & \text{falls }y > 0.
                \end{cases}
            \]
            Damit gilt für die Dichte von \(Y\) für \(y > 0\):
            \[
                f^Y\parentheses*{y} = \frac{\d}{\d y}F^Y\parentheses*{y} = 2\varphi\parentheses*{\sqrt{y}} \cdot \frac{1}{2\sqrt{y}} = \frac{1}{\sqrt{2\pi}}y^{-\frac{1}{2}}e^{-\frac{y}{2}}.
            \]
            Wegen \(\Gamma\parentheses*{\frac{1}{2}} = \sqrt{\pi}\) ist \(Y \sim \chi^2\parentheses*{1}\) bzw. \(Y \sim \Gamma\parentheses*{\frac{1}{2}, \frac{1}{2}}\).
            Damit ist auch die Verteilung einer Summe \(X_1^2 + \cdots + X_n^2\) stochastisch unabhängiger \(N\parentheses*{0, 1}\)-verteilter Zufallsvariablen \(X_1, \ldots, X_n\) wiederum \(\chi^2\)-verteilt: \(\sum_{i = 1}^n X_i^2 \sim \chi^2\parentheses*{n}\) (bzw. \(\sim \Gamma\parentheses*{\frac{1}{2}, \frac{n}{2}}\)).
            \item Analog zu 1) resultiert für \(X \sim \Gamma\parentheses*{\alpha, \beta}\) und \(a > 0\) die Eigenschaft
            \[
                aX \sim \Gamma\parentheses*{\frac{\alpha}{a}, \beta}.
            \] 
        \end{enumerate}
    \end{example}

    \begin{theorem}
        Die Zufallsvariable \(X\) habe die Riemann-Dichte \(f^X\) mit
        \[
            f^X\parentheses*{x}\begin{cases}
                > 0, & \text{falls }x \in \parentheses*{a, b},\\
                = 0, & \text{sonst}.
            \end{cases}
        \]
        Weiterhin sei die Funktion \(g: \parentheses*{a, b} \to \parentheses*{c, d}\) bijektiv und stetig differenzierbar mit stetig differenzierbarer Umkehrfunktion \(g^{-1}\).
        Dann besitzt die transformierte Zufallsvariable \(Y = g\parentheses*{X}\) die Dichte
        \[
            f^Y\parentheses*{y} = \begin{cases}
                \absolute*{\parentheses*{g^{-1}}'\parentheses*{y}}f^X\parentheses*{g^{-1}\parentheses*{y}}, & \text{falls }y \in \parentheses*{c, d},\\
                0, & \text{sonst}.
            \end{cases}
        \]
    \end{theorem}

    \begin{remark}
        Die obige Formel enthält die Ableitung der Umkehrfunktion.
        Unter den Voraussetzungen des Theorem gilt mit einer Formel für die Ableitung der Umkehrfunktion \(\parentheses*{g^{-1}}'\parentheses*{y} = \frac{1}{g'\parentheses*{g^{-1}\parentheses*{y}}}\) die Darstellung
        \[
            f^Y\parentheses*{y} = \frac{f^X\parentheses*{g^{-1}\parentheses*{y}}}{\absolute*{g'\parentheses*{g^{-1}\parentheses*{y}}}}, \quad y \in \parentheses*{c, d}.
        \]
    \end{remark}

    Auf die linearen Transformationen in Beispiel 2, 1) und 3) ist der Satz direkt anwendbar mit \(g\parentheses*{x} = \sigma x + \mu\) und \(g\parentheses*{x} = ax, x \in \R\).
    Die Funktion \(g\parentheses*{x} = x^2\) ist auf \(\R\) nicht bijektiv, sodass Theorem 4 in Beispiel 2, 2) nicht von Nutzen ist.
\end{document}
