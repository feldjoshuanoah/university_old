\documentclass{lecture}

\institute{Institut für Statistik und Wirtschaftsmathematik}
\title{Vorlesung 20}
\author{Joshua Feld, 406718}
\course{Statistik}
\professor{Cramer}
\semester{Sommersemester 2022}
\program{CES (Bachelor)}

\begin{document}
    \maketitle

    
    \section*{Schätzungen bei Normalverteilung}
    
    In dieser Vorlesung werden nur Normalverteilungsmodelle betrachtet, die in der Praxis breite Verwendung finden.
    Dabei werden folgende Situationen unterschieden:
    \begin{align}
        X_1, \ldots, X_n &\stackrel{\text{iid}}{\sim} N\parentheses*{\mu, \sigma^2}\text{ mit }\mu \in \R\text{ unbekannt und }\sigma^2 > 0\text{ bekannt},\label{eq:1}\\
        X_1, \ldots, X_n &\stackrel{\text{iid}}{\sim} N\parentheses*{\mu, \sigma^2}\text{ mit }\mu \in \R\text{ bekannt und }\sigma^2 > 0\text{ unbekannt},\label{eq:2}\\
        X_1, \ldots, X_n &\stackrel{\text{iid}}{\sim} N\parentheses*{\mu, \sigma^2}\text{ mit }\mu \in \R\text{ unbekannt und }\sigma^2 > 0\text{ unbekannt}.\label{eq:3}
    \end{align}
    
    
    \section*{Punktschätzung}
    
    Das folgende Theorem enthält die Maximum-Likelihood-Schätzer für die obigen Normalverteilungsmodelle.
    
    \begin{theorem}
        \begin{enumerate}
            \item Der Maximum-Likelihood-Schätzer für \(\mu\) in \eqref{eq:1} ist gegeben durch \(\hat{\mu} = \bar{X}\).
            \item Der Maximum-Likelihood-Schätzer für \(\sigma^2\) in \eqref{eq:2} ist gegeben durch \(\hat{\sigma}_\mu^2 = \frac{1}{n}\sum_{i = 1}^n \parentheses*{X_i - \mu}^2\).
            \item Der Maximum-Likelihood-Schätzer für \(\parentheses*{\mu, \sigma^2}\) in \eqref{eq:3} ist gegeben durch \(\parentheses*{\hat{\mu}, \tilde{\sigma}^2}\) mit
            \[
                \hat{\mu} = \bar{X}, \quad \tilde{\sigma}^2 = S^2 = \frac{1}{n}\sum_{i = 1}^n \parentheses*{X_i - \bar{X}}^2.
            \]
        \end{enumerate}
    \end{theorem}
    
    \begin{proof}
        Im Folgenden werden nur (i) und (iii) bewiesen.
        \(x_1, \ldots, x_n \in \R\) sei ein Stichprobenergebnis.
        Sei \(X_1, \ldots, X_n \stackrel{\text{iid}}{\sim} N\parentheses*{\mu, \sigma_0^2}, \mu \in \R\) mit \(\sigma_0^2 > 0\) bekannt.
        Dann ist die Likelihoodfunktion gegeben durch
        \[
            L\parentheses*{\mu} = \frac{1}{\parentheses*{\sqrt{2\pi}}^n \sigma_0^n}\exp\parentheses*{-\frac{1}{2\sigma_0^2}\sum_{i = 1}^n \parentheses*{x_i - \mu}^2}.
        \]
        Aus der \(\log\)-Likelihoodfunktion \(l\parentheses*{\mu} = -n\ln\parentheses*{\sqrt{2\pi}\sigma_0} - \frac{1}{2\sigma_0^2}\sum_{i = 1}^n \parentheses*{x_i - \mu}^2\) resultiert nach dem Verschiebungssatz die Darstellung
        \[
            l\parentheses*{u} = -n\ln\parentheses*{\sqrt{2\pi}\sigma_0} - \frac{1}{2\sigma_0^2}\sum_{i = 1}^n \parentheses*{x_i - \bar{x}}^2 - \frac{n}{2\sigma_0^2}\parentheses*{\mu - \bar{x}}^2 = l\parentheses*{\bar{x}} - \frac{n}{2\sigma_0^2}\parentheses*{\mu - \bar{x}}^2.
        \]
        Für \(\mu \in \R\) ist \(l\parentheses*{\mu}\) maximal für \(\mu = \bar{x}\), sodass \(\hat{\mu} = \bar{X}\) Maximum-Likelihood-Schätzer für \(\mu\) ist.
        Sei \(X_1, \ldots, X_n \stackrel{\text{iid}}{\sim} N\parentheses*{\mu, \sigma^2}, \mu \in \R, \sigma^2 > 0\).
        Dann ist die Likelihoodfunktion gegeben durch
        \[
            L\parentheses*{\mu, \sigma} = \frac{1}{\parentheses*{\sqrt{2\pi}}^n \sigma^n}\exp\parentheses*{-\frac{1}{2\sigma^2}\sum_{i = 1}^n \parentheses*{x_i - \mu}^2}.
        \]
        Aus der \(\log\)-Likelihoodfunktion \(l\parentheses*{\mu, \sigma} = -n\ln\parentheses*{\sqrt{2\pi}} - n\ln\sigma - \frac{1}{2\sigma_0^2}\sum_{i = 1}^n \parentheses*{x_i - \mu}^2\) resultiert nach dem Verschiebungssatz die Darstellung
        \[
            l\parentheses*{\mu, \sigma} = -n\ln\parentheses*{\sqrt{2\pi}} - n\ln\parentheses*{\sigma} - \frac{n}{2\sigma^2}s^2 - \frac{n}{2\sigma^2}\parentheses*{\mu - \bar{x}}^2.
        \]
        Dieser Wert kann für jedes \(\sigma > 0\) nach oben abgeschätzt werden durch \(h\parentheses*{\sigma} = -n\ln\parentheses*{\sqrt{2\pi}} - n\ln\parentheses*{\sigma} - \frac{n}{2\sigma^2}s^2\), d.h.
        \[
            l\parentheses*{\mu, \sigma} \le h\parentheses*{\sigma},
        \]
        mit Gleichheit genau dann, wenn \(\mu = \bar{x}\).
        Ableiten von \(h\) bzgl. \(\sigma\) ergibt
        \[
            h'\parentheses*{\sigma} = -\frac{n}{\sigma} + \frac{ns^2}{\sigma^3},
        \]
        sodass \(\sigma^2 = s^2\) der einzige Kandidat für eine lokale Maximalstelle ist.
        Die Untersuchung des Monotonieverhaltens von \(h\) liefert, dass \(\sigma^2 = s^2\) das eindeutige Maximum von \(h\) und wegen
        \[
            l\parentheses*{\mu, \sigma} \le h\parentheses*{\sigma} \le h\parentheses*{s}
        \]
        auch das eindeutige (globale) Maximum von \(l\) liefert.
        Somit sind die genannten Schätzer die Maximum-Likelihood-Schätzer.
    \end{proof}
    
    Die in Theorem 1 hergeleiteten Maximum-Likelihood-Schätzer haben schöne Eigenschaften.
    Beispielsweise sind sie (bis auf \(\tilde{\sigma}^2 = S^2\)) erwartungstreu für \(\mu\) bzw. \(\sigma^2\).
    Das Beispiel zeigt auch, dass die Stichprobenvarianz eine erwartungstreue Schätzung für \(\sigma^2\) ist.
    Dieser Schätzer wird \(\tilde{\sigma}^2\) daher meist vorgezogen.
    Eine weitere wichtige Eigenschaft von \(\hat{\mu}\) und \(\hat{\sigma}^2\) ist ihre stochastische Unabhängigkeit.
    
    \begin{theorem}
        Sei \(X_1, \ldots, X_n \stackrel{\text{iid}}{\sim} N\parentheses*{\mu, \sigma^2}, \mu \in \R, \sigma^2 > 0\).
        Die Maximum-Likelihood-Schätzer \(\hat{\mu}, \tilde{\sigma}^2\) bzw. das Stichprobenmittel \(\hat{\mu} = \bar{X}\) und die Stichprobenvarianz \(\hat{\sigma}^2 = \frac{1}{n - 1}\sum_{i = 1}^n \parentheses*{X_i - \bar{X}}^2\) sind stochastisch unabhängig.
    \end{theorem}
    
    
    \section*{Konfidenzintervalle}
    
    Basierend auf einer Stichprobe \(X_1, \ldots, X_n \stackrel{\text{iid}}{\sim} N\parentheses*{\mu, \sigma^2}\) werden \(\parentheses*{1 - \alpha}\)-Konfidenzintervalle für die Parameter \(\mu\) und \(\sigma^2\) angegeben.
    Als Punktschätzer werden hierbei das Stichprobenmittel \(\bar{X} = \frac{1}{n}\sum_{i = 1}^n X_i\) und die Stichprobenvarianz \(\hat{\sigma}^2 = \frac{1}{n - 1}\sum_{i = 1}^n \parentheses*{X_i -\bar{X}}^2\) verwendet.
    Bei der Konstruktuion der Konfidenzintervalle wird jeweils berücksichtigt, welche der Parameter als bekannt bzw. unbekannt angenommen werden.
    
    \begin{theorem}
        Seien \(\alpha \in \parentheses*{0, 1}\), \(u_\beta\) das \(\beta\)-Quantil der Standardnormalverteilung und \(X_1, \ldots, X_n \stackrel{\text{iid}}{\sim} N\parentheses*{\mu, \sigma_0^2}, \mu \in \R\) mit \(\sigma_0^2 > 0\) wie in Modell \eqref{eq:1}.
        Dann sind \(\parentheses*{1 - \alpha}\)-Konfidenzintervalle für \(\mu\) gegeben durch:
        \begin{enumerate}
            \item Zweiseitiges Konfidenzintervall: \(\brackets*{\bar{X} - u_{1 - \frac{\alpha}{2}}\frac{\sigma_0}{\sqrt{n}}, \bar{X} + u_{1 - \frac{\alpha}{2}}\frac{\sigma_0}{\sqrt{n}}}\),
            \item Einseitiges, unteres Konfidenzintervall: \(\left(-\infty, \bar{X} + u_{1 - \alpha}\frac{\sigma_0}{\sqrt{n}}\right]\),
            \item Einseitiges, oberes Konfidenzintervall: \(\left[\bar{X} - u_{1 - \alpha}\frac{\sigma_0}{\sqrt{n}}, \infty\right)\).
        \end{enumerate}
    \end{theorem}
    
    \begin{proof}
        Zum Nachweis wird die Eigenschaft \(\bar{X} \sim N\parentheses*{\mu, \frac{\sigma_0^2}{n}}\) benutzt.
    \end{proof}

    Das in Theorem 3 aufgeführte zweiseitige Konfidenzintervall kann zur Versuchsplanung im folgenden Sinn genutzt werden.
    Liegt die Vertrauenswahrscheinlichkeit \(1 - \alpha\) fest, so kann vor einer Erhebung der Daten ein Stichprobenumfang \(n\) festgelegt werden, dass das zweiseitige Konfidenzintervall eine vorgegebene Länge \(\mathcal{L}_0\) nicht überschreitet.

    \begin{theorem}
        Seien \(\alpha \in \parentheses*{0, 1}\) und \(\mathcal{L}_0 > 0\).
        Dann hat das zweiseitige Konfidenzintervall aus Theorem 3 höchstens die Länge \(\mathcal{L}_0\), falls der Stichprobenumfang \(n\) die folgende Ungleichung erfüllt:
        \[
            n \ge \frac{4u_{1 - \frac{\alpha}{2}}^2 \sigma_0^2}{\mathcal{L}_0^2}.
        \]
        Der erforderliche Mindeststichprobenumfang ist daher durch die kleinste natürliche Zahl gegeben, die größer oder gleich der rechten Seite der Ungleichung ist.
    \end{theorem}

    \begin{proof}
        Die Länge des zweiseitigen Intervalls ist gegeben durch
        \[
            \hat{o} - \hat{u} = \bar{X} + u_{1 - \frac{\alpha}{2}}\frac{\sigma_0}{\sqrt{n}} - \parentheses*{\bar{X} - u_{1 - \frac{\alpha}{2}}\frac{\sigma_0}{\sqrt{n}}} = 2u_{1 - \frac{\alpha}{2}}\frac{\sigma_0}{\sqrt{n}},
        \]
        sodass die resultierende Bedingung lautet:
        \[
            2u_{1 - \frac{\alpha}{2}}\frac{\sigma_0}{\sqrt{n}} \le \mathcal{L}_0 \iff 2u_{1 - \frac{\alpha}{2}}\frac{\sigma_0}{\mathcal{L}_0} \le \sqrt{n} \iff 4u_{1 - \frac{\alpha}{2}}^2 \frac{\sigma_0^2}{\mathcal{L}_0^2} \le n.
        \]
    \end{proof}

    Durch eine geeignete ``Versuchsplanung'' kann also die Güte des Ergebnisses beeinflusst werden.

    \begin{theorem}
        Seien \(\alpha \in \parentheses*{0, 1}\), \(t_\beta\paretheses*{n - 1}\) das \(\beta\)-Quantil der \(t\parentheses*{n - 1}\)-Verteilung und \(X_1, \ldots, X_n \stackrel{\text{iid}}{\sim} N\parentheses*{\mu, \sigma_0^2}, \mu \in \R, \sigma_0^2 > 0\) wie in Modell \eqref{eq:3}.
        Dann sind \(\parentheses*{1 - \alpha}\)-Konfidenzintervalle für \(\mu\) gegeben durch:
        \begin{enumerate}
            \item Zweiseitiges Konfidenzintervall: \(\brackets*{\bar{X} - t_{1 - \frac{\alpha}{2}}\parentheses*{n - 1}\frac{\hat{\sigma}}{\sqrt{n}}, \bar{X} + t_{1 - \frac{\alpha}{2}}\parentheses*{n - 1}\frac{\hat{\sigma}}{\sqrt{n}}}\),
            \item Einseitiges, unteres Konfidenzintervall: \(\left(-\infty, \bar{X} + t_{1 - \frac{\alpha}{2}}\parentheses*{n - 1}\frac{\hat{\sigma}}{\sqrt{n}}\right]\),
            \item Einseitiges, oberes Konfidenzintervall: \(\left[\bar{X} - t_{1 - \frac{\alpha}{2}}\parentheses*{n - 1}\frac{\hat{\sigma}}{\sqrt{n}}, \infty\right)\).
        \end{enumerate}
    \end{theorem}

    Die obigen Konfidenzintervalle werden analog zu den Konfidenzintervallen aus Theorem 3 mit \(\hat{\sigma}\) anstelle von \(\sigma\) konstruiert.
    Die Quantile der Standardnormalverteilung werden durch die entsprechenden Quantile der \(t\parentheses*{n - 1}\)-Verteilung ersetzt.
    Die Aussagen beruhen auf der Verteilungseigenschaft
    \[
        T = \sqrt{n}\frac{\bar{X} - \mu}{\hat{\sigma}} \sim t\parentheses*{n - 1}.
    \]
    Für dem Parameter \(\sigma^2\) lassen sich ebenfalls ein- bzw. zweiseitige Konfidenzintervalle bestimmen.
    In völliger Analogie zur obigen Vorgehensweise erhält man folgende Intervallschätzungen. Nach Beispiel 5 der siebzehnten Vorlesung werden die Quantile der \(\chi^2\)-Verteilung mit \(n - 1\) Freiheitsgraden verwendet.

    \begin{theorem}
        Seien \(\alpha \in \parentheses*{0, 1}\), \(\chi_\beta^2\parentheses*{n - 1}\) das \(\beta\)-Quantil der \(\chi^2\parentheses*{n - 1}\)-Verteilung und \(X_1, \ldots, X_n \stackrel{\text{iid}}{\sim} N\parentheses*{\mu, \sigma_0^2}, \mu \in \R, \sigma_0^2 > 0\) wie in Modell \eqref{eq:3}.
        Dann sind \(\parentheses*{1 - \alpha}\)-Konfidenzintervalle für \(\mu\) gegeben durch:
        \begin{enumerate}
            \item Zweiseitiges Konfidenzintervall: \(\brackets*{\frac{n - 1}{\chi_{1 - \frac{\alpha}{2}}^2\parentheses*{n - 1}}\hat{\sigma}^2, \frac{n - 1}{\chi_{\frac{\alpha}{2}}^2\parentheses*{n - 1}}\hat{\sigma}^2}\),
            \item Einseitiges, unteres Konfidenzintervall: \(\left(-\infty, \frac{n - 1}{\chi_{\frac{\alpha}{2}}^2\parentheses*{n - 1}}\hat{\sigma}^2\right]\),
            \item Einseitiges, oberes Konfidenzintervall: \(\left[\frac{n - 1}{\chi_{1 - \frac{\alpha}{2}}^2\parentheses*{n - 1}}\hat{\sigma}^2, \infty\right)\).
        \end{enumerate}
    \end{theorem}

    Konfidenzintervalle für \(\sigma^2\) im Modell \eqref{eq:2} erhält man, indem in den Intervallen aus Theorem 6 die Schätzung \(\parentheses*{n - 1}\hat{\sigma}^2\) durch die Statistik \(n\hat{\sigma}_{\mu_0}^2 = \sum_{i = 1}^n \parentheses*{X_i - \mu_0}^2\) ersetzt wird und bei den Quantilen die Anzahl der Freiheitsgrade um \(1\) erhöht wird, also \(\chi_\beta^2\parentheses*{n - 1}\) jeweils durch \(\chi_\beta^2\parentheses*{n}\) ersetzt wird.

    \begin{theorem}
        Seien \(\alpha \in \parentheses*{0, 1}\), \(\chi_\beta^2\parentheses*{n}\) das \(\beta\)-Quantil der \(\chi^2\parentheses*{n}\)-Verteilung und \(X_1, \ldots, X_n \stackrel{\text{iid}}{\sim} N\parentheses*{\mu, \sigma_0^2}, \sigma^2 > 0\) mit \(\mu_0 \in \R\) bekannt wie in Modell \eqref{eq:2}.
        Dann sind \(\parentheses*{1 - \alpha}\)-Konfidenzintervalle für \(\sigma^2\) gegeben durch:
        \begin{enumerate}
            \item Zweiseitiges Konfidenzintervall: \(\brackets*{\frac{n}{\chi_{1 - \frac{\alpha}{2}}^2\parentheses*{n}}\hat{\sigma}_{\mu_0}^2, \frac{n}{\chi_{\frac{\alpha}{2}}^2\parentheses*{n}}\hat{\sigma}_{\mu_0}^2}\),
            \item Einseitiges, unteres Konfidenzintervall: \(\left(-\infty, \frac{n}{\chi_{\frac{\alpha}{2}}^2\parentheses*{n}}\hat{\sigma}_{\mu_0}^2\right]\),
            \item Einseitiges, oberes Konfidenzintervall: \(\left[\frac{n}{\chi_{1 - \frac{\alpha}{2}}^2\parentheses*{n}}\hat{\sigma}_{\mu_0}^2, \infty\right)\).
        \end{enumerate}
    \end{theorem}

    \begin{theorem}
        Für \(\sigma\) gewinnt man geeignete Konfidenzintervalle durch Ziehen der Quadratwurzel aus den entsprechenden Intervallgrenzen \(\hat{u}\) bzw. \(\hat{o}\).
    \end{theorem}


    \section*{Konfidenzintervall für die Differenz \(\delta = \mu_1 - \mu_2\) der Erwartungswerte zweier Normalverteilungen bei unbekannter (gleicher) Varianz \(\sigma^2\)}

    In diesem Abschnitt wird eine Statistik vorgestellt, mit der zwei normalverteilte, stochastisch unabhängige Stichproben miteinander hinsichtlich ihrer Mittelwerte verglichen werden können.
    Eine derartige Vorgehensweise ist dann von Interesse, wenn man sich für Unterschiede in den Erwartungswerten zweier Teilgruppen (z.B. Männer -- Frauen, Produkte zweier (unabhängiger) Anlagen, etc.) interessiert.
    Das betrachtete Modell ist eine Variante des Zweistichprobenmodells mit normalverteilten Stichprobenvariablen, wobei die Varianz \(\sigma^2\) der Stichprobenvariablen in beiden Populationen als gleich unterstellt wird.
    Dies ist eine wesentliche Annahme für die Konstruktion der Konfidenzintervalle in Theorem 8.

    \(X_1, \ldots, X_{n_1} \stackrel{\text{iid}}{\sim} N\parentheses*{\mu_1, \sigma^2}\) und \(Y_1, \ldots, Y_{n_2} \stackrel{\text{iid}}{\sim} N\parentheses*{\mu_2, \sigma^2}\) seien stochastisch unabhängige Stichproben mit \(n_1, n_2 \ge 2\).
    Die Parameter \(\mu_1, \mu_2 \in \R\) und \(\sigma^2 > 0\) seien unbekannt.

    \begin{theorem}
        Sei \(\alpha \in \parentheses*{0, 1}\).
        Ein zweiseitiges \(\parentheses*{1 - \alpha}\)-Konfidenzintervall \(\brackets*{\hat{u}, \hat{o}}\) im gegebenen Modell für die Differenz der Erwartungswerte \(\delta = \mu_1 - \mu_2\) ist gegeben durch
        \[
            \brackets*{\hat{\Delta} - t_{1 - \frac{\alpha}{2}}\parentheses*{n_1 + n_2 - 2}\hat{\sigma}_{\text{pool}} \cdot \sqrt{\frac{1}{n_1} + \frac{1}{n_2}}, \hat{\Delta} + t_{1 - \frac{\alpha}{2}}\parentheses*{n_1 + n_2 - 2}\hat{\sigma}_{\text{pool}} \cdot \sqrt{\frac{1}{n_1} + \frac{1}{n_2}}},
        \]
        wobei \(\hat{\Delta} = \bar{X} - \bar{Y}\) eine Punktschätzung für die Differenz der Erwartungswerte \(\delta\),
        \[
            \hat{\sigma}_{\text{pool}}^2 = \frac{1}{n_1 + n_2 - 2}\parentheses*{\sum_{i = 1}^{n_1}\parentheses*{X_i - \bar{X}}^2 + \sum_{j = 1}^{n_2}\parentheses*{Y_j - \bar{Y}}^2} = \frac{n_1 - 1}{n_1 + n_2 - 2}\hat{\sigma}_1^2 + \frac{n_2 - 1}{n_1 + n_2 - 2}\hat{\sigma}_2^2
        \]
        eine kombinierte Varianzschätzung mit den Stichprobenvarianzen \(\hat{\sigma}_1^2\) und \(\hat{\sigma}_2^2\) der entsprechenden Stichproben sowie \(t_\beta\parentheses*{n_1 + n_2 - 2}\) das \(\beta\)-Quantil der \(t\parentheses*{n_1 + n_2 - 2}\)-Verteilung sind.
        Einseitige Konfidenzintervalle ergeben sich unter Verwendung derselben Größen als
        \[
            \left(-\infty, \hat{\Delta} + t_{1 - \alpha}\parentheses*{n_1 + n_2 - 2}\hat{\sigma}_{\text{pool}} \cdot \sqrt{\frac{1}{n_1} + \frac{1}{n_2}}\right], \quad \left[\hat{\Delta} - t_{1 - \frac{\alpha}{2}}\parentheses*{n_1 + n_2 - 2}\hat{\sigma}_{\text{pool}} \cdot \sqrt{\frac{1}{n_1} + \frac{1}{n_2}}, \infty\right)
        \]
    \end{theorem}

    Die obigen Aussagen beruhen auf der Eigenschaft
    \[
        \parentheses*{\sqrt{\frac{1}{n_1} + \frac{1}{n_2}}}^{-1}\frac{\hat{\Delta} - \parentheses*{\mu_1 - \mu_2}}{\hat{\sigma}_{\text{pool}}} \sim t\parentheses*{n_1 + n_2 - 2}.
    \]
    Zu beachten ist, dass das vorgestellte Konfidenzintervall auf der Annahme beruht, dass die Varianzen in den Stichproben identisch sind.
    Auf diese Annahme kann verzichtet werden, wenn die Varianzen \(\sigma_1^2\) und \(\sigma_2^2\) als gegeben unterstellt werden.
    Gilt \(X_1, \ldots, X_{n_1} \stackrel{\text{iid}}{\sim} N\parentheses*{\mu_1, \sigma_1^2}, Y_1, \ldots, Y_{n_2} \stackrel{\text{iid}}{\sim} N\parentheses*{\mu_2, \sigma_2^2}\), so können unter Verwendung der Statistik
    \[
        \parentheses*{\sqrt{\frac{\sigma_1^2}{n_1} + \frac{\sigma_2^2}{n_2}}}^{-1}\parentheses*{\hat{\Delta} - \parentheses*{\mu_1 - \mu_2}} \sim N\parentheses*{0, 1}
    \]
    entsprechende Konfidenzintervalle konstruiert werden.
\end{document}
