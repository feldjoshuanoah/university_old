\documentclass{exercise}

\institute{Institut für Statistik und Wirtschaftsmathematik}
\title{Präsenzübung 2}
\author{Joshua Feld, 406718}
\course{Statistik}
\professor{Cramer}
\semester{Sommersemester 2022}
\program{CES (Bachelor)}

\begin{document}
    \maketitle


    \section*{Aufgabe 1}

    \begin{problem}
        An einer Klausur haben insgesamt zwanzig Studierende teilgenommen. Für die Lösungen wurden nur ganzzahlige Punkte vergeben, und maximal konnten \(40\) Punkte erreicht werden.
        Hierbei ergaben sich die folgenden Punktzahlen für die einzelnen Studierenden
        \[
            10, 20, 32, 24, 20, 26, 40, 26, 10, 6, 26, 32, 32, 20, 30, 40, 32, 10, 26, 20.
        \]
        \begin{enumerate}
            \item Bestimmen Sie die empirische Verteilungsfunktion zu den erzielten Punktzahlen, und stellen Sie diese graphisch dar.
            \item Zum Bestehen der Klausur waren mindestens \(16\) Punkte erforderlich, und ab \(32\) Punkten gab es die Note \emph{Gut}.
            Bestimmen Sie mithilfe der empirischen Verteilungsfunktion den Anteil der Studierenden, die
            \begin{enumerate}
                \item die Klausur nicht bestanden haben,
                \item mindestens die Note \emph{Gut} erhielten,
                \item die Klausur bestanden haben, aber eine schlechtere Note als \emph{Gut} erhielten.
            \end{enumerate}
        \end{enumerate}
    \end{problem}

    \subsection*{Lösung}
    \begin{enumerate}
        \item
        \item
        \begin{enumerate}
            \item 
            \item 
            \item 
        \end{enumerate} 
    \end{enumerate}


    \section*{Aufgabe 2}

    \begin{problem}
        Bei einem Preisvergleich für einen DVD-Festplatten-Recorder eines bestimmten Fabrikats wurden in neun Geschäften die folgenden Verkaufspreise (in \euro) ermittelt:
        \[
            319, 284, 299, 289, 299, 299, 319, 329, 299.
        \]
        \begin{enumerate}
            \item Berechnen Sie zu den gegebenen Verkaufspreisen
            \begin{enumerate}
                \item den Median,
                \item das arithmetische Mittel,
                \item die empirische Varianz,
                \item die empirische Standardabweichung.
            \end{enumerate}
            \item In einem weiteren Kaufhaus wird der DVD-Festplatten-Recorder ins Sortiment aufgenommen und zum Sonderpreis von \(249\text{\euro}\) angeboten.
            Berechnen Sie für die sich hiermit insgesamt ergebenden zehn Verkaufspreise
            \begin{enumerate}
                \item den Median,
                \item das arithmetische Mittel. 
            \end{enumerate}
        \end{enumerate}
    \end{problem}

    \subsection*{Lösung}
    \begin{enumerate}
        \item
        \begin{enumerate}
            \item 
            \item 
            \item 
            \item 
        \end{enumerate}
        \item
        \begin{enumerate}
            \item 
            \item 
        \end{enumerate}
    \end{enumerate}


    \section*{Aufgabe 3}

    \begin{problem}
        Im Sommer 2014 interessierte sich der Fußball-Fan Benno Wadenkrampf für die Temperaturen der brasilianischen Stadt Recife, in der die deutsche Nationalmannschaft ihr drittes Gruppenspiel bei der Fußball-Weltmeisterschaft 2014 austrug.
        In einer Internet-Quelle fand er hierzu die nachfolgend angegebenen monatlichen (durchschnittlichen) Tageshöchsttemperaturen im Verlauf eines Jahres.
        Diese Temperaturwerte waren allerdings in Grad Fahrenheit statt in Grad Celsius angegeben.
        \begin{center}
            \begin{tabular}{lcccccccccccc}
                \toprule
                Monat & Jan. & Feb. & März & April & Mai & Juni & Juli & Aug. & Sep. & Okt. & Nov. & Dez.\\
                \midrule
                \(T_{\text{max}}\) & \(86,4\) & \(86,4\) & \(86\) & \(85,5\) & \(84\) & \(83,8\) & \(81,1\) & \(81,5\) & \(82,6\) & \(84,2\) & \(86,2\) & \(86,4\)\\
                \bottomrule
            \end{tabular}
        \end{center}
        \begin{enumerate}
            \item Berechnen Sie zu den angegebenen Temperaturwerten (in \si{\degreeFahrenheit}) das arithmetische Mittel, die empirische Varianz und die empirische Standardabweichung.
            \item Berechnen Sie das arithmetische Mittel, die empirische Varianz und die empirische Standardabweichung für die zugehörigen Temperaturwerte in \si{\degreeCelsius}.
        \end{enumerate}
    \end{problem}

    \subsection*{Lösung}
    \begin{enumerate}
        \item
        \item 
    \end{enumerate}


    \section*{Aufgabe 4}

    \begin{problem}
        In einer Einrichtung, die eine Verhaltenstherapie zur Gewichtsreduktion anbietet, meldeten sich innerhalb einer Woche dreißig Personen an.
        Von jeder Person wurde bei der Aufnahme der Body-Mass-Index (kurz: BMI) ermittelt.
        Dieser wird als Quotient aus dem Körpergewicht und dem Quadrat der Körpergröße berechnet.
        Man erhielt folgende Messwerte (in \si{\kilo\gram\per\meter\squared}):
        \[
            28,1, 25, 27,4, 40,9, 40, 29,8, 27,2, 27,4, 26,4, 42,4,\\
        \]
        \[
            28,5, 27,9, 34,1, 34,7, 44,8, 21,3, 26,1, 27,6, 29,1, 26,9,
        \]
        \[
            25,2, 29,3, 36,8, 36,4, 38,5, 26,8, 29,7, 25,7, 28,8, 24,9
        \]
        Gemäß einer in der Medizin üblichen Klassifikation werden BMI-Werte in folgende vier Klassen eingeteilt:
        \begin{center}
            \begin{tabular}{lcccc}
                \toprule
                Klasse & \(1\) & \(2\) & \(3\) & \(4\)\\
                \midrule
                BMI & \(\brackets*{20, 25}\) & \(\left(25, 30\right]\) & \(\left(30, 40\right]\) & \(\left(40, 45\right]\)\\
                \bottomrule
            \end{tabular}
        \end{center}
        Berechnen Sie zu dieser Klasseneinteilung der dreißig gemessenen BMI-Werte die zugehörigen relativen Klassenhäufigkeiten, und erstellen Sie das zugehörige Histogramm.
        Wählen Sie hierbei den Maßstab für die vertikale Achse so, dass die maximale Höhe des Histogramms in Ihrer graphischen Darstellung \(15\sis{\centi\meter}\) beträgt.
    \end{problem}

    \subsection*{Lösung}
\end{document}
