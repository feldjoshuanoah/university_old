\documentclass{exercise}

\institute{Institut für Statistik und Wirtschaftsmathematik}
\title{Präsenzübung 4}
\author{Joshua Feld, 406718}
\course{Statistik}
\professor{Cramer}
\semester{Sommersemester 2022}
\program{CES (Bachelor)}

\begin{document}
    \maketitle


    \section*{Aufgabe 1}

    \begin{problem}
        Beim Würfeln mit einem \emph{unsymmetrischen} sechsseitigen Würfel interessiert man sich für die Wahrscheinlichkeiten, mit denen die einzelnen Augenzahlen \(1, \ldots, 6\) geworfen werden.
        Es sind aber lediglich zu den Ereignissen
        \begin{align*}
            A &\hat{=} \text{``Es wird eine gerade Zahl geworfen''},\\
            B &\hat{=} \text{``Es wird eine Augenzahl kleiner oder gleich }3\text{ geworfen''},\\
            C &\hat{=} \text{``Es wird eine Eins oder eine Vier geworfen''}
        \end{align*}
        die folgenden Wahrscheinlichkeiten bekannt (mit \(P\) als zugrundeliegender Wahrscheinlichkeitsverteilung):
        \[
            P\parentheses*{A} = \frac{5}{8}, \quad P\parentheses*{C} = \frac{5}{12}, \quad P\parentheses*{A \cup B} = \frac{23}{24}, \quad P\parentheses*{A \cap B} = \frac{1}{6}, \quad P\parentheses*{A \cap B} = \frac{1}{3}.
        \]
        Bestimmen Sie aus diesen Angaben die zugehörige Zähldichte, d.h. \(P\parentheses*{\braces*{i}}\) für \(i \in \braces*{1, \ldots, 6}\).
    \end{problem}

    \subsection*{Lösung}
    Mit der Ergebnismenge
    \[
        \Omega := \braces*{1, \ldots, 6}
    \]
    erhält man aus der Aufgabenstellung
    \[
        A = \braces*{2, 4, 6}, \quad B = \braces*{1, 2, 3}, \quad C = \braces*{1, 4}, \quad A \cup B = \braces*{1, 2, 3, 4, 6}, \quad A \cap B = \braces*{2}, \quad A \cap C = \braces*{4}.
    \]
    Es bezeichne \(p: \Omega \to \brackets*{0, 1}\) die gesuchte Zähldichte, d.h. es gilt
    \[
        p\parentheses*{\omega} = P\parentheses*{\braces*{\omega}}, \quad \omega \in \Omega.
    \]
    Dann folgt zunächst aus den zuvor angegebenen Mengendarstellungen:
    \begin{align*}
        p\parentheses*{2} &= P\parentheses*{\braces*{2}} = P\parentheses*{A \cap B} = \frac{1}{6},\\
        p\parentheses*{4} &= P\parentheses*{\braces*{4}} = P\parentheses*{A \cap C} = \frac{1}{3}.
    \end{align*}
    Weiter folgt
    \begin{align*}
        P\parentheses*{A} &= \sum_{\omega \in A}p\parentheses*{\omega} = p\parentheses*{2} + p\parentheses*{4} + p\parentheses*{6},\\
        P\parentheses*{C} &= \sum_{\omega \in C}p\parentheses*{\omega} = p\parentheses*{1} + p\parentheses*{4}\\
        P\parentheses*{A \cup B} &= \sum_{\omega \in A \cup B}p\parentheses*{\omega} = p\parentheses*{1} + p\parentheses*{2} + p\parentheses*{3} + p\parentheses*{4} + p\parentheses*{6}.
    \end{align*}
    Hieraus erhält man
    \begin{align*}
        p\parentheses*{6} &= P\parentheses*{A} - p\parentheses*{2} - p\parentheses*{4} = \frac{5}{8} - \frac{1}{6} - \frac{1}{3} = \frac{1}{8},\\
        p\parentheses*{1} &= P\parentheses*{C} - p\parentheses*{4} = \frac{5}{12} - \frac{1}{3} = \frac{1}{12},\\
        p\parentheses*{3} &= P\parentheses*{A \cup B} - p\parentheses*{1} - p\parentheses*{2} - p\parentheses*{4} - p\parentheses*{6} = \frac{23}{24} - \frac{1}{12} - \frac{1}{6} - \frac{1}{3} - \frac{1}{8} = \frac{1}{4}.
    \end{align*}
    Schließlich gilt mit der Normierungseigenschaft für Zähldichten
    \[
        1 = P\parentheses*{\Omega} = \sum_{\omega \in \Omega}p\parentheses*{\omega} = \sum_{i = 1}^6 p\parentheses*{i} \implies p\parentheses*{5} = 1 - \sum_{\substack{1 \le i \le 6,\\i \ne 5}} p\parentheses*{i} = 1 - \frac{1}{12} - \frac{1}{6} - \frac{1}{4} - \frac{1}{3} - \frac{1}{8} = \frac{1}{24}.
    \]


    \section*{Aufgabe 2}

    \begin{problem}
        Bei einem Sportfest nimmt ein gedopter Sportler an zwei aufeinanderfolgenden Wettkämpfen teil.
        Hierbei bezeichne \(A\) das Ereignis, dass er den ersten Wettkampf gewinnt, \(B\) das Ereignis, dass er den zweiten Wettkampf gewinnt, und \(C\) das Ereignis, dass er \emph{vor} den Wettkämpfen wegen Dopings ausgeschlossen wird.
        Es seien folgende Wahrscheinlichkeiten bekannt:
        \[
            P\parentheses*{A} = \frac{3}{5}, \quad P\parentheses*{B} = \frac{3}{10}, \quad P\parentheses*{C^c} = \frac{9}{10}, \quad P\parentheses*{A \cup B} = \frac{7}{10}.
        \]
        Berechnen Sie aus diesen Angaben die folgenden Wahrscheinlichkeiten:
        \begin{align*}
            \text{a) }&P\parentheses*{A \cap B}, & \text{c) }&P\parentheses*{C \cap A}, & \text{e) }&P\parentheses*{A \cap \parentheses*{A^c \cup B}},\\
            \text{b) }&P\parentheses*{\parentheses*{A \cup B}^c}, & \text{d) }&P\parentheses*{A \cup B \cup C}, & \text{f) }&P\parentheses*{C^c \cap A}.
        \end{align*}
    \end{problem}

    \subsection*{Lösung}
    \begin{enumerate}
        \item Es gilt
        \[
            P\parentheses*{A \cup B} = P\parentheses*{A} + P\parentheses*{B} - P\parentheses*{A \cap B}.
        \]
        Hieraus folgt
        \[
            P\parentheses*{A \cap B} = P\parentheses*{A} + P\parentheses*{B} - P\parentheses*{A \cup B} = \frac{3}{5} + \frac{3}{10} - \frac{7}{10} = \frac{1}{5}.
        \]
        \item Man erhält
        \[
            \parentheses*{\parentheses*{A \cup B}^c} = 1 - P\parentheses*{A \cup B} = 1 - \frac{7}{10} = \frac{3}{10}.
        \]
        \item Zunächst folgt aus der Aufgabenstellung
        \[
            C \cap A = \emptyset \quad \text{und} \quad C \cap B = \emptyset.
        \]
        Hiermit erhält man
        \[
            P\parentheses*{C \cap A} = P\parentheses*{\emptyset} = 0,
        \]
        da \(P\parentheses*{\emptyset} = P\parentheses*{\Omega^c} = 1 - P\parentheses*{\Omega} = 1 - 1 = 0\).
        \item Man erhält
        \begin{align*}
            P\parentheses*{A \cup B \cup C} &= P\parentheses*{\parentheses*{A \cup B} \cup C}\\
            &= P\parentheses*{A \cup B} + P\parentheses*{C} - P\parentheses*{\parentheses*{A \cup B} \cap C}\\
            &= P\parentheses*{A \cup B} + P\parentheses*{C} - P\parentheses*{\parentheses*{A \cap C} \cup \parentheses*{B \cap C}}\\
            &= P\parentheses*{A \cup B} + 1 - P\parentheses*{C^c} - P\parentheses*{\emptyset}\\
            &= \frac{7}{10} + 1 - \frac{9}{10}\\
            &= \frac{4}{5}.
        \end{align*}
        \item Man erhält
        \[
            P\parentheses*{A \cap \parentheses*{A^c \cup B}} = P\parentheses*{\parentheses*{A \cap A^c} \cup \parentheses*{A \cap B}} = P\parentheses*{A \cap B} = \frac{1}{5}.
        \]
        \item Wie bereits in c) festgestellt, ist \(A \cap C = \emptyset\).
        Hiermit folgt zunächst
        \[
            A \subseteq C^c \implies A \cap C^c = A.
        \]
        Hiermit erhält man
        \[
            P\parentheses*{C^c \cap A} = P\parentheses*{A} = \frac{3}{5}.
        \]
    \end{enumerate}


    \section*{Aufgabe 3}

    \begin{problem}
        Die erste Glücksspirale wurde 1971 folgendermaßen durchgeführt: In einer einzigen Trommel befanden sich 70 gleichartige Kugeln, von denen jeweils 7 mit einer der Ziffern \(0, 1, 2, \ldots, 9\) beschriftet waren.
        Um eine 7-stellige Gewinnzahl zu ermitteln, wurden aus der Trommel -- jeweils nach gründlichem Mischen -- nacheinander 7 Kugeln ohne Zurücklegen gezogen.
        
        Berechnen Sie jeweils die Wahrscheinlichkeit dafür, dass bei dieser Ausspielung eine der folgenden Gewinnzahlen gezogen wurde.
        \begin{enumerate}
            \item \(4444444\),
            \item \(9876543\),
            \item \(2224755\).
        \end{enumerate}
        Was dieses Ausspielungsverfahren fair?
    \end{problem}

    \subsection*{Lösung}
    In unserem Modell nehmen wir an, dass alle Kugeln unterscheidbar sind und nummerieren alle 70 Kugeln folgendermaßen mit den Zahlen \(1, \ldots, 70\) durch:
    \begin{align*}
        1, \ldots, 7&: \quad \text{Kugeln mit der Ziffer }0,\\
        8, \ldots, 14&: \quad \text{Kugeln mit der Ziffer }1,\\
        &\vdots\\
        64, \ldots, 70&: \quad \text{Kugeln mit der Ziffer }9.
    \end{align*}
    Aus der Aufgabenstellung folgt:
    \begin{itemize}
        \item Die Ziehung der Kugeln erfolgt \emph{ohne Zurücklegen}.
        \item Die \emph{Reihenfolge} der Ziehung ist \emph{wesentlich} (da z.B. \(9876543\) eine andere Gewinnzahl als \(3456789\) ist).
    \end{itemize}
    Eine geeignete Ergebnismenge ist somit
    \[
        \Omega = \braces*{\parentheses*{\omega_1, \ldots, \omega_7} : \omega_1, \ldots, \omega_7 \in \braces*{1, \ldots, 70}, \omega_i \ne \omega_j\text{ für }i \ne j},
    \]
    für die \(\absolute*{\Omega} = 70 \cdot 69 \cdot \ldots \cdot 64 = \frac{70!}{63!}\) gilt.
    Da in der Aufgabenstellung von ``gründlichem Mischen'' die Rede ist, können wir die Glücksspirale als Laplace-Experiment betrachten, d.h.
    \[
        P\parentheses*{E} = \frac{\absolute*{E}}{\absolute*{\Omega}} = \frac{\absolute*{E}}{70 \cdot 69 \cdot \ldots \cdot 64}\text{ für }E \subseteq \Omega.
    \]
    \begin{enumerate}
        \item Bezeichne \(A\) das Ereignis ``Gewinnzahl \(4444444\)'', dann ist
        \[
            A = \braces*{\parentheses*{\omega_1, \ldots, \omega_7} \in \Omega : \omega_1, \ldots, \omega_7 \in \braces*{29, \ldots, 35}},
        \]
        also die Menge der \(\parentheses*{7, 7}\)-Permutationen aus \(\braces*{29, \ldots, 35}\) ohne Wiederholung.
        Damit folgt
        \[
            \absolute*{A} = 7 \cdot 6 \cdot \ldots \cdot 1 = 7!
        \]
        und somit
        \[
            P\parentheses*{A} = \frac{\absolute*{A}}{\absolute*{\Omega}} = \frac{7!}{\frac{70!}{63!}} \approx 0,83 \cdot 10^{-9}.
        \]
        \item Bezeichne \(B\) das Ereignis ``Gewinnzahl \(9876543\)'', dann ist
        \[
            B = \braces*{\parentheses*{\omega_1, \ldots, \omega_7} \in \Omega : \omega_1 \in \braces*{64, \ldots, 70}, \omega_2 \in \braces*{57, \ldots, 63}, \ldots, \omega_7 \in \braces*{22, \ldots, 28}}.
        \]
        Für jede Komponente von \(\parentheses*{\omega_1, \ldots, \omega_7}\) gibt es (unabhängig voneinander) \(7\) Auswahlmöglichkeiten.
        Damit folgt
        \[
            \absolute*{B} = 7 \cdot \ldots \cdot 7 = 7^7
        \]
        und somit
        \[
            P\parentheses*{B} = \frac{\absolute*{B}}{\absolute*{\Omega}} = \frac{7^7}{\frac{70!}{63!}} \approx 136,31 \cdot 10^{-9}.
        \]
        \item Bezeichne \(C\) das Ereignis ``Gewinnzahl \(2224755\)'', dann ist
        \begin{align*}
            C = \{\parentheses*{\omega_1, \ldots, \omega_7} \in \Omega : \omega_1, \omega_2, \omega_3 &\in \braces*{15, \ldots, 31},\\
            \omega_4 &\in \braces*{29, \ldots, 35},\\
            \omega_5 &\in \braces*{50, \ldots, 56},\\
            \omega_6, \omega_7 &\in \braces*{36, \ldots, 42}\}.
        \end{align*}
        Die einzelnen Komponenten von \(\parentheses*{\omega_1, \ldots, \omega_7}\) werden jeweils aus einer \(7\)-elementigen Menge unter Berücksichtigung der Reihenfolge ohne Wiederholung ausgewählt (\(\parentheses*{7, k}\)-Permutationen ohne Wiederholung mit \(k = 3\), \(k = 1\) bzw. \(k = 2\)).
        Damit folgt
        \[
            \absolute*{C} = \parentheses*{7 \cdot 6 \cdot 5} \cdot 7 \cdot 7 \cdot \parentheses*{7 \cdot 6} = 7^4 \cdot 6^2 \cdot 5
        \]
        und somit
        \[
            P\parentheses*{C} = \frac{\absolute*{C}}{\absolute*{\Omega}} = \frac{7^4 \cdot 6^2 \cdot 5}{\frac{70!}{63!}} \approx 71,53 \cdot 10^{-9}.
        \]
    \end{enumerate}
    Aus a) und b) folgt
    \[
        \frac{P\parentheses*{\text{``Gewinnzahl }9876543\text{''}}}{P\parentheses*{\text{``Gewinnzahl }4444444\text{''}}} = \frac{P\parentheses*{B}}{P\parentheses*{A}} = \frac{7^7}{7!} \approx 163,4.
    \]
    Damit war das Ziehungsverfahren offenbar nicht fair in dem Sinne, dass jede Ziffernfolge die gleiche Gewinnchance hatte.
    Gewinnzahlen mit lauter verschiedenen Ziffern (wie oben in b)) besaßen die höchste Gewinnwahrscheinlichkeit, Gewinnzahlen mit identischen Ziffern (wie oben in a)) die geringste Gewinnwahrscheinlichkeit.


    \section*{Aufgabe 4}

    \begin{problem}
        Die Funktion \(f: \R \to \R\) sei gegeben durch
        \[
            f\parentheses*{x} = \begin{cases}
                \frac{c}{\sqrt{x + 2}}, & \text{falls }2 \le x \le 7,\\
                0, & \text{sonst},
            \end{cases}
        \]
        mit einer Konstanten \(c \in \R\).
        \begin{enumerate}
            \item Bestimmen Sie \(c\) so, dass \(f\) eine Riemann-Dichte ist.
            \item Bestimmen Sie für die in a) ermittelte Konstante \(c\) die zu \(f\) gehörige Verteilungsfunktion \(F\).
            \item Berechnen Sie für die in a) ermittelte Konstante \(c\) die folgenden Wahrscheinlichkeiten:
            \[
                \text{(i) }P\parentheses*{\left(-\infty, 5\right]}, \quad \text{(ii) }P\parentheses*{\left(3, 5\right]}, \quad \text{(iii) }P\parentheses*{\parentheses*{5, \infty}}.
            \]
        \end{enumerate}
    \end{problem}

    \subsection*{Lösung}
    \begin{enumerate}
        \item Die gegebene Funktion \(f: \R \to \R\) ist eine Riemann-Dichte, falls die folgenden beiden Bedingungen erfüllt sind:
        \begin{enumerate}
            \item \(f\parentheses*{x} \ge 0\) für alle \(x \in \R\),
            \item \(\int_{-\infty}^\infty f\parentheses*{x}\d x = 1\).
        \end{enumerate}
        Aufgrund der Definition von \(f\) und (i) muss zunächst \(c \ge 0\) sein.
        Weiter folgt aus (ii):
        \[
            1 = \int_{-\infty}^\infty f\parentheses*{x}\d x = \int_2^7 \frac{c}{\sqrt{x + 2}}\d x = c\int_2^7 \parentheses*{x + 2}^{-\frac{1}{2}}\d x = 2c\left.\parentheses*{x + 2}^{\frac{1}{2}}\right|_2^7 = 2c\parentheses*{\sqrt{9} - \sqrt{4}} = 2c \implies c = \frac{1}{2}.
        \]
        Für \(c = \frac{1}{2}\) ist somit \(f\) eine Riemann-Dichte.
        \item Die zu \(f\) gehörige Verteilungsfunktion \(F\) ist gegeben durch
        \[
            F\parentheses*{x} = \int_{-\infty}^x f\parentheses*{t}\d t, \quad x \in \R.
        \]
        Für \(x < 2\) gilt
        \[
            F\parentheses*{x} = \int_{-\infty}^x f\parentheses*{t}\d t = \int_{-\infty}^x 0\d t = 0.
        \]
        Für \(2 \le x \le 7\) gilt
        \[
            F\parentheses*{x} = \int_{-\infty}^x f\parentheses*{t}\d t = \int_2^x \frac{1}{2\sqrt{t + 2}}\d t = \frac{1}{2}\int_2^x \parentheses*{t + 2}^{-\frac{1}{2}}\d t = \frac{1}{2} \cdot \left.2 \cdot \parentheses*{t + 2}^{\frac{1}{2}}\right|_2^x = \sqrt{x + 2} - \sqrt{4} = \sqrt{x + 2} - 2.
        \]
        Für \(x > 7\) gilt
        \[
            F\parentheses*{x} = \int_{-\infty}^x f\parentheses*{t}\d t = \int_2^7 \frac{1}{2\sqrt{t + 2}}\d t = \sqrt{9} - 2 = 1.
        \]
        Insgesamt erhalten wir somit folgende Darstellung für die Verteilungsfunktion:
        \[
            F\parentheses*{x} = \begin{cases}
                0, & \text{falls }x < 2,\\
                \sqrt{x + 2} - 2, & \text{falls }2 \le x \le 7,\\
                1, & \text{falls }x > 7.
            \end{cases}
        \]
        \item Mit \(c = \frac{1}{2}\) erhält man die folgenden Wahrscheinlichkeiten:
        \begin{enumerate}
            \item \(P\parentheses*{\left(-\infty, 5\right]} = F\parentheses*{5} = \sqrt{7} - 2 \approx 0,646\),
            \item \(P\parentheses*{\left(3, 5\right]} = \int_3^5 f\parentheses*{t}\d t = \int_{-\infty}^5 f\parentheses*{t}\d t - \int_{-\infty}^3 f\parentheses*{t}\d t = F\parentheses*{5} - F\parentheses*{3} = \sqrt{7} - 2 - \parentheses*{\sqrt{5} - \sqrt{2}} \approx 0,41\),
            \item
            \begin{align*}
                P\parentheses*{\parentheses*{5, \infty}} &= P\parentheses*{\parentheses*{-\infty, \infty} \setminus \left(-\infty, 5\right]}\\
                &= P\parentheses*{\parentheses*{-\infty, \infty}} - P\parentheses*{\left(-\infty, 5\right]}\\
                &= \int_{-\infty}^\infty f\parentheses*{t}\d t - \int_{-\infty}^5 f\parentheses*{t}\d t\\
                &= 1 - F\parentheses*{5}\\
                &= 1 - \parentheses*{\sqrt{7} - 2}\\
                &\approx 0,354.
            \end{align*}
        \end{enumerate}
    \end{enumerate}
\end{document}
