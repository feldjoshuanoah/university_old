\documentclass{exercise}

\institute{Institut für Statistik und Wirtschaftsmathematik}
\title{Präsenzübung 6}
\author{Joshua Feld, 406718}
\course{Statistik}
\professor{Cramer}
\semester{Sommersemester 2022}
\program{CES (Bachelor)}

\begin{document}
    \maketitle


    \section*{Aufgabe 1}

    \begin{problem}
        Bäcker H. backt jeden Morgen frische Brötchen für ein Kurhotel.
        Die zufällige Anzahl \(X\) der jeweils nachgefragten Brötchen werde durch die folgende Zähldichte \(p^X: \R \to \R\) beschrieben:
        \[
            p^X\parentheses*{x} = \begin{cases}
                \frac{1}{10000}, & \text{falls }x \in \braces*{1, \ldots, 1000},\\
                \frac{1}{500}, & \text{falls }x \in \braces*{1001, \ldots, 1400},\\
                \frac{1}{6000}, & \text{falls }x \in \braces*{1401, \ldots, 2000},\\
                0, & \text{sonst}.
            \end{cases}
        \]
        \begin{enumerate}
            \item Zeigen Sie, dass durch \(p^X\) tatsächlich die Zähldichte einer diskreten Zufallsvariablen \(X\) gegeben ist.
            Bestimmen Sie darüber hinaus den \emph{Träger}
            \[
                T^X = \braces*{x \in \R : p^X\parentheses*{x} > 0}
            \]
            der (diskreten) Zufallsvariablen \(X\).
            \item Bäcker H. vermag täglich maximal \(1500\) Brötchen zu liefern.
            Berechnen Sie die Wahrscheinlichkeit dafür, dass die Nachfrage des Hotels die Lieferkapazität des Bäckers übersteigt.
        \end{enumerate}
    \end{problem}

    \subsection*{Lösung}
    \begin{enumerate}
        \item
        \item 
    \end{enumerate}


    \section*{Aufgabe 2}

    \begin{problem}
        Die Anzahl der täglich in der Abteilung einer Haftpflichtversicherung eintreffenden Schadensmeldungen kann durch eine Poisson-verteilte Zufallsvariable mit Parameter \(\lambda = 6\) beschrieben werden.

        Berechnen Sie die Wahrscheinlichkeit dafür, dass an einem Tag in dieser Abteilung
        \begin{enumerate}
            \item genau fünf Schadensmeldungen,
            \item mindestens vier Schadensmeldungen eintreffen.
        \end{enumerate}
    \end{problem}

    \subsection*{Lösung}
    \begin{enumerate}
        \item
        \item 
    \end{enumerate}


    \section*{Aufgabe 3}

    \begin{problem}
        Die Lebensdauer (in Stunden) von Energiesparlampen eines bestimmten Fabrikats kann durch eine mit Parameter \(\lambda > 0\) exponentialverteilte Zufallsvariable \(X\) beschrieben werden.
        Die zugehörige Verteilungsfunktion \(F^X: \R \to \brackets*{0, 1}\) ist damit gegeben durch
        \[
            F^X\parentheses*{x} = \begin{cases}
                0, &\text{falls }x < 0,\\
                1 - e^{-\lambda x}, & \text{falls }x \ge 0.
            \end{cases}
        \]
        \begin{enumerate}
            \item Berechnen Sie für \(\lambda = \frac{1}{800}\) die Wahrscheinlichkeit dafür, dass die Lebensdauer einer derartigen Energiesparlampe
            \begin{enumerate}
                \item höchstens \(300\) Stunden,
                \item mehr als \(120\) Stunden,
                \item mindestens \(240\) und höchstens \(360\) Stunden
            \end{enumerate}
            beträgt?
            \item Für welchen Wert des Parameters \(\lambda\) ergibt sich eine Lebensdauerverteilung, bei der mit Wahrscheinlichkeit \(0,99\) die Lebensdauer einer derartigen Energiesparlampe mindestens \(100\) Stunden beträgt?
        \end{enumerate}
    \end{problem}

    \subsection*{Lösung}
    \begin{enumerate}
        \item
        \begin{enumerate}
            \item
            \item
            \item
        \end{enumerate}
        \item
    \end{enumerate}


    \section*{Aufgabe 4}

    \begin{problem}
        Es seien \(X\) und \(Y\) stochastisch unabhängige, diskrete Zufallsvariablen auf einem Wahrscheinlichkeitsraum \(\parentheses*{\Omega, \mathfrak{U}, P}\) mit \(\mathcal{X} = \braces*{0, 1, 2, 3}\) für \(X\) und \(\mathcal{Y} = \braces*{1, 2, 3, 4}\) für \(Y\).
        Die zugehörigen Zähldichten \(p^X\) und \(p^Y\) seien gegeben durch:
        \begin{align*}
            p^X\parentheses*{k} = P\parentheses*{X = k} = \frac{1}{4}\text{ für }k \in \mathcal{X},\\
            p^Y\parentheses*{k} = P\parentheses*{Y = k} = \frac{1}{4}\text{ für }k \in \mathcal{Y}.
        \end{align*}
        Bestimmen Sie die Verteilung (d.h. die zugehörige Zähldichte) von \(Z = X + Y\).
    \end{problem}

    \subsection*{Lösung}
\end{document}
