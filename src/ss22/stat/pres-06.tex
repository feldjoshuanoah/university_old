\documentclass{exercise}

\institute{Institut für Statistik und Wirtschaftsmathematik}
\title{Präsenzübung 6}
\author{Joshua Feld, 406718}
\course{Statistik}
\professor{Cramer}
\semester{Sommersemester 2022}
\program{CES (Bachelor)}

\begin{document}
    \maketitle


    \section*{Aufgabe 1}

    \begin{problem}
        Bäcker H. backt jeden Morgen frische Brötchen für ein Kurhotel.
        Die zufällige Anzahl \(X\) der jeweils nachgefragten Brötchen werde durch die folgende Zähldichte \(p^X: \R \to \R\) beschrieben:
        \[
            p^X\parentheses*{x} = \begin{cases}
                \frac{1}{10000}, & \text{falls }x \in \braces*{1, \ldots, 1000},\\
                \frac{1}{500}, & \text{falls }x \in \braces*{1001, \ldots, 1400},\\
                \frac{1}{6000}, & \text{falls }x \in \braces*{1401, \ldots, 2000},\\
                0, & \text{sonst}.
            \end{cases}
        \]
        \begin{enumerate}
            \item Zeigen Sie, dass durch \(p^X\) tatsächlich die Zähldichte einer diskreten Zufallsvariablen \(X\) gegeben ist.
            Bestimmen Sie darüber hinaus den \emph{Träger}
            \[
                T^X = \braces*{x \in \R : p^X\parentheses*{x} > 0}
            \]
            der (diskreten) Zufallsvariablen \(X\).
            \item Bäcker H. vermag täglich maximal \(1500\) Brötchen zu liefern.
            Berechnen Sie die Wahrscheinlichkeit dafür, dass die Nachfrage des Hotels die Lieferkapazität des Bäckers übersteigt.
        \end{enumerate}
    \end{problem}

    \subsection*{Lösung}
    Eine Funktion \(p: \R \to \R\) ist Zähldichte einer diskreten Zufallsvariablen genau dann, wenn folgende beiden Bedingungen erfüllt sind:
    \begin{enumerate}[label=(\roman*)]
        \item \(p\parentheses*{x} \ge 0\) für alle \(x \in \R\),
        \item \(\sum_{x \in T}p\parentheses*{x} = 1\), wobei \(T = \braces*{x \in \R : p\parentheses*{x} > 0}\) der zugehörige (höchstens abzählbare) Träger ist.
    \end{enumerate}
    \begin{enumerate}
        \item Gemäß Aufgabenstellung ist der Träger von \(X\) gegeben durch
        \[
            T^X = \braces*{x \in \R : p^X\parentheses*{x} > 0} = \braces*{1, \ldots, 2000}.
        \]
        \begin{enumerate}
            \item \(p^X\parentheses*{x} \ge 0\) für alle \(x \in \R\) gemäß Definition.
            \item
            \begin{align*}
                \sum_{x \in T^X}p^X\parentheses*{x} &= \sum_{x = 1}^{2000}p^X\parentheses*{x}\\
                &= \sum_{x = 1}^{1000}\frac{1}{10000} + \sum_{x = 1001}^{1400}\frac{1}{500} + \sum_{x = 1401}^{2000}\frac{1}{6000}\\
                &= 1000 \cdot \frac{1}{10000} + 400 \cdot \frac{1}{500} + 600 \cdot \frac{1}{6000}\\
                &= \frac{1}{10} + \frac{4}{5} + \frac{1}{10} = 1.
            \end{align*}
        \end{enumerate}
        Somit ist \(p^X\) wirklich eine Zähldichte.
        \item Es bezeichne \(P\) die zugrundeliegende Wahrscheinlichkeitsverteilung.
        Es gilt
        \[
            P\parentheses*{X > 1500} = P\parentheses*{X \ge 1501} = \sum_{x = 1501}^{2000}\underbrace{P\parentheses*{X = x}}_{= p^X\parentheses*{x}} = \sum_{x = 1501}^{2000}p^X\parentheses*{x} = \sum_{x = 1501}^{2000}\frac{1}{6000} = \frac{1}{12} \approx 0,083.
        \]
    \end{enumerate}


    \section*{Aufgabe 2}

    \begin{problem}
        Die Anzahl der täglich in der Abteilung einer Haftpflichtversicherung eintreffenden Schadensmeldungen kann durch eine Poisson-verteilte Zufallsvariable mit Parameter \(\lambda = 6\) beschrieben werden.

        Berechnen Sie die Wahrscheinlichkeit dafür, dass an einem Tag in dieser Abteilung
        \begin{enumerate}
            \item genau fünf Schadensmeldungen,
            \item mindestens vier Schadensmeldungen eintreffen.
        \end{enumerate}
    \end{problem}

    \subsection*{Lösung}
    Es bezeichnen \(X\) die Anzahl der täglich eintreffenden Schadensmeldungen und \(P\) die zugrundeliegende Wahrscheinlichkeitsverteilung.
    Dann gilt gemäß Aufgabenstellung
    \[
        P\parentheses*{X = k} = p^X\parentheses*{k} = \frac{\lambda^k}{k!}e^{-\lambda} = \frac{6^k}{k!}e^{-6}, \quad k \in \N_0.
    \]
    \begin{enumerate}
        \item Die Wahrscheinlichkeit dafür, dass an einem Tag genau fünf Schäden gemeldet werden ist gegeben durch
        \[
            P\parentheses*{X = 5} = p^X\parentheses*{5} = \frac{6^5}{5!}e^{-6} \approx 0,161.
        \]
        \item Die Wahrscheinlichkeit dafür, dass an einem Tag mindestens vier Schäden gemeldet werden, ist gegeben durch
        \begin{align*}
            P\parentheses*{X \ge 4} &= 1 - P\parentheses*{X < 4}\\
            &= 1 - P\parentheses*{X \le 3}\\
            &= 1 - \parentheses*{P\parentheses*{X = 0} + P\parentheses*{X = 1} + P\parentheses*{X = 2} + P\parentheses*{X = 3}}\\
            &= 1 - \parentheses*{\frac{6^0}{0!}e^{-6} + \frac{6^1}{1!}e^{-6} + \frac{6^2}{2!}e^{-6} + \frac{6^3}{3!}e^{-6}}\\
            &= 1 - e^{-6} \cdot \parentheses*{1 + 6 + 18 + 36}\\
            &= 1- 61e^{-6} \approx 0,849.
        \end{align*}
    \end{enumerate}


    \section*{Aufgabe 3}

    \begin{problem}
        Die Lebensdauer (in Stunden) von Energiesparlampen eines bestimmten Fabrikats kann durch eine mit Parameter \(\lambda > 0\) exponentialverteilte Zufallsvariable \(X\) beschrieben werden.
        Die zugehörige Verteilungsfunktion \(F^X: \R \to \brackets*{0, 1}\) ist damit gegeben durch
        \[
            F^X\parentheses*{x} = \begin{cases}
                0, &\text{falls }x < 0,\\
                1 - e^{-\lambda x}, & \text{falls }x \ge 0.
            \end{cases}
        \]
        \begin{enumerate}
            \item Berechnen Sie für \(\lambda = \frac{1}{800}\) die Wahrscheinlichkeit dafür, dass die Lebensdauer einer derartigen Energiesparlampe
            \begin{enumerate}
                \item höchstens \(300\) Stunden,
                \item mehr als \(120\) Stunden,
                \item mindestens \(240\) und höchstens \(360\) Stunden
            \end{enumerate}
            beträgt?
            \item Für welchen Wert des Parameters \(\lambda\) ergibt sich eine Lebensdauerverteilung, bei der mit Wahrscheinlichkeit \(0,99\) die Lebensdauer einer derartigen Energiesparlampe mindestens \(100\) Stunden beträgt?
        \end{enumerate}
    \end{problem}

    \subsection*{Lösung}
    Da \(X \sim \Exp\parentheses*{\lambda}\) mit \(\lambda > 0\) ergibt sich die zugehörige Verteilungsfunktion
    \[
        F^X\parentheses*{x} = \begin{cases}
            0, & \text{falls }x < 0,\\
            1 - e^{-\lambda x}, & \text{falls }y \ge 0.
        \end{cases}
    \]
    \begin{enumerate}
        \item Seien \(\lambda = \frac{1}{800}\) und \(P\) die zugrundeliegende Wahrscheinlichkeitsverteilung.
        Dann gilt:
        \begin{enumerate}
            \item \(P\parentheses*{X \le 300} = F^X\parentheses*{300} = 1 - e^{-\frac{300}{800}} = 1 - e^{-\frac{3}{8}} \approx 0,313\),
            \item \(P\parentheses*{X > 120} = 1 - P\parentheses*{X \le 120} = 1 - F^X\parentheses*{120} = e^{-\frac{120}{800}} = e^{-\frac{3}{20}} \approx 0,861\),
            \item \(P\parentheses*{240 \le X \le 360} = P\parentheses*{240 < X \le 360} = F^X\parentheses*{360} - F^X\parentheses*{240} = 1 - e^{-\frac{360}{800}} - \parentheses*{1 - e^{-\frac{240}{800}}} = e^{-\frac{3}{10}} - e^{-\frac{9}{20}} \approx 0,103\).
        \end{enumerate}
        \item
        \[
            P\parentheses*{X \ge 100} = P\parentheses*{X > 100} = 1 - P\parentheses*{X \le 100} = 1 - F^X\parentheses*{100} = e^{-100} \stackrel{!}{=} 0,99
        \]
        \[
            \iff -100\lambda = \ln\parentheses*{0,99} \iff \lambda = -\frac{\ln\parentheses*{0,99}}{100} \approx 1,005 \cdot 10^{-4}.
        \]
    \end{enumerate}


    \section*{Aufgabe 4}

    \begin{problem}
        Es seien \(X\) und \(Y\) stochastisch unabhängige, diskrete Zufallsvariablen auf einem Wahrscheinlichkeitsraum \(\parentheses*{\Omega, \mathfrak{A}, P}\) mit \(\mathcal{X} = \braces*{0, 1, 2, 3}\) für \(X\) und \(\mathcal{Y} = \braces*{1, 2, 3, 4}\) für \(Y\).
        Die zugehörigen Zähldichten \(p^X\) und \(p^Y\) seien gegeben durch:
        \begin{align*}
            p^X\parentheses*{k} = P\parentheses*{X = k} = \frac{1}{4}\text{ für }k \in \mathcal{X},\\
            p^Y\parentheses*{k} = P\parentheses*{Y = k} = \frac{1}{4}\text{ für }k \in \mathcal{Y}.
        \end{align*}
        Bestimmen Sie die Verteilung (d.h. die zugehörige Zähldichte) von \(Z = X + Y\).
    \end{problem}

    \subsection*{Lösung}
    Gemäß Aufgabenstellung besitzt die Zufallsvariable \(Z = X + Y\) folgenden Wertebereich:
    \[
        \mathcal{Z} = \braces*{x + y : x \in \mathcal{X}, y \in \mathcal{Y}} = \braces*{x + y : x \in \braces*{0, 1, 2, 3}, y \in \braces*{1, 2, 3, 4}} = \braces*{1, \ldots, 7}.
    \]
    Da \(X\) und \(Y\) nach Voraussetzung stochastisch unabhängig sind, folgt mit dem Faltungssatz für diskrete, (stochastisch) unabhängige Zufallsvariablen
    \begin{equation}\label{eq:1}
        p^Z\parentheses*{z} = P\parentheses*{Z = z} = \sum_{y \in \mathcal{Y}}p^X\parentheses*{z - y}p^Y\parentheses*{y} = \sum_{y = 1}^4 p^X\parentheses*{z - y}\underbrace{p^Y\parentheses*{y}}_{= \frac{1}{4}} = \frac{1}{4}\sum_{y = 1}^4 p^X\parentheses*{z - y}, \quad z \in \mathcal{Z} = \braces*{1, \ldots, 7}.
    \end{equation}
    Man beachte, dass laut Aufgabenstellung \(p^X\parentheses*{x} = 0\) für \(x \in \R \setminus \mathcal{X}\) und \(p^Y\parentheses*{y} = 0\) für \(y \in \R \setminus \mathcal{Y}\) ist.
    Aus \eqref{eq:1} erhält man mit der Definition von \(p^X\):
    \begin{align*}
        p^Z\parentheses*{1} &= \frac{1}{4}\sum_{y = 1}^4 p^X\parentheses*{1 - y} = \frac{1}{4}\parentheses*{p^X\parentheses*{0} + p^X\parentheses*{-1} + p^X\parentheses*{-2} + p^X\parentheses*{-3}} = \frac{1}{4} \cdot \parentheses*{\frac{1}{4} + 0 + 0 + 0} = \frac{1}{16},\\
        p^Z\parentheses*{2} &= \frac{1}{4}\sum_{y = 1}^4 p^X\parentheses*{2 - y} = \frac{1}{4}\parentheses*{p^X\parentheses*{1} + p^X\parentheses*{0} + p^X\parentheses*{-1} + p^X\parentheses*{-2}} = \frac{1}{4} \cdot \parentheses*{\frac{1}{4} + \frac{1}{4} + 0 + 0} = \frac{1}{8},\\
        p^Z\parentheses*{3} &= \frac{1}{4}\sum_{y = 1}^4 p^X\parentheses*{3 - y} = \frac{1}{4}\parentheses*{p^X\parentheses*{2} + p^X\parentheses*{1} + p^X\parentheses*{0} + p^X\parentheses*{-1}} = \frac{1}{4} \cdot \parentheses*{\frac{1}{4} + \frac{1}{4} + \frac{1}{4} + 0} = \frac{3}{16},\\
        p^Z\parentheses*{4} &= \frac{1}{4}\sum_{y = 1}^4 p^X\parentheses*{4 - y} = \frac{1}{4}\parentheses*{p^X\parentheses*{3} + p^X\parentheses*{2} + p^X\parentheses*{1} + p^X\parentheses*{0}} = \frac{1}{4} \cdot \parentheses*{\frac{1}{4} + \frac{1}{4} + \frac{1}{4} + \frac{1}{4}} = \frac{1}{4},\\
        p^Z\parentheses*{5} &= \frac{1}{4}\sum_{y = 1}^4 p^X\parentheses*{5 - y} = \frac{1}{4}\parentheses*{p^X\parentheses*{4} + p^X\parentheses*{3} + p^X\parentheses*{2} + p^X\parentheses*{1}} = \frac{1}{4} \cdot \parentheses*{0 + \frac{1}{4} + \frac{1}{4} + \frac{1}{4}} = \frac{3}{16},\\
        p^Z\parentheses*{6} &= \frac{1}{4}\sum_{y = 1}^4 p^X\parentheses*{6 - y} = \frac{1}{4}\parentheses*{p^X\parentheses*{5} + p^X\parentheses*{4} + p^X\parentheses*{3} + p^X\parentheses*{2}} = \frac{1}{4} \cdot \parentheses*{0 + 0 + \frac{1}{4} + \frac{1}{4}} = \frac{1}{8},\\
        p^Z\parentheses*{7} &= \frac{1}{4}\sum_{y = 1}^4 p^X\parentheses*{7 - y} = \frac{1}{4}\parentheses*{p^X\parentheses*{6} + p^X\parentheses*{5} + p^X\parentheses*{4} + p^X\parentheses*{3}} = \frac{1}{4} \cdot \parentheses*{0 + 0 + 0 + \frac{1}{4}} = \frac{1}{16}.
    \end{align*}
    Man erkennt deutlich die Symmetrie der Verteilung von \(Z\) (bzgl. \(z = 4\)).
\end{document}
