\documentclass{exercise}

\usepackage{multirow}

\institute{Institut für Statistik und Wirtschaftsmathematik}
\title{Präsenzübung 7}
\author{Joshua Feld, 406718}
\course{Statistik}
\professor{Cramer}
\semester{Sommersemester 2022}
\program{CES (Bachelor)}

\begin{document}
    \maketitle


    \section*{Aufgabe 1}
    
    \begin{problem}
        Die Verteilungsfunktion \(F^X\) einer stetigen Zufallsvariablen \(X\) sei gegeben durch
        \[
            F^X\parentheses*{x} = \begin{cases}
                1 - \frac{1}{2}e^{-\frac{x}{2}}\parentheses*{x + 2}, & \text{falls }x > 0,\\
                0, & \text{falls }x \le 0.
            \end{cases}
        \]
        Bestimmen Sie eine Dichtefunktion \(f^X\) von \(X\).

        \emph{Hinweis: Sie können (ohne eigenen Nachweis) verwenden:
        \[
            \braces*{x \in \R : 0 < F^X\parentheses*{x} < 1} = \parentheses*{0, \infty}.
        \]}
    \end{problem}
    
    \subsection*{Lösung}
    Ist die Verteilungsfunktion \(F^X\) der Zufallsvariablen \(X\) auf
    \[
        \braces*{x \in \R : 0 < F^X\parentheses*{x} < 1} = \parentheses*{a, b}
    \]
    stetig differenzierbar (mit \(a \in \R \cup \braces*{-\infty}\) und \(b \in \R \cup \braces*{\infty}\)), so ist eine Dichtefunktion \(f^X\) von \(X\) gegeben durch:
    \[
        f^X\parentheses*{x} = \begin{cases}
            0, & \text{falls }x \le a,\\
            \parentheses*{F^X}'\parentheses*{x}, & \text{falls }a < x < b,\\
            0, & \text{falls }x \ge b.
        \end{cases}
    \]
    Die gegebene Verteilungsfunktion \(F^X\) ist stetig differenzierbar auf \(\parentheses*{0, \infty}\) mit
    \[
        \parentheses*{F^X}'\parentheses*{x} = -\frac{1}{2}e^{-\frac{x}{2}}\parentheses*{-\frac{1}{2}}\parentheses*{x + 2} - \frac{1}{2}e^{-\frac{x}{2}} \cdot 1 = \frac{1}{4}e^{-\frac{x}{2}}\parentheses*{x + 2 - 2} = \frac{1}{4}xe^{-\frac{x}{2}}, \quad x \in \parentheses*{0, \infty}.
    \]
    Eine Dichtefunktion \(f^X\) von \(X\) ist, mit dem Hinweis, dass
    \[
        \braces*{x \in \R : 0 < F^X\parentheses*{x} < 1} = \parentheses*{0, \infty}
    \]
    gilt, dann gegeben durch
    \[
        f^X\parentheses*{x} = \begin{cases}
            0, & \text{falls }x \le 0,\\
            \frac{1}{4}xe^{-\frac{x}{2}}, & \text{falls }x > 0.
        \end{cases}
    \]


    \section*{Aufgabe 2}
    
    \begin{problem}
        Gegeben sei die folgende (unvollständige) Wahrscheinlichkeitstabelle eines diskreten Zufallsvektors \(\parentheses*{X, Y}\), der auf einem Wahrscheinlichkeitsraum \(\parentheses*{\Omega, \mathfrak{U}, P}\) definiert ist.
        Hierbei nehmen \(X\) die Werte \(1\) und \(2\) und \(Y\) die Werte \(1\), \(2\) und \(3\) an.
        \begin{center}
            \begin{tabular}{cc|ccc|c}
                \toprule
                \multicolumn{2}{c|}{\multirow{2}{*}{\(P\parentheses*{X = i, Y = j}\)}} & \multicolumn{3}{c|}{\(j\)} & \multirow{2}{*}{\(P\parentheses*{X = i}\)}\\
                \multicolumn{2}{c|}{} & \(1\) & \(2\) & \(3\) &\\
                \midrule
                \multirow{2}{*}{\(i\)} & \(1\) & \(0,2\) & ? & ? & \(0,4\)\\
                & \(2\) & ? & ? & ? & ?\\
                \midrule
                \multicolumn{2}{c|}{\(P\parentheses*{Y = j}\)} & ? & \(0,3\) & ? &\\
                \bottomrule
            \end{tabular}
        \end{center}
    \end{problem}
    
    \subsection*{Lösung}
    Aufgrund der geforderten stochastischen Unabhängigkeit von \(X\) und \(Y\) muss gelten:
    \[
        P\parentheses*{X = i, Y = j} = P\parentheses*{X = i} \cdot P\parentheses*{Y = j}, \quad j \in \braces*{1, 2}, j \in \braces*{1, 2, 3}.
    \]
    Weiter gilt aufgrund der Eigenschaften von Zähldichten:
    \[
        P\parentheses*{X = 1} + P\parentheses*{X = 2} = 1, \quad P\parentheses*{Y = 1} + P\parentheses*{Y = 2} + P\parentheses*{Y = 3} = 1.
    \]
    Hiermit erhält man
    \[
        0,2 = P\parentheses*{X = 1, Y = 1} = P\parentheses*{X = 1} \cdot P\parentheses*{Y = 2} = 0,4 \cdot P\parentheses*{Y = 1} \iff P\parentheses*{Y = 1} = \frac{0,2}{0,4} = 0,5,
    \]
    und dann sukzessive
    \begin{align*}
        P\parentheses*{Y = 3} &= 1 - P\parentheses*{Y = 1} - P\parentheses*{Y = 2} = 1 - 0,5 - 0,3 = 0,2,\\
        P\parentheses*{X = 2} &= 1 - P\parentheses*{X = 1} = 1 - 0,4 = 0,6,\\
        P\parentheses*{X = 1, Y = 2} &= P\parentheses*{X = 1} \cdot P\parentheses*{Y = 2} = 0,4 \cdot 0,3 = 0,12,\\
        P\parentheses*{X = 1, Y = 3} &= P\parentheses*{X = 1} \cdot P\parentheses*{Y = 3} = 0,4 \cdot 0,2 = 0,08,\\
        P\parentheses*{X = 2, Y = 1} &= P\parentheses*{X = 2} \cdot P\parentheses*{Y = 1} = 0,6 \cdot 0,5 = 0,3,\\
        P\parentheses*{X = 2, Y = 2} &= P\parentheses*{X = 2} \cdot P\parentheses*{Y = 2} = 0,6 \cdot 0,3 = 0,18,\\
        P\parentheses*{X = 2, Y = 3} &= P\parentheses*{X = 2} \cdot P\parentheses*{Y = 3} = 0,6 \cdot 0,2 = 0,12.
    \end{align*}
    \begin{center}
        \begin{tabular}{cc|ccc|c}
            \toprule
            \multicolumn{2}{c|}{\multirow{2}{*}{\(P\parentheses*{X = i, Y = j}\)}} & \multicolumn{3}{c|}{\(j\)} & \multirow{2}{*}{\(P\parentheses*{X = i}\)}\\
            \multicolumn{2}{c|}{} & \(1\) & \(2\) & \(3\) &\\
            \midrule
            \multirow{2}{*}{\(i\)} & \(1\) & \(0,2\) & \(0,12\) & \(0,08\) & \(0,4\)\\
            & \(2\) & \(0,3\) & \(0,18\) & \(0,12\) & \(0,6\)\\
            \midrule
            \multicolumn{2}{c|}{\(P\parentheses*{Y = j}\)} & \(0,5\) & \(0,3\) & \(0,2\) &\\
            \bottomrule
        \end{tabular}
    \end{center}
        

    \section*{Aufgabe 3}
    
    \begin{problem}
        Das Abwassersystem einer Gemeinde, an das \(1332\) Haushalte angeschlossen sind, ist für eine maximale Last von \(13500\) Litern pro Stunde ausgelegt.

        Nehmen Sie an, dass die einzelnen Abwassermengen (pro Stunde) von \(n\) angeschlossenen Haushalten durch stochastisch unabhängige Zufallsvariablen \(X_1, \ldots, X_n\) beschrieben werden können, wobei \(X_i\) für \(i \in \braces*{1, \ldots, n}\) normalverteilt ist, mit erstem Parameter \(\mu = 10\) (\(\frac{\text{Liter}}{\text{Stunde}}\)) und zweitem Parameter \(\sigma^2 = 4\) (\(\parentheses*{\frac{\text{Liter}}{\text{Stunde}}}^2\)).

        Berechnen Sie die Wahrscheinlichkeit einer Überlastung des Abwassersystems
        \begin{enumerate}
            \item für die angeschlossenen \(1332\) Haushalte,
            \item für den Fall, dass \(8\) weitere Haushalte an das Abwassersystem angeschlossen werden.
        \end{enumerate}
        \emph{Hinweis: Beachten Sie, dass unter den gegebenen Voraussetzungen gemäß Vorlesung gilt
        \[
            \sum_{i = 1}^n X_i \sim N\parentheses*{n\mu, n\sigma^2}.
        \]}
    \end{problem}
    
    \subsection*{Lösung}
    Gemäß Voraussetzung wird die gesamte Abwassermenge von \(n\) angeschlossenen Haushalten beschrieben durch
    \[
        S = \sum_{i = 1}^n X_i.
    \]
    Da \(X_1, \ldots, X_n\) nach Voraussetzung stochastisch unabhängig sind, folgt
    \[
        S = \sum_{i = 1}^n X_i = n\bar{X}_n \sim N\parentheses*{n\mu, n\sigma^2}
    \]
    Hieraus folgt weiter
    \[
        \frac{S - n\mu}{\sqrt{n\sigma^2}} \sim\parentheses*{0, 1}.
    \]
    Mit \(c = 13500\) und \(\Phi\) als Verteilungsfunktion zu \(N\parentheses*{0, 1}\) (sowie mit \(P\) als zugrundeliegender Wahrscheinlichkeitsverteilung) erhält man dann
    \[
        P\parentheses*{S > c} = 1 - P\parentheses*{S \le c} = 1 - P\parentheses*{\frac{S - n\mu}{\sqrt{n\sigma^2}} \le \frac{c - n\mu}{\sqrt{n\sigma^2}}} = 1 - \Phi\parentheses*{\frac{c - n\mu}{\sqrt{n\sigma^2}}}.
    \]
    Einsetzen der Zahlenwerte:
    \begin{enumerate}
        \item \(n = 1332\), \(\mu = 10\), \(\sigma^2 = 4\), \(c = 13500\)
        \[
            P\parentheses*{S > c} = 1 - \Phi\parentheses*{\frac{13500 - 1332 \cdot 10}{\sqrt{1332 \cdot 4}}} \approx 1 - \Phi\parentheses*{2,47} \approx 1 - 0,993 = 0,007 = 0,7\%,
        \]
        \item \(n = 1340\), \(\mu = 10\), \(\sigma^2 = 4\), \(c = 13500\)
        \[
            P\parentheses*{S > c} = 1 - \Phi\parentheses*{\frac{13500 - 1340 \cdot 10}{\sqrt{1340 \cdot 4}}} \approx 1 - \Phi\parentheses*{1,37} \approx 1 - 0,915 = 0,085 = 8,5\%.
        \]
    \end{enumerate}


    \section*{Aufgabe 4}
    
    \begin{problem}
        Gegeben sei eine mit Parameter \(\lambda = 1\) exponentialverteilte Zufallsvariable (kurz: \(X \sim \Exp\parentheses*{1}\)).
        Bestimmen Sie mithilfe der Transformationsformel für Dichtefunktionen eine Riemann-Dichte \(f^Y\) der Zufallsvariablen \(Y = e^{-X}\).
    \end{problem}
    
    \subsection*{Lösung}
    Vor Anwendung der Transformationsformel für Dichtefunktionen müssen zunächst sämtliche zugehörige Voraussetzungen verifiziert werden.

    Gemäß Aufgabenstellung ist eine Riemann-Dichte \(f^X: \R \to \left[0, \infty\right)\) von \(X\) gegeben durch
    \[
        f^X\parentheses*{x} = \begin{cases}
            e^{-x}, & \text{falls }x > 0,\\
            0, & \text{falls }x \le 0.
        \end{cases}
    \]
    Insbesondere ist hiermit
    \[
        f^X\parentheses*{x} > 0\text{ für }x \in \parentheses*{a, b} = \parentheses*{0, \infty} \quad \text{und} \quad f^X\parentheses*{x} = 0\text{ sonst}.
    \]
    Weiterhin ist \(Y = e^{-X} = g\parentheses*{X}\) mit \(g: \parentheses*{0, \infty} \to \parentheses*{0, 1}\), definiert durch
    \[
        g\parentheses*{x} = e^{-x}, \quad x \in \parentheses*{0, \infty}.
    \]
    Hieraus folgt, dass die Funktion \(g: \parentheses*{0, \infty} \to \parentheses*{0, 1}\) bijektiv ist mit Umkehrfunktion \(g^{-1}: \parentheses*{0, 1} \to \parentheses*{0, \infty}\), die gegeben ist durch
    \[
        g^{-1}\parentheses*{y} = -\ln\parentheses*{y}, \quad y \in \parentheses*{0, 1}.
    \]
    Diese Umkehrfunktion \(g^{-1}\) ist stetig differenzierbar auf \(\parentheses*{0, 1}\) mit
    \[
        \parentheses*{g^{-1}}'\parentheses*{y} = -\frac{1}{y}, \quad y \in \parentheses*{0, 1}.
    \]
    (Wegen \(\parentheses*{g^{-1}}'\parentheses*{y} \ne 0, y \in \parentheses*{0, 1}\) ist damit auch \(g\) stetig differenzierbar auf \(\parentheses*{0, \infty}\).)

    Damit sind alle Voraussetzungen erfüllt und hiermit folgt für \(y \in \parentheses*{x, d} = \parentheses*{0, 1}\)
    \[
        f^Y\parentheses*{y} = \absolute*{\parentheses*{g^{-1}}'\parentheses*{y}}f^X\parentheses*{g^{-1}\parentheses*{y}} = \absolute*{-\frac{1}{y}}e^{-\parentheses*{-\ln\parentheses*{y}}} = \frac{y}{y} = 1.
    \]
    Insgesamt erhält man damit aus der vorherigen Gleichung und der Transformationsformel
    \[
        f^Y\parentheses*{y} = \begin{cases}
            1, & \text{falls }y \in \parentheses*{0, 1},\\
            0, & \text{falls }y \in \R \setminus \parentheses*{0, 1}.
        \end{cases}
    \]
    Aus der Vorlesung ergibt sich daraus \(Y \sim R\parentheses*{0, 1}\).
\end{document}
