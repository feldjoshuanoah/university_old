\documentclass{exercise}

\institute{Institut für Statistik und Wirtschaftsmathematik}
\title{Präsenzübung 8}
\author{Joshua Feld, 406718}
\course{Statistik}
\professor{Cramer}
\semester{Sommersemester 2022}
\program{CES (Bachelor)}

\begin{document}
    \maketitle


    \section*{Aufgabe 1}
    
    \begin{problem}
        Es bezeichnen X und Y die zufäälligen Anzahlen der Kunden, die während einer Stunde an zwei Schaltern \(A\) bzw. \(B\) ankommen.
        Hierbei werde angenommen, dass die beiden Zufallsvariablen \(X\) und \(Y\) stochastisch unabhängig und jeweils mit Parameter \(\lambda = 10\) Poisson-verteilt sind.
        Dann gibt \(Z = X + Y\) die Gesamtzahl der Kunden an, die während einer Stunde an den beiden Schaltern \(A\) bzw. \(B\) ankommen.
        Berechnen Sie:
        \begin{enumerate}
            \item \(P\parentheses*{19 \le Z \le 21}\),
            \item \(E\parentheses*{Z}\).
        \end{enumerate}
        Hierbei bezeichnet \(P\) die zugrundeliegende Wahrscheinlichkeitsverteilung.
    \end{problem}
    
    \subsection*{Lösung}
    \begin{enumerate}
        \item Nach Voraussetzung sind \(X\) und \(Y\) stochastisch unabhängig.
        Folglich ist \(Z = X + Y\) Poisson-verteilt mit Parameter \(\mu = 2\lambda = 20\).
        Es folgt
        \[
            P\parentheses*{Z = k} = \frac{\mu^k}{k!}e^{-\mu} = \frac{20^k}{k!}e^{-20}, \quad k \in \N_0.
        \]
        Damit erhält man
        \[
            P\parentheses*{19 \le Z \le 20} = P\parentheses*{Z = 19} + P\parentheses*{Z = 20} + P\parentheses*{Z = 21} = \parentheses*{\frac{20^{19}}{19!} + \frac{20^{20}}{20!} + \frac{20^{21}}{21!}}e^{-20} \approx 0,262.
        \]
        \item Zunächst gilt gemäß Vorlesung mit \(\lambda = 10\)
        \[
            E\parentheses*{X} = E\parentheses*{Y} = \lambda = 10.
        \]
        Hiermit folgt
        \[
            E\parentheses*{Z} = E\parentheses*{X + Y} = E\parentheses*{X} + E\parentheses*{Y} = 2\lambda = 20.
        \]
    \end{enumerate}


    \section*{Aufgabe 2}
    
    \begin{problem}
        Die Dichtefunktion \(f^{\parentheses*{X, Y}}: \R^2 \to \R\) des zweidimensionalen stetigen Zufallsvektors \(\parentheses*{X, Y}\) sei gegeben durch
        \[
            f^{\parentheses*{X, Y}}\parentheses*{x, y} = \begin{cases}
                \frac{1}{36}, & \text{falls }0 \le x \le 12\text{ und }5 \le y \le 8,\\
                0, & \text{sonst}.
            \end{cases}
        \]
        \begin{enumerate}
            \item Bestimmen Sie Randdichten \(f^X\) und \(f^Y\) von \(X\) bzw. \(Y\) und entscheiden Sie (mit Begründung), ob die Zufallsvariablen \(X\) und \(Y\) stochastisch unabhängig sind.
            \item Bestimmen Sie die Verteilungsfunktionen \(F^X\) und \(F^Y\) von \(X\) bzw. \(Y\).
            \item Bestimmen Sie die Verteilungsfunktion \(F^{\parentheses*{X, Y}}\) des Zufallsvektors \(\parentheses*{X, Y}\).
            \item Berechnen Sie \(\Korr\parentheses*{X, Y}\).
            \item Berechnen Sie \(E\parentheses*{\frac{X}{Y^2}}\).
        \end{enumerate}
    \end{problem}
    
    \subsection*{Lösung}
    \begin{enumerate}
        \item Zunächst bestimmen wir \(f^X\):
        \begin{itemize}
            \item Fall 1: \(x < 0\) oder \(x > 12\):
            \[
                f^X\parentheses*{x} = \int_{-\infty}^\infty\underbrace{f^{\parentheses*{X, Y}}\parentheses*{x, y}}_{= 0\text{, da }x \not\in \brackets*{0, 12}}\d y = \int_{-\infty}^\infty 0\d y = 0.
            \]
            \item Fall 2: \(0 \le x \le 12\):
            \[
                f^X\parentheses*{x} = \int_{-\infty}^\infty f^{\parentheses*{X, Y}}\parentheses*{x, y}\d y = \int_5^8 \frac{1}{36}\d y = \frac{8 - 5}{36} = \frac{1}{12}.
            \]
        \end{itemize}
        Nun bestimmen wir noch \(f^Y\):
        \begin{itemize}
            \item Fall 1: \(y < 5\) oder \(y > 8\):
            \[
                f^Y\parentheses*{y} = \int_{-\infty}^\infty\underbrace{f^{\parentheses*{X, Y}}\parentheses*{x, y}}_{= 0\text{, da }y \not\in \brackets*{5, 8}}\d y = \int_{-\infty}^\infty 0\d y = 0.
            \]
            \item Fall 2: \(5 \le y \le 8\):
            \[
                f^Y\parentheses*{y} = \int_{-\infty}^\infty f^{\parentheses*{X, Y}}\parentheses*{x, y}\d y = \int_0^12 \frac{1}{36}\d y = \frac{12}{36} = \frac{1}{3}.
            \]
        \end{itemize}
        Insgesamt erhält man
        \begin{align}
            f^X\parentheses*{x} &= \begin{cases}
                \frac{1}{12}, & \text{falls }x \in \brackets*{0, 12},\\
                0, & \text{sonst},
            \end{cases}\label{eq:1}\\
            f^Y\parentheses*{y} &= \begin{cases}
                \frac{1}{3}, & \text{falls }y \in \brackets*{5, 8},\\
                0, & \text{sonst}.
            \end{cases}\label{eq:2}
        \end{align}
        Der Vergleich von \eqref{eq:1} bzw. \eqref{eq:2} mit Definition 5 aus Vorlesung 7 zeigt
        \[
            X \sim R\parentheses*{0, 12} \quad \text{und} \quad Y \sim R\parentheses*{5, 8}.
        \]
        Weiter folgt für \(x, y \in \R\):
        \begin{equation}\label{eq:3}
            f^X\parentheses*{x}f^Y\parentheses*{y} = \begin{cases}
                \frac{1}{36}, & \text{falls }0 \le x \le 12\text{ und }5 \le y \le 8,\\
                0, & \text{sonst}
            \end{cases} = f^{\parentheses*{X, Y}}\parentheses*{x, y}.
        \end{equation}
        Gemäß \eqref{eq:3} und Satz C3.14 sind \(X\) und \(Y\) stochastisch unabhängig.
        \item Wir bestimmen zunächst \(F^X\) mittels \(f^X\):
        \begin{itemize}
            \item Fall 1: \(x < 0\):
            \[
                F^X\parentheses*{x} = \int_{-\infty}^x f^X\parentheses*{t}\d t = \int_{-\infty}^x 0\d t = 0.
            \]
            \item Fall 2: \(0 \le x \le 12\):
            \[
                F^X\parentheses*{x} = \int_{-\infty}^x f^X\parentheses*{t}\d t = \int_0^x \frac{1}{12}\d t = \frac{1}{12}x.
            \]
            \item Fall 3: \(x > 12\):
            \[
                F^X\parentheses*{x} = \int_{-\infty}^x f^X\parentheses*{t}\d t = \int_0^{12}\frac{1}{12}\d t = 1.
            \]
        \end{itemize}
        Nun können wir noch \(F^Y\) mittels \(f^Y\) bestimmen:
        \begin{itemize}
            \item Fall 1: \(y < 5\):
            \[
                F^Y\parentheses*{y} = \int_{-\infty}^y f^Y\parentheses*{t}\d t = \int_{-\infty}^y 0\d t = 0.
            \]
            \item Fall 2: \(5 \le y \le 8\):
            \[
                F^Y\parentheses*{y} = \int_{-\infty}^y f^Y\parentheses*{t}\d t = \int_5^y \frac{1}{3}\d t = \frac{1}{3}\parentheses*{y - 5}.
            \]
            \item Fall 3: \(y > 8\):
            \[
                F^Y\parentheses*{y} = \int_{-\infty}^y f^Y\parentheses*{t}\d t = \int_5^8 \frac{1}{3}\d t = 1.
            \]
        \end{itemize}
        Insegsamt erhalten wir
        \begin{align}
            F^X\parentheses*{x} &= \begin{cases}
                0, & \text{falls }x < 0,\\
                \frac{1}{12}x, & \text{falls }0 \le x \le 12,\\
                1, & \text{falls }x > 12,
            \end{cases}\label{eq:4}\\
            F^Y\parentheses*{y} &= \begin{cases}
                0, & \text{falls }y < 5,\\
                \frac{1}{12}\parentheses*{y - 5}, & \text{falls }5 \le y \le 8,\\
                1, & \text{falls }y > 8.
            \end{cases}\label{eq:5}
        \end{align}
        Auch hier zeigt der Vergleich von \eqref{eq:4} bzw. \eqref{eq:5} mit Definition 5 der siebten Vorlesung das bereits aus a) bekannte Ergebnis
        \[
            X \sim R\parentheses*{0, 12} \quad \text{und} \quad Y \sim R\parentheses*{5, 8}.
        \]
        Wenn diese Verteilungen von \(X\) bzw. \(Y\) bekannt sind und verwendet werden dürfen, können alternativ zur obigen Herleitung die Darstellungen \eqref{eq:4} und \eqref{eq:5} von \(F^X\) bzw. \(F^Y\) direkt mithilfe von Definition 5 der siebten Vorlesung angegeben werden.
        \item Aufgrund der in a) festgestellten stochastischen Unabhängigkeit von \(X\) und \(Y\) gilt zunächst mit Satz C 3.6:
        \[
            F^{\parentheses*{X, Y}}\parentheses*{x, y} = F^X\parentheses*{x}F^Y\parentheses*{y}, \quad x, y \in \R.
        \]
        \begin{itemize}
            \item Fall 1: \(x < 0\) oder \(y < 5\):
            \[
                F^{\parentheses*{X, Y}}\parentheses*{x, y} = F^X\parentheses*{x}F^Y\parentheses*{y} = 0.
            \]
            \item Fall 2: \(x \in \brackets*{0, 12}\) und \(y \in \brackets*{5, 8}\):
            \[
                F^{\parentheses*{X, Y}}\parentheses*{x, y} = F^X\parentheses*{x}F^Y\parentheses*{y} = \frac{1}{12}x \cdot \frac{1}{3}\parentheses*{y - 5} = \frac{1}{36}x\parentheses*{y - 5}.
            \]
            \item Fall 3: \(x \in \brackets*{0, 12}\) und \(y > 8\):
            \[
                F^{\parentheses*{X, Y}}\parentheses*{x, y} = F^X\parentheses*{x}F^Y\parentheses*{y} = \frac{1}{12}x \cdot 1 = \frac{1}{12}x.
            \]
            \item Fall 4: \(x > 12\) und \(y \in \brackets*{5, 8}\):
            \[
                F^{\parentheses*{X, Y}}\parentheses*{x, y} = F^X\parentheses*{x}F^Y\parentheses*{y} = 1 \cdot \frac{1}{3}\parentheses*{y - 5} = \frac{1}{3}\parentheses*{y - 5}.
            \]
            \item Fall 5: \(x > 12\) und \(y > 8\):
            \[
                F^{\parentheses*{X, Y}}\parentheses*{x, y} = F^X\parentheses*{x}F^Y\parentheses*{y} = 1 \cdot 1 = 1.
            \]
        \end{itemize}
        Insgesamt erhält man
        \[
            F^{\parentheses*{X, Y}}\parentheses*{x, y} = \begin{cases}
                0, & \text{falls }x < 0\text{ oder }y < 5,\\
                \frac{1}{36}x\parentheses*{y - 5}, & \text{falls }x \in \brackets*{0, 12}\text{ und }y \in \brackets*{5, 8},\\
                \frac{1}{12}x, & \text{falls }x \in \brackets*{0, 12}\text{ und }y > 8,\\
                \frac{1}{3}\parentheses*{y - 5}, & \text{falls }x > 12\text{ und }y \in \brackets*{5, 8},\\
                1, & \text{falls }x > 12\text{ und }y > 8.
            \end{cases}
        \]
        \item Gemäß a) sind \(X\) und \(Y\) stochastisch unabhängig und damit auch unkorreliert.
        Mit \(\Kov\parentheses*{X, Y} = 0\) und der Definition von \(\Korr\parentheses*{X, Y}\) folgt
        \[
            \Korr\parentheses*{X, Y} = 0.
        \]
        \item Unter Anwendung von Satz 5.6 mit \(g: \R^2 \to \R, \parentheses*{x, y} \mapsto g\parentheses*{x} = \frac{x}{y^2}\) erhält man
        \begin{align*}
            E\parentheses*{\frac{X}{Y^2}} &= \int_{-\infty}^\infty \int_{-\infty}^\infty \frac{x}{y^2}f^{\parentheses*{X, Y}}\parentheses*{x, y}\d x\d y\\
            &= \int_5^8 \int_0^{12}\frac{x}{y^2}\frac{1}{36}\d x\d y\\
            &= \frac{1}{36}\int_5^8 y^{-2}\parentheses*{\int_0^{12}x\d x}\d y\\
            &= \frac{1}{36}\int_5^8 y^{-2}\brackets*{\frac{x^2}{2}}_{x = 0}^{x = 12}\d y\\
            &= \frac{1}{36}\int_5^8 72y^{-2}\d y\\
            &= 2\int_5^8 y^{-2}\d y\\
            &= 2\brackets*{-y^{-1}}_{y = 5}^{y = 8}\\
            &= 2 \cdot \frac{3}{40} = \frac{3}{20}.
        \end{align*}
    \end{enumerate}


    \section*{Aufgabe 3}
    
    \begin{problem}
        Es seien \(X\) und \(Y\) stochastisch unabhängige Zufallsvariablen mit \(X \sim R\parentheses*{-2, 4}\) und \(Y \sim \Exp\parentheses*{0,5}\).
        (Somit ist \(X\) (stetig) gleichverteilt auf dem Intervall \(\brackets*{-2, 4}\) und \(Y\) exponentialverteilt mit Parameter \(\lambda = 0,5\).)
        Weiter sei \(V = 3X + Y - 5\).
        \begin{enumerate}
            \item Berechnen Sie \(E\parentheses*{V}\).
            \item Berechnen Sie \(\Var\parentheses*{V}\).
            \item Berechnen Sie \(\Kov\parentheses*{V, X}\).
        \end{enumerate}
    \end{problem}
    
    \subsection*{Lösung}
    Zunächst gilt gemäß Beispiel 1, 3) und 4) der fünfzehnten Vorlesung sowie Beispiel 3, 3) und 4) selbiger Vorlesung:
    \begin{align*}
        E\parentheses*{X} &= \frac{-2 + 4}{2} = 1, & \Var\parentheses*{X} &= \frac{\parentheses*{4 - \parentheses*{-2}}^2}{12} = 3,\\
        E\parentheses*{Y} &= \frac{1}{0,5} = 2, & \Var\parentheses*{Y} &= \frac{1}{0,5^2} = 4.
    \end{align*}
    \begin{enumerate}
        \item Mit den Rechenregeln für Erwartungswerte folgt
        \[
            E\parentheses*{V} = E\parentheses*{3X + Y - 5} = 3E\parentheses*{X} + E\parentheses*{Y} - 5 = 3 + 2 - 5 = 0.
        \]
        \item Mit den Rechenregeln für Varianzen erhält man
        \[
            \Var\parentheses*{V} = \Var\parentheses*{3X + Y - 5} = \Var\parentheses*{3X} + \Var\parentheses*{Y} = 3^2 \cdot \Var\parentheses*{X} + \Var\parentheses*{Y} = 9 \cdot 3 + 4 = 31.
        \]
        \item Mit den Rechenregeln für Kovarianzen erhält man
        \[
            \Kov\parentheses*{V, X} = \Kov\parentheses*{3X + Y - 5, X} = 3\underbrace{\Kov\parentheses*{X, X}}_{= \Var\parentheses*{X}} + \underbrace{\Kov\parentheses*{Y, X}}_{= 0\text{, da }X, Y\text{ unabhängig}} = 3\Var\parentheses*{X} = 3 \cdot 3 = 9.
        \]
    \end{enumerate}
\end{document}
