\documentclass{exercise}

\institute{Institut für Statistik und Wirtschaftsmathematik}
\title{Präsenzübung 9}
\author{Joshua Feld, 406718}
\course{Statistik}
\professor{Cramer}
\semester{Sommersemester 2022}
\program{CES (Bachelor)}

\begin{document}
    \maketitle


    \section*{Aufgabe 1}
    
    \begin{problem}
        Der Betreiber eines Getränkestands auf einem alljährlich stattfindenden Frühjahrsfest setzt morgens \(200\sis{\liter}\) Bowle an, die er in \(0,2\sis{\liter}\)-Bechern verkauft.
        
        Aus langjähriger Erfahrung weiß er, dass die Anzahl der Bowle-Becher, die ein Kunde an seinem Stand trinkt, als (diskrete) Zufallsvariable mit Erwartungswert \(\mu = 2,6\) und Varianz \(\sigma^2 = 3\) aufgefasst werden kann.
        (``Im Mittel'' sollte der Vorrat also für \(\frac{1000}{2,6} \approx 384\) Kunden reichen.)
        
        Nehmen Sie bei der Bearbeitung der folgenden Aufgabenteile an, dass der Bowle-Konsum der einzelnen Kunden des Getränkestands unabhängig voneinander erfolgt.
        \begin{enumerate}
            \item Wie groß ist (approximativ) die Wahrscheinlichkeit dafür, dass der Vorrat sogar für \(400\) Kunden ausreicht?
            \item Wie groß ist (approximativ) die Wahrscheinlichkeit dafür, dass der Vorrat \emph{nicht} ausreicht, obwohl nur \(360\) Kunden den Stand besuchen?
            \item Wie groß muss die Anzahl der Kunden mindestens sein, damit die (approximativ berechnete) Wahrscheinlichkeit dafür, dass der Betreiber abends noch einen (zu entsorgenden) Restbestand hat, höchstens \(0,01\) beträgt?
        \end{enumerate}
    \end{problem}
    
    \subsection*{Lösung}
    Der Bowle-Konsum an dem Getränkestand kann beschrieben werden durch identisch verteilte Zufallsvariablen \(X_1, \ldots, X_n\) mit folgender Interpretation:
    \[
        X_i = \text{Anzahl der Bowle-Becher des }i\text{-ten Kunden}, \quad i \in \braces*{1, \ldots, n}.
    \]
    Zusätzliche Annahme (gemäß Aufgabenstellung): \(X_1, \ldots, X_n\) sind stochastisch unabhängig.
    (Hiermit wird angenommen, dass der Bowle-Konsum der einzelnen Kunden unabhängig voneinander erfolgt!)
    Weiter gilt gemäß Aufgabenstellung:
    \[
        \mu = E\parentheses*{X_1} = 2,6, \quad \sigma^2 = \Var\parentheses*{X_1} = 3.
    \]
    Die Gesamtzahl der Becher, die von den \(n\) Kunden gekauft werden, ist dann gegeben durch
    \[
        S_n = \sum_{i = 1}^n X_i.
    \]
    In den folgenden Lösung ist jeweils zu beachten, dass die angesetzte Bowle-Menge von \(200\sis{\liter}\) der Anzahl von \(1000\) Bechern à \(0,2\sis{\liter}\) entspricht.
    \begin{enumerate}
        \item Da der Stichprobenumfang \(n = 400\) sehr hoch ist (und die sonstigen Voraussetzungen erfüllt sind), kann der zentrale Grenzwertsatz angewendet werden.
        Man erhält mit \(\Phi\) als Bezeichnung für die Verteilungsfunktion der Standard-Normalverteilung:
        \begin{align*}
            P\parentheses*{S_n \le 1000} &= P\parentheses*{\frac{S_n - n\mu}{\sqrt{n\sigma^2}} \le \frac{1000 - n\mu}{\sqrt{n\sigma^2}}}\\
            &\approx \Phi\parentheses*{\frac{1000 - n\mu}{\sqrt{n\sigma^2}}}\\
            &= \Phi\parentheses*{\frac{1000 - 400 \cdot 2,6}{\sqrt{400 \cdot 3}}}\\
            &\approx \Phi\parentheses*{-1,15}\\
            &= 1 - \Phi\parentheses*{1,15}\\
            &\approx 1 - 0,875 = 0,125
        \end{align*}
        \item Wiederum kann der zentrale Grenzwertsatz angewendet werden, da auch der Stichprobenumfang \(n = 360\) hinreichend groß ist.
        Man erhält analog zu a):
        \begin{align*}
            P\parentheses*{S_n > 1000} &= 1 - P\parentheses*{S_n \le 1000}\\
            &= 1 - P\parentheses*{\frac{S_n - n\mu}{\sqrt{n\sigma^2}} \le \frac{1000 - n\mu}{\sqrt{n\sigma^2}}}\\
            &\approx 1 - \Phi\parentheses*{\frac{1000 - n\mu}{\sqrt{n\sigma^2}}}\\
            &= 1 - \Phi\parentheses*{\frac{1000 - 360 \cdot 2,6}{\sqrt{360 \cdot 3}}}\\
            &\approx 1 - \Phi\parentheses*{1,95}\\
            &\approx 1- 0,974 = 0,026.
        \end{align*}
        \item Auch hierbei kann der zentrale Grenzwertsatz angewendet werden, da der resultierende Stichprobenumfang hinreichend hoch sein wird.
        Zunächst gilt für \(n \in \N\) analog zu a) bzw. b):
        \[
            P\parentheses*{S_n < 1000} = P\parentheses*{S_n \le 999} = P\parentheses*{\frac{S_n - n\mu}{\sqrt{n\sigma^2}} \le \frac{999 - n\mu}{\sqrt{n\sigma^2}}} \approx \Phi\parentheses*{\frac{999 - n\mu}{\sqrt{n\sigma^2}}}.
        \]
        Mit dieser Approximaziom folgt für \(n \in \N\) mit \(t = 999\) (als Abkürzung):
        \begin{align}
            P\parentheses*{S_n < 1000} \le 0,01 &\iff \Phi\parentheses*{\frac{t - n\mu}{\sqrt{n\sigma^2}}} \le 0,01\nonumber\\
            &\iff \frac{t - n\mu}{\sqrt{n}\sigma} \le \Phi^{-1}\parentheses*{0,01} = c\nonumber\\
            &\iff t - n\mu \le c\sigma\sqrt{n}\nonumber\\
            &\iff \underbrace{n + \frac{c\sigma}{\mu}\sqrt{n} + \frac{t}{\mu} \ge 0}_{\text{Quadratische Ungleichung in }\sqrt{n}}\nonumber\\
            &\iff \parentheses*{\sqrt{n}}^2 + 2\frac{c\sigma}{2\mu}\sqrt{n} + \parentheses*{\frac{c\sigma}{2\mu}}^2 \ge \frac{t}{\mu} + \parentheses*{\frac{c\sigma}{2\mu}}^2\nonumber\\
            &\iff \parentheses*{\sqrt{n} + \frac{c\sigma}{2\mu}}^2 \ge \underbrace{\frac{4t\mu + c^2 \sigma^2}{4\mu^2}}_{> 0\text{, da }t > 0\text{ und }\mu > 0}\nonumber\\
            &\iff \absolute*{\sqrt{n} + \frac{c\sigma}{2\mu}} \ge \sqrt{\frac{4t\mu + c^2 \sigma^2}{4\mu^2}} = \frac{\sqrt{4t\mu + c^2 \sigma^2}}{2\mu}\nonumber\\
            &\iff \sqrt{n} + \frac{c\sigma}{2\mu} \ge \frac{\sqrt{4t\mu + c^2 \sigma^2}}{2\mu}\text{ oder }\sqrt{n} + \frac{c\sigma}{2\mu} \le -\frac{\sqrt{4t\mu + c^2 \sigma^2}}{2\mu}\nonumber\\
            &\iff \sqrt{n} \ge \frac{-c\sigma + \sqrt{4t\mu + c^2 \sigma^2}}{2\mu}\text{ oder }\sqrt{n} \le \frac{-c\sigma - \sqrt{4t\mu + c^2 \sigma^2}}{2\mu}\label{eq:1}
        \end{align}
        Mit \(\mu = 2,6\), \(\sigma = \sqrt{3}\) und \(c = \Phi^{-1}\parentheses*{0,01} \approx -2,326\) erhält man
        \begin{align}
            \frac{-c\sigma + \sqrt{4t\mu + c^2 \sigma^2}}{2\mu} & \approx \frac{-\parentheses*{-2,326} \cdot \sqrt{3} + \sqrt{4 \cdot 999 \cdot 2,6 + \parentheses*{-2,326}^2 \cdot 3}}{2 \cdot 2,6} \approx 20,392,\label{eq:2}\\
            \frac{-c\sigma - \sqrt{4t\mu + c^2 \sigma^2}}{2\mu} & \approx \frac{-\parentheses*{-2,326} \cdot \sqrt{3} - \sqrt{4 \cdot 999 \cdot 2,6 + \parentheses*{-2,326}^2 \cdot 3}}{2 \cdot 2,6} \approx -18,842.\label{eq:3}
        \end{align}
        Wegen \(\sqrt{n} \ge 0\) ist gemäß \eqref{eq:2} und \eqref{eq:3} nur die erste Ungleichung von \eqref{eq:1} erfüllbar.
        Es folgt daher
        \begin{equation}\label{eq:4}
            P\parentheses*{S_n < 1000} \le 0,01 \iff \sqrt{n} \ge \frac{-c\sigma + \sqrt{4t\mu + c^2 \sigma^2}}{2\mu} \approx 20,392 \iff n \ge 20,392^2 \approx 415,834.
        \end{equation}
        Mit \(n \in \N\) erhält man somit aus \eqref{eq:4} die Lösung \(n = 416\), d.h.: Die Anzahl der Kunden an dem Getränkestand muss mindestens \(416\) betragen, damit die (approximativ berechnete) Wahrscheinlichkeit dafür, dass der Betreiber abends noch einen (zu entsorgenden) Restbestand hat, höchstens \(0,01\) beträgt.
    \end{enumerate}
    
    
    \section*{Aufgabe 2}
    
    \begin{problem}
        Zu gegebenem \(n \in \N\) seien \(X_1, \ldots, X_n\) stochastisch unabhängige Zufallsvariablen, die jeweils auf dem Intervall \(\brackets*{0, b}\) (stetig) gleichverteilt seien.
        Hierbei sei die obere Intervallgrenze \(b > 0\) unbekannt.
        Zur Schätzung von \(b\) betrachtet man folgende Schätzfunktion:
        \[
            \hat{b} = 2\bar{X}_n = \frac{2}{n}\sum_{i = 1}^n X_i.
        \]
        \begin{enumerate}
            \item Ist \(\hat{b}\) ein erwartungstreuer Schätzer für den Parameter \(b\)?
            \item Bestimmen Sie den mittleren quadratischen Fehler \(\MSE_b\parentheses*{\hat{b}}\) des Schätzers \(\hat{b}\) für \(b > 0\).
        \end{enumerate}
    \end{problem}
    
    \subsection*{Lösung}
    \begin{enumerate}
        \item Gemäß Aufgabenstellung sind die Zufallsvariablen \(X_1, \ldots, X_n\) jeweils (stetig) gleichverteilt auf dem Intervall \(\brackets*{0, b}\) (kurz: \(X_i \sim R\parentheses*{0, b}\) für \(i \in \braces*{1, \ldots, n}\)).
        Hierbei ist der Parameter \(b > 0\) unbekannt.
        Damit erhält man mit Beispiel 1, 3) der fünfzehnten Vorlesung für \(b > 0\):
        \[
            E_b\parentheses*{X_i} = \frac{b}{2}, \quad i \in \braces*{1, \ldots, n}.
        \]
        Hiermit folgt für \(b > 0\):
        \begin{equation}\label{eq:5}
            E_b\parentheses*{\hat{b}} = E_b\parentheses*{\frac{2}{n}\sum_{i = 1}^n X_i} = \frac{2}{n}\sum_{i = 1}^n \underbrace{E_b\parentheses*{X_i}}_{= \frac{b}{2}} = \frac{2}{n}n\frac{b}{2} = b.
        \end{equation}
        Da Gleichung \eqref{eq:5} für beliebige, also \emph{alle} Parameter \(b > 0\) gilt, ist somit der Schätzer \(\hat{b}\) erwartungstreu für die Schätzung von \(b\) gemäß Definition 1 der achtzehnten Vorlesung.
        \item Zunächst erhält man mit a) für \(b > 0\):
        \[
            \MSE_b\parentheses*{\hat{b}} = E_b\parentheses*{\hat{b} - b}^2 = E_b\parentheses*{\hat{b} - E_b\parentheses*{\hat{b}}}^2 = \Var_b\parentheses*{\hat{b}}.
        \]
        Weiter gilt für \(b > 0\) gemäß Voraussetzung und Beispiel 3, 3) der fünfzehnten Vorlesung:
        \[
            \Var_b\parentheses*{X_i} = \frac{\parentheses*{b - 0}^2}{12} = \frac{b^2}{12}, \quad i \in \braces*{1, \ldots, n}.
        \]
        Hiermit folgt für \(b > 0\):
        \[
            \MSE_b\parentheses*{\hat{b}} = \Var_b\parentheses*{\hat{b}} = \Var_b\parentheses*{\frac{2}{n}\sum_{i = 1}^n X_i} = \frac{4}{n^2}\sum_{i = 1}^n \Var_b\parentheses*{X_i} = \frac{4}{n^2}\sum_{i = 1}^n \frac{b^2}{12} = \frac{4}{n^2}n\frac{b^2}{12} = \frac{b^2}{3n}.
        \]
    \end{enumerate}
    
    
    \section*{Aufgabe 3}
    
    \begin{problem}
        Im Rahmen einer statistischen Untersuchung wurden die folgenden Beobachtungswerte ermittelt:
        \[
            3,8, \quad 2,3, \quad 1,3, \quad 1,7, \quad 1,6, \quad 2,1, \quad -0,2, \quad 0,8.
        \]
        Diese Daten können aufgefasst werden als Realisationen \(x_1, \ldots, x_8\) stochastisch unabhängiger, identisch verteilter Zufallsvariablen \(X_1, \ldots, X_8\) mit unbekannter Verteilungsfunktion \(F\).
        Berechnen Sie aus den gegebenen Beobachtungen einen Schätzwert für \(F\parentheses*{2}\).
    \end{problem}
    
    \subsection*{Lösung}
    Gemäß der Vorlesung ist allgemein für ein \(t \in \R\) ein geeigneter Schätzwert für den Funktionswert \(F\parentheses*{t}\) der (unbekannten) Verteilungsfunktion \(F\) gegeben durch den entsprechenden Funktionswert der \emph{empirischen} Verteilungsfunktion \(\hat{F}_n\parentheses*{t}\).
    Damit erhält man hier mit den gegebenen Beobachtungen \(x_1, \ldots, x_8\) folgenden Schätzwert für \(F\parentheses*{2}\):
    \[
        \hat{F}_8\parentheses*{2} = \underbrace{\frac{1}{8}\sum_{i = 1}^8 \mathbb{I}_{\left(-\infty, 2\right]}\parentheses*{x_i}}_{\substack{\text{Relativer Anteil}\\\text{aller Beob. }\le 2}} = \frac{5}{8} = 0,625.
    \]
\end{document}
