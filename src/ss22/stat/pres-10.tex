\documentclass{exercise}

\institute{Institut für Statistik und Wirtschaftsmathematik}
\title{Präsenzübung 10}
\author{Joshua Feld, 406718}
\course{Statistik}
\professor{Cramer}
\semester{Sommersemester 2022}
\program{CES (Bachelor)}

\begin{document}
    \maketitle


    \section*{Aufgabe 1}
    
    \begin{problem}
        Es seien \(X_1, \ldots, X_n\) stochastisch unabhängige, jeweils \(\betadist\parentheses*{\alpha, 1}\)-verteilte Zufallsvariablen mit \(\alpha > 0\).
        Die zugehörige Dichtefunktion \(f_\alpha\) der Zufallsvariablen \(X_i\) für \(i \in \braces*{1, \ldots, n}\) in Abhängigkeit vom Parameter \(\alpha\) ist dann gemäß Definition 10 der siebten Vorlesung gegeben durch
        \[
            f_\alpha\parentheses*{x} = \begin{cases}
                \alpha x^{\alpha - 1}, & \text{falls }x \in \parentheses*{0, 1},\\
                0, & \text{falls }x \in \R \setminus \parentheses*{0, 1}.
            \end{cases}
        \]
        Bestimmen Sie zu gegebenen Realisationen \(x_1, \ldots, x_n \in \parentheses*{0, 1}\) von \(X_1, \ldots, X_n\) eine Maximum-Likelihood-Schätzung \(\hat{\alpha}\) für den unbekannten Parameter \(\alpha\).
    \end{problem}
    
    \subsection*{Lösung}


    \section*{Aufgabe 2}
    
    \begin{problem}
        Peter nimmt an zehn aufeinanderfolgenden Tagen an einem Glücksspiel mit Gewinnwahrscheinlichkeit \(p \in \parentheses*{0, 1}\) teil.
        Dabei spielt er das Spiel jeden Tag so oft, bis er einmal gewinnt.
        Danach hört er für diesen Tag auf.
        Nach dieser Strategie verfährt er an jedem der zehn Tage.
        Nachfolgend sind die Anzahlen der Spiele angegeben, die Peter an den einzelnen Tagen spielt:
        \[
            13, \quad 7, \quad 10, \quad 2, \quad 11, \quad 17, \quad 15, \quad 9, \quad 19, \quad 11.
        \]
        Berechnen Sie den zugehörigen Maximum-Likelihood-Schätzwert \(\hat{p}\) von \(p\), der sich für die oben angegebenen Anzahlen ergibt.

        \emph{Hinweis: Die gegebenen Anzahlen können aufgefasst werden als Realisationen stochastisch unabhängiger Zufallsvariablen \(X_1, \ldots, X_n\) mit \(Z_i = X_i - 1 \sim \geo\parentheses*{p}\) für \(i \in \braces*{1, \ldots, n}\).}
    \end{problem}
    
    \subsection*{Lösung}


    \section*{Aufgabe 3}
    
    \begin{problem}
        Im Rahmen einer Qualitätskontrolle wurden für \(7\) Energiesparlampen eines bestimmten Fabrikats die folgenden Lebensdauern (in Betriebsstunden) ermittelt:
        \[
            3122, \quad 7784, \quad 3870, \quad 5617, \quad 12771, \quad 6201, \quad 10605.
        \]
        Diese gemessenen Lebensdauern können als Realisationen stochastisch unabhängiger, jeweils \(\Exp\parentheses*{\lambda}\)-verteilter Zufallsvariablen aufgefasst werden mit \(\lambda > 0\).
        Ermitteln Sie ein zweiseitiges \(90\%\)-Konfidenzintervall für den unbekannten Parameter \(\lambda\).
    \end{problem}
    
    \subsection*{Lösung}
\end{document}