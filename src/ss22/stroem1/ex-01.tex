\documentclass{exercise}

\institute{Lehrstuhl für Strömungslehre und Aerodynamisches Institut}
\title{Übung 1}
\author{Joshua Feld, 406718}
\course{Strömungsmechanik I}
\professor{Schröder}
\semester{Sommersemester 2022}
\program{CES (Bachelor)}

\begin{document}
    \maketitle


    \section*{Aufgabe 1}
    
    \begin{problem}
        In zwei übereinandergeschichteten Flüssigkeiten schwebt ein Würfel.
        Bestimmen Sie die Höhe \(h\).
        \begin{center}
            \begin{tikzpicture}
                \draw (0,0) -- (5,0);
                \draw (0,1.5) -- (5,1.5);
                \draw (0,3) -- (5,3);
                \draw[fill=white] (2,1) rectangle (3,2) node[pos=.5] {\(\rho_K\)};
                \draw[<->|] (1.75,1.5) -- (1.75,2) node[midway,left] {\(h\)};
                \draw[|<->|] (2,.75) -- (3,.75) node[midway,below] {\(a\)};
                \node at (4,2.25) {\(\rho_1\)};
                \node at (4,.75) {\(\rho_2\)};
                \draw[->] (5.5,2) -- (5.5,1) node[midway,right] {\(g\)};
            \end{tikzpicture}
        \end{center}
        Gegeben: \(\rho_1\), \(\rho_2\), \(\rho_K\), \(a\)
    \end{problem}
    
    \subsection*{Lösung}
    Der Körper schwebt in den beiden Flüssigkeiten, d.h. es existiert ein Kräftegleichgewicht zwischen Auftriebs- und Gewichtkraft:
    \begin{equation}\label{eq:1}
        F_A = G.
    \end{equation}
    Es gilt
    \[
        F_A = \rho_1 ha^2 g + \rho_2\parentheses*{a - h}a^2 g
    \]
    und
    \[
        G = \rho_K a^3 g.
    \]
    Setzen wir dies in \eqref{eq:1} ein, so erhalten wir
    \[
        h = \frac{\rho_2 - \rho_K}{\rho_2 - \rho_1}a = 6,67 \cdot 10^{-2}\sis{\meter}.
    \]


    \section*{Aufgabe 2}
    
    \begin{problem}
        Eine unten offene Boje der Masse \(m_B\), die mit einem Seil am Boden eines Sees befestigt ist, ragt mit einem Drittel ihrer Höhe aus dem Wasser, wenn das Seil nicht gespannt ist.
        Durch steigenden Wasserstand wird die Boje unter die Wasseroberfläche gezogen und sinkt bei der Eintauchtiefe \(H\) zu Boden.
        Bestimmen Sie \(H\).

        Gegeben: \(m_B\), \(p_a\), \(\rho_W\), \(h\), \(g\), \(\rho_L \ll \rho_W\)

        \emph{Hinweis: Nehmen Sie an, dass die Temperatur im Inneren der Boje konstant ist und die eingeschlossene Luft als ideales Gas betrachtet werden kann. Das Gewicht des Seiles ist zu vernachlässigen.}
    \end{problem}
    
    \subsection*{Lösung}
    Sei \(\tau\) das Volumen der Boje und \(\tau_H\) das eingeschlossene Volumen der Luft.
    In beiden Situationen existiert ein Kräftegleichgewicht
    \[
        gm_B = \frac{1}{3}\tau g\rho_W, \quad gm_B = \tau_H g\rho_W.
    \]
    Aus Gleichsetzen der beiden Gleichgewichtsbedingungen folgt direkt
    \[
        \tau_H = \frac{1}{3}\tau.
    \]
    Da die eingeschlossene Luftmasse konstant ist, folgt
    \[
        \frac{2}{3}\tau\rho_{L_0} = \tau_H\rho_{LH} = \frac{1}{3}\tau\rho_{LH} \iff 2\rho_{L_0} = \rho_{LH}.
    \]
    Das ideale Gasgesetz liefert
    \[
        2\frac{p_a + \frac{1}{3}\rho_W gh}{R_L T_L} = \frac{p_a + \rho_W g\parentheses*{H + \frac{1}{3}h}}{R_L T_L}
    \]
    und wir erhalten schlussendlich durch Umformen
    \[
        H = \frac{p_a}{\rho_W g} + \frac{1}{3}h.
    \]
\end{document}
