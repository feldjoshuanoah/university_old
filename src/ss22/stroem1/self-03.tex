\documentclass{exercise}

\institute{Lehrstuhl für Strömungslehre und Aerodynamisches Institut}
\title{Selbstrechenübung 3}
\author{Joshua Feld, 406718}
\course{Strömungsmechanik I}
\professor{Schröder}
\semester{Sommersemester 2022}
\program{CES (Bachelor)}

\begin{document}
    \maketitle


    \section*{Aufgabe 1}
    
    \begin{problem}
        Ein großes Becken ist mit einem großen geschlossenen Behälter verbunden.
        Der Einlass ist gut gerundet.
        Die Verbindung besteht aus einem gekrümmten Rohr.
        Jedes Teilstück des Rohres mit der Länge \(l\) hat einen Rohrreibungskoeffizienten \(\lambda\).
        Das Rohr dreht sich mit konstanter Winkelgeschwindigkeit \(\omega\).
        Bei der Winkelgeschwindigkeit \(\omega_0\) ist der Massenstrom gerade Null.
        \begin{enumerate}
            \item Berechnen Sie den Druck \(p_i\) im Behälter.
            \item Berechnen Sie den Volumenstrom \(\dot{Q}\) durch das Rohr für \(\omega = 2\omega_0\).
            \item Skizzieren Sie die mechanischen Energieanteile für eine Stromlinie, beginnend an der Oberfläche.
        \end{enumerate}
        Gegeben: \(l, g, d, \rho, \omega_0, p_a, \lambda\)
        
        \emph{Hinweis: \(\frac{\partial p}{\partial r} = \rho\omega^2 r\).}
    \end{problem}
    
    \subsection*{Lösung}
    \begin{enumerate}
        \item Bernoullische Gleichung mit Strömungsverlusten und Energiezufuhr:
        \[
            \rho\frac{v_1^2}{2} + \rho gh_1 + p_1 + \Delta p = \rho\int_1^2 \frac{\partial v}{\partial t}\d s + \rho\frac{v_2^2}{2} + \rho gh_2 + p_2 + \Delta p_v.
        \]
        Für \(\omega_0\) soll \(\dot{m} = 0\) gelten mit \(\dot{m} = \rho Av \implies v = 0\).
        Nun können die Bernoullischen Gleichungen aufgestellt werden, wobei zusätzlich von einer stationären Strömung ausgegangen wird.
        
        Bernoulli von \(1\) nach \(5\) (entspricht der Anwendung der hydrostatischen Grundgleichung):
        \[
            p_5 = p_a + 4\rho gl.
        \]
        Bernoulli von \(5\) nach \(6\):
        \[
            p_6 = p_5 + \frac{1}{2}\rho\parentheses*{\omega_0 l}^2.
        \]
        Insgesamt ergibt sich also
        \[
            p_i = p_6 = p_a + 4\rho gl + \frac{1}{2}\rho\parentheses*{\omega_0 l}^2.
        \]
        Die Addition des Terms \(\frac{1}{2}\rho\parentheses*{\omega_0 l}^2\) auf der Stromaufseite entspricht der Energiezufuhr durch die Rotation und ergibt sich aus der Integration des Hinweises vom Zustand \(5\) nach \(6\).
        \item Bernoulli von \(1\) nach \(6\):
        \[
            p_a + 4\rho gl + \frac{1}{2}\rho \cdot \parentheses*{2\omega_0 l}^2 = \frac{1}{2}\rho v_6^2 + p_i + \frac{1}{2}\rho\parentheses*{v_{2^*}^2 \lambda\frac{l}{2d} + v_{3^*}^2 \lambda\frac{l}{4d} + v_6^2 \lambda\frac{2l}{d}}.
        \]
        Da der Behälter laut Aufgabenstellung groß ist, kann angenommen werden, dass sich der Druck \(p_i\) im Behälter durch das einströmende Wasser nicht ändert.
        Somit kann \(p_i\) aus Aufgabenteil a) eingesetzt werden:
        \begin{align*}
            p_a + 4\rho gl + \frac{1}{2}\rho \cdot \parentheses*{2\omega_0 l}^2 &= \frac{1}{2}\rho v_6^2 + p_a + 4\rho gl + \frac{1}{2}\rho\parentheses*{\omega_0 l}^2 + \frac{1}{2}\rho\parentheses*{v_{2^*}^2 \lambda\frac{l}{2d} + v_{3^*}^2 \lambda\frac{l}{4d} + v_6^2 \lambda\frac{2l}{d}}\\
            \implies 0 &= -\frac{3}{2}\rho\omega_0^2 l^2 + \frac{1}{2}\rho v_6^2 + \frac{1}{2}\rho\parentheses*{v_{2^*}^2 \lambda\frac{l}{2d} + v_{3^*}^2 \lambda\frac{l}{4d} + v_6^2 \lambda\frac{2l}{d}}.
        \end{align*}
        Zur Bestimmung der Geschwindigkeiten wird die Kontinuitätsgleichung benutzt:
        \[
            v_{2^*}\pi d^2 = v_{3^*}\pi \cdot \parentheses*{2d}^2 = v_6 \pi\parentheses*{\frac{d}{2}}^2 \implies v_{2^*} = \frac{v_6}{4}, v_{3^*} = \frac{v_6}{16}.
        \]
        Einsetzen der Geschwindigkeiten ergibt
        \[
            0 = -\frac{3}{2}\rho\omega_0^2 l^2 + \frac{1}{2}\rho v_6^2 \parentheses*{1 + \lambda\frac{l}{d} \cdot \parentheses*{2 + \frac{1}{32} + \frac{1}{1024}}}.
        \]
        Somit folgt für die Ausströmgeschwindigkeit
        \[
            v_6 = \sqrt{\frac{3\omega_0^2 l^2}{1 + \lambda\frac{l}{d} \cdot \frac{2081}{1024}}}.
        \]
        Mit der Ausströmgeschwindigkeit lässt sich nun der gesuchte Volumenstrom bestimmen zu
        \[
            \dot{V} = \frac{\pi d^2}{4}v_6 = \frac{\pi d^2}{4}\sqrt{\frac{3\omega_0^2 l^2}{1 + \lambda\frac{l}{d} \cdot \frac{2081}{1024}}}.
        \]
    \end{enumerate}
    
    
    \section*{Aufgabe 2}
    
    \begin{problem}
        Aus einem großen (\(D \gg d_{1, 2}\)) Industrieofen strömt Abgas (\(\rho_G = \text{konst.}\)) stationär durch zwei Schornsteine mit den Durchmessern \(d_1\) und \(d_2\) in den Höhen \(h_1\) und \(h_2\) in die Atmosphäre.
        In dem Ofen steht die durch den offenen Boden eintretende Luft (\(\rho_L = \text{konst.}, \rho_L > \rho_G\)) bis zur konstanten Höhe \(h_0\).
        \begin{enumerate}
            \item Bestimmen Sie das Verhältnis \(\frac{d_1}{d_2}\), bei dem die Volumenströme in den beiden Schornsteinen gleich groß sind.
            \item Eine Drossel mit dem Drosselverlustbeiwert \(\zeta_{\text{Dr}}\) wird im Schornstein \(2\) eingebaut.
            Bestimmen Sie für \(d_1 = d_2\) den Drosselverlustbeiwert \(\zeta_{\text{Dr}}\), der zu gleich großen Volumenströmen in den Schornsteinen führt.
        \end{enumerate}
        Gegeben: \(h_0, h_1, h_2\)
        
        \emph{Hinweis: Bis auf die Drosseldurchströmung ist die Strömung als reibungsfrei anzusehen.}
    \end{problem}
    
    \subsection*{Lösung}
    \begin{enumerate}
        \item Volumenströme sollen gleich groß sein, d.h.
        \[
            v_{G_1}\frac{\pi d_1^2}{4} = v_{G_2}\frac{\pi d_2^2}{4} \iff \frac{v_{G_1}}{v_{G_2}} = \frac{d_2^2}{d_1^2} \iff \frac{d_1}{d_2} = \sqrt{\frac{v_{G_2}}{v_{G_1}}}.
        \]
        Somit lässt sich das Verhältnis der Durchmesser durch das Verhältnis der Geschwindigkeiten ausdrücken, welche im Folgenden durch die hydrostatische Grundgleichung und Bernoullische Gleichung bestimmt werden.
        
        Bernoulli von \(0\) nach \(k\) (\(k \equiv \text{``Index des jeweiligen Schornsteins''}\)):
        \[
            p_{a_0} = p_{a_k} + \frac{1}{2}\rho_G v_{G_k}^2 + \rho_G g\parentheses*{h_k - h_0}, \quad k = 1, 2.
        \]
        Die Drücke können durch die Anwendung des hydrostatischen Grundgesetzes außerhalb des Ofens bestimmt werden.
        Dies ist möglich, da der Ofen unten offen ist und oben am Austritt der Umgebungsdruck herrschen muss:
        \[
            p_{a_0} = p_{a_k} + \rho_L g\parentheses*{h_k - h_0}.
        \]
        Einsetzen ergibt für die beiden Geschwindigkeiten
        \[
            p_{a_k} + \rho_L g\parentheses*{h_k - h_0} = p_{a_k} + \frac{1}{2}\rho_G v_{G_k}^2 + \rho_G g\parentheses*{h_k - h_0} \iff v_{G_k} = \sqrt{2g\parentheses*{h_k - h_0}\parentheses*{\frac{\rho_L}{\rho_G} - 1}}.
        \]
        Einsetzen der Geschwindigkeiten in das Durchmesserverhältnis ergibt letztlich
        \[
            \frac{d_1}{d_2} = \sqrt[4]{\frac{h_2 - h_0}{h_1 - h_0}}.
        \]
        \item Bernoulli von \(0\) nach \(2\):
        \[
            p_{a_0} = p_{a_2} + \frac{1}{2}\rho_G v_{G_2}^2 \cdot \parentheses*{1 + \zeta_{\text{Dr}}} + \rho_G g\parentheses*{h_2 - h_0}.
        \]
        Aus den Voraussetzungen, dass die Durchmesser sowie Volumenströme beider Schornsteine gleich groß sein sollen, folgt, dass die Geschwindigkeiten gleich groß sein müssen.
        Demnach kann \(v_{G_1}\) aus a) für \(v_{G_2}\) eingesetzt werden.
        Außerdem ist \(p_{a_0}\) aus a) bekannt:
        \begin{align*}
            p_{a_0} &= p_{a_2} + \rho_L g\parentheses*{h_2 - h_0},\\
            v_{G_2} = v_{G_1} &= \sqrt{2g\parentheses*{h_1 - h_0}\parentheses*{\frac{\rho_L}{\rho_G} - 1}}
        \end{align*}
        \[
            \implies p_{a_2} + \rho_L g\parentheses*{h_2 - h_0} = p_{a_2} + \rho_G g\parentheses*{h_1 - h_0}\parentheses*{\frac{\rho_L}{\rho_G} - 1} \cdot \parentheses*{1 + \zeta_{\text{Dr}}} + \rho_G g\parentheses*{h_2 - h_0}.
        \]
        Weiteres Umformen ergibt schließlich die Lösung
        \begin{align*}
            g\parentheses*{\rho_L - \rho_G}\parentheses*{h_2 - h_0} &= \rho_G g\parentheses*{h_1 - h_0}\parentheses*{\frac{\rho_L}{\rho_G} - 1} \cdot \parentheses*{1 + \zeta_{\text{Dr}}}\\
            \iff \parentheses*{\rho_L - \rho_G}\parentheses*{h_2 - h_0} &= \parentheses*{h_1 - h_0}\parentheses*{\rho_L - \rho_G} \cdot \parentheses*{1 + \zeta_{\text{Dr}}}\\
            \iff \frac{h_2 - h_0}{h_1 - h_0} &= 1 + \zeta_{\text{Dr}}\\
            \iff \frac{h_2 - h_0}{h_1 - h_0} - 1 &= \zeta_{\text{Dr}}\\
            \iff \frac{h_2 - h_0 -\parentheses*{h_1 - h_0}}{h_1 - h_0} = \frac{h_2 - h_1}{h_1 - h_0} &= \zeta_{\text{Dr}}.
        \end{align*}
    \end{enumerate}
\end{document}
