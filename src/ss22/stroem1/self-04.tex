\documentclass{exercise}

\institute{Lehrstuhl für Strömungslehre und Aerodynamisches Institut}
\title{Selbstrechenübung 4}
\author{Joshua Feld, 406718}
\course{Strömungsmechanik I}
\professor{Schröder}
\semester{Sommersemester 2022}
\program{CES (Bachelor)}

\begin{document}
    \maketitle


    \section*{Aufgabe 1}
    
    \begin{problem}
        Der Wasserspiegel eines Obersees soll mithilfe einer Pumpe reguliert werden.
        Bei andauernder Trockenheit kann über eine Pumpe Wasser (Volumenstrom \(\dot{V}\)) dem Untersee entnommen werden.
        Dieses strömt dabei durch drei Rohrelemente der Länge \(L\) und dem Durchmesser \(D\).
        Der Rohrreibungsbeiwert sei jeweils \(\lambda\).
        \begin{enumerate}
            \item Bestimmen Sie die erforderliche Pumpenleistung \(P\).
            \item Ein Überlaufen des Sees kann durch Ablassen von Wasser durch einen Ablauf (links) verhindert werden.
            Nach welcher Zeit \(\Delta T\) erreicht das ausströmende Wasser die Hälfte der stationären Endgeschwindigkeit, wenn die Klappe am Ende der Leitung plötzlich geöffnet wird?
            Hierbei sei die Widerstandszahl am scharfkantigen Übergang von \(D_2\) auf \(D_3\) gegeben zu \(\zeta\).
        \end{enumerate}
        Gegeben: \(\dot{V}, \lambda, D_1, D_2, D_3 = 2D_2, L_1, L_2 \gg D_3, L_3 \gg D_3, h, H, \rho, g, p_a, \zeta\)
    \end{problem}
    
    \subsection*{Lösung}
    \begin{enumerate}
        \item Es gilt
        \begin{equation}\label{eq:1}
            P = \dot{V}\Delta p, \quad \text{mit} \quad \Delta p = p_2 - p_1.
        \end{equation}
        Bernoulli von \(0\) nach \(1\):
        \begin{equation}\label{eq:2}
            p_a = p_1 + \frac{\rho}{2}v_1^2 \parentheses*{1 + 2\lambda\frac{L_1}{D_1}} + \rho gL_1 \iff p_1 = p_a - \frac{\rho}{2}v_1^2 \parentheses*{1 + 2\lambda\frac{L_1}{D_1}} - \rho gL_1,
        \end{equation}
        wobei \(v_1\) noch unbekannt ist.
        Bernoulli von \(2\) nach \(3\):
        \begin{equation}
            p_2 + \frac{\rho}{2}v_2^2 = p_3 + \frac{\rho}{2}v_3^2 \parentheses*{1 + \lambda\frac{L_1}{D_1}} \iff p_2 = p_3 + \frac{\rho}{2}v_3^2 \parentheses*{1 + \lambda\frac{L_1}{D_1}} - \frac{\rho}{2}v_2^2,
        \end{equation}
        wobei \(v_2\), \(v_3\) und \(p_3\) noch unbekannt sind.
        Nun verwenden wir die hydrostatische Grundgleichung zur Bestimmung von
        \begin{equation}
            p_3 = p_a + \rho gh.
        \end{equation}
        Aus der Kontinuitätsgleichung
        \[
            \dot{m} = \dot{m}_3 = \dot{m}_2 = \dot{m}_1 = \rho v_1 A_1 = \rho v_1 \frac{\pi}{4}D_1^2
        \]
        folgt mit \(\rho = \text{konst.}\) die unbekannte Geschwindigkeit
        \[
            \dot{V} = \frac{\dot{m}}{\rho} = v_1 \frac{\pi}{4}D_1^2 \iff v_1 = \frac{4\dot{V}}{\pi D_1^2}
        \]
        und, da sich die Punkte \(1\), \(2\) und \(3\) im selben Rohr befinden, gilt
        \begin{equation}\label{eq:5}
            v = v_3 = v_2 = v_1 = \frac{4\dot{V}}{\pi D_1^2}.
        \end{equation}
        Setzen wir nun die Gleichungen \eqref{eq:2} - \eqref{eq:5} in \eqref{eq:1} ein, so folgt
        \[
            P = \dot{V}\parentheses*{\rho g\parentheses*{h + L_1} + 8\rho\parentheses*{\frac{\dot{V}}{\pi D_1^2}}^2\parentheses*{1 + 3\lambda\frac{L_1}{D_1}}}.
        \]
        \item Bernoulli für instationäre Strömungen von \(4\) nach \(6\):
        \begin{equation}\label{eq:6}
            p_a + \rho gH = p_6 + \frac{\rho}{2}v_6^2 + \rho\int_4^6 \frac{\partial v}{\partial t}\d s + \Delta p_V
        \end{equation}
        Da bei \(t = 0\) der Auslauf des Trichters freigegeben wird, gilt \(p_6\parentheses*{t \ge 0} = p_a\).
        Zwischen Punkt \(4\) und Punkt \(6\) existiert somit ein Druckverlust durch Rohrreibung (\(\Delta p_V\))
        \begin{equation}\label{eq:7}
            \Delta p_V = \frac{\rho}{2}v_5^2 \lambda\frac{L_2}{D_2} + \frac{\rho}{2}v_6^2 \lambda\frac{L_3}{D_3} + \frac{\rho}{2}v_5^2 \zeta.
        \end{equation}
        Aus der Kontinuitätsgleichung zwischen \(5\) und \(6\)
        \[
            \dot{m} = \rho\frac{\pi}{4}D_2^2 v_5 = \rho\frac{\pi}{4}D_3^2 v_6
        \]
        folgt mit \(D_3 = 2D_2\)
        \[
            v_5 = 4v_6,
        \]
        was wir in \eqref{eq:7} einsetzen und damit
        \[
            \Delta p_V = \frac{\rho}{2}v_6^2 \frac{\lambda}{D_2}\parentheses*{\frac{L_3}{2} + 16L_2} + \frac{\rho}{2}v_6^2 \cdot 16\zeta,
        \]
        wobei \(v_6\) nun die einzige unbekannte Variable ist.
        Dies setzen wir nun wiederum in \eqref{eq:6} ein.
        Für stationäres Ausströmen gilt \(\frac{\partial v}{\partial t} = 0\) und es folgt
        \[
            v_{6, \text{stat}} = \parentheses*{\frac{2gH}{1 + \frac{\lambda}{D_2}\parentheses*{\frac{L_3}{2} + 16L_2} + 16\zeta}}^{\frac{1}{2}}.
        \]
        Gesucht ist nun das Zeitintervall \(\Delta T\) nach dem \(v_5 = \frac{v_{6, \text{stat}}}\) gilt.
        Für den instationären Term gilt
        \[
            \rho\int_4^6 \frac{\partial v}{\partial t}\d s = \rho\int_{L_2}\frac{\partial v_5}{\partial t}\d s + \rho\int_{L_3}\frac{\partial v_6}{\partial t}\d s = \rho\int_{L_2}\frac{4\d v_6}{\d t}\d s + \rho \int_{L_3}\frac{\d v_6}{\d t}\d s = \rho\parentheses*{4L_2 + L_3}\frac{\d v_6}{\d t}.
        \]
        Setzen wir dies nun in \eqref{eq:6} ein und formen nach den Integrationsvariablen um, so erhalten wir
        \begin{align*}
            p_a + \rho gH &= p_a + \frac{\rho}{2}v
        \end{align*}
    \end{enumerate}
\end{document}
